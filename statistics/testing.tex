% Copyright (c) 2013–2014, authors of "Core Econometrics;" a
% complete list of authors is available in the file AUTHORS.tex.

% Permission is granted to copy, distribute and/or modify this
% document under the terms of the GNU Free Documentation License,
% Version 1.3 or any later version published by the Free Software
% Foundation; with no Invariant Sections, no Front-Cover Texts, and no
% Back-Cover Texts.  A copy of the license is included in the file
% LICENSE.tex and is also available online at
% <http://www.gnu.org/copyleft/fdl.html>.

\section{Hypothesis testing and confidence intervals}

\subsection{Introduction to interval estimation}

\begin{itemize}[leftmargin=0pt]  

\item Note that our discussion of the sampling distribution lets us
  think rigorously about the precision associated with a particular
  estimator of an unknown parameter $\theta$.  But it hasn't helped us
  model uncertainty about the parameter value itself.  We'll see in
  this section that confidence intervals are one way to think
  rigorously about uncertainty of the parameter value, and we'll see
  how to generate confidence intervals from point estimators.

\item Let's introduce some definitions
  \begin{defn}
    An \emph{interval estimator} of a parameter $\theta$ is any pair of
    functions $L(X)$ and $U(X)$ s.t. $L(x) \leq U(x)$ for all $x$ and,
    when we observe $x$ we make the inference $\theta \in [L(x), U(x)]$.
  \end{defn}

  \begin{defn}
    The \emph{coverage probability} of an interval estimator
    $[L(X), U(X)]$ is the probability $\Pr_\theta[\theta \in [L(X), U(X)]]$
  \end{defn}

  \begin{defn}
    The \emph{confidence coefficient} is $\inf_\theta P_\theta[\theta \in [L(X), U(X)]]$
  \end{defn}

  Generally, the idea behind an interval estimator is that we want to
  estimate $\theta$ in a way that accounts for uncertainty.  So the
  researcher ``agrees'' not to distinguish between values in the
  interval.  These intervals can be one-sided or two-sided.

\item The idea of a confidence interval can be extended to
  \emph{confidence regions} or \emph{confidence sets} when the
  parameter has more than one element.  The extension is obvious.

\item If we have a statistic with known sampling distribution, we can
  often use that distribution to generate a confidence interval.  This
  is the idea: suppose you have a statistic $\thetah$ and know that its
  distribution function, $F(t; \theta)$, is a monotone function of $\theta$.  We
  can define a $1-\alpha$ confidence interval of the form $[\theta_L(\thetah),
  \theta_U(\thetah)]$ by solving the following equations for $\theta_L, \theta_U$:

  \begin{itemize}
  \item If $F(t; \theta)$ is decreasing in $\theta$:
    \begin{align*}
      F(\thetah; \theta_U) &= \alpha/2 & F(\thetah; \theta_L) &= 1 - \alpha/2
    \end{align*}
  \item If $F(t; \theta)$ is increasing in $\theta$:
    \begin{align*}
      F(\thetah; \theta_L) &= \alpha/2 & F(\thetah; \theta_U) &= 1 - \alpha/2
    \end{align*}
  \item Note that the $L$ and the $U$ are switched in the two formula.
  \end{itemize}

  \begin{ex} Let $X_1,...,X_n$ be iid $N(\theta, 1)$ and we want to
    construct a two-sided 95\% interval for $\theta$.  Let be $\Xb$, which
    is $N(\theta, 1/n)$.  Then
    \begin{equation*}
      \Pr_\theta[\Xb \leq t] = \Phi(\sqrt{n} (t - \theta)) = F(t; \theta)
    \end{equation*}
    which is decreasing in $\theta$.

    To construct an interval, we need to solve
    \begin{align*}
      \Phi(\sqrt{n} (\Xb - \theta_U)) &= 0.025, &
      \Phi(\sqrt{n} (\Xb - \theta_L)) &= 0.975
    \end{align*}
    which becomes
    \begin{align*}
      \Xb - \theta_U &= \Phi^{-1}(0.025) / \sqrt{n}, &
      \Xb - \theta_L &= \Phi^{-1}(0.975) / \sqrt{n}
    \end{align*}
    which then becomes
    \begin{align*}
      \Xb - \Phi^{-1}(0.025) / \sqrt{n} &= \theta_U, &
      \Xb - \Phi^{-1}(0.975) / \sqrt{n} &= \theta_L.
    \end{align*}
    Since $\Phi^{-1}(0.025) = -1.96$ and $\Phi^{-1}(0.975) = 1.96$, this
    finally becomes
    \begin{align*}
      \theta_U &= \Xb + 1.96 / \sqrt{n}, &
      \theta_L &= \Xb - 1.96 / \sqrt{n}
    \end{align*}
  \end{ex}

\end{itemize}

\subsection{Interval optimality}

\begin{itemize}[leftmargin=0pt]

\item Generally, as you might assume from above, if you have several
  estimators to use, you should choose the estimator with the tighter
  sampling distribution---the more efficient estimator.

\item This is going to lead to a principle better for asymptotic
  intervals.

\item A second question, for a two sided interval, is how to spit the
  mass between the tails.  The following theorem can help (this
  version is taken from \citealp{CB02}):

  \begin{thm}
    Let $f(x)$ be a unimodal pdf with mode $x^*$.  If $[a,b]$
    satisfies
    \begin{enumerate}
    \item $\int_a^b f(x) dx = 1 - \alpha$,
    \item $f(a) = f(b) > 0$, and
    \item $a \leq x^* \leq b$
    \end{enumerate}
    then $[a,b]$ is the shortest interval with coverage $1-\alpha$.
  \end{thm}

  The proof works by showing that any shorter interval has coverage
  strictly less than $1-\alpha$.  It's pretty easy to see from pictures,
  which I need to draw.

  \begin{itemize}
  \item Suppose $b'-a' < b-a$ and $a', b' \leq a$
  \item Suppose $b'-a' < b-a$ and $a' \leq a < b'$ (so then $b' < b$).
  \item Suppose $b'-a' < b-a$ and $a < a'$ and $b' < b$.
  \item Suppose $b'-a' < b-a$ and $a < a'$ and $b \leq b'$.
  \end{itemize}

\end{itemize}

\subsection{Basic setup for testing}

\begin{itemize}[leftmargin=0pt]

\item Often we don't care about the exact value of a particular
  unknown parameter, but only care about whether or not it falls in a
  particular range.  For example, we may only care whether a
  particular program is cost effective or not.  This is similar
  conceptually to confidence intervals, and we will see that it is a
  nearly identical problem mathematically, but the emphasis is
  somewhat different.  When constructing confidence intervals, the
  idea is to avoid unnecessarily precise (and wrong) estimates, and
  the end points are determined by the data.  Here, in hypothesis
  testing, we know the points we care about in advance, and then look
  at the data to see whether it is consistent with those parameter
  values.

\item So the setup is that we have a parameter $\theta$ and are interested
  in knowing which of two scenarios is true: if $\theta \in \Theta_0$ or if $\theta \in
  \Theta_a$.  The first is the \emph{null hypothesis}, which is what we
  would believe (and act on) in the absence of any data, and the
  second is the \emph{alternative hypothesis}, which is a candidate
  belief that might be true.  A \emph{hypothesis test} is a statistic
  with two possible outcomes: (1) reject the null hypothesis or (2)
  fail to reject/accept the null hypothesis.  It makes the math much
  easier if we let the test statistic take on the values 0 or 1, where
  1 indicates rejection.

  There are four possible things that can happen when we conduct a
  hypothesis test
  \begin{enumerate}
  \item We reject the null hypothesis, but it is really true.  This is
    an error.
  \item We fail to reject the null hypothesis, but it is false.  This
    is another error.
  \item We correctly reject the null hypothesis; it really is false.
  \item We correctly fail to reject the null hypothesis; it really is true.
  \end{enumerate}
  The convention is that the first error, rejecting a true hypothesis,
  is the worst possible outcome and that failing to reject a false
  hypothesis is a much less serious mistake.  When I say that this is
  the convention, it means that you need to set up your problem to
  conform with this prioritization.  If the consequences of both types
  of mistake are roughly equal, a hypothesis test may not be
  appropriate.

  These are called (unhelpfully) \emph{Type I error} and \emph{Type II
  error}.  One way to remember the difference is to let the Roman
  numeral ``I'' in Type I error stand for ``I'm screwed,'' because it
  is the worse one.

\item Some more terminology:
  \begin{defn}
    A test's \emph{power function} is the probability that the test
    will reject for each possible value of the true parameter, $\theta$,
    i.e., if $\delta$ is a test statistic, then its power function $\beta$
    satisfies
    \begin{equation*}
      \beta(\theta) = \Pr_\theta[\delta = 1].
    \end{equation*}
    
    The test has \emph{size $\alpha$} if
    \begin{equation*}
      \sup_{\theta \in \Theta_0} \beta(\theta) = \alpha
    \end{equation*}
    and \emph{level $\alpha$} if
    \begin{equation*}
      \sup_{\theta \in \Theta_0} \beta(\theta) \leq \alpha.
    \end{equation*}

    A test is \emph{unbiased} if $\inf_{\theta \in \Theta_a} \beta(\theta) \geq \alpha$
  \end{defn}

  Note that $\alpha$ should be set in advance of the experiment.  By
  convention, hypothesis tests that reject with size or level 0.05 are
  considered \emph{significant} rejections.  A test is \emph{valid} if
  it actually has the correct size.

\item An example may be helpful.
  \begin{ex}
    Suppose $X_1,...,X_n \sim \iid N(\theta,1)$ and we want to test the
    hypothesis $\theta = 0$ against $\theta = 1$, so $\Theta_0 = \{0\}$ and $\Theta_a =
    \{1\}$.  We can use the  standard one-sided test statistic that
    you've seen in undergrad:
    \begin{equation*}
      \delta(X) = \ind\{\sqrt{n} \Xb > c\}
    \end{equation*}
    where $c$ is a critical value chosen to ensure that $\delta$ has size
    $\alpha$ (which we haven't set yet).

    We can find $c$ for a given value of $\alpha$, say $\alpha = 0.05$.  Then
    $c$ must satisfy
    \begin{equation*}
      \Pr_\theta[\sqrt{n} \Xb > c] = \alpha
    \end{equation*}
    Since $\Pr_\theta[\sqrt{n} \Xb > c] = 1 - \Phi(c)$, this becomes
    \begin{equation*}
      c = \Phi^{-1}(1 - \alpha) \approx 1.64
    \end{equation*}
    and so $\delta(X) = \ind\{\sqrt{n} \Xb > 1.64\}$ which is probably
    something you've seen before.

    We can now find the power function, $\beta(\theta)$.  For any $\theta \in \RR$,
    \begin{align*}
      \beta(\theta) &= \Pr_\theta[\sqrt{n} \Xb > 1.64] \\
      &= \Pr_\theta[\sqrt{n} (\Xb - \theta) > 1.64 - \sqrt{n} \theta] \\
      &= 1 - \Phi(1.64 - \sqrt{n} \theta).
    \end{align*}
    Note that as $\theta$ grows towards $\infty$, $\Phi(1.64 - \sqrt{n} \theta)$
    converges to 0 and the power converges to 1; as $\theta$ goes towards
    $-\infty$, $\Phi(1.64 - \sqrt{n} \theta)$ converges to 1 and the power
    converges to 0.  Monotonicity implies that this test is unbiased
    (the probability that it rejects if $\theta = 1$ is greater than $\alpha$.
  \end{ex}

\item A brief aside on \pvalue s.  These are useful when you want to
  present results to an audience which may have people who have
  different ideas about the correct size of the test.  A \pvalue\
  tells you the largest size for which the test would reject.

\item There is a fundamental relationship between testing hypotheses
  and constructing confidence intervals.  Given any hypothesis test, we
  can construct a corresponding confidence interval as
  follows \citep[As described in][Section 9.2]{CB02}:

  For each $\theta_0$, let $A(\theta_0)$ be the acceptance region of a level $\alpha$
  test of $H_0: \theta = \theta_0$.  For each sample $x$, define a set $C(x)$ in
  the parameter space by \[C(x) = \{\theta_0: x \in A(\theta_0)\}.\]  Then the
  random set $C(X)$ is a $1-\alpha$ confidence set.\footnote{You should
    prove this on your own for homework.  The proof follows
    immediately from the definition of each set, but it can take some
    thought and work to realize it.  Start by writing out the
    definition of a $1-\alpha$ confidence set and then verify by set
    relationships that $C(X)$ satisfies the definition.}

  \begin{ex}
    Suppose that $X_1,\dots,X_n \sim N(\mu,\sigma^2)$ and we want to construct a
    two-sided $1-\alpha$ confidence interval for $\mu$; $\sigma^2$ is known.  We
    can start with the family of tests for the null hypothesis
    \begin{equation*}
      H_0:\ \mu = \mu_0 \text{\quad vs \quad} \mu \neq \mu_0
    \end{equation*}
    for which we want to use the statistic
    \begin{equation*}
      \delta(X) = \ind\{\sqrt{n} |\Xb - \mu_0| / \sigma > c_\alpha\}
    \end{equation*}
    where $c_\alpha$ is a critical value chosen so that the test has
    correct size (see the previous example).

    Then
    \begin{equation*}
      A(\mu_0) = \{X : \sqrt{n} |\Xb - \mu_0| / \sigma \leq c_\alpha\}
    \end{equation*}
    and
    \begin{equation*}
      C(X) = \{\mu_0 : \sqrt{n} |\Xb - \mu_0| / \sigma \leq c_\alpha\}.
    \end{equation*}
  \end{ex}

  Notice that
  \begin{itemize}
  \item $A(\mu_0)$ has samples as its elements and is different for
    different parameter values.
  \item $C(x)$ has parameter values as its elements and is different
    for different samples.
  \end{itemize}

\end{itemize}

\subsection{Hypothesis testing and interval estimation in general}

\begin{itemize}[leftmargin=0pt]

\item Let $\{T_n\}$ be a sequence of test statistics for $\theta \in \theta_0$
  against $\theta \in \theta_A$.  The sequence of tests has asymptotic size $\alpha$ if
  \begin{equation*}
    \lim_{n \to \infty} \sup_{\theta \in \theta_0} \Pr_\theta[T_n \text{ rejects}] = \alpha
  \end{equation*}
  It would be great to discuss local alternatives here.

\item Let $\{[L_n, U_n\}$ be a sequence of interval estimators for a
  parameter $\theta$.  The asymptotic confidence level for this sequence of
  statistics is
  \begin{equation*}
    \lim_{n \to \infty} \inf_\theta \Pr_{\theta}[\theta \in [L_n, U_n]].
  \end{equation*}
  It would be great to discuss uniformity here.

\end{itemize}

\subsection{Specific methods of producing test statistics}

\begin{itemize}[leftmargin=0pt]

\item There are a few key principles that can be used to produce test
  statistics in very general conditions.

\item The Wald tests is built on the CLT explicitly.  Suppose we want
  to test the null $\theta = \theta_0$ and we have an asymptotically normal
  estimator $\thetah_n$, so $\sqrt{n} (\thetah_n - \theta) \to^d N(0,\Sigma)$ for all $\theta$
  and that $\Sigmah$ is a consistent estimator of $\Sigma$.  Then if the null
  hypothesis is $\theta = \theta_0$, we know that
  \begin{equation*}
    n (\thetah_n - \theta_0)' \Sigmah^{-1} (\thetah_n - \theta_0) \to^d \chi^2_p
  \end{equation*}
  where $p$ is the dimension of $\theta_0$.

\item Under appropriate regularity conditions, the Likelihood Ratio
  Test is asymptotically chi-square as well.  So is a test based on
  the first order conditions, which is called the \emph{score} or
  \emph{LM} test.

\end{itemize}

\subsection{Likelihood Ratio Test}

\begin{itemize}[leftmargin=0pt]

\item If we know the distribution of the random sample, and want to
  test a hypothesis about the unknown parameters, we can use the
  likelihood ratio test to get a test statistic.  This can also be
  effective asymptotically when the likelihood function isn't known.

  The \emph{Likelihood Ratio Test statistic} is given by
  \begin{equation*}
    \Lambda(x_1,\dots,x_n) =
    \frac{\sup_{\theta \in \Theta_0} L(\theta; x_1,\dots,x_n)}{\sup_{\theta \in \Theta_0 \cup \Theta_a} L(\theta; x_1,\dots,x_n)}
  \end{equation*}
  and the test rejects if $\Lambda(x_1,...,x_n) \leq c$ where $c$ is chosen so
  that the test has the correct prespecified size.

  [You can draw some pictures connecting this to the maximum
  likelihood estimator]

  This is a useful method because it usually gives you a starting
  approach for a statistic.  As we'll see later, it often gives you
  the best test.

\item An example can help:
  \begin{ex}
    Suppose that $X_1,...,X_n \sim \iid\ \uniform(0, b)$ and we want to
    test the null hypothesis $b \geq 1$ vs $b < 1$.  (add some pictures).
    Intuitively, the test should always reject if $\max_i X_i > 1$,
    but it should also reject for values below but very close to 1.
    
    To start, calculate the likelihood:
    \begin{equation*}
      L(b; x_1,...,x_n) = b^{-n} \ind\{b \geq \max_i X_i\}.
    \end{equation*}
    
    Then (for the numerator)
    \begin{equation*}
      \sup_{b \geq 1} L(b; x_1,\dots,x_n) =
      \begin{cases}
        1 & \text{if } 1 > \max_i X_i \\
        (\max_i X_i)^{-n} & \text{if } 1 \leq \max_i X_i.
      \end{cases}
    \end{equation*}
    and (for the denominator)
    \begin{equation*}
      \sup_{b > 0} L(b; x_1,\dots,x_n) = (\max_i X_i)^{-n}.
    \end{equation*}
    Putting them together, we get
    \begin{equation*}
      \Lambda(x) =
      \begin{cases}
        (\max_i X_i)^n & \text{if } \max_i X_i \leq 1. \\
        1 & \text{if } \max_i X_i > 1 \\
      \end{cases}
    \end{equation*}

    Now, to find $c$ so that $\sup_{b \geq 1} \Pr_b[(\max_i X_i)^n \leq c] = \alpha$
    \begin{align*}
      \sup_{b \geq 1} \Pr_b[(\max_i X_i)^n \leq c]
      &= \sup_{b\geq 1} \Pr_b[\max_i X_i/b \leq c^{1/n}/b] \\
      &= \sup_{b \geq 1} \Pr_b[X_1/b \leq c^{1/n}/b, ..., X_n/b \leq c^{1/n}/b] \\
      &= \sup_{b \geq 1} \Pr_b[X_1/b \leq c^{1/n}/b] \cdots \Pr_b[X_n/b \leq c^{1/n}/b] \\
      &= \sup_{b \geq 1} c/b^n \\
      &= c
    \end{align*}
    where we're using the fact that $X_i/b \sim \uniform(0,1)$.  So
    $c=\alpha$.  This test is obviously equivalent to comparing $\max_i
    X_i$ to $\alpha^{1/n}$.
  \end{ex}

\item We can run some simulations to verify that this works and makes
  sense (code in Julia, \citealp{BKS12}).
  \begin{Verbatim}
using Distributions; b = 1; n = 10; crit = 0.05

lrt_stats = [maximum(rand(Uniform(0, b), n)) for i in 1:10000]
mean(lrt_stats .<= crit^(1/n))

### to plot:
### plot(x = lrt_stats, Geom.histogram)
  \end{Verbatim}
  And repeat for $n = 100$ and $n = 1000$ as well.
\end{itemize}

\subsection{Neyman-Pearson Lemma}

\begin{itemize}[leftmargin=0pt]

\item Helps us think about the best possible power;
  \begin{defn}
    Let $C$ be a class of tests for testing $\theta \in \Theta_0$
    against $\theta \in \Theta_0^c$.  A test in $C$ with power
    function $\beta(\theta)$ is a \emph{uniformly most powerful} (UMP)
    class $C$ test if $\beta(\theta) \geq \beta'(\theta)$ for all
    $\theta \in \Theta^c$ and all $\beta'$ that are power functions of
    tests in $C$.
  \end{defn}
  (There's a good chance that this statement is verbatim from
  \citealp{CB02}.)

  The usefulness of this concept really shows up in asymptotic
  arguments. In finite samples, there is often no UMP test, but one
  can sometimes be constructed if the class is restricted.

\item A statement of the famous Lemma (this may also be from
  \citealp{CB02}):
  \begin{thm}[Neyman-Pearson Lemma]
    Consider testing $\theta = \theta_0$ vs $\theta = \theta_1$ where
    the pdf for each parameter value is $f(x; \theta_i)$, $i = 1,2$,
    using a test with rejection region $R$ such that, for some $k \geq
    0$,
    \begin{align*}
      x &\in R    &&\text{if } f(x; \theta_1) > k f(x; \theta_0) \\
      x &\notin R &&\text{if } f(x; \theta_1) < k f(x; \theta_0)
    \end{align*}
    Then:
    \begin{enumerate}
    \item Any size $\alpha$ test with such a rejection region is a UMP
      level $\alpha$-test.
    \item If there is a size $\alpha$ test with such a rejection
      region for $k > 0$, then every UMP level $\alpha$ test has size
      $\alpha$ and every UMP level $\alpha$ test has this rejection
      region almost surely.
    \end{enumerate}
  \end{thm}

\item Both inequalities are strict to allow for discrete random
  variables.

\item This Lemma justifies the LRT (at least for simple tests).

\item Most UMP results for more complicated settings are extensions of
  this lemma.

\item This lemma is more or less how we think of power in econometrics

\end{itemize}

%%% Local Variables:
%%% mode: latex
%%% TeX-master: "../core_econometrics"
%%% End:
