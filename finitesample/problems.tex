% Copyright (c) 2013, authors of "Core Econometrics;" a
% complete list of authors is available in the file AUTHORS.tex.

% Permission is granted to copy, distribute and/or modify this
% document under the terms of the GNU Free Documentation License,
% Version 1.3 or any later version published by the Free Software
% Foundation; with no Invariant Sections, no Front-Cover Texts, and no
% Back-Cover Texts.  A copy of the license is included in the file
% LICENSE.tex and is also available online at
% <http://www.gnu.org/copyleft/fdl.html>.

\chapter{Problems}

\begin{enumerate}

\item Write an R function that calculates the maximum likelihood
  estimator of $\alpha$ and $\beta$ for the gamma$(\alpha,\beta)$
  distribution. Maximize the likelihood function numerically even if
  you are able to analytically derive the MLE.  You may find the R
  functions \texttt{optim} and \texttt{dgamma} helpful.  You should
  pay attention to the initial values of the numerical optimization
  procedure, and may want to use the method of moments to derive
  sensible initial values.

\item We often want to understand the properties of a statistical
  estimator that has too complicated of a distribution to calculate
  directly.  Or we might know that the distribution of a statistical
  estimator is approximately equal to particular formula, but don't
  know how accurate the approximation will be in any application.  For
  either situation, it can be useful to use a computer to simulate the
  estimator under different assumptions.

  It can be shown (i.e. \citet[p. 483]{CB02}) that if $X_1,\dots,X_n$
  is an i.i.d. sample with distribution and density $F$ and $f$
  respectively, then the sample median of $X_1,\dots,X_n$ is
  approximately (i.e. this holds at $n = \infty$) distributed $\N(\theta,
  (4 n f^2(\theta))^{-1})$, where $\theta$ is the population median.
  \begin{enumerate}
  \item Simulate 1000 samples of size 10 with $X \sim N(1,2)$ and plot
    a histogram of the sample median.  Also plot the density of the
    limiting normal distribution on the same graph.  How well does the
    approximation match the simulated density?  Hint: you will find it
    easier if you use the R function \texttt{replicate}.
  \item Repeat the previous question with $n=100$.
  \item Repeat the first question with different distribution
    functions.  What features of the distribution function affect the
    quality of the approximation?
  \end{enumerate}

\item Suppose that $X_1,\dots,X_n \sim i.i.d. N(\mu, \sigma^2)$.  Is
  the sample median consistent for $\mu$?  Is it asymptotically
  normal?  Do these answers require the $X_i$ to be Normal?

\item Suppose that $X_1,\dots,X_n$ are distributed uniform$(0,b)$.
  Derive the LRT for the null hypothesis $b \geq b_0$ against the
  alternative $b < b_0$ and also for the null hypothesis $b \leq b_0$
  against $b > b_0$.  Please discuss and compare the tests.

\item Suppose that $X_1,\dots,X_n$ are i.i.d. uniform($a$,$b$).
  Derive the LRT of the null hypothesis $\E X_i = 0$ against the
  two-sided alternative $\E X_i \neq 0$.

\item Let $X = (X_1,\dots,X_n)$ be a random sample and let $\theta$ be
  some parameter of interest.  For each $\theta_0$, let $A(\theta_0)$
  be the acceptance region of a level $\alpha$ test of the null
  hypothesis that $\theta = \theta_0$.  For each sample $x$, define a
  set $C(x)$ in the parameter space by
  \begin{equation}
     C(x) = \{ \theta_0 : x \in A(\theta_0) \}.
  \end{equation}
  Prove that the random set $C(X)$ is a $1-\alpha$ confidence set for the
  parameter $\theta$.

\end{enumerate}

%%% Local Variables: 
%%% mode: latex
%%% TeX-master: "../core_econometrics"
%%% End: 
