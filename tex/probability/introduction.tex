% Copyright © 2013, authors of the "Econometrics Core" textbook; a
% complete list of authors is available in the file AUTHORS.tex.

% Permission is granted to copy, distribute and/or modify this
% document under the terms of the GNU Free Documentation License,
% Version 1.3 or any later version published by the Free Software
% Foundation; with no Invariant Sections, no Front-Cover Texts, and no
% Back-Cover Texts.  A copy of the license is included in the file
% LICENSE.tex and is also available online at
% <http://www.gnu.org/copyleft/fdl.html>.

\part*{Introduction to the probability lectures}%
\addcontentsline{toc}{part}{Quick overview of the probability material}

This motivation is largely taken from \citet{Res_1999} but needs to be
removed/rewritten.  There are a few key theoretical concepts that lead
to many econometric tools.

\section{Law of Large Numbers}

     Suppose that $X₁,...,X_n$ is a sequence of independent,
     identically distributed (iid from here on) random variables with
     mean $μ$.  Then $\Xb → μ$ as $n → ∞$
\begin{itemize}
\item average payoff at a 25-cent slot machine gets close to the
       expected payoff as I play for longer
\item haven't specified what the arrow denotes formally
\item justifies using the average payoff as an estimator of expected
       payoff
\end{itemize}
\section{Central Limit Theorem}

If $X₁,⋯,X_n$ are iid with mean $μ$ and variance $σ$$²$ then
\[
\Pr[n^{1/2}(\Xb - μ)/σ ≤ c] → Φ(c) ≡ ∫_{-∞}^c (2π)^{-1/2} e^{-u²/2}du
\]
\begin{itemize}
\item this is a stronger result than the law of large numbers
\item I can use this result to find out the probability of winning more
       than (say) \$15 at a slot machine after playing for a long time.
\end{itemize}

%%% Local Variables: 
%%% mode: latex
%%% TeX-master: "../../probability"
%%% End: 