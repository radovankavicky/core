% Copyright © 2013, authors of the "Econometrics Core" textbook; a
% complete list of authors is available in the file AUTHORS.tex.

% Permission is granted to copy, distribute and/or modify this
% document under the terms of the GNU Free Documentation License,
% Version 1.3 or any later version published by the Free Software
% Foundation; with no Invariant Sections, no Front-Cover Texts, and no
% Back-Cover Texts.  A copy of the license is included in the file
% LICENSE.tex and is also available online at
% <http://www.gnu.org/copyleft/fdl.html>.

\part*{Confidence intervals}%
\addcontentsline{toc}{part}{Confidence intervals}

\section{Introduction to interval estimation}

\begin{itemize}
\item We can take the ``hypothesis testing mentality'' and apply it to
      point estimation
\begin{itemize}
\item apply the philosophy of hypothesis testing
\item apply the mathematics of hypothesis testing
\end{itemize}
\item point estimation
\begin{itemize}
\item want to get a value
\begin{itemize}
\item have an unknown parameter $θ$
\item use that data to construct a best guess for $θ$
\item we're going to use that value as though it were true
\end{itemize}
\end{itemize}
\item confidence intervals
\begin{itemize}
\item want to get a range of values that are \emph{possible}
\item we're going to consider those that are inside the interval
        indistinguishable from eachother and those outside
        indistinguishable from eachother too.
\end{itemize}
\end{itemize}

\section{Definitions}

\begin{itemize}
\item An \emph{interval estimator} of a parameter $θ$ is any pair of
      functions $L(X)$ and $U(X)$ s.t. $L(x) ≤ U(x)$ for all $x$
      and, when we observe $x$ we make the inference $θ ∈
      [L(x), U(x)]$
\item The \emph{coverage probability} of an interval estimator
      $[L(X),U(X)]$ is the probability $P_θ[θ ∈ [L(X), U(X)]]$
\item The \emph{confidence coefficient} is $\inf_θ P_θ[θ ∈ [L(X), U(X)]]$
\end{itemize}

\section{Pivoting the CDF}

\begin{itemize}
\item works for one parameter, continuous rv.
\item suppose you have a statistic $\θh$ and know that its distribution
      function, $F(t; θ)$, is a monotone function of $θ$
\begin{itemize}
\item e.g. if $\θh$ is normal with mean $θ$ and variance one, then
  $F(t; θ) = Φ(t - θ)$ which is decreasing in $θ$.
\end{itemize}
\item we can define a $1-α$ confidence interval of the form
  $[θ_L(\θh), θ_U(\θh)]$ by solving the following equations for $θ_L, θ_U$
\begin{description}
\item[decreasing in $θ$] \[F(\θh; θ_U) = α/2\] and \[F(\θh; θ_L) = 1 - α/2\]
\item[increasing in $θ$] \[F(\θh; θ_L) = α/2\] and \[F(\θh; θ_U) = 1 - α/2\]
\end{description}
note that the L and U are switched in the two formulae.
\item You can also do this with asymptotic distributions (and that's the
     usual approach in economics for presenting asymptotic CI
\item example: Let $X₁,...,X_n$ be iid $N(θ, 1)$ and we want to
     construct a two-sided 95\% interval for $θ$.
\begin{itemize}
\item know that $\sqrt{n} (\Xb - θ) → N(0,1)$ in distribution.
\item define $\θh = \sqrt{n} \Xb$
\begin{itemize}
\item $F(t; θ) = Φ(t - θ \sqrt{n})$
\item obviously, decreasing in $θ$, so solve
\begin{itemize}
\item $Φ(\sqrt{n} \Xb - θ_U \sqrt{n}) = 0.025$
\item $Φ(\sqrt{n} \Xb - θ_L \sqrt{n}) = 0.975$
\end{itemize}
\item this becomes:
\begin{itemize}
\item $\sqrt{n} \Xb - θ_U \sqrt{n} = Φ^{-1}(0.025)$
\item $\sqrt{n} \Xb - θ_L \sqrt{n} = Φ^{-1}(0.975)$
\end{itemize}
\item which in turn becomes:
\begin{itemize}
\item $θ_U = \Xb - Φ^{-1}(0.025) / \sqrt{n} = \Xb + 1.96/\sqrt{n}$
\item $θ_L = \Xb - Φ^{-1}(0.975) / \sqrt{n} = \Xb - 1.96/\sqrt{n}$
\end{itemize}
\item which is the standard confidence interval.
\end{itemize}
\end{itemize}
\end{itemize}

%%% Local Variables:
%%% mode: latex
%%% TeX-master: "../../finitesample"
%%% End:
