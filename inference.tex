\documentclass[nohyper]{external/tufte-handout}
% The hyperref is loaded in the preamble with additional arguments to
% avoid some xelatex warnings.

\title[Inference in finite samples]%
{Inference in Finite Samples \\
  Core Econometrics Textbook, part 3}
% Copyright © 2013, authors of the "Core Econometrics;" a
% complete list of authors is available in the file AUTHORS.tex.

% Permission is granted to copy, distribute and/or modify this
% document under the terms of the GNU Free Documentation License,
% Version 1.3 or any later version published by the Free Software
% Foundation; with no Invariant Sections, no Front-Cover Texts, and no
% Back-Cover Texts.  A copy of the license is included in the file
% LICENSE.tex and is also available online at
% <http://www.gnu.org/copyleft/fdl.html>.

\newcommand{\version}{0.7.1}
\newcommand{\releasedate}{10 Dec. 2013}

%%% Local Variables:
%%% mode: latex
%%% TeX-master: "core_econometrics"
%%% End:

\author{Gray Calhoun} 
% (This comment is repeated in the Makefile)
% I'm still not sure the best way to do author information; I'm much
% more concerned in the long run about how different attributation
% styles would make someone more or less likely to contribute to an
% existing text or to license an existing draft.  For now, there's
% only one author, so I'll put myself as the author.  If someone else
% contributes any edits, etc., I'll change it to {Gray Calhoun and
% EFLP}.  If anyone wants to contribute a lot of original material and
% wants named authorship, please email the mailing list so we can
% discuss merging projects.

\usepackage{amssymb,amsmath,amsthm,verbatim}
\usepackage{fontspec,unicode-math,xltxtra,xunicode,booktabs}
\setromanfont[Ligatures=TeX]{TeX Gyre Pagella}
\setsansfont[Ligatures=TeX,Scale=MatchLowercase]{TeX Gyre Heros}
% \setmonofont[Scale=MatchLowercase]{Inconsolata}
\setmathfont{Asana-Math}

\frenchspacing
\setcounter{secnumdepth}{1}
\setcounter{tocdepth}{1}
\renewcommand\bibname{}
\renewcommand\refname{}
\renewcommand\contentsname{}
\bibliographystyle{abbrvnat}
\setcitestyle{round}
\newcommand{\email}[1]{\href{mailto:#1}{\nolinkurl{#1}}}
\newcommand{\homepage}{\url{http://www.econometricslibrary.org}}
\newcommand{\maillist}{\email{econometricslibrary@librelist.com}}
\newcommand{\bugtrack}%
{\url{https://github.com/EconometricsLibrary/CoreEconometricsText/issues}}

% Workaround for bugs in the tufte-latex class
\renewcommand\smallcapsspacing[1]{{\addfontfeature{LetterSpace = 8}\scshape#1}}
\renewcommand\allcapsspacing[1]{{\addfontfeature{LetterSpace = 15}#1}}
% Getting waringings from latexmk with default tufte-latex hyperref
\usepackage[unicode,pdfencoding=auto,hyperfootnotes=false,hidelinks]{hyperref}

\newcommand{\BibTeX}{Bib\!\TeX}
\newcommand{\pvalue}{\ensuremath{p}-value}
\newcommand{\ftest}{\ensuremath{F}-test}
\newcommand{\ttest}{\ensuremath{t}-test}

% Math shortcuts
\renewcommand{\Pr}{\operatorname{Pr}}

\DeclareMathOperator{\1}{1}
\DeclareMathOperator{\abs}{abs}
\DeclareMathOperator{\avar}{avar}
\DeclareMathOperator{\bias}{bias}
\DeclareMathOperator{\corr}{corr}
\DeclareMathOperator{\cov}{cov}
\DeclareMathOperator{\E}{E}
\DeclareMathOperator{\median}{median}
\DeclareMathOperator{\mse}{mse}
\DeclareMathOperator{\rank}{rank}
\DeclareMathOperator{\range}{range}
\DeclareMathOperator{\sd}{sd}
\DeclareMathOperator{\tr}{tr}
\DeclareMathOperator{\var}{var}

\DeclareMathOperator*{\argmax}{arg\,max}
\DeclareMathOperator*{\argmin}{arg\,min}
\DeclareMathOperator*{\plim}{plim}

\DeclareMathOperator{\binomial}{binomial}
\DeclareMathOperator{\bernoulli}{bernoulli}
\DeclareMathOperator{\invWishart}{inverse\ Wishart}
\DeclareMathOperator{\N}{N}
\DeclareMathOperator{\uniform}{uniform}

\newcommand{\BB}{\ensuremath{\mathbb{B}}}
\newcommand{\NN}{\ensuremath{\mathbb{N}}}
\newcommand{\PP}{\ensuremath{\mathbb{P}}}
\newcommand{\QQ}{\ensuremath{\mathbb{Q}}}
\newcommand{\RR}{\ensuremath{\mathbb{R}}}
\newcommand{\RRᵏ}{\ensuremath{\mathbb{R}ᵏ}}
\newcommand{\RRⁿ}{\ensuremath{\mathbb{R}ⁿ}}
\newcommand{\RRb}{\ensuremath{\bar{\mathbb{R}}}}
\newcommand{\ZZ}{\ensuremath{\mathbb{Z}}}

\newcommand{\Fs}{\ensuremath{\mathcal{F}}}
\newcommand{\Gs}{\ensuremath{\mathcal{G}}}
\newcommand{\Ps}{\ensuremath{\mathcal{P}}}

\newcommand{\ov}[2][1]{\tfrac{#1}{#2}}
\newcommand{\iid}{i.i.d.}

\newcommand{\ep}{\varepsilon}
\newcommand{\eph}{\hat{\varepsilon}}

\newcommand{\ah}{\hat{a}}
\newcommand{\αh}{\hat{α}}
\newcommand{\bh}{\hat{b}}
\newcommand{\βb}{\bar{β}}
\newcommand{\βh}{\hat{β}}
\newcommand{\βt}{\tilde{β}}
\newcommand{\eh}{\hat{e}}
\newcommand{\εb}{\bar{ε}}
\newcommand{\εh}{\hat{ε}}
\newcommand{\εt}{\tilde{ε}}
\newcommand{\ηh}{\hat{η}}
\newcommand{\Fh}{\hat{F}}
\newcommand{\λh}{\hat{λ}}
\newcommand{\μb}{\bar{μ}}
\newcommand{\μh}{\hat{μ}}
\newcommand{\Ωh}{\hat{Ω}}
\newcommand{\Qh}{\hat{Q}}
\newcommand{\Σh}{\hat{Σ}}
\newcommand{\sh}{\hat{s}}
\newcommand{\σh}{\hat{σ}}
\newcommand{\σb}{\bar{σ}}
\newcommand{\θh}{\hat{θ}}
\newcommand{\Vh}{\hat{V}}
\newcommand{\Xb}{\bar{X}}
\newcommand{\Xc}{\mathcal{X}}
\newcommand{\Xh}{\hat{X}}
\newcommand{\Xt}{\tilde{X}}
\newcommand{\Yb}{\bar{Y}}
\newcommand{\Yh}{\hat{Y}}
\newcommand{\Yt}{\tilde{Y}}
\newcommand{\yh}{\hat{y}}

\newcommand{\dx}{\,dx}
\newcommand{\dy}{\,dy}
\newcommand{\dμ}{\,dμ}
\newcommand{\dθ}{\,dθ}
\renewcommand{\dz}{\,dz}

\newtheorem{thm}{Theorem}[section]
\newtheorem{defn}{Definition}[section]
\newtheorem{ex}{Example}[section]
\newtheorem{asmp}{Assumption}[section]


\begin{document}
\maketitle

\bigskip\noindent%
Copyright © 2013, the authors of the \textit{Core Econometrics Textbook}.
A complete author list is included in Appendix A of this document.

Permission is granted to copy, distribute and/or modify this document
under the terms of the GNU Free Documentation License, Version 1.3 or
any later version published by the Free Software Foundation; with no
Invariant Sections, no Front-Cover Texts, and no Back-Cover Texts.  A
copy of the license is included in Appendix B of this document and is
also available online at \url{http://www.gnu.org/copyleft/fdl.html}.

This text was produced as part of the Econometrics Free Library
Project.  The project's goal is to produce high-quality free and
open access econometrics textbooks and reference material.  More
information about this project is available at its homepage,
\homepage, including information on how
to participate.  This text is typeset in \LaTeX\ and there is a link
to the source of the document at the project's homepage as well.

Please cite this document as
\begin{itemize}
\item[] \bibentry{eflp-core}
\item[] \verbatiminput{CITATION.bib}
\end{itemize}
The \BibTeX\ record is provided for your convenience.
If the year and version are not specified, please see the files
\texttt{README.md} or \texttt{VERSION.tex} for directions on how to
find them.

When you find errors in this text, please email the project mailing
list, \maillist, or log the error in the project issue tracker,
\bugtrack.  It would be helpful if you could refer to the version
number listed above as well.  Thanks!

\addcontentsline{toc}{part}{Table of Contents}
\tableofcontents

% Copyright © 2013, authors of the "Econometrics Core" textbook; a
% complete list of authors is available in the file AUTHORS.tex.

% Permission is granted to copy, distribute and/or modify this
% document under the terms of the GNU Free Documentation License,
% Version 1.3 or any later version published by the Free Software
% Foundation; with no Invariant Sections, no Front-Cover Texts, and no
% Back-Cover Texts.  A copy of the license is included in the file
% LICENSE.tex and is also available online at
% <http://www.gnu.org/copyleft/fdl.html>.

\part*{Introduction to hypothesis testing}%
\addcontentsline{toc}{part}{Introduction to hypothesis testing}

\begin{itemize}
\item Motivation
\begin{itemize}
\item there are many settings where we care what the unknown
       coefficients are.
\begin{itemize}
\item forecasting
\item setting policy
\end{itemize}
\item there are plenty of other settings where we don't really care what
       the coefficients are, but just want to know whether the
       coefficient is above some threshold
\end{itemize}
\item Example: have a basic relationship: $wage_t = β₀ + β₁
     × education_t + \ep_t$
\begin{itemize}
\item may want to know whether $β₁$ is positive
\item may want to know whether $β₁$ is above some threshold that
       makes funding a program worthwhile
\item probably don't care about whether or not $β₁$ is exactly
       2.95 vs 2.97 -- they probably mean exactly the same thing as far
       as your decision is concerned.  But $-0.01$ vs 0.01 probably don't
       mean exactly the same thing as far as decisions are concerned
\item For (most) estimation, though, we would treat mismeasuring the
       true parameter by 0.02 the same, whether the true value was 2.95
       or $-0.01$
\end{itemize}
\end{itemize}

\section{Basic setup for testing}

\subsection{Definition of test statistic}

\begin{itemize}
\item Suppose that you have two different hypotheses about the
       unknown parameters of the DGP
\begin{description}
\item[null] $θ ∈ Θ₀$
\begin{itemize}
\item the null hypothesis is what we would believe about $θ$ if we
           had no data about the problem
\end{itemize}
\item[alternative] $θ ∈ Θ_a$
\begin{itemize}
\item the alternaive hypothesis is a candidate belief that might be true.
\item but we need \textbf{strong} evidence before we believe it
\end{itemize}
\end{description}
\item A \emph{test statistc} is a statistic that specifies the following:
\begin{enumerate}
\item For which samples do we make the decision to accept $θ_a$
          (our alternative) as true
\item For which samples do we \textbf{not} make that decision, and instead
          act as though $θ₀$ were true.
\end{enumerate}
\end{itemize}

\subsection{Example of a test statistic}

\begin{itemize}
\item Suppose $X₁,...,X_n ∼ iid N(μ,1)$ and we have the hypotheses:
\begin{description}
\item[null] $μ = 0$
\item[alt] $μ = 1$
\end{description}
\item we can use the standard one-sided test statistic:
\begin{description}
\item[reject 0] if $\sqrt{n} (\Xb - μ₀) / σ = \sqrt{n}
                     \Xb ≥ 1.64$
\item[accept 0] if $\sqrt{n} \Xb < 1.64$
\end{description}
\item tells us that we reject null for any sample that gives us a mean
       greater than $1.64 / \sqrt{n}$ and do not reject for other
       samples
\end{itemize}

\subsection{Error types and probabilities}

\begin{itemize}
\item Test statistics can't be foolproof.
\begin{itemize}
\item Even if $μ = 0$, we'll see some samples that have a sample
         mean greater than $1.64 / \sqrt{n}$
\item If $μ = 1$, we'll also see samples with a small mean
\item The key aspect of statistical testing is that we can calculate
         the probabilities of making different sorts of errors, and
         design procedures that make errors that we're comfortable with.
\end{itemize}
\item Possible outcomes:
\begin{itemize}
\item null is true, test statistic does not reject
\begin{itemize}
\item great!
\end{itemize}
\item null is true, test statistic does reject
\begin{itemize}
\item problem
\end{itemize}
\item null is false, test statistic does not reject
\begin{itemize}
\item problem
\end{itemize}
\item null is false, test statistic rejects
\begin{itemize}
\item great!
\end{itemize}
\end{itemize}
\item key philosophy behind testing:
\begin{itemize}
\item we set up the problems so that the null is what we'd do
         without data.
\item consequently, if the null is false, but the test statistic
         does not reject, it's not so bad
\item \textbf{but} if the null is true and we reject it, that is bad.
\begin{itemize}
\item we should need more evidence to change our behaviour than we
           would need to keep doing the same thing we were.
\end{itemize}
\item advantages of this approach:
\begin{itemize}
\item simplifies the mathematics considerably
\item intuitively appealing.
\item matches up with actual practice in ``academic'' scientific experiments
\end{itemize}
\end{itemize}
\item terminology
\begin{itemize}
\item type one error: reject when you shouldn't
\item type two error: don't reject when you should
\item power: probability that the test rejects (usually will depend
         on the value of the true parameter).
\begin{itemize}
\item formally, let $β(θ) = \Pr_θ[reject]$.  $β(θ)$ is the power function.
\end{itemize}
\item size: probability that the test rejects if the null hypothesis is true;
\begin{itemize}
\item i.e. probability of type I error
\item a test is \emph{size} $α$ if $\sup_{θ ∈ Θ₀} β(θ) = α$
\item a test is \emph{level} $α$ if $\sup_{θ ∈ Θ₀} β(θ) ≤ α$
\item \textbf{size} is what we care most about.  We want to set the size
\end{itemize}
\item valid: a test that has correct size -- ie rejects if the null
         is true at the percentages we want
\item unbiased: a test is unbiased if $\inf_{θ ∈ Θ_A} β(θ) ≥ α$
\end{itemize}
\end{itemize}

\subsection{Motivating example}

\begin{itemize}
\item sometimes we can monkey with a statistic to get a test
\item think back to normal distribution with $H₀: μ = 0$ vs
       $H_a: μ = 1$ and we want to test at 5\%
\begin{itemize}
\item statistic was $\sqrt{n} \Xb$.
\item if $μ = 0$, then we know that $\sqrt{n} \Xb$ is standard normal.
\item $\Xb > 0$ is evidence against the null, so let's try to
         find a threshold that preserves size
\begin{itemize}
\item find $c$ such that $\Pr[\sqrt{n} \Xb ≥ c] = 0.05$
  \begin{align*}
    \Pr[\sqrt{n} \Xb ≥ c] &= 1 - \Pr[\sqrt{n} \Xb < c] \\
    &= 1 - \Pr[\sqrt{n} \Xb ≤ c] \text{ (continuity)} \\
    &= 1 - Φ(c)
  \end{align*}
  so, $Φ(c) = 0.95$ and $c = Φ^{-1}(0.95) \approx 1.64$
\end{itemize}
\item So the probability that $\sqrt{n} \Xb ≥ 1.64 = 0.05$
\item so, the probability of rejecting if the null hypothesis is
         true is 0.05.
\begin{itemize}
\item the size of the test is 0.05
\item if we want to change the size of the test, all we have to
           do is change 1.64 to a different number.
\end{itemize}
\end{itemize}
\item power
\begin{itemize}
\item knowing the distribution under the null gives us size
\item to find out the power, we want to know the distribution under
         the alternative
\begin{itemize}
\item if $μ = 1$ then $\sqrt{n}(\Xb - 1)$ is standard normal.
\item $β(1) = P[\sqrt{n} \Xb ≥ 1.64] = P[\sqrt{n}(\bar
           X - 1) ≥ 1.64 - \sqrt{n}] = 1 - Φ(1.64 - \sqrt{n}) → 1$
\end{itemize}
\end{itemize}
\end{itemize}

\subsection{p-values}

\begin{itemize}
\item Say you're working, do a test at 5\% and want to show your advisor
\item maybe your advisor is more or less tolerant of type 1 error
\item so you test at 10\%, 9.9\%, 9.8\%, etc and record rejections and
  acceptances
\item but, you could summarize that by looking at the smallest
  critical value for which you would reject
\item this critical value is the p-value
\end{itemize}

\section{Relationship between hypothesis testing and confidence
  intervals}

\begin{itemize}

\item There is a fundamental relationship between testing hypotheses
  and constrcting confidence intervals.  Given any hypothesis test, we
  can construct a corresponding confidence interval as
  follows \citep[As described in][Section 9.2]{CaB_2001}:

  For each $θ₀$, let $A(θ₀)$ be the acceptance region of a level $α$
  test of $H₀: θ = θ₀$.  For each sample $x$, define a set $C(x)$ in
  the parameter space by \[C(x) = \{θ₀: x ∈ A(θ₀)\}\].  Then the
  random set $C(X)$ is a $1-α$ confidence set.\sidenote{You should
    prove this on your own for homework.  The proof follows
    immediately from the definition of each set, but it can take some
    thought and work to realize it.  Start by writing out the
    definition of a $1-α$ confidence set and then verify by set
    relationships that $C(X)$ satistifies the definition.}

\item Here's an example using this result:
  \begin{itemize}
  \item suppose $X₁,…,X_n ∼ N(μ,σ²)$ and we want to construct a
    two-sided $1-α$ confidence interval for $μ$.
    \begin{itemize}
    \item $σ²$ is known
    \end{itemize}
  \item start by getting the family of tests:
    \begin{itemize}
    \item test of $H₀: μ = μ₀$ vs $μ ≠ μ₀$ is given by:
      \begin{description}
      \item[reject] if $μ₀ > \Xb + z_{α/2} σ / \sqrt{n}$ or $μ₀ < \Xb
        - z_{α/2} σ / \sqrt{n}$.
      \item[accept] otherwise
      \end{description}
    \item gives $A(μ₀) = \{x :\Xb - z_{α/2} σ / \sqrt{n} ≤ μ₀ ≤ \Xb +
      z_{α/2} σ / \sqrt{n}\}$
    \item then $C(x) = \{μ :\Xb - z_{α/2} σ / \sqrt{n} ≤ μ ≤ \Xb +
      z_{α/2} σ / \sqrt{n}\}$
      \begin{itemize}
      \item we're changing two things
        \begin{description}
        \item[.] the set $A(μ₀)$ has samples as its elements
        \item[.] the set $C(x)$ has parameter values as its elements
        \item[.] $A(μ₀)$ is different for different parameter values
        \item[.] $C(x)$ is different for different samples
        \end{description}
      \end{itemize}
    \end{itemize}
  \end{itemize}
\end{itemize}

%%% Local Variables:
%%% mode: latex
%%% TeX-master: "../inference"
%%% End:

% Copyright © 2013, authors of the "Econometrics Core" textbook; a
% complete list of authors is available in the file AUTHORS.tex.

% Permission is granted to copy, distribute and/or modify this
% document under the terms of the GNU Free Documentation License,
% Version 1.3 or any later version published by the Free Software
% Foundation; with no Invariant Sections, no Front-Cover Texts, and no
% Back-Cover Texts.  A copy of the license is included in the file
% LICENSE.tex and is also available online at
% <http://www.gnu.org/copyleft/fdl.html>.

\part*{Motivation for the Likelihood Ratio Test}%
\addcontentsline{toc}{part}{Motivation for the Likelihood Ratio Test}

\begin{itemize}
\item If we know the distribution of the random sample, and want to
  test a hypothesis about the unknown parameters, we can use the
  likelihood ratio test to get a test statistic.
\item test statistic is given by
  \begin{equation*}
    Λ(x₁,…,x_n) =
    \frac{\sup_{θ ∈ Θ₀} L(θ; x₁,…,x_n)}{\sup_{θ ∈ Θ₀ ∪ Θ_a} L(θ; x₁,…,x_n)}
  \end{equation*}
\begin{itemize}
\item $L(θ; x₁,...,x_n)$ is the likelihood function
\item reject if $Λ(x₁,...,x_n) ≤ c$ where $c$ is chosen
        so that the test has the correct prespecified size.
\end{itemize}
\item you can draw some pictures connecting this to the maximum
      likelihood estimator
\end{itemize}

\subsection{Uniform LRT}

\begin{itemize}
\item suppose that $X₁,...,X_n$ is iid $uniform(0,b)$ and we want
      to test the null hypothesis $b ≥ 1$ vs $b < 1$.
\begin{itemize}
\item use pictures to motivate a lot of this
\item get the statistic
\begin{itemize}
\item calculate the likelihood: $L(b; x₁,...,x_n) = b^{-n}$ if
  $b ≥ \max_i X_i$, zero otherwise.
\begin{itemize}
\item introduce the order statistics:
\begin{description}
\item[.] $x_{(1)} = \min_i x_i$
\item[.] $x_{(2)}$ is the 2nd smallest
\item[.] through $x_{(n)} = \max_i x_i$
\end{description}
\end{itemize}
\item The denominator of the test statistic is just
  $\sup_{b > 0} L(b; x₁,...,x_n) = x_{(n)}^{-n}$
\item The numerator is also $x_{(n)}^{-n}$ as long as $x_{(n)} ≥ 1$.
  Otherwise, the numerator is 1
\begin{itemize}
\item $\sup_{b ≥ 1} L(b; x₁,...,x_n)$
\end{itemize}
\item so the likelihood ratio test statistic is
\begin{itemize}
\item 1 if $x_{(n)} ≥ 1$
\item $x_{(n)}^{-n}$ if $x_{(n)} < 1$
\end{itemize}
\item we're going to reject if $x_{(n)} ≤ c$ for some value of
          $c$ that we're going to calculate
\begin{itemize}
\item this makes intuitive sense -- if the maximum value we
            see is much less than 1, it is unlikely that $b ≥
            1$.
\item if $x_{(n)} ≥ 1$ we know that the alternative can't be true
\end{itemize}
\end{itemize}
\item get the critical value from the distribution of $x_{(n)}$
\begin{itemize}
\item we're going to reject if $x_{(n)}$ is small
\item since we know $X₁,...,X_n$ is uniform, we know the
          distribution of $X_{(n)}$
\item want to find $c$ so that $\sup_b \Pr_b[X_{(n)} ≤ c] = α$
  \begin{align*}
    \sup_{b ≥ 1} \Pr_b[X_{(n)} ≤ c]
    &= \sup_{b≥ 1} \Pr_b[X_{(n)}/b ≤ c/b] \\
    &= \sup_{b ≥ 1} \Pr_b[X₁/b ≤ c/b, ..., X_n/b ≤ c/b] \\
    &= \sup_{b ≥ 1} \Pr_b[X₁/b ≤ c/b] ⋯ \Pr_b[X_n/b ≤ c/b] \\
    &= \sup_{b ≥ 1} cⁿ/bⁿ \\
    &= cⁿ
  \end{align*}
  where we're using the fact that $X_i/b ∼ uniform(0,1)$.  So
  $c = α^{1/n}$
\end{itemize}
\item plugging in specific numbers
\begin{itemize}
\item code:
\begin{itemize}
\item n <- 10
\item crit <- 0.05\^(1/n)
\item b <- 1
\item stats <- replicate(10000, max(runif(n, 0, b)))
\item hist(stats, 150)
\item mean(stats <= crit)
\item repeat for n = 100, n = 1000, and discuss different critical values
\end{itemize}
\item suppose $n = 10$ and $α = 0.05$
\begin{itemize}
\item gives critical value of $0.05^{0.10} = 0.74$
\end{itemize}
\item calculate the maximum of the 10 numbers we draw and
          compare it to $0.74$
\begin{itemize}
\item if it is less than 0.74, we reject the null hypothesis
            that $b ≥ 1$
\item otherwise, we don't reject
\end{itemize}
\item if $n = 100$, the critical value becomes 0.97
\item as we get more observations, the rejection region grows
          and the test becomes more powerful while preserving the
          correct size.
\end{itemize}
\end{itemize}
\end{itemize}

\subsection{Linear regression LRT \textbf{:hw:}}

\section{Neyman-Pearson Lemma}

\begin{itemize}
\item Helps us think about the best possible power
\item Def of UMP: Let C be a class of tests for testing $θ ∈ Θ_0$
  against $θ ∈ Θ_0^c$.  A test in C with power function $β(θ)$ is a
  \emph{uniformly most powerful} (UMP) class C test if $β(θ) ≥ β'(θ)$
  for all $θ ∈ Θ^c$ and all $β'$ that are power functions of tests in
  C.
\item Usefulness really shows up in asymptotics
\item Often there is no UMP class C test
\begin{itemize}
\item sometimes we can get a UMP test by restricting C further
\end{itemize}
\end{itemize}

\subsection{Statement of Lemma}

\begin{itemize}
\item Consider testing $θ = θ_0$ vs $θ = θ₁$ where the pdf for each
  parameter value is $f(x; θ_i)$, $i = 1,2$, using a test with
  rejection region $R$ such that, for some $k ≥ 0$,
\begin{itemize}
\item $x ∈ R$ if $f(x; θ₁) > k f(x; θ_0)$
\item $x ∉ R$ if $f(x; θ₁) < k f(x; θ_0)$
\end{itemize}
\item Then
\begin{enumerate}
\item Any size $α$ test with such a rejection region is a UMP level $α$-test
\item If there is a size $α$ test with such a rejection region for $k
  > 0$, then every UMP level $α$ test has size $α$ and every UMP level
  $α$ test has this rejection region almost surely.
\end{enumerate}
\item Notes
\begin{itemize}
\item both are strict inequalities to allow for discrete random
         variables
\item justifies the LRT (at least for simple tests)
\item most UMP results for more complicated settings are extensions
         of this lemma
\item more or less how we think of power in econometrics
\end{itemize}
\end{itemize}

%%% Local Variables:
%%% mode: latex
%%% TeX-master: "../inference"
%%% End:

% Copyright © 2013, authors of the "Core Econometrics Textbook;" a
% complete list of authors is available in the file AUTHORS.tex.

% Permission is granted to copy, distribute and/or modify this
% document under the terms of the GNU Free Documentation License,
% Version 1.3 or any later version published by the Free Software
% Foundation; with no Invariant Sections, no Front-Cover Texts, and no
% Back-Cover Texts.  A copy of the license is included in the file
% LICENSE.tex and is also available online at
% <http://www.gnu.org/copyleft/fdl.html>.

\part*{Inference on linear regression}%
\addcontentsline{toc}{part}{Inference on linear regression}

\section{Estimation of linear regression under restrictions}

\begin{itemize}

\item Let's start with a simple example (one that uses economic
  variables instead of letters; coincedentally, this is the same
  example used in \citealp[p. 81]{Gre12}).  Suppose we have a simple
  model of investment,
  \begin{equation}\label{eq:1}
    \log I_t = β₁ + β₂ i_t + β₃ Δp_t + β₄ \log Y_t + β_5 t + \ep_t
  \end{equation}
  where
  \begin{itemize}
  \item $I_t$ is investment in period $t$
  \item $i_t$ is the nominal interest rate
  \item $Δp_t$ is the rate of inflation
  \item $Y_t$ is real output
  \item $t$ is a time trend.
  \end{itemize}
  We might think that the inflation rate and the nominal interest rate
  don't matter individually, but that the real interest rate is a key
  driver of investment decisions.  That would imply that the
  investment equation should be (to a rough approximation),
  \begin{equation}\label{eq:2}
    \log I_t = β₁ + β₂ (i_t - Δp_t) + β₄ \log Y_t + β_5 t + \ep_t.
  \end{equation}
  
  We can estimate this new model by using $i_t - Δp_t$ as a regressor
  instead of $i_t$ and $Δp_t$, but we can also estimate the model by
  estimating~\eqref{eq:1} under the constraint $β₂ = - β₃$.

\item In general, this amounts to solving an optimization problem
  under a constraint.  We'll focus on linear constraints here for
  simplicity, so the estimator becomes
  \begin{equation}\label{eq:4}
    \βh = \argmin_β ∑_{i=1}^n (y_i - x_i'β)² = \argmin_β (Y - Xβ)(Y-Xβ)
  \end{equation}
  subject to the constraint
  \begin{equation}\label{eq:5}
    R β = q
  \end{equation}
  where $R$ and $q$ are an arbitrary $j × k$ matrix ($j ≤ k$) and $j ×
  1$ vector respectively that are known and set by the researcher and
  $R$ is assumed to have rank $j$.

  Most economics graduate students will have solved dozens of
  constrained optimization problems by the time they read this
  passage, so we'll just do a sketch of the
  solution.\footnote{\citet{SB94} is a reasonably comprehensive
    resource for results like these and it is typically required
    reading by graduate economics programs.}  Set up the Lagrangian
  \begin{equation}\label{eq:6}
    (Y - Xβ)'(Y - Xβ) + (Rβ - q)' λ
  \end{equation}
  and take derivatives with respect to $β$ to get the first order
  conditions
  \begin{equation}\label{eq:3}
    0 = 2X'X β^* - 2 X'Y + R'λ^*
  \end{equation}
  and the original constraint~\eqref{eq:5}, where the star indicates
  that the variable solves the constrained optimization problem.
  
  We can rewrite Equation~\eqref{eq:3} as
  \begin{equation}\label{eq:8}
    β^* = \βh - (1/2) (X'X)^{-1} R'λ^*,
  \end{equation}
  where $\βh$ is the usual OLS estimator, and premultiplying by $R$
  gives
  \begin{equation}\label{eq:7}
    Rβ^* = R \βh - (1/2) R (X'X)^{-1} R'λ^*.
  \end{equation}
  Since $Rβ^* = q$,~\eqref{eq:7} determines $λ^*$:
  \begin{equation}\label{eq:9}
    (1/2) λ^* = (R (X'X)^{-1} R')^{-1} (R \βh - q)
  \end{equation}
  and substituting~\eqref{eq:9} into~\eqref{eq:8} gives the solution,
  \begin{equation}\label{eq:10}
    β^* = \βh - (X'X)^{-1} R' (R (X'X)^{-1} R')^{-1} (R \βh - q)
  \end{equation}

\item Notice that $Xβ^*$ can be interpreted as a projection onto a
  subspace of the columns of $X$.  If we define $Z = X (X'X)^{-1} R'$
  then
  \begin{equation*}
    X β^* = (X(X'X)^{-1}X' - Z(Z'Z)Z') Y ≡ (P_X - P_Z) Y,
  \end{equation*}
  and notice that $P_X P_Z = P_Z$.  On reflection this should not be
  suprising.  If the restriction amounts to imposing that some of the
  coefficients are zero, it's obvious.  Otherwise, there's a rotation
  of the design matrix $X$ such that the restriction is equivalent to
  imposing zero on some of the coefficients in the rotated design
  matrix.

\item The restricted estimator of $σ²$ under this restriction is going
  to be $(1/(n - k + \rank(R))) ∑_i (y_i - x_i'β^*)²$

\end{itemize}

\section{Finite-sample hypothesis testing in linear regression}

\begin{itemize}

\item Suppose that we want to test an arbitrary linear hypothesis about
  $β$; ie \[R β = q\] against the alternative \[R β ≠ q\]
\begin{itemize}
\item ie $R = (1, 0, 0, ...)$ and $q=0$ gives us a test that $β₀=0$
\item $R = I$ and $q = (0,0,...,0)$ gives us a test that all of the
         coefficients are equal to zero.
\end{itemize}
\item for now, assume we have normal errors
\end{itemize}

\paragraph{change in SSR under the null}
      we can look at the change in the SSR when we impose the null
        hypothesis
\begin{itemize}
\item ie ${SSR_R - SSR \over SSR}$
\begin{itemize}
\item $SSR = ∑_i \eph²_i$
\item don't need to present this, but can be written as \[
  {(R² - R²_R) / J \over (1-R²) / (n-k-1)} \]
\end{itemize}
\item Our test is actually a scaled version of that:
  \[ F = {(SSR_R - SSR)/J \over SSR / (n-k-1)} \]
\begin{itemize}
\item $J$ is the number of restrictions (ie dimension of $q$)
\end{itemize}
\end{itemize}

\paragraph{Distribution of F under null}
      now, suppose the null is true, what is the distribution of $F$?

\paragraph{distribution of numerator}
\begin{itemize}
\item reexpress the numerator
  \begin{align*}
    SSR_R - SSR
    &= (Y - X\βh_R)'(Y - X\βh_R) - (Y - X\βh)'(Y - X\βh) \\
    &= (\βh - \βh_R)'X'X(\βh - \βh_R)
  \end{align*}
  (you're proving this for homework).  Remember that
  \begin{align*}
    \βh_R &= \βh + (X'X)^{-1} R'(R(X'X)^{-1}R')(q - R\βh) \\
    &= (q - R\βh)' (R(X'X)^{-1}R')^{-1} R(X'X)^{-1}X'X(X'X)^{-1}R' (R(X'X)^{-1}R')^{-1} (q - R\βh) \\
    &= (q - R\βh)' (R(X'X)^{-1}R')^{-1} R(X'X)^{-1} R'(R(X'X)^{-1}R')^{-1} (q - R\βh) \\
    &= (q - R\βh)' (R(X'X)^{-1}R')^{-1} (q - R\βh)
  \end{align*}
\item distribution of numerator
\begin{itemize}
\item $q - R\βh$ is normal with mean $q - Rβ$ and variance $σ²
  R(X'X)^{-1}R'$
\item under the null, this mean is zero.
\item so we have a normal divided by its variance covariance matrix...
\item so $(q - R\βh)'(R(X'X)^{-1}R')^{-1}(q - R\βh)$ equals $σ²$
  chi-square r.v. with $J$ degrees of freedom
\end{itemize}
\end{itemize}

\paragraph{distribution of denominator}
\begin{itemize}
\item just like earlier, we know that $SSR = (n-k-1) s²$ and $s²$
         and $\βh$ are independent given $X$.
\item denominator is $σ²$ times a chi-square with $n-k$
         degrees of freedom and indpendent of the numerator.
\end{itemize}

\paragraph{distribution of statistic}
       distribution of $F$
\begin{itemize}
\item $F = {σ² χ²_J / J \over σ² χ²_{n-k-1} / (n-k-1)}$ in
  distribution.
\item numerator and denominator are independent
\item so this has an $F_{J, n-k-1}$ distribution under the null.
\end{itemize}

\paragraph{Distribution of F under alternative}
\begin{itemize}
\item Denominator is not affected
\item Numerator is
\begin{itemize}
\item for $R\βh - q$ to have mean zero, we need the null to
          be true
\item otherwise the numerator will get larger
\end{itemize}
\end{itemize}

\subsection{t-test}

\begin{itemize}

\item Suppose you wanted to test a single restriction, say $β_i = b$
  for some known $b$.
\item We know that ${\βh_i - β_i \over \sqrt{s² q_i}}$ is the ratio of
  a standard normal r.v. and a chi-square/(n-k-1) random variable
\item so it is $t$ with (n-k-1) degrees of freedom
\item under the null, we know $β_i = b$, so we also have ${\βh_i - b
    \over \sqrt{s² q_i}}$
\item we can use this as a test statistic:
  \begin{itemize}
  \item calculate the value of the r.v.
  \item get the appropriate critical values from the t-distribution
    table (or the computer)
  \item reject if the statistic is farther from zero than the critical
    value
  \end{itemize}
\item If we're testing an equality, this will give us exactly the same
  test as the F-test with 1 and $n-k-1$ degrees of freedom
\item if we're testing an inequality, this test can be a little easier
  to work with.
\end{itemize}

%%% Local Variables:
%%% mode: latex
%%% TeX-master: "../inference"
%%% End:

% Copyright © 2013, authors of the "Core Econometrics Textbook;" a
% complete list of authors is available in the file AUTHORS.tex.

% Permission is granted to copy, distribute and/or modify this
% document under the terms of the GNU Free Documentation License,
% Version 1.3 or any later version published by the Free Software
% Foundation; with no Invariant Sections, no Front-Cover Texts, and no
% Back-Cover Texts.  A copy of the license is included in the file
% LICENSE.tex and is also available online at
% <http://www.gnu.org/copyleft/fdl.html>.

\part*{Bayesian statistics}%
\addcontentsline{toc}{part}{Bayesian statistics}

\section{Introduction}

\begin{itemize}

\item We're going to take a fairly moderate perspective on Bayesian
  statistics.  In contrast to point estimation or hypothesis testing,
  here we characterize our uncertainty about the unknown parameters of
  the likelihood function by modeling them as random parameters as
  well.  Note that this does ont require us to actually think that
  they are random; it's just a convenient way to characterize our
  uncertainty.

\item The density before we observe any data is called our
  \emph{prior} density.  The goal of the statistical exercise now is
  to update that density based on the observed data, giving us a
  \emph{posterior} density.  In the problems that we study in this
  book, both Bayesian and Frequentist estimators will be relatively
  well-behaved, and the prior will not have much of an effect when the
  samples are large.  That's not always the case, though.

\end{itemize}

\subsection{mathematics}

\begin{itemize}
\item dead simple.
\item denote the whole sample as $X$ (ie $X₁,...,X_n$)
\begin{itemize}
\item we'll assume that this is iid, but that's not important
\end{itemize}
\item suppose that $X_i ∼ f_X(·; θ)$
\begin{itemize}
\item this specification is part of the model
\item this part looks like what we've done
\end{itemize}
\item now, suppose that $θ ∼ f_θ$
\begin{itemize}
\item so the previous line really should be
  \[X_i ∣ θ ∼ f_X(·, θ)\]
\item this density represents what we believe about the unknown
  parameters $θ$ before we look at any data
\item called the \emph{prior} distribution
\item parameters on this density are called ``hyperparameters''
\end{itemize}
\item We're going to base decisions on the posterior distribution of
  $θ$, which is updated to reflect what we've learned from the data
\begin{itemize}
\item posterior distribution is the conditional distribution
  $f_θ(t ∣ X)$
\item we get this using Bayes's rule:
  \[f_θ(t ∣ X = x) = \frac{f_X(x ∣ θ = t) f_θ(t)}{f_X(x)}\]
\end{itemize}
\item If we need to represent the posterior by a single number, we
       can do this as long as we have a loss function
\begin{itemize}
\item Remember, a loss function is some function $L$
\begin{itemize}
\item convex
\item $L(0) = 0$
\item these conditions can be generalized
\end{itemize}
\item estimator is the value $\θh$ that minimizes
  \[ \E(L(θ - \θh) ∣ X₁,...,X_n) \]
\begin{itemize}
\item expectation is over $θ$
\end{itemize}
\item similar to certainty-equivalence.
\end{itemize}
\end{itemize}

\subsection{example}

\begin{itemize}
\item Suppose $X_i ∼ i.i.d.\ bernoulli(θ)$
\begin{itemize}
\item we're interested in $∑_i X_i ∼\ bernoulli(n, θ)$
\item $f(x ∣ θ) = \binom{n}{x} θ^x (1-θ)^{n-x}$
\end{itemize}
\item A prior might be that $θ ∼\ uniform(0,1)$
\end{itemize}

\section{Choice of prior}

\begin{itemize}
\item There are three ``standard'' methods for choosing a priod
\begin{itemize}
\item Uninformative prior
\begin{itemize}
\item ad hoc
\item formalized
\end{itemize}
\item Conjugate prior
\item carefully considered prior (this is less common)
\end{itemize}
\end{itemize}

\subsection{Uninformative/objective prior}

\begin{itemize}
\item One chosen to (in a way) minimize the knowledge embodied in
       the prior
\item Often people use ad hoc uninformative priors
\begin{itemize}
\item can give misleading results
\end{itemize}
\item Formal method of getting uninformative prior:
\begin{itemize}
\item prior density should be proportional to $\sqrt{|I(θ)|}$
\begin{itemize}
\item ``abs'' is determinant
\item $I(θ)$ is the `information matrix' from the Cramer-Rao lower bound
  \[I_{ij}(θ) = \E\Big[ \frac{∂}{∂ θ_i} \log f_X(X ∣ θ) · \frac{∂}{∂ θ_j}
  \log f_X(X ∣ θ) \Big]\]
\end{itemize}
\end{itemize}
\item Reference prior
\begin{itemize}
\item Chooses the prior distribution that maximizes the expected
  distance between prior ($f_θ(·)$) and posterior ($f_θ(· ∣ X)$)
  distributions
\item often gives same result as Jeffreys Prior
\item \citet{Ber05} gives a useful review of these priors.
\end{itemize}
\end{itemize}

\subsection{Conjugate prior}

\begin{itemize}

\item A conjugate prior is one that is chosen so that the prior and
  the posterior are in the same family.  This can be computationally
  and analytically convenient, since it means that the posterior has a
  closed-form solution.

\item Definition: \citep[From][7.2.3]{CB02} Let $\mathcal{F}$
  denote the class of pdfs $f_X(x ∣ θ)$.  A class $Π$ of prior
  distributions is a \emph{conjugate family} for $\mathcal F$ if the
  posterior is in the class $Π$ for all $f$ in $\mathcal F$, all
  priors in $Π$, and all $x$.

\item The uniform distribution (which we looked at earlier) is a
  special case of beta distribution with parameters (1,1).  We'll show
  here that the beta distribution is a conjugate prior for our earlier
  example.  Consider beta prior with parameters $α$ and $β$ and $X ∼
  \binomial(n,p)$ as the DGP, so
  \begin{align*}
    p_θ(t)      &= \frac{Γ(α + β) t^{α-1} (1-t)^{β-1}}{Γ(α) Γ(β)}&
    p_{X}(S ∣ θ) &= \binom{n}{S} θ^S (1-θ)^{n-S} \\ 
    &&          &= \frac{Γ(n + 1)}{Γ(S + Γ(n - S + 1)} θ^S (1-θ)^{n-s}
  \end{align*}
  
  To get the posterior, combine the distributions as before:
  \begin{align*}
    p_θ(t ∣ data)
    &= K × \frac{Γ(n + 1)}{Γ(S + 1) Γ(n - S + 1)}
       t^S (1-t)^{n-S} × \frac{Γ(α + β) t^{α-1} (1-t)^{β-1}}{Γ(α) Γ(β)} \\
    &= K × \frac{Γ(n + 1) Γ(α + β)}{Γ(S + 1) Γ(n - S + 1) Γ(α) Γ(β)}
       t^{α + S - 1} (1 - t)^{β + n - S - 1}
  \end{align*}

  We can recognize that this is the kernel of a beta$(α+S, β+n-S)$
  density again; so for it to be a density, we have to have the
  normalizing factor,
  \begin{equation*}
    K = \Big(\frac{Γ(n + 1) Γ(α + β)}{Γ(S + 1) Γ(n - S +1)Γ(α)Γ(β)}\Big)^{-1}
        \frac{Γ(α + β + n)}{Γ(α + S)Γ(β + n)}
  \end{equation*}
  And the posterior mean is $(S + α) / (n + α + β)$.

  In general, for any bernoulli, if the prior is $beta(α,β)$ the
  posterior is beta$(S + α, N - S + β)$ Note that this analysis
  suggests that the uniform is not a good way to model ``no
  information'' in this problem; it is equivalent to having observed a
  single success and a single failure in addition to the real data
  set.  The beta(0,0) would be equivalent to having observed no
  additional data beyond the sample.
\end{itemize}

\paragraph{advantages}
\begin{itemize}
\item mathematical convenience
\item interpretation as an augmented dataset
\begin{itemize}
\item Another way to interpret the prior in this case
\begin{itemize}
\item suppose we'd observed other data previously
\item $α$ is the number of successes we observed
\item $β$ is the number of failures
\end{itemize}
\item Increasing $α$ and $β$ decreases the variance of the prior
\begin{itemize}
\item makes the prior have a larger effect on our estimator
\end{itemize}
\item we see this for some families of distributions (exponential family)
\end{itemize}
\end{itemize}

\subsection{Asymptotics}

\begin{itemize}
\item What happens as $N → ∞$ (for conjugate prior)
\item $α$ and $β$ are fixed, so $\frac{R + α}{R + G + α + β}$ behaves
  like $\frac{R}{R+G}$
\item ie the prior doesn't matter and the Bayesian estimator
       (squared error loss) and MLE behave the same
\item typically the case for general priors, but one can construct
       counterexamples
\end{itemize}

\section{Implementing the prior numerically}

\begin{itemize}
\item for conjugate priors, everything is easy (if you know the analytical form)
\item otherwise, finding the density is kind of a bitch:
  \[f_θ(t ∣ X = x) ∝ f_X(x ∣ θ = t) f_θ(t)\]
\item On the computer, though, it's easy using metropolis-hastings algorithm
\begin{itemize}
\item remember the accept-reject algorithm we used for homework
\begin{itemize}
\item want to draw from some $X ∼ f$ w/ support between 0 and 1,
         height between zero and one.
\begin{itemize}
\item draw candidate $X$ from uniform distribution
\item draw another variable $V$ from uniform distribution
\item if $V ≤ f(X)$, keep $X$, otherwise draw another $X$ and $V$.
\end{itemize}
\item can be extended to non-uniform for both
\item problem, though:
\begin{itemize}
\item requires candidate density to be ``larger'' than real density
\begin{description}
\item[.] really talking about in the tails.
\item[.] draw
\end{description}
\end{itemize}
\item metropolis algorithm gets around this restriction
\begin{itemize}
\item one problem: gives asymptotic/approximate results
\end{itemize}
\end{itemize}
\end{itemize}
\end{itemize}

\subsection{description of algorithm \citep[page 254]{CB02}}

Suppose you can generate $V ∼ f_V$, and the density $f_Y$ has the same
support
\begin{enumerate}
\item Generate $Z₀ ∼ f_V$
\item To generate $Z_i$, $i > 0$:
\begin{itemize}
\item Generate $V_i ∼ f_V$
\item Set $Z_i$
\begin{itemize}
\item $= V_i$ with probability $\min\{ f_Y(V_i)/ f_V(V_i)
  × f_V(Z_{i-1}) / f_Y(Z_{i-1}), 1\}$
\end{itemize}
\item $= Z_{i-1}$ otherwise
\end{itemize}
\item As $i → ∞$, $Z_i → f_Y$ in distribution.
\end{enumerate}

\subsection{simple example}

\paragraph{ergodicity}
\begin{itemize}
\item let $V =$
\begin{itemize}
\item 0 with probability 1/4
\item 1 with probability 3/4
\end{itemize}
\item let $Y =$
\begin{itemize}
\item 0 with probability 1/2
\item 1 with probability 1/2
\end{itemize}
\item suppose $Z_{i-1} = 0$ and draw $V_i$
\begin{itemize}
\item if $V_i = 0$, $Z_i = 0$ always
\item if $V_i = 1$, $Z_i$ equals
\begin{itemize}
\item 1 with probability 1/3
\item 0 with probability 2/3
\end{itemize}
\item so $Z_i ∣ Z_{i-1} = 0$ is
\begin{itemize}
\item 0 with prob $1/4 + 2/3 × 3/4 = 3/4$
\item 1 with prob $1/3 × 3/4 = 1/4$
\end{itemize}
\end{itemize}
\item suppose $Z_{i-1} = 1$ and draw $V_i$
\begin{itemize}
\item if $V_i = 0$, $Z_i$ is
\begin{itemize}
\item 0 with probability $\min(.5/.25 × .75/.5, 1) =
            \min(3,1) = 1$
\item 1 with prob 0
\end{itemize}
\item if $V_i = 1$, $Z_i$ is 1 always
\item so $Z_i ∣ Z_{i-1} = 1$ is
\begin{itemize}
\item 0 with prob 1/4
\item 1 with prob 3/4
\end{itemize}
\end{itemize}
\item Now, suppose we have $Z_{i-1} ∼ f_Y$
\begin{itemize}
\item 0 with prob 1/2
\item 1 with prob 1/2
\end{itemize}
\item Then $\Pr[Z_i = 0] = \Pr[Z_i = 0 ∣ Z_{i-1}=0] \Pr[Z_{i-1} = 0] + \Pr[Z_i = 0 ∣ Z_{i-1}=1] \Pr[Z_{i-1} = 1]$
\begin{itemize}
\item $= \frac{3}{4} \frac{1}{2} + \frac{1}{4} \frac{1}{2} = \frac12$
\end{itemize}
\item So once we have a draw from $f_Y$, we can use the
        metropolis-hastings algorithm to generate another draw from $f_y$
\end{itemize}

\paragraph{getting the initial draw from $f_Y$}
\begin{itemize}
\item we can represent the ``transitions'' as a matrix
\item set \[P = \begin{pmatrix} 3/4 & 1/4 \\ 1/4 & 3/4 \end{pmatrix}\]
\item each row represents $\Pr[Z_i = z ∣ Z_{i-1} = z']$
\begin{itemize}
\item different rows give different $z'$
\end{itemize}
\item we can let the row-vector $Z₀ = (1-p, p)$ represent the
  probabilities that the initial $Z₀$ is 0 or 1.
\item Then, $Z₀ P$ gives probabilities that $Z₁$ is zero or 1
\begin{itemize}
\item $Z₀ P²$ gives probs that $Z₂$ is zero or 1
\item dot dot dot
\item $Z₀ P^i$ gives probs that $Z_i$ is zero or 1
\end{itemize}
\item we can diagonalize $P$
        \[P = Q D Q^{-1}\]
        with
        \[Q = \begin{pmatrix} 1 & -1/4 \\ 1 &
        1/4 \end{pmatrix},\qquad 
        Q^{-1} = \begin{pmatrix} 1/2 & 1/2 \\ -2 & 2 \end{pmatrix},
        \qquad 
        D = \begin{pmatrix} 1 & 0 \\ 0 & 1/2 \end{pmatrix}\]
\begin{itemize}
\item Q gives the right-eigenvectors of P
\item $Q^{-1}$ gives the left eigenvectors of $P$
\item D gives the eigenvalues of P
\end{itemize}
\item note that 
  \[P^i = (Q D Q^{-1})ⁿ = Q Dⁿ Q^{-1} = Q \begin{pmatrix} 1ⁿ & 0 \\
    0 & 0.5ⁿ \end{pmatrix} Q^{-1} → Q \begin{pmatrix} 1ⁿ & 0 \\ 0 &
    0 \end{pmatrix} = \begin{pmatrix} 0.5 & 0.5 \\ 0.5 &
    0.5 \end{pmatrix}\]
\item so $Z₀ Pⁿ = (1-p, p) Pⁿ → ((1-p)/2 + p/2, (1-p)/2 + p/2) =
  (1/2, 1/2)$
\begin{itemize}
\item no matter what density we start with, after a large number
          of steps we'll get draws from close to the marginal
          distribution.
\item quick numeric example (going to be much better behaved than
          typical mcmc application)
\begin{itemize}
\item \texttt{mpower <- function(M, power) \{}
\item \texttt{Mp <- M}
\item \texttt{for (i in seq\_len(power-1)) Mp <- Mp \%*\% M}
\item \texttt{Mp\}}
\item \texttt{P <- matrix(c(3/4,1/4,1/4,3/4), 2, 2)}
\item \texttt{c(1,0) \%*\% P}
\item \texttt{c(1,0) \%*\% mpower(P, 3)}
\item \texttt{c(1,0) \%*\% mpower(P, 10)}
\end{itemize}
\end{itemize}
\end{itemize}

\paragraph{wrap up of example}
\begin{itemize}
\item we made some simplifications
\begin{itemize}
\item finite number of possible values
\end{itemize}
\item basic result holds much more generally (ie continuous rv)
\end{itemize}

\subsection{numeric example}

\begin{itemize}
\item data: $X ∣ θ ∼$ binom($10, θ$)
\begin{itemize}
\item \texttt{n <- 10}
\item \texttt{x <- rbinom(1, n, runif(1))}
\item \texttt{x}
\end{itemize}
\item want to find
\begin{enumerate}
\item posterior distribution
\item minimum risk estimtor for squared-error (ie the mean of the posterior)
\item estimator that minimizes $\E(L(p - \hat p) ∣ X$ where
  $L(e) = \exp(-e) - 1$ if $e < 0$, \emph{e} if $e ≥ 0$
\begin{itemize}
\item \texttt{L2 <- function(e) ifelse(e < 0, exp(-e) - 1, e)}
\item \texttt{curve(L2(x), from = -1, to = 1)}
\end{itemize}
\end{enumerate}
\item prior: $θ ∼ k φ(x)$ if $x ∈ (1/4, 3/4)$, 0 otherwise
\begin{itemize}
\item $k$ chosen so that it integrates to 1
\item not necessarily realistic
\item note that Bayes rule tells us that anywhere the prior is zero,
         the posterior will be zero as well, so the posterior's going to be 
         between 1/4 and 3/4
\item need a function that will generate from prior:
\begin{itemize}
\item \texttt{rprior <- function() \{}
\begin{itemize}
\item \texttt{repeat \{}
\item \texttt{x <- rnorm(1)}
\item \texttt{if (x > 0.25 \& x < 0.75) break}
\item \texttt{\}}
\item \texttt{x}
\item \texttt{\}}
\end{itemize}
\item \texttt{hist(replicate(600, rprior()), 40, freq = FALSE, xlim = c(0,1))}
\end{itemize}
\item need a function that will evaluate (proportional to) prior:
\begin{itemize}
\item \texttt{dprior <- function(p) ifelse(p > 0.25 \& p < 0.75, dnorm(p), 0)}
\item \texttt{curve(dprior(x), from = 0, to = 1, n = 1000, add = TRUE)}
\end{itemize}
\item need to evaluate the density of the posterior:
\begin{itemize}
\item \texttt{dposterior <- function(p, x, n) dbinom(x, n, p) * dprior(p)}
\item \texttt{curve(dposterior(x), from = 0, to = 1, col = "red", add = TRUE, n = 1000)}
\end{itemize}
\item note: this is a toy example, so we really could just use this 
         function to calculate the mean, minimal loss estimator, etc.
\begin{itemize}
\item \texttt{mass <- integrate(function(p) dposterior(p, x, n), lower = 0.25, upper = 0.75)}
\item expected value: \texttt{integrate(function(p) p * dposterior(p, x, n) / mass\$value, lower = 0.25, upper = 0.75)}
\item minimum loss: \texttt{optimize(f = function(phat) integrate(function(p) L2(p - phat) * dposterior(p, x, n) / mass\$value, lower = 0.25, upper = 0.75)\$value, lower = 0.25, upper = 0.75)}
\item \textbf{but} in bigger problems (more parameters, more complicated distributions), these
           direct solutions won't be feasible... they'll take too long.
\item metropolis hastings can still work in those bigger problems
\end{itemize}
\end{itemize}
\item to implement, initialize vector of draws (``burn'' first 1000, use second 1000)
\begin{itemize}
\item \texttt{nsim <- 2000}
\item \texttt{Z <- c(rprior(), rep(NaN, nsim))}
\end{itemize}
\item make function that generates $Z_i$ given $Z_{i-1}$
\begin{itemize}
\item \texttt{rZ <- function(Zprev, rv, fv, fy) \{}
\begin{itemize}
\item \texttt{V <- rv()}
\item \texttt{U <- runif(1)}
\item \texttt{probV <- fy(V) * fv(Zprev) / (fv(V) * fy(Zprev))}
\item \texttt{if (U < probV) return(V)}
\item \texttt{else return(Zprev)\}}
\end{itemize}
\end{itemize}
\item run mcmc
\begin{itemize}
\item \texttt{for (i in seq\_len(nsim) + 1)}
\begin{itemize}
\item \texttt{Z[i] <- rZ(Z[i-1], rv = rprior, fv = dprior, fy = function(p) dposterior(p, x, n))}
\end{itemize}
\end{itemize}
\item get posterior:
\begin{itemize}
\item \texttt{hist(Z[-(1:1000)], 40)}
\item \texttt{curve(dposterior(x)/mass\$value, from = 0, to = 1, col = "red", add = TRUE, n = 1000)}
\end{itemize}
\item get estimates:
\begin{itemize}
\item expected value: \texttt{mean(Z[-(1:1000)])}
\item minimum loss: \texttt{optimize(function(phat) mean(L2(Z[-(1:1000)] - phat)), lower = 0.25, upper=0.75)}
\end{itemize}
\item look at the actual draws of $Z$
\begin{itemize}
\item \texttt{plot(Z[1500:1700], type = "o")}
\item really dependent (which makes sense), ie not iid draws
\item doesn't matter, because there is weak enough dependence that we
         can still get averages, etc.
\end{itemize}
\item what happens if our prior doesn't actually contain the ``true'' $p$?
\begin{itemize}
\item draw $x$
\begin{itemize}
\item \texttt{x <- rbinom(1, n, .9)}
\end{itemize}
\item reset $Z$ and rerun mcmc (same as before)
\begin{itemize}
\item \texttt{Z <- c(rprior(), rep(NaN, nsim))}
\item \texttt{for (i in seq\_len(nsim) + 1)}
\begin{itemize}
\item \texttt{Z[i] <- rZ(Z[i-1], rv = rprior, fv = dprior, fy = function(p) dposterior(p, x, n))}
\end{itemize}
\end{itemize}
\item \texttt{hist(Z[-(1:1000)], 40)}
\item bunches up at edge of support
\end{itemize}
\end{itemize}

\section{Bayesian modeling for OLS}

\begin{itemize}
\item For OLS, we have to specify a joint prior for
\begin{itemize}
\item $σ$ (ie $\E(\ep²_i ∣ X)$)
\item $β$
\item For convenience, we'll specify a prior for $β$ conditional on $σ²$.
\end{itemize}
\item We'll discuss conjugate and noninformative priors
\item Need to specify distribution of the random variables (Y): $N(Xβ,
  σ² I)$ given $X$
\begin{itemize}
\item We'll continue to condition on $X$ (assume X and $β$ and $γ$ are
  independent).
\end{itemize}
\end{itemize}

\subsection{densities for beta}

\begin{itemize}
\item Finite sample theory and asymptotics indicate that
  $\βh$ is going to behave as though it is approximately normal.
\item Suggests that getting a normal posterior might be a good idea
\item suggests that starting with a normal prior (for congugate) is too.
\item Prior :: $p(β ∣ 1/σ²) = N(β₀, σ² Σ₀)$
\begin{itemize}
\item ie prior is normal with known mean and variance
\end{itemize}
\end{itemize}

\paragraph{posterior distribution}
\begin{itemize}
\item Use Bayes' rule to get posterior:
  \[ p(β ∣ X, Y, σ^{2}) = \frac{p(Y ∣ X, β, σ^{2}) p(β ∣ X,
    σ^{2})}{p(Y ∣ X, σ^{2})}\]
\item $Y$ given $X$, $β$, and $σ^{2}$ is (obviously) normal
\begin{itemize}
\item mean $Xβ$
\item variance $σ² I$
\end{itemize}
\item posterior is
  \begin{multline}
    p(β ∣ X, Y, σ²) ∝ \big(\ov{\sqrt{2 π σ²}}\big)^n
    \exp\big(-\ov[(Y - Xβ)'(Y - Xβ)]{2σ²}\big) \\
    × \ov{\sqrt{2^k π^k \det(σ² Σ₀)}}
    \exp\big(- \ov{2} (β - β₀)'(σ² Σ₀)^{-1} (β - β₀)\big)
  \end{multline}
\begin{itemize}
\item terms inside the exponentials can be written as ($\σh²$ and
  $\βh$ are the MLE estimators)
  \[\exp\big(- \ov{2 σ²} (\eh - X(β - \βh))' (\eh - X(β - \βh))\big)\]
  and $(β - β₀)'Σ₀^{-1}(β - β₀)$,
  which can be rewritten as
  \begin{equation*}
    \exp\big(- \ov[n \σh²]{2 σ²}\big)
    \exp\big(- \ov{2σ²} (β - \βh)'X'X(β - \βh) + (β - β₀)Σ₀^{-1}(β - β₀)\big)
  \end{equation*}
\item To get this into something more useful, we want to rewrite it as
  \[(β - \βh)'X'X(β - \βh) + (β - β₀)Σ₀^{-1}(β - β₀) = (β - a β₀ - b \βh)'V(β - a β₀ - b \βh),\]
  then solve for $a$, $b$, and $V$.  For $V$ we have
  \begin{align*}
    β'Vβ &= β'X'Xβ + β Σ₀^{-1} β & ⇔ \\
    V    &= X'X + Σ₀^{-1},
  \end{align*}
  for $a$,
  \begin{align*}
    β'V a β₀ &= β'Σ₀^{-1}β₀ & ⇔ \\
    V a      &= Σ₀^{-1}     & ⇔ \\
    a        &= V^{-1} Σ₀^{-1} \\
             &= (X'X + Σ₀^{-1})^{-1} Σ₀^{-1},
  \end{align*}
  and for $b$,
  \begin{align*}
    β'V b \βh &= β'X'X \βh & ⇔ \\
    V b       &= X'X & ⇔ \\
    b         &= V^{-1} X'X \\
              &= (X'X + Σ₀^{-1})^{-1} X'X \\
              &= I - a.
  \end{align*}
  So we have 
  \[β'X'Xβ + β Σ₀^{-1} β  = (β - a β₀ - (I - a) \βh)'(X'X + Σ₀^{-1}) (β - a β₀ - (I - a) \βh)\] 
  with $a = (X'X + Σ₀^{-1})^{-1}Σ₀^{-1}$.
\item The posterior of $β$ is $N(a β₀ + (I - a)\βh, σ² (X'X + Σ₀^{-1})^{-1})$
\end{itemize}
\end{itemize}

\paragraph{interpretation from a classical perspective}
\begin{itemize}
\item We can get a classical estimator by using the posterior mean:
  \[ \βh_{bayes} = aβ₀ + (I-a)\βh \]
\item shrinkage estimator
\begin{itemize}
\item ridge is a special case with $β₀ = 0$ and $Σ₀ ∝ I$
\end{itemize}
\item Can think of the posterior as an average of the prior and MLE
\begin{itemize}
\item Will be biased (in general)
\item will have smaller variance than the OLS estimator.
\end{itemize}
\item In general, as sample size increases and as certainty in
        prior decreases, the ``Bayesian'' estimator behaves more like the
        OLS estimator
\end{itemize}

\paragraph{uninformative prior}
\begin{itemize}
\item $Σ₀$ denotes how strong our beliefs are
\begin{itemize}
\item small value indicates that we are very confident in our
           prior mean
\item large value indicates that we don't have a lot of
           confidence in our prior mean
\end{itemize}
\item As $Σ₀ → ∞$, $Σ₀^{-1}$ converges to zero
\begin{itemize}
\item mean of posterior converges to $\βh$
\item variance of posterior converges to $σ²(X'X)^{-1}$
\end{itemize}
\item ie, the Bayesian estimator converges to OLS as the prior
         becomes less informative
\item interpretation makes sense: when you don't have strong
         beliefs, you should weight the data heavily
\end{itemize}

\paragraph{asymptotics}
\begin{itemize}
\item As $n → ∞$, $X'X → ∞$ and $a → 0$
\item Same as with uninformative prior; bayesian estimator
         converges to OLS estimator as $n$ gets large
\item makes sense: when you have a lot of data, you shouldn't rely
         too heavily on your prior beliefs.
\end{itemize}

\subsection{densities for sigma-squared}

\begin{itemize}
\item Conjugate prior for $1/σ²$ is the gamma distribution, so $σ²$
  has an inverse-gamma distribution:
  \[p_{σ²}(σ² ∣ α, δ) = \ov[δ^α]{Γ(α)} (\ov{σ²})^{α+1} \exp\big(-\ov[β]{σ²}\big)\]
  with $α,δ > 0$.
\begin{itemize}
\item mean is $δ / (α - 1)$ for $α > 1$
\item variance is $β² / (α-1)²(α-2)$ for $α > 2$
\end{itemize}
\item Joint prior for $β$ and $1/σ²$ is called the Normal-Gamma prior:
  \begin{multline*}
    p_{β,σ²}(β, σ² ∣ Σ₀, α, δ) = \Big(\ov{\sqrt{2^k π^k \det(σ² Σ₀)}}\Big)
    \exp\big(- \ov{2} (β - β₀)'(σ² Σ₀)^{-1} (β - β₀)\big)
    \\ × \ov[δ^α]{Γ(α)} (\ov{σ²})^{α+1} \exp(-\ov[δ]{σ²})
  \end{multline*}
\item We can work out the posterior as before:
  \begin{align*}
    p(β, σ² ∣ X, Y) 
    &∝ \big(\ov{\sqrt{2 π σ²}}\big)^n
       \exp(-\ov{2σ²} (Y - Xβ)'(Y - Xβ)) \\
    &\quad × \big(\ov{\sqrt{2^k π^k \det(σ² Σ₀)}}\big) \exp(-\ov{2} (β - β₀)'(σ² Σ₀)^{-1} (β - β₀)) \\
    &\quad × \ov[δ^α]{Γ(α)} \big(\ov{σ²}\big)^{α+1} \exp(-\ov[δ]{σ²}) \\
    &= \big(\ov{\sqrt{2 π σ²}}\big)^n \big(\ov{\sqrt{2^k π^k \det(σ² Σ₀)}}\big)
       \exp(- \ov[n \σh²]{2 σ²})
       \ov[δ^α]{Γ(α)} \big(\ov{σ²}\big)^{α+1} \exp\big(- \ov[δ]{σ²}\big) \\
    &\quad × \exp\big(- \ov{2σ²} (β - \βh)'X'X(β - \βh) + (β - β₀)Σ₀^{-1}(β - β₀)\big)
  \end{align*}
  which is proportional to 
  \begin{multline*}
    \Big\{\Big(\ov{2πσ²}\Big)^{k/2} \det(X'X + Σ₀^{-1}) \\
          \exp\Big( -\ov{2σ²} (β - aβ₀ - (I-a)\βh)' (X'X + Σ₀^{-1})(β - aβ₀ - (I-a)\βh)) \Big\} \\
    \Big\{\Big(\ov{2πσ²}\Big)^{n/2} 
           \exp\Big(- \ov[n\σh²]{2σ²}\Big) 
           \ov[δ^α]{Γ(α)} \Big(\ov{σ²}\Big)^{α+1} \exp\Big(-\ov[δ]{σ²}\Big) \Big\}
  \end{multline*}
\begin{itemize}
\item The first part is the posterior of $β$ conditional on $σ²$.
\item We can rewrite the second part to be more interpretable:
\item combining terms and dropping scale constants gives:
  \[(\frac{1}{σ²})^{n/2 + α + 1} \exp(- \frac{\σh² n/2 + δ}{σ²})\]
\item after thinking for a bit, we can see that this is the kernel of an inverse
         gamma with parameters
\begin{itemize}
\item $n/2+α$ and
\item $\σh² n/2 + δ$
\end{itemize}
\item mean is
  \[\frac{\σh² n/2 + δ}{n/2 + α - 1}
  = \σh² \frac{n}{n + 2α - 2} + δ(\frac{2}{n + 2α - 2})\]
\item to interpret this,
\begin{itemize}
\item define prior mean: $σ₀² = \frac{δ}{α - 1}$
\item $δ = σ²₀ (α - 1)$
\item posterior mean becomes 
  \[\σh² \frac{n}{n + 2α - 2} + σ₀²(\frac{2 α - 2}{n + 2α - 2}) 
  = w \σh² + (1-w) σ₀²\]
\item Bayes estimate is a weighted average of the MLE and the
           prior mean
\item as $n → ∞$, $w → 1$ and MLE dominates
\item we can write prior variance as $σ₀^4 / (α - 2)$
\begin{itemize}
\item for $σ₀$ fixed, as $α → ∞$ prior
             variance converges to zero and $w → 0$ as well, so
             posterior mean dominates.
\item for $σ₀$ fixed, as $α → 1$ from above, $w$
             converges to 1 again and MLE dominates
\end{itemize}
\end{itemize}
\end{itemize}
\end{itemize}

\subsection{putting it all together}

After the analysis above, we can write the posterior density of $β$
and $σ²$ explicitly as
\begin{multline*}
  \Big(\ov{2πσ²}\Big)^{k/2} \det(X'X + Σ₀^{-1})
  \exp\Big(-\ov{2σ²} (β-aβ₀-(I-a)\βh)'(X'X+Σ₀^{-1})(β-aβ₀-(I-a)\βh)\Big) \\
  × \frac{(\σh² n/2 + δ)^{α + n/2}}{Γ(n/2 + α)}
    \Big(\ov{σ²}\Big)^{n/2+α+1}\exp\Big(-\ov{\σh² n/2 + δ}{σ²}\Big)
\end{multline*}

%%% Local Variables:
%%% mode: latex
%%% TeX-master: "../inference"
%%% End:

% Copyright © 2013, authors of the "Econometrics Core" textbook; a
% complete list of authors is available in the file AUTHORS.tex.

% Permission is granted to copy, distribute and/or modify this
% document under the terms of the GNU Free Documentation License,
% Version 1.3 or any later version published by the Free Software
% Foundation; with no Invariant Sections, no Front-Cover Texts, and no
% Back-Cover Texts.  A copy of the license is included in the file
% LICENSE.tex and is also available online at
% <http://www.gnu.org/copyleft/fdl.html>.

\part*{Problem set, hypothesis testing}%
\addcontentsline{toc}{part}{Problem set, hypothesis testing}

\begin{enumerate}

\item Let $\{y_t\}$ be an iid $N(0,σ²)$ sequence.  Define $S_T =
  (T-1)^{-1} ∑_{t=1}^T (y_t - \bar y)²$.
  \begin{enumerate}
  \item Prove that $\sqrt{n} (S_T - σ²) → N(0,2σ⁴)$ in
    distribution.
  \item Calculate the asymptotic distribution of $\log S_T$.
  \item Do your results depend on the normality of $y_t$?
  \item Calculate the 90th and 95th percentile for $S_T$ with
    arbitrary values of $σ²$ using both of these asymptotic
    distributions.
  \item Use each of these results to derive a test of the null
    hypothesis $σ² = σ₀²$ against the null $σ² > σ₀²$, where $σ₀²$ is
    a known but arbitrary value.  Your answer should give two formulas
    for the test's critical value---each one depends on $σ₀²$ and $α$
    (the nominal size of the test) and you should have a separate
    answer for each asymptotic approximation.
  \item Let $c₁$ and $c₂$ denote the 90th percentiles that you
    calculated in question ?.  Simulate 1000 i.i.d. standard normal
    samples with $n = 50$ and calculate the probability that $S_T$ is
    less than each of these percentiles.
  \item Repeat the previous question for the 95th percentiles.
  \item Plot a histogram of your 1000 simulated $S_T$ along with each
    approximate density for $S_T$.  What do these results tell you
    about the quality of the approximations?
  \item You can also prove that $(T-1) S_T$ has a chi-square
    distribution with $T-1$ degrees of freedom in finite samples.
    Calculate the 90th and 95th percentiles of $S_T$ using this
    chi-square distribution and repeat the previous two simulations.
    How do these simulations compare to the previous simulations?
  \item Repeat the previous three questions using a skewed
    distribution and a heavy-tailed distribution (changing the
    approximation as necessary).  How do the results change?  How do
    they change if you use different values of $n$?
  \item What do these simulations tell you about using these
    approximations for testing.  Focus on the usual confidence levels
    (i.e. 10\%, 5\%, and 1\% tests).
  \end{enumerate}

\item Suppose that $X₁,…,X_n$ are distributed uniform$(0,b)$.  Derive
  the LRT for the null hypothesis $b ≥ b₀$ against the alternative $b
  < b₀$ and also for the null hypothesis $b ≤ b₀$ against $b > b₀$.
  Please discuss and compare the tests.

\item Suppose that $X₁,…,X_n$ are i.i.d. uniform($a$,$b$).  Derive the
  LRT of the null hypothesis $\E X_i = 0$ against the two-sided
  alternative $\E X_i ≠ 0$.

\item Let $X = (X₁,…,X_n)$ be a random sample and let $θ$ be some
  parameter of interest.  For each $θ₀$, let $A(θ₀)$ be the acceptance
  region of a level $α$ test of the null hypothesis that $θ = θ₀$.
  For each sample $x$, define a set $C(x)$ in the parameter space
  by
  \begin{equation}
     C(x) = \{ θ₀ : x ∈ A(θ₀) \}.
  \end{equation}
  Prove that the random set $C(X)$ is a $1-α$ confidence set for the
  parameter $θ$.

\end{enumerate}

%%% Local Variables: 
%%% mode: latex
%%% TeX-master: "../inference"
%%% End: 


% The files `AUTHORS_standalone.tex` and `LICENSE_standalone.tex` are
% available if you want to distribute the author list and the FDL on
% their own.
\addtocontents{toc}{\protect\setcounter{tocdepth}{0}}

\newpage
\part*{Complete list of authors}%
\addcontentsline{toc}{part}{Appendix A: Complete list of authors}
% Copyright © 2013, authors of the "Econometrics Core" textbook; a
% complete list of authors is available in the file AUTHORS.tex.

% Permission is granted to copy, distribute and/or modify this
% document under the terms of the GNU Free Documentation License,
% Version 1.3 or any later version published by the Free Software
% Foundation; with no Invariant Sections, no Front-Cover Texts, and no
% Back-Cover Texts.  A copy of the license is included in the file
% LICENSE.tex and is also available online at
% <http://www.gnu.org/copyleft/fdl.html>.

% Remove the next two lines if you are distributing the author list as
% a standalone pdf.
\noindent%
The following is a list of the contributors to the Econometrics Free
Library Project's \textit{Econometrics Core}, in order of their date
of first involvement (yes, I'm aware that it's a little ridiculous to
have this as a separate file when there is only a single contributor,
but let's dream big, shall we).

\begin{description}
\item[2009-07-01] Gray Calhoun, \email{gcalhoun@iastate.edu}
\end{description}

%%% Local Variables:
%%% mode: latex
%%% TeX-master: "AUTHORS_standalone"
%%% End:

\newpage
\part*{GNU Free Documentation License}%
\addcontentsline{toc}{part}{Appendix B: GNU Free Documentation License}
% Remove the next two lines if you are distributing the author list as
% a standalone pdf.
\part*{GNU Free Documentation License}%
\addcontentsline{toc}{part}{Appendix B: GNU Free Documentation License}
\setcounter{section}{-1}%
\renewcommand\thesection{\arabic{section}}%
\noindent Version 1.3, 3 November 2008

\noindent Copyright \copyright\ 2000, 2001, 2002, 2007, 2008 Free
Software Foundation, Inc.

\noindent \texttt{<http://fsf.org/>}
 
\noindent Everyone is permitted to copy and distribute verbatim copies
of this license document, but changing it is not allowed.

\section{Preamble}

The purpose of this License is to make a manual, textbook, or other
functional and useful document ``free'' in the sense of freedom: to
assure everyone the effective freedom to copy and redistribute it,
with or without modifying it, either commercially or noncommercially.
Secondarily, this License preserves for the author and publisher a way
to get credit for their work, while not being considered responsible
for modifications made by others.

This License is a kind of ``copyleft'', which means that derivative
works of the document must themselves be free in the same sense.  It
complements the GNU General Public License, which is a copyleft
license designed for free software.

We have designed this License in order to use it for manuals for free
software, because free software needs free documentation: a free
program should come with manuals providing the same freedoms that the
software does.  But this License is not limited to software manuals;
it can be used for any textual work, regardless of subject matter or
whether it is published as a printed book.  We recommend this License
principally for works whose purpose is instruction or reference.


\section{APPLICABILITY AND DEFINITIONS}

This License applies to any manual or other work, in any medium, that
contains a notice placed by the copyright holder saying it can be
distributed under the terms of this License.  Such a notice grants a
world-wide, royalty-free license, unlimited in duration, to use that
work under the conditions stated herein.  The ``\textbf{Document}'',
below, refers to any such manual or work.  Any member of the public is
a licensee, and is addressed as ``\textbf{you}''.  You accept the
license if you copy, modify or distribute the work in a way requiring
permission under copyright law.

A ``\textbf{Modified Version}'' of the Document means any work containing the
Document or a portion of it, either copied verbatim, or with
modifications and/or translated into another language.

A ``\textbf{Secondary Section}'' is a named appendix or a front-matter section of
the Document that deals exclusively with the relationship of the
publishers or authors of the Document to the Document's overall subject
(or to related matters) and contains nothing that could fall directly
within that overall subject.  (Thus, if the Document is in part a
textbook of mathematics, a Secondary Section may not explain any
mathematics.)  The relationship could be a matter of historical
connection with the subject or with related matters, or of legal,
commercial, philosophical, ethical or political position regarding
them.

The ``\textbf{Invariant Sections}'' are certain Secondary Sections whose titles
are designated, as being those of Invariant Sections, in the notice
that says that the Document is released under this License.  If a
section does not fit the above definition of Secondary then it is not
allowed to be designated as Invariant.  The Document may contain zero
Invariant Sections.  If the Document does not identify any Invariant
Sections then there are none.

The ``\textbf{Cover Texts}'' are certain short passages of text that are listed,
as Front-Cover Texts or Back-Cover Texts, in the notice that says that
the Document is released under this License.  A Front-Cover Text may
be at most 5 words, and a Back-Cover Text may be at most 25 words.

A ``\textbf{Transparent}'' copy of the Document means a machine-readable copy,
represented in a format whose specification is available to the
general public, that is suitable for revising the document
straightforwardly with generic text editors or (for images composed of
pixels) generic paint programs or (for drawings) some widely available
drawing editor, and that is suitable for input to text formatters or
for automatic translation to a variety of formats suitable for input
to text formatters.  A copy made in an otherwise Transparent file
format whose markup, or absence of markup, has been arranged to thwart
or discourage subsequent modification by readers is not Transparent.
An image format is not Transparent if used for any substantial amount
of text.  A copy that is not ``Transparent'' is called ``\textbf{Opaque}''.

Examples of suitable formats for Transparent copies include plain
ASCII without markup, Texinfo input format, LaTeX input format, SGML
or XML using a publicly available DTD, and standard-conforming simple
HTML, PostScript or PDF designed for human modification.  Examples of
transparent image formats include PNG, XCF and JPG.  Opaque formats
include proprietary formats that can be read and edited only by
proprietary word processors, SGML or XML for which the DTD and/or
processing tools are not generally available, and the
machine-generated HTML, PostScript or PDF produced by some word
processors for output purposes only.

The ``\textbf{Title Page}'' means, for a printed book, the title page itself,
plus such following pages as are needed to hold, legibly, the material
this License requires to appear in the title page.  For works in
formats which do not have any title page as such, ``Title Page'' means
the text near the most prominent appearance of the work's title,
preceding the beginning of the body of the text.

The ``\textbf{publisher}'' means any person or entity that distributes
copies of the Document to the public.

A section ``\textbf{Entitled XYZ}'' means a named subunit of the Document whose
title either is precisely XYZ or contains XYZ in parentheses following
text that translates XYZ in another language.  (Here XYZ stands for a
specific section name mentioned below, such as ``\textbf{Acknowledgements}'',
``\textbf{Dedications}'', ``\textbf{Endorsements}'', or ``\textbf{History}''.)  
To ``\textbf{Preserve the Title}''
of such a section when you modify the Document means that it remains a
section ``Entitled XYZ'' according to this definition.

The Document may include Warranty Disclaimers next to the notice which
states that this License applies to the Document.  These Warranty
Disclaimers are considered to be included by reference in this
License, but only as regards disclaiming warranties: any other
implication that these Warranty Disclaimers may have is void and has
no effect on the meaning of this License.


\section{VERBATIM COPYING}

You may copy and distribute the Document in any medium, either
commercially or noncommercially, provided that this License, the
copyright notices, and the license notice saying this License applies
to the Document are reproduced in all copies, and that you add no other
conditions whatsoever to those of this License.  You may not use
technical measures to obstruct or control the reading or further
copying of the copies you make or distribute.  However, you may accept
compensation in exchange for copies.  If you distribute a large enough
number of copies you must also follow the conditions in section~3.

You may also lend copies, under the same conditions stated above, and
you may publicly display copies.


\section{COPYING IN QUANTITY}

If you publish printed copies (or copies in media that commonly have
printed covers) of the Document, numbering more than 100, and the
Document's license notice requires Cover Texts, you must enclose the
copies in covers that carry, clearly and legibly, all these Cover
Texts: Front-Cover Texts on the front cover, and Back-Cover Texts on
the back cover.  Both covers must also clearly and legibly identify
you as the publisher of these copies.  The front cover must present
the full title with all words of the title equally prominent and
visible.  You may add other material on the covers in addition.
Copying with changes limited to the covers, as long as they preserve
the title of the Document and satisfy these conditions, can be treated
as verbatim copying in other respects.

If the required texts for either cover are too voluminous to fit
legibly, you should put the first ones listed (as many as fit
reasonably) on the actual cover, and continue the rest onto adjacent
pages.

If you publish or distribute Opaque copies of the Document numbering
more than 100, you must either include a machine-readable Transparent
copy along with each Opaque copy, or state in or with each Opaque copy
a computer-network location from which the general network-using
public has access to download using public-standard network protocols
a complete Transparent copy of the Document, free of added material.
If you use the latter option, you must take reasonably prudent steps,
when you begin distribution of Opaque copies in quantity, to ensure
that this Transparent copy will remain thus accessible at the stated
location until at least one year after the last time you distribute an
Opaque copy (directly or through your agents or retailers) of that
edition to the public.

It is requested, but not required, that you contact the authors of the
Document well before redistributing any large number of copies, to give
them a chance to provide you with an updated version of the Document.


\section{MODIFICATIONS}

You may copy and distribute a Modified Version of the Document under
the conditions of sections 2 and 3 above, provided that you release
the Modified Version under precisely this License, with the Modified
Version filling the role of the Document, thus licensing distribution
and modification of the Modified Version to whoever possesses a copy
of it.  In addition, you must do these things in the Modified Version:

\begin{itemize}
\item[A.] 
   Use in the Title Page (and on the covers, if any) a title distinct
   from that of the Document, and from those of previous versions
   (which should, if there were any, be listed in the History section
   of the Document).  You may use the same title as a previous version
   if the original publisher of that version gives permission.
   
\item[B.]
   List on the Title Page, as authors, one or more persons or entities
   responsible for authorship of the modifications in the Modified
   Version, together with at least five of the principal authors of the
   Document (all of its principal authors, if it has fewer than five),
   unless they release you from this requirement.
   
\item[C.]
   State on the Title page the name of the publisher of the
   Modified Version, as the publisher.
   
\item[D.]
   Preserve all the copyright notices of the Document.
   
\item[E.]
   Add an appropriate copyright notice for your modifications
   adjacent to the other copyright notices.
   
\item[F.]
   Include, immediately after the copyright notices, a license notice
   giving the public permission to use the Modified Version under the
   terms of this License, in the form shown in the Addendum below.
   
\item[G.]
   Preserve in that license notice the full lists of Invariant Sections
   and required Cover Texts given in the Document's license notice.
   
\item[H.]
   Include an unaltered copy of this License.
   
\item[I.]
   Preserve the section Entitled ``History'', Preserve its Title, and add
   to it an item stating at least the title, year, new authors, and
   publisher of the Modified Version as given on the Title Page.  If
   there is no section Entitled ``History'' in the Document, create one
   stating the title, year, authors, and publisher of the Document as
   given on its Title Page, then add an item describing the Modified
   Version as stated in the previous sentence.
   
\item[J.]
   Preserve the network location, if any, given in the Document for
   public access to a Transparent copy of the Document, and likewise
   the network locations given in the Document for previous versions
   it was based on.  These may be placed in the ``History'' section.
   You may omit a network location for a work that was published at
   least four years before the Document itself, or if the original
   publisher of the version it refers to gives permission.
   
\item[K.]
   For any section Entitled ``Acknowledgements'' or ``Dedications'',
   Preserve the Title of the section, and preserve in the section all
   the substance and tone of each of the contributor acknowledgements
   and/or dedications given therein.
   
\item[L.]
   Preserve all the Invariant Sections of the Document,
   unaltered in their text and in their titles.  Section numbers
   or the equivalent are not considered part of the section titles.
   
\item[M.]
   Delete any section Entitled ``Endorsements''.  Such a section
   may not be included in the Modified Version.
   
\item[N.]
   Do not retitle any existing section to be Entitled ``Endorsements''
   or to conflict in title with any Invariant Section.
   
\item[O.]
   Preserve any Warranty Disclaimers.
\end{itemize}

If the Modified Version includes new front-matter sections or
appendices that qualify as Secondary Sections and contain no material
copied from the Document, you may at your option designate some or all
of these sections as invariant.  To do this, add their titles to the
list of Invariant Sections in the Modified Version's license notice.
These titles must be distinct from any other section titles.

You may add a section Entitled ``Endorsements'', provided it contains
nothing but endorsements of your Modified Version by various
parties---for example, statements of peer review or that the text has
been approved by an organization as the authoritative definition of a
standard.

You may add a passage of up to five words as a Front-Cover Text, and a
passage of up to 25 words as a Back-Cover Text, to the end of the list
of Cover Texts in the Modified Version.  Only one passage of
Front-Cover Text and one of Back-Cover Text may be added by (or
through arrangements made by) any one entity.  If the Document already
includes a cover text for the same cover, previously added by you or
by arrangement made by the same entity you are acting on behalf of,
you may not add another; but you may replace the old one, on explicit
permission from the previous publisher that added the old one.

The author(s) and publisher(s) of the Document do not by this License
give permission to use their names for publicity for or to assert or
imply endorsement of any Modified Version.


\section{COMBINING DOCUMENTS}

You may combine the Document with other documents released under this
License, under the terms defined in section~4 above for modified
versions, provided that you include in the combination all of the
Invariant Sections of all of the original documents, unmodified, and
list them all as Invariant Sections of your combined work in its
license notice, and that you preserve all their Warranty Disclaimers.

The combined work need only contain one copy of this License, and
multiple identical Invariant Sections may be replaced with a single
copy.  If there are multiple Invariant Sections with the same name but
different contents, make the title of each such section unique by
adding at the end of it, in parentheses, the name of the original
author or publisher of that section if known, or else a unique number.
Make the same adjustment to the section titles in the list of
Invariant Sections in the license notice of the combined work.

In the combination, you must combine any sections Entitled ``History''
in the various original documents, forming one section Entitled
``History''; likewise combine any sections Entitled ``Acknowledgements'',
and any sections Entitled ``Dedications''.  You must delete all sections
Entitled ``Endorsements''.

\section{COLLECTIONS OF DOCUMENTS}

You may make a collection consisting of the Document and other documents
released under this License, and replace the individual copies of this
License in the various documents with a single copy that is included in
the collection, provided that you follow the rules of this License for
verbatim copying of each of the documents in all other respects.

You may extract a single document from such a collection, and distribute
it individually under this License, provided you insert a copy of this
License into the extracted document, and follow this License in all
other respects regarding verbatim copying of that document.


\section{AGGREGATION WITH INDEPENDENT WORKS}

A compilation of the Document or its derivatives with other separate
and independent documents or works, in or on a volume of a storage or
distribution medium, is called an ``aggregate'' if the copyright
resulting from the compilation is not used to limit the legal rights
of the compilation's users beyond what the individual works permit.
When the Document is included in an aggregate, this License does not
apply to the other works in the aggregate which are not themselves
derivative works of the Document.

If the Cover Text requirement of section~3 is applicable to these
copies of the Document, then if the Document is less than one half of
the entire aggregate, the Document's Cover Texts may be placed on
covers that bracket the Document within the aggregate, or the
electronic equivalent of covers if the Document is in electronic form.
Otherwise they must appear on printed covers that bracket the whole
aggregate.


\section{TRANSLATION}

Translation is considered a kind of modification, so you may
distribute translations of the Document under the terms of section~4.
Replacing Invariant Sections with translations requires special
permission from their copyright holders, but you may include
translations of some or all Invariant Sections in addition to the
original versions of these Invariant Sections.  You may include a
translation of this License, and all the license notices in the
Document, and any Warranty Disclaimers, provided that you also include
the original English version of this License and the original versions
of those notices and disclaimers.  In case of a disagreement between
the translation and the original version of this License or a notice
or disclaimer, the original version will prevail.

If a section in the Document is Entitled ``Acknowledgements'',
``Dedications'', or ``History'', the requirement (section~4) to Preserve
its Title (section~1) will typically require changing the actual
title.


\section{TERMINATION}

You may not copy, modify, sublicense, or distribute the Document
except as expressly provided under this License.  Any attempt
otherwise to copy, modify, sublicense, or distribute it is void, and
will automatically terminate your rights under this License.

However, if you cease all violation of this License, then your license
from a particular copyright holder is reinstated (a) provisionally,
unless and until the copyright holder explicitly and finally
terminates your license, and (b) permanently, if the copyright holder
fails to notify you of the violation by some reasonable means prior to
60 days after the cessation.

Moreover, your license from a particular copyright holder is
reinstated permanently if the copyright holder notifies you of the
violation by some reasonable means, this is the first time you have
received notice of violation of this License (for any work) from that
copyright holder, and you cure the violation prior to 30 days after
your receipt of the notice.

Termination of your rights under this section does not terminate the
licenses of parties who have received copies or rights from you under
this License.  If your rights have been terminated and not permanently
reinstated, receipt of a copy of some or all of the same material does
not give you any rights to use it.


\section{REVISIONS OF THIS LICENSE}

The Free Software Foundation may publish new, revised versions
of the GNU Free Documentation License from time to time.  Such new
versions will be similar in spirit to the present version, but may
differ in detail to address new problems or concerns.  See
\texttt{http://www.gnu.org/copyleft/}.

Each version of the License is given a distinguishing version number.
If the Document specifies that a particular numbered version of this
License ``or any later version'' applies to it, you have the option of
following the terms and conditions either of that specified version or
of any later version that has been published (not as a draft) by the
Free Software Foundation.  If the Document does not specify a version
number of this License, you may choose any version ever published (not
as a draft) by the Free Software Foundation.  If the Document
specifies that a proxy can decide which future versions of this
License can be used, that proxy's public statement of acceptance of a
version permanently authorizes you to choose that version for the
Document.

\section{RELICENSING}

``Massive Multiauthor Collaboration Site'' (or ``MMC Site'') means any
World Wide Web server that publishes copyrightable works and also
provides prominent facilities for anybody to edit those works.  A
public wiki that anybody can edit is an example of such a server.  A
``Massive Multiauthor Collaboration'' (or ``MMC'') contained in the
site means any set of copyrightable works thus published on the MMC
site.

``CC-BY-SA'' means the Creative Commons Attribution-Share Alike 3.0
license published by Creative Commons Corporation, a not-for-profit
corporation with a principal place of business in San Francisco,
California, as well as future copyleft versions of that license
published by that same organization.

``Incorporate'' means to publish or republish a Document, in whole or
in part, as part of another Document.

An MMC is ``eligible for relicensing'' if it is licensed under this
License, and if all works that were first published under this License
somewhere other than this MMC, and subsequently incorporated in whole
or in part into the MMC, (1) had no cover texts or invariant sections,
and (2) were thus incorporated prior to November 1, 2008.

The operator of an MMC Site may republish an MMC contained in the site
under CC-BY-SA on the same site at any time before August 1, 2009,
provided the MMC is eligible for relicensing.


\section*{ADDENDUM: How to use this License for your documents}
\addcontentsline{toc}{section}{ADDENDUM: How to use this License for your documents}

To use this License in a document you have written, include a copy of
the License in the document and put the following copyright and
license notices just after the title page:

\bigskip
\begin{quote}
    Copyright \copyright{}  YEAR  YOUR NAME.
    Permission is granted to copy, distribute and/or modify this document
    under the terms of the GNU Free Documentation License, Version 1.3
    or any later version published by the Free Software Foundation;
    with no Invariant Sections, no Front-Cover Texts, and no Back-Cover Texts.
    A copy of the license is included in the section entitled ``GNU
    Free Documentation License''.
\end{quote}
\bigskip
    
If you have Invariant Sections, Front-Cover Texts and Back-Cover Texts,
replace the ``with \dots\ Texts.''\ line with this:

\bigskip
\begin{quote}
    with the Invariant Sections being LIST THEIR TITLES, with the
    Front-Cover Texts being LIST, and with the Back-Cover Texts being LIST.
\end{quote}
\bigskip
    
If you have Invariant Sections without Cover Texts, or some other
combination of the three, merge those two alternatives to suit the
situation.

If your document contains nontrivial examples of program code, we
recommend releasing these examples in parallel under your choice of
free software license, such as the GNU General Public License,
to permit their use in free software.
\newpage
\part*{References}%
\addcontentsline{toc}{part}{References}
\bibliography{common/references,CITATION}
\end{document}

%%% Local Variables:
%%% mode: latex
%%% TeX-master: "inference"
%%% End:
