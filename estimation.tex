\documentclass[nohyper]{external/tufte-handout}
% The hyperref is loaded in the preamble with additional arguments to
% avoid some xelatex warnings.

\title{Core Econometrics Textbook Part 2: Point Estimation in Finite Samples}
% Copyright © 2013, authors of the "Core Econometrics;" a
% complete list of authors is available in the file AUTHORS.tex.

% Permission is granted to copy, distribute and/or modify this
% document under the terms of the GNU Free Documentation License,
% Version 1.3 or any later version published by the Free Software
% Foundation; with no Invariant Sections, no Front-Cover Texts, and no
% Back-Cover Texts.  A copy of the license is included in the file
% LICENSE.tex and is also available online at
% <http://www.gnu.org/copyleft/fdl.html>.

\newcommand{\version}{0.7.1}
\newcommand{\releasedate}{10 Dec. 2013}

%%% Local Variables:
%%% mode: latex
%%% TeX-master: "core_econometrics"
%%% End:

\author{Gray Calhoun} 
% (This comment is repeated in the Makefile)
% I'm still not sure the best way to do author information; I'm much
% more concerned in the long run about how different attributation
% styles would make someone more or less likely to contribute to an
% existing text or to license an existing draft.  For now, there's
% only one author, so I'll put myself as the author.  If someone else
% contributes any edits, etc., I'll change it to {Gray Calhoun and
% EFLP}.  If anyone wants to contribute a lot of original material and
% wants named authorship, please email the mailing list so we can
% discuss merging projects.

\usepackage{amssymb,amsmath,amsthm,verbatim}
\usepackage{fontspec,unicode-math,xltxtra,xunicode,booktabs}
\setromanfont[Ligatures=TeX]{TeX Gyre Pagella}
\setsansfont[Ligatures=TeX,Scale=MatchLowercase]{TeX Gyre Heros}
% \setmonofont[Scale=MatchLowercase]{Inconsolata}
\setmathfont{Asana-Math}

\frenchspacing
\setcounter{secnumdepth}{1}
\setcounter{tocdepth}{1}
\renewcommand\bibname{}
\renewcommand\refname{}
\renewcommand\contentsname{}
\bibliographystyle{abbrvnat}
\setcitestyle{round}
\newcommand{\email}[1]{\href{mailto:#1}{\nolinkurl{#1}}}
\newcommand{\homepage}{\url{http://www.econometricslibrary.org}}
\newcommand{\maillist}{\email{econometricslibrary@librelist.com}}
\newcommand{\bugtrack}%
{\url{https://github.com/EconometricsLibrary/CoreEconometricsText/issues}}

% Workaround for bugs in the tufte-latex class
\renewcommand\smallcapsspacing[1]{{\addfontfeature{LetterSpace = 8}\scshape#1}}
\renewcommand\allcapsspacing[1]{{\addfontfeature{LetterSpace = 15}#1}}
% Getting waringings from latexmk with default tufte-latex hyperref
\usepackage[unicode,pdfencoding=auto,hyperfootnotes=false,hidelinks]{hyperref}

\newcommand{\BibTeX}{Bib\!\TeX}
\newcommand{\pvalue}{\ensuremath{p}-value}
\newcommand{\ftest}{\ensuremath{F}-test}
\newcommand{\ttest}{\ensuremath{t}-test}

% Math shortcuts
\renewcommand{\Pr}{\operatorname{Pr}}

\DeclareMathOperator{\1}{1}
\DeclareMathOperator{\abs}{abs}
\DeclareMathOperator{\avar}{avar}
\DeclareMathOperator{\bias}{bias}
\DeclareMathOperator{\corr}{corr}
\DeclareMathOperator{\cov}{cov}
\DeclareMathOperator{\E}{E}
\DeclareMathOperator{\median}{median}
\DeclareMathOperator{\mse}{mse}
\DeclareMathOperator{\rank}{rank}
\DeclareMathOperator{\range}{range}
\DeclareMathOperator{\sd}{sd}
\DeclareMathOperator{\tr}{tr}
\DeclareMathOperator{\var}{var}

\DeclareMathOperator*{\argmax}{arg\,max}
\DeclareMathOperator*{\argmin}{arg\,min}
\DeclareMathOperator*{\plim}{plim}

\DeclareMathOperator{\binomial}{binomial}
\DeclareMathOperator{\bernoulli}{bernoulli}
\DeclareMathOperator{\invWishart}{inverse\ Wishart}
\DeclareMathOperator{\N}{N}
\DeclareMathOperator{\uniform}{uniform}

\newcommand{\BB}{\ensuremath{\mathbb{B}}}
\newcommand{\NN}{\ensuremath{\mathbb{N}}}
\newcommand{\PP}{\ensuremath{\mathbb{P}}}
\newcommand{\QQ}{\ensuremath{\mathbb{Q}}}
\newcommand{\RR}{\ensuremath{\mathbb{R}}}
\newcommand{\RRᵏ}{\ensuremath{\mathbb{R}ᵏ}}
\newcommand{\RRⁿ}{\ensuremath{\mathbb{R}ⁿ}}
\newcommand{\RRb}{\ensuremath{\bar{\mathbb{R}}}}
\newcommand{\ZZ}{\ensuremath{\mathbb{Z}}}

\newcommand{\Fs}{\ensuremath{\mathcal{F}}}
\newcommand{\Gs}{\ensuremath{\mathcal{G}}}
\newcommand{\Ps}{\ensuremath{\mathcal{P}}}

\newcommand{\ov}[2][1]{\tfrac{#1}{#2}}
\newcommand{\iid}{i.i.d.}

\newcommand{\ep}{\varepsilon}
\newcommand{\eph}{\hat{\varepsilon}}

\newcommand{\ah}{\hat{a}}
\newcommand{\αh}{\hat{α}}
\newcommand{\bh}{\hat{b}}
\newcommand{\βb}{\bar{β}}
\newcommand{\βh}{\hat{β}}
\newcommand{\βt}{\tilde{β}}
\newcommand{\eh}{\hat{e}}
\newcommand{\εb}{\bar{ε}}
\newcommand{\εh}{\hat{ε}}
\newcommand{\εt}{\tilde{ε}}
\newcommand{\ηh}{\hat{η}}
\newcommand{\Fh}{\hat{F}}
\newcommand{\λh}{\hat{λ}}
\newcommand{\μb}{\bar{μ}}
\newcommand{\μh}{\hat{μ}}
\newcommand{\Ωh}{\hat{Ω}}
\newcommand{\Qh}{\hat{Q}}
\newcommand{\Σh}{\hat{Σ}}
\newcommand{\sh}{\hat{s}}
\newcommand{\σh}{\hat{σ}}
\newcommand{\σb}{\bar{σ}}
\newcommand{\θh}{\hat{θ}}
\newcommand{\Vh}{\hat{V}}
\newcommand{\Xb}{\bar{X}}
\newcommand{\Xc}{\mathcal{X}}
\newcommand{\Xh}{\hat{X}}
\newcommand{\Xt}{\tilde{X}}
\newcommand{\Yb}{\bar{Y}}
\newcommand{\Yh}{\hat{Y}}
\newcommand{\Yt}{\tilde{Y}}
\newcommand{\yh}{\hat{y}}

\newcommand{\dx}{\,dx}
\newcommand{\dy}{\,dy}
\newcommand{\dμ}{\,dμ}
\newcommand{\dθ}{\,dθ}
\renewcommand{\dz}{\,dz}

\newtheorem{thm}{Theorem}[section]
\newtheorem{defn}{Definition}[section]
\newtheorem{ex}{Example}[section]
\newtheorem{asmp}{Assumption}[section]


\begin{document}
\maketitle

\bigskip\noindent%
Copyright © 2013, the authors of the \textit{Core Econometrics Textbook}.
A complete author list is included in Appendix A of this document.

Permission is granted to copy, distribute and/or modify this document
under the terms of the GNU Free Documentation License, Version 1.3 or
any later version published by the Free Software Foundation; with no
Invariant Sections, no Front-Cover Texts, and no Back-Cover Texts.  A
copy of the license is included in Appendix B of this document and is
also available online at \url{http://www.gnu.org/copyleft/fdl.html}.

This text was produced as part of the Econometrics Free Library
Project.  The project's goal is to produce high-quality free (and
open-source) econometrics textbooks and reference material.  More
information about this project is available at its homepage,
\homepage, including information on how
to participate.  This text is typeset in \LaTeX\ and there is a link
to the source of the document at the project's homepage as well.

Please cite this document as
\begin{itemize}
\item[] \bibentry{eflp-core}
\item[] \verbatiminput{CITATION.bib}
\end{itemize}
The \BibTeX\ record is provided for your convenience.
If the year and version are not specified, please see the files
\texttt{README.md} or \texttt{VERSION.tex} for directions on how to
find them.

\addcontentsline{toc}{part}{Table of Contents}
\tableofcontents

% Copyright © 2013, authors of the "Core Econometrics Textbook;" a
% complete list of authors is available in the file AUTHORS.tex.

% Permission is granted to copy, distribute and/or modify this
% document under the terms of the GNU Free Documentation License,
% Version 1.3 or any later version published by the Free Software
% Foundation; with no Invariant Sections, no Front-Cover Texts, and no
% Back-Cover Texts.  A copy of the license is included in the file
% LICENSE.tex and is also available online at
% <http://www.gnu.org/copyleft/fdl.html>.

\part*{Introduction}%
\addcontentsline{toc}{part}{Introduction}

\begin{itemize}

\item The \emph{Probability Overview} deals with hypothetical and
  theoretical relationships—where $X$ is a random vector that
  summarizes some experiment.  In this document, we imagine that some
  aspects of that experiment are known and some are unknown and that
  we observe $n$ outcomes, $X₁,…,X_n$, drawn from that experiment.
  The question we'll start to answer is, what can we learn about the
  unknown aspects of the experiment from the observations?

\item Give a concrete example from linear regression.

\item Finite sample, so we need to make implausibly strong assumptions
  about the DGP.  We'll look at ways to address this in the
  \emph{Asymptotics} document.  We'll find that a lot carries over.

\item There are many ways that one could split estimation strategies
  into different groups.  We'll find it useful to use these three
  categories,
  \begin{description}
  \item[Point Estimation:] An estimate of the parameter of interest,
    say $β$ in the linear regression examplem, ideally coupled with a
    measure of the precision of that estimate.

  \item[Hypothesis testing:] A binary outcome, yes or no, indicating
    whether or not the data are consistent with a proposed
    hypothesis.  For the regression example, we might have a
    hypothesis that the slope is zero: the mean of $y$ is the same,
    regardless of the value of $x$.  Note that, in any given data set,
    we might see a positive slope even if this hypothesis is true,
    since the data are generated by a random process.  This means that
    simply looking at the slope and rejecting whenever it's positive
    is a \emph{poor} strategy.

  \item[Bayesian inference:] In contrast to the other two, Bayesian
    inference is concerned with communicating the uncertainty about
    the unknown parameter (again, here it's $β$).  To do this, the
    researcher treats $β$ itself as another random variable with its
    own density, called a \emph{prior density}.  After observing
    $X₁,…,X_n$, the researcher then computes the \emph{posterior
      density}, $f_β(· ∣ X₁,…,X_n)$, which describes uncertainty over
    $β$ after observing the data.  As you can imagine, the choice of
    the prior density sometimes has a strong effect on the analysis
    and sometimes does not, and researchers are usually relieved when
    it does not.

    Sometimes it is natural to view $β$ as a random variable and
    sometimes it is quite unnatural.  Whether or not this assumption
    is realistic or not is irrelevant as far as applying a Bayesian
    analysis goes.  You can think of it as a mathematical model of
    uncertainty without taking it as an assumption about the real
    world.
  \end{description}
  
\item If $x₁,…,x_n$ is a sequence of random variables, any function of
  that sequence, $T(x₁,…,x_n)$ is called a \emph{statistic} or
  \emph{estimator}.  The important part of this defnintion is that an
  estimator can feasibly be constructed from any sequence possible
  outcomes of the random sequence, it can not depend on any additional
  information about the DGP.  For example, the sample mean $\Xb =
  n^{-1} ∑_{i=1}ⁿ X_i$ is an estimator because it is only a function
  of the observed data; the statistic $\max(\Xb, 10)$ is as well,
  because the number 10 is arbitrary and unconnected with the DGP.
  But the random variable $\max(\Xb, \E X₁ + 1)$ is not an estimator,
  because it depends on unknown information about the DGP
  (specifically, $\E X₁$).

\item Also note that some things that are probably pretty bad
  estimators meet our definition of \emph{estimator}.  For example,
  the constants ``30'' and ``1/2'' are both estimators, since they
  don't depend on any unknown features of the DGP.

\item When it is important to distinguish between the algorithm for
  generating a statistic from a dataset and the value that the
  statistic takes on for a particular dataset, we will call the
  algorithm \emph{the estimator} and the particular realization
  \emph{the estimate}.

\end{itemize}

%%% Local Variables: 
%%% mode: latex
%%% TeX-master: "../estimation"
%%% End: 

% Copyright © 2013, authors of the "Econometrics Core" textbook; a
% complete list of authors is available in the file AUTHORS.tex.

% Permission is granted to copy, distribute and/or modify this
% document under the terms of the GNU Free Documentation License,
% Version 1.3 or any later version published by the Free Software
% Foundation; with no Invariant Sections, no Front-Cover Texts, and no
% Back-Cover Texts.  A copy of the license is included in the file
% LICENSE.tex and is also available online at
% <http://www.gnu.org/copyleft/fdl.html>.

\part*{Deriving point estimators}%
\addcontentsline{toc}{part}{Overview of point estimation}

\section{Summarizing data sets}

\begin{itemize}

\item We'll start with a few very basic definitions.
  \begin{defn}
    The random variables $X₁,...,X_n$ are a \emph{random sample} if
    they form an i.i.d. draw from some distribution $F$.
  \end{defn}

  \begin{defn}
    If $T$ is a function from the sample space of $(X₁,...,X_n)$ to
    $\RR^k$ then the random variable $T(X₁,...,X_n)$ is a
    \emph{statistic}.
  \end{defn}

  Note that statistics are random variables and have distributions.  A
  statistic is anything that can be calculated from the data.  It can
  not depend on unknown parameters of the distribution.

\item It's worth spending a little while thinking about ``reasonable''
  ways to summarize datasets.  For now, we're only going to worry
  about mathematical characterizations of ``summarizing'' a dataset,
  we're not going to get into statistical graphics, etc.  One natural
  way to think about a data summary is through the idea of
  ``sufficient'' statistics.

  \begin{defn}
    If $X₁,...,X_n ∼ f(·; θ)$ is a draw from a known family of
    distributions with unknown parameter $θ$, a statistic $T(X)$ is a
    \emph{sufficient statistic for $θ$} if the conditional
    density/distribution $f(· ∣ T(X))$ does not depend on $θ$.
  \end{defn}

  The order statistics are always going to be a sufficient statistic
  for i.i.d. random variables, but there are often better (as in
  smaller) options, and we'll present a result soon that can help in
  constructing sufficient statistics.

  \begin{defn}
    $T(X)$ is a \emph{minimal sufficient statistic} if it is a
    function of any other sufficient statistic.
  \end{defn}

  First, though, there's the related concept of ancillarity.

  \begin{defn}
    If $X₁,...,X_n ∼ f(·; θ)$ is a draw from a known family of
    distributions with unknown parameter $θ$, a statistic $T(X)$ is an
    \emph{ancillary statistic for $θ$} if the sampling distribution of
    $T(X)$ does not depend on $θ$.
  \end{defn}

  Ancillary statistics can be informative about the precision of an
  estimate: for example, if $X₁,\ldots,X_n ∼ \uniform(θ,θ+1)$, we can
  see that $\range(X_i) = \max_i X_i - \min_i X_i$ is ancillary:
  \begin{align*}
    \max_i X_i - \min_i X_i
    &= \max_i (X_i - θ) - \min_i (X_i - θ) \\
    &=^d \max_i U_i - \min_i U_i
  \end{align*}
  where $U_i ∼ \uniform(0, 1)$.  But, if
  \begin{equation*}
    \θh = (1/2) \min_i X_i + (1/2) \max_i (X_i - 1)
  \end{equation*}
  this estimator will clearly be more precise if the range is close to
  one than if it is close to zero.  So samples with a large range are
  more informative about $θ$ than those with a small range.  This
  observation can sometimes make inference \emph{conditional on
  ancillary statistics} attractive, but we won't cover that topic in
  this document.  See \citet[Chapter 10]{LR:05} for further discussion.

\item The ``factorization theorem'' is useful in determining whether
  or not a given statistic is sufficient for a parameter.
  \begin{thm}
    If $f(x; θ)$ is the joint pdf of $(X₁,...,X_n)$, $T(X)$ is a
    sufficient statistic for $θ$ if and only if
    \begin{equation*}
      f(x; θ) = g(T(x); θ) h(x)
    \end{equation*}
    for some functions $g$ and $h$.
  \end{thm}
  The important thing to note is that $T(x)$ and $θ$ are both in $g$
  and $θ$ is not in $h$.  The other thing to note is the ``if and only
  if'' part of the proof.

  It is straightforward to prove that sufficiency implies the
  factorization holds: let $h(x)$ be the joint density of $x$ given
  $T(X)$.  It is less straightforward, but more important in practice,
  to prove that existence of the factorization implies sufficiency.
  The result still follows from the conditional densities (as it seems
  that it must).
  \begin{align*}
    f_X(x ∣ T(X) = T(x); θ)
    &= f(x, T(x); θ) / f_T(T(x); θ) \\
    &= f_X(x; θ) / f_T(T(x); θ) \\
    &= g(T(x); θ) h(x) / f_T(T(x); θ)
  \end{align*}
  where the second equality holds because $T(x)$ is redundant when we
  know $x$, and the third equality holds because we're assuming that
  this factorization exists.

  Now we can write the denominator as
  \begin{align*}
    f_T(T(x); θ)
    &= ∫_A f(T(y) ∣ X = y; θ) f_X(y) \d y \\
    &= ∫_A f_X(y; θ) \d y \\
    &= ∫_A g(T(x); θ) h(y) \d y \\
    &= g(T(x); θ) ∫_A h(y) \d y
  \end{align*}
  where $A$ is the set of points s.t. $T(y) = T(x)$ for all $y ∈ A$
  and the equalities hold for the same reasons as before.

  Now just merge the two equations to get
  \begin{equation*}
    f_X(x ∣ T(X) = T(x); θ)
    = \frac{g(T(x); θ) h(x)}{g(T(x); θ) ∫_A h(y) \d y}
    = \frac{h(x)}{∫_A h(y) \d y}.
  \end{equation*}
  This last quantity does not depend on $θ$, so we're done.

  We can use the factorization theorem to prove that $\min_i X_i$ and
  $\max_i X_i$ are sufficient statistics for the $\uniform(a,b)$.
  \begin{ex}
    The joint pdf of $X₁,...,X_n$ is
    \begin{align*}
      f(x₁,...,x_n; a, b)
      &= ∏_{i=1}^n \tfrac{1}{b-a} \1\{x_i ∈ [a,b]\} \\
      &= \Big(\tfrac{1}{b-a}\Big)^n \1\{a ≤ \min_i x_i \text{ and } \max_i x_i ≤ b\},
    \end{align*}
    so we can define $h(x) = 1$ and
    \begin{equation*}
      g(T(x); a, b) = \Big(\tfrac{1}{b-a}\Big)^n \1\{a ≤ \min_i x_i \text{ and } \max_i x_i ≤ b\}.
    \end{equation*}
  \end{ex}

\item We can also think of the density function itself as giving a
  summary of the dataset.\sidenote{Note that we're implicitly assuming
    that we know the density function, but that's not really true.}
  Earlier, though, we viewed the densities as functions of the
  possible values of the random variables, $x$, given particular
  parameter values.  Now we're going to treat the random variables as
  known/observed and view the densities as functions of the possible
  parameterizations.  When we do that, we call them \emph{likelihoods}
  and will typically write them as $L(θ; x)$.

  [PLOT PICTURE OF UNIFORM LIKELIHOOD.]

  Regions where the likelihood is high are to some degree more
  plausible than regions where it is low.  If we look at the uniform
  dist; when the likelihood is higher, its corresponding density is
  more concentrated around the observed data.

\end{itemize}

\section{The method of moments}

\begin{itemize}
\item Suppose we have $X₁,...,X_n ∼ \iid\ f(x; θ)$ where $f$ is known
  and $θ$ is an unknown $p$-vector that we want to estimate.  The
  \emph{method of moments} estimator is based on a straightforward
  idea often called the ``analogy principle''—replace population
  expectations with averages to derive an estimator.  A good estimator
  of $\Pr[X_i ≤ c]$ is often $(1/n) ∑_i \1\{X_i ≤ c\}$, a good
  estimator of $\E X_i$ is often $(1/n) ∑_i X_i$, etc.

  To estimate $θ$, we first relate it to the first $p$ moments of
  $X_i$:
  \begin{equation}\label{eq:1}
    (μ₁,...,μ_p)' = (g₁(θ),...,g_p(θ))'
  \end{equation}
  where $μ_ℓ = \E X_i^ℓ$ and $g$ depends on the density $f$.  Then the
  estimator of $θ$ is the population vector that solves~\eqref{eq:1},
  so
  \begin{equation}\label{eq:2}
    \big((1/n) ∑_i X_i,...,(1/n) ∑_i X_i^p\big)
    = (g₁(\θh),...,g_p(\θh))'
  \end{equation}
  or
  \begin{equation*}
    \θh = g^{-1}\big((1/n) ∑_i X_i,...,(1/n) ∑_i X_i^p\big).
  \end{equation*}
  Obviously $g$ has to be inevitable for this to be a feasible
  estimator.  For this to be a reasonable estimator, we want each
  $μ_ℓ$ to be close (in some as of yet unspecified way) to $μ_ℓ$ and
  $g^{-1}$ to be continuous so that $\θh$ is correspondingly close to
  $θ$.

\item If it is difficult or impossible to get a closed form solution
  for $\θh$ you can solve~\eqref{eq:2} numerically.

\item We will see later that averages are easy to analyze with
  asymptotics (as the number of observations gets arbitrarily large)
  and, if $g^{-1}$ is smooth, $\θh$ will also be easy to analyze.

\item It might help to do a few examples.

\item %
  \begin{ex}
    Suppose that $X_i ∼ \iid\ \N(μ, σ²)$ where both $μ$ and $σ²$ are
    unknown.  The first two moments are
    \begin{equation*}
      (\E X_i, \E X_i²) = (μ, μ² + σ²)
    \end{equation*}
    and so the method of moments estimator of $(μ, σ²)$ is given by
    \begin{align*}
      (\μh, \σh^2)
      &= (\μh_1, \μh_2 - \μh_1^2) \\
      &= \big( (1/n) ∑_i X_i, (1/n) ∑_i (X_i - \Xb)² \big)
    \end{align*}
  \end{ex}

\item %
  \begin{ex}
    Suppose that $X_i ∼ \iid\ \uniform(a,b)$ where the parameters $a$
    and $b$ are both unknown.  We can again calculate the first two
    moments of $X_i$,
    \begin{align*}
      \E X_i
      &= \ov{b-a} ∫_a^b x \dx \\
      &= (b+a) / 2
    \intertext{and}
      \E X_i^2
      &= \ov{b-a} ∫_a^b x² \dx \\
      &= \frac{b^3 - a^3}{3(b - a)} \\
      &= (1/3) × (b² + ab + a²).
    \end{align*}
    These can be solved for $\ah$ and $\bh$:
    \begin{equation*}
      \ah = \bh - 2 \μh_1
    \end{equation*}
    and
    \begin{equation*}
      \bh^2 + (\bh - 2 \μh_1) \bh + (\bh - 2 \μh_1)² - 3 \μh_2 = 0
    \end{equation*}
    so
    \begin{equation*}
      (\ah, \bh) = (\μh_1 - \sh \sqrt{3}, \μh_1 + \sh \sqrt{3})
    \end{equation*}
    where $\sh = \sqrt{(1/n) ∑_i (X_i - \Xb)²}$.
  \end{ex}

  This is probably not a good estimator of $a$ and $b$.

\item %
  Suppose now that $(Y_i, X_i) ∼ \iid$ where the distribution of $X_i$
  is unspecified and $Y_i ∣ X_i ∼ N(β₀ + β₁ X_i, σ²)$.  We want to estimate
  $β₀$ and $β₁$ and will treat $σ²$ as a known constant for now.

  The method of moments estimator is based on $\E Y_i$ and $\E X_i
  Y_i$:
  \begin{equation*}
    \begin{pmatrix} \E Y_i \\ \E X_i Y_i  \end{pmatrix}
    =
    \begin{pmatrix}
      β₀ + β₁ \E X_i \\ β₀ \E X₁ + β₁ \E X₁²
    \end{pmatrix}
    =
    \begin{pmatrix}
      1 & \E X_i \\ \E X_i & \E X_i²
    \end{pmatrix}
    \begin{pmatrix} β₀ \\ β₁ \end{pmatrix}
  \end{equation*}
  So
  \begin{equation*}
    \begin{pmatrix} β₀ \\ β₁ \end{pmatrix}
    =
    \begin{pmatrix} 1 & \E X_i \\ \E X_i & \E X_i² \end{pmatrix}^{-1}
    \begin{pmatrix} \E Y_i \\ \E X_i Y_i  \end{pmatrix}
  \end{equation*}
  Assuming invertability of the matrix.  The estimator is (again,
  assuming invertability)
  \begin{equation*}
    \begin{pmatrix} \βh_0 \\ \βh_1 \end{pmatrix}
    = \begin{pmatrix} 1 & (1/n) ∑_i X_i \\ 
      (1/n) ∑_i X_i & (1/n) ∑_i X_i² \end{pmatrix}^{-1}
    \begin{pmatrix} (1/n) ∑_i Y_i \\ (1/n) ∑_i X_i Y_i  \end{pmatrix}
  \end{equation*}

\item The method of moments estimator has some advantages: it is
  usually a viable way to get an estimator and the estimator is
  usually easy to analyze, at least asymptotically.

\item The estimator has some disadvantages too.  It is usually going
  to be inefficient.  It should be clear that the estimator is
  reliable only when the first few moments have a lot of information
  about the distribution.  This is true for the Normal distribution
  (in the univariate case as well as the linear regression case), but
  not for the uniform.

  The estimators are not even guaranteed to satisfy distributional
  constraints: we could easily estimate values of $\ah$ and $\bh$ that
  are smaller than the smallest and largest observed values.

\item There is a variation of method of moments, \emph{Generalized
    Method of Moments} (usually abbreviated to GMM) \citep[derived
  by][as part of his Ph.D. dissertation]{Han:82}.  Many models in
  economics give implication that the period $t$ conditional
  expectation satisfies 
  \begin{equation*}
    \E_t g(X_t, θ) = 0
  \end{equation*}
  where $g(x_t, θ)$ comes from agent rationality—essentially,
  $g(X_t,θ)$ measures the agent's surprise, and if agents choose
  rationally in period $t$, then the difference between their
  prediction and the actual outcome must be unforecastable (if it
  weren't, they would have already acted).

  The LIE tells us that
  \begin{equation*}
    \E g(X_t, θ) = 0
  \end{equation*}
  unconditionally as well, so $θ$ can be estimated by solving
  \begin{equation*}
    (1/n) ∑_t g(X_t, θ) = 0
  \end{equation*}
  for $θ$.  If there are more equations than parameters, $θ$ can be
  estimated using a weighting scheme, and the additional equations can
  be used to test the adequacy model.  The key difference between GMM
  and the method of moments is that the equations here are generally
  derived from economic theory and not from distributional assumptions
  on the random variables.  Of course, the setup is very general, so
  it turns out that many estimators are special cases of GMM and can
  be analyzed using the same theory.\sidenote{\citet{Hay:00} gives a
    treatment of much of the material we cover, using GMM as the
    underlying analysis.}

\end{itemize}

\section{Maximum Likelihood Estimators}

\begin{itemize}

\item We discussed the likelihood function earlier as a method of
  summarizing a data set.  Notice that parameter values corresponding
  to higher values of the likelihood function also correspond to
  densities that are more tightly concentrated around the observed
  data.  This suggests that a good estimate might be the parameter
  value associated with the tightest density, i.e. the largest value
  of the likelihood function.

  \begin{ex} Suppose that $X₁,...,X_n ∼ \uniform(0, b)$.  Then the
    likelihood function (as before) is
    \begin{equation*}
      L(b; x₁,...,x_n) = ∏_i (1/b) \1\{0 ≤ x_i ≤ b\}
    \end{equation*}
    [Draw the likelihood and several densities]

    The maximum is clearly $b = \max x_i$.
  \end{ex}

\item The definition is unsurprising:
  \begin{defn}
    Suppose that $X₁,...,X_n ∼ f(x₁,...,x_n; θ)$.  The \emph{Maximum
      Likelihood Estimator} (MLE) of $θ$ is
    \begin{equation*}
      \θh = \argmax_θ L(θ; x₁,...,x_n).
    \end{equation*}
  \end{defn}

\item We can see that the usual OLS estimator is the MLE.
  \begin{ex} Suppose that $(Y_i,X_i)$ are \iid\ where $X_i$ is a $k ×
    1$ random vector with unspecified density $f$ and
    \begin{equation*}
      Y_i ∣ X_i ∼ N(X_i'β, σ²),
    \end{equation*}
    $β$ and $σ²$ are the parameters of interest.  The maximization
    problem will be easier of we take logs (remember, log is
    monotonic, so the maximizer of the log likelihood will be the same
    as that of the likelihood.  The log likelihood is
    \begin{equation*}
      \log L(β,σ²; x, y) = const - n\log (\sqrt{σ²}) -
      (1/2σ²) ∑_i (y_i - x_i'β)² + ∑_i f(x_i).
    \end{equation*}
    
    The first order conditions for $β$ are
    \begin{equation*}
      0 = (∂/∂β) \log L(β, σ²; x, y) = (1/σ²) ∑_i x_i (y_i - x_i'β)
    \end{equation*}
    which gives
    \begin{equation*}
      \βh = \big(∑_i x_i x_i'\big)^{-1} ∑_i x_i y_i.
    \end{equation*}
    The first order conditions for $σ²$ are
    \begin{equation*}
      0 = (∂/∂σ²) \log L(β, σ²; x, y) = -(n/2σ²) + (1/2σ⁴) ∑_i (y_i - x_i'β)²
    \end{equation*}
    which gives
    \begin{equation*}
      \σh² = (1/n) ∑_i (y_i - x_i'\βh)².
    \end{equation*}

    You should verify that this is a maximum using the second order
    conditions.
  \end{ex}

\item The derivative of the log likelihood shows up often and is
  called the \emph{score}.

\item Unlike method of moments, where we connect our parameters to the
  mean, variance, etc. regardless of the distribution; here we look at
  the features of the data that the distribution tells us are the most
  relevant.  For the normal, these features \emph{are} the mean and
  variance, so MLE and MoM give us the same statistics.  For the
  uniform, and others, this is \emph{not} the mean and variance.

\item The MLE has a nice invariance property: say you're not
  interested in the parameters per se, but care about a transformation
  of the parameters $T(θ)$.  If $\θh$ is the maximum likelihood
  estimator of $θ$, then $T(\θh)$ is the MLE of $T(θ)$.
\item We'll see later that you can use MLE to get an estimator, even
  if you don't believe that the distribution is true.  This estimator
  is called the \emph{quasi-maximum likelihood} estimator and will be
  (asymptotically) valid as long as the quasi MLE first order
  conditions are true for the correct distribution.

\end{itemize}

%%% Local Variables: 
%%% mode: latex
%%% TeX-master: "../estimation"
%%% End: 

%  LocalWords:  datasets dataset ancillarity parameterizations GMM





% Copyright © 2013, authors of "Core Econometrics Notes;" a
% complete list of authors is available in the file AUTHORS.tex.

% Permission is granted to copy, distribute and/or modify this
% document under the terms of the GNU Free Documentation License,
% Version 1.3 or any later version published by the Free Software
% Foundation; with no Invariant Sections, no Front-Cover Texts, and no
% Back-Cover Texts.  A copy of the license is included in the file
% LICENSE.tex and is also available online at
% <http://www.gnu.org/copyleft/fdl.html>.

\part*{Optimality properties of point estimation}%
\addcontentsline{toc}{part}{Optimality properties of point estimation}

\section{Optimality focusing on mean and variance}

\begin{itemize}

\item In this section, we're going to view estimators as algorithms
  and evaluate them based on how well they typically perform.  This
  means that we're going to evaluate estimators on the basis of their
  distributions, called their \emph{sampling distribution}.  

\item Typically, the class of possible point estimators is too large
  to make general statements about optimality.  But we can make
  statements about a restricted class of estimators.  A natural
  restriction to impose is that the estimator must be correct ``on
  average,'' i.e. that it should not make systemic and predictable
  errors.  Such an estimator is called a ``unbiased'' estimator.

  \begin{defn}
    If $\θh$ is an estimator of $θ₀$, the \emph{bias} of $\θh$ is the
    quantity $\E \θh - θ₀$.  An estimator is \emph{unbiased} if its
    bias equals zero for every value of $θ₀$
  \end{defn}

  It can make sense to compare unbiased estimators through their
  dispersion: a more disperse estimator is probably worse than one
  that's more tightly concentrated around the true parameter value
  $θ₀$.

  \begin{defn}
    If $\θh₁$ and $\θh₂$ are two unbiased estimators of a parameter
    $θ$, $\θh₁$ is \emph{more efficient} than $\θh₂$ if the variance
    of $\θh₁$ is less than that of $\θh₂$.
  \end{defn}

  An estimator $\θh$ of $θ$ is said to be the \emph{best unbiased
  estimator of $θ$} if it is unbiased and has smaller variance than
  any other unbiased estimator

\item For parameter vectors, we need the variance of $γ'\θh₁$ to be
  smaller than that of $γ'\θh₂$ for all nonzero $γ$.  This is the same
  as requiring that $\var(\θh₂) - \var(\θh₁)$ is positive definite
  (you should verify this on your own).

\item A particular result comes up in many different contexts
  \begin{thm}
    Suppose $\E \θh₁ = θ$; $\θh₁$ is the best unbiased estimator of
    $θ$ iff $\θh₁$ and $\θh₁ - \θh₂$ are uncorrelated for any other
    unbiased estimator $\θh₂$ of $θ$.
  \end{thm}
  This also holds when restricted to other classes.

  The first part of the proof is easy.  Assume that $\θh₁$ and $\θh₁ -
  \θh₂$ are uncorrelated.  Then
  \begin{align*}
    \var(\θh₂) &= \var(\θh₂ - \θh₁ + \θh₁) \\
    &= \var(\θh₂ - \θh₁) + \var(\θh₁) \\
    &≥ \var(\θh₁)
  \end{align*}
  as required.

  For the second part of the proof, assume $\θh₁$ is the best unbiased
  estimator and consider the variance of the unbiased estimator $a
  \θh₁ + (1 - a) \θh₂$.  Then
  \begin{align*}
    \var(a \θh₁ + (1 - a) \θh₂)
    &= \var(\θh₁ + (1 - a) (\θh₂ - \θh₁)) \\
    &= \var(\θh₁) + (1 - a)² \var(\θh₂ - \θh₁)
    + 2 (1 - a) \cov(\θh₁, \θh₂ - \θh₁).
  \end{align*}

  We can choose $a$ to minimize this variance and find that it is
  smallest for
  \begin{equation*}
    a = 1 + \frac{\cov(\θh₁, \θh₂ - \θh₁)}{\var(\θh₂ - \θh₁)}.
  \end{equation*}
  Plugging in above gives
  \begin{equation*}
    \var(a \θh₁ + (1 - a) \θh₂)
    = \var(\θh₁) - \frac{(\cov(\θh₁, \θh₂ - \θh₁))²}{\var(\θh₂ - \θh₁)}.
  \end{equation*}
  This last quantity is less than $\var(\θh₁)$ (a contradiction)
  unless $\cov(\θh₁, \θh₂ - \θh₁) = 0$, completing the proof.

\item It can also be useful to know the smallest possible variance
  that an unbiased estimator (or estimator in general) could have in a
  particular problem.  Then, if we have an unbiased estimator that has
  that variance, we know that it is ``the best'' estimator possible in
  a certain sense and can stop looking for better estimators.

  The Cramer-Rao lower bound gives us such a bound.
  \begin{thm}
    Suppose $X₁,...,X_n$ have a joint density with likelihood $L(θ;
    X)$ and let $\θh$ be an estimator of $θ$ with finite variance and
    \begin{equation*}
      (d/dθ) ∫_\Xc \θh(x) L(θ; x) \dx = ∫_\Xc \θh(x) (d/dθ) L(θ; x) \dx
    \end{equation*}
    where $\Xc$ is the support of $X₁,...,X_n$.
    Then 
    \begin{equation*}
      \var(\θh) ≥  \frac{((d/dθ) \E \θh(X))²}{\E ((d/dθ) \log L(θ; X))²}.
    \end{equation*}
  \end{thm}
  If the data are \iid\ then this result can be simplified.

  The proof is relatively straightforward and follows from the
  Cauchy-Schwarz inequality.  For any random variables $Y$ and $Z$,
  the CS inequality implies that
  \begin{equation*}
    \cov(Y, Z)² ≤ \var(Y) \var(Z).
  \end{equation*}
  If we let $Y = \θh$ and $Z = (d/dθ) \log L(θ; X)$, we get
  \begin{equation*}
    \var(\θh) ≥
    \frac{\cov(\θh, (d/dθ) \log L(θ; X))²}
         {\var((d/dθ) \log L(θ; X))}
  \end{equation*}
  (as long as $\var((d/dθ) \log L(θ; X)) > 0$), and the result holds
  after we show that
  \begin{equation*}
    \cov(\θh, (d/dθ) \log L(θ; X)) = (d/dθ) \E \θh
  \end{equation*}
  and
  \begin{equation*}
    \E (d/dθ) \log L(θ; X)) = 0,
  \end{equation*}
  which I need to fill in in the future.

  \begin{ex}
    Suppose that $X₁,...,X_n ∼ \iid N(μ, σ²)$ with $σ$² known.  We can
    use the CR lower bound to find the best-case variance of an
    unbiased estimator.  Since
    \begin{equation*}
      d/dμ \log L(μ, σ²; X) = ∑_i (x_i - μ)/σ²,
    \end{equation*}
    we have
    \begin{equation*}
      \E(d/dμ \log L(μ, σ²; X))² = n/σ²,
    \end{equation*}
    and the CR lower bound for any unbiased estimator of $μ$ is
    $σ²/n$.  The sample mean is unbiased and has this variance, so
    it's the best unbiased estimator.
  \end{ex}

\item It is possible that no estimator, even the best one, achieves
  the CR lower bound.

\end{itemize}

\section{Loss-function based optimality}

\begin{itemize}

\item We can also consider other measures of optimality.  One is to
  consider the average loss associated with an estimator.  A
  \emph{loss function} $L$ is a convex function s.t. $L(0) = 0$ that
  measures the cost associated with misestimating $\theta$: the cost
  associated with the error $\θh - θ$ is $L(\θh - θ)$.  An estimator
  $\θh₁$ has lower risk than $\θh₂$ if
  \begin{equation*}
    \E L(\θh₁ - θ) ≤ \E L(\θh₂ - θ)
  \end{equation*}
  
  A standard loss criterion is MSE, where $L(x) = x²$.  Notice that
  MSE weights (squared) bias and variance equally:
  \begin{align*}
    \E (\θh - θ)² &= \E((\θh - \E \θh) + (\E \θh - θ))² \\
    &= \E(\θh - \E \θh)² + 2 \E (\θh - \E \θh) (\E \θh - θ) + (\E \θh - θ)² \\
    &= \var(\θh) + \bias(\θh)²
  \end{align*}
  since the middle term in the second to last equation is zero.  Loss
  functions can also be derived directly from economic criteria like
  utility functions.

\item Loss-function based criteria are rarely decisive, in the sense
  that the best estimator can depend on the parameter values, as shown
  in the next example.
  \begin{ex}
    Suppose that $X₁,...,X_n ∼ \binomial(p)$ and consider two
    estimators of $p$, $\Xb$ and $1/2$.  Then
    \begin{equation*}
      \mse(\Xb) = \bias(\Xb)² + \var(\Xb) = p(1-p)/n
    \end{equation*}
    and
    \begin{equation*}
      \mse(1/2) = \bias(1/2)² + \var(1/2) = (1/2 - p)².
    \end{equation*}
    When $p$ is near $1/2$, the second estimator has lower MSE.  When
    $p$ is not near $1/2$, the first estimator will typically have
    lower MSE.
  \end{ex}

  For loss-function based optimality to be useful, the researcher
  needs to weight the parameter space by importance; if $W(θ)$ is a
  weighting function for the parameter space, we can choose the
  estimator $\θh$ associated with the lowest average risk:
  \begin{equation*}
    ∫ \E L(\θh - θ) W(θ) \dθ.
  \end{equation*}

\end{itemize}

%%% Local Variables: 
%%% mode: latex
%%% TeX-master: "../estimation"
%%% End: 

%  LocalWords:  iff MSE

% Copyright © 2013, authors of the "Core Econometrics Textbook;" a
% complete list of authors is available in the file AUTHORS.tex.

% Permission is granted to copy, distribute and/or modify this
% document under the terms of the GNU Free Documentation License,
% Version 1.3 or any later version published by the Free Software
% Foundation; with no Invariant Sections, no Front-Cover Texts, and no
% Back-Cover Texts.  A copy of the license is included in the file
% LICENSE.tex and is also available online at
% <http://www.gnu.org/copyleft/fdl.html>.

\part*{Confidence intervals}%
\addcontentsline{toc}{part}{Confidence intervals}

\section{Introduction to interval estimation}

\begin{itemize}  

\item Note that our discussion of the sampling distribution lets us
  think rigorously about the precision associated with a particular
  estimator of an unknown parameter $θ$.  But it hasn't helped us
  model uncertainty about the parameter value itself.  We'll see in
  this section that confidence intervals are one way to think
  rigorously about uncertainty of the parameter value, and we'll see
  how to generate confidence intervals from point estimators.

\item Let's introduce some definitions
  \begin{defn}
    An \emph{interval estimator} of a parameter $θ$ is any pair of
    functions $L(X)$ and $U(X)$ s.t. $L(x) ≤ U(x)$ for all $x$ and,
    when we observe $x$ we make the inference $θ ∈ [L(x), U(x)]$.
  \end{defn}

  \begin{defn}
    The \emph{coverage probability} of an interval estimator
    $[L(X), U(X)]$ is the probability $\Pr_θ[θ ∈ [L(X), U(X)]]$
  \end{defn}

  \begin{defn}
    The \emph{confidence coefficient} is $\inf_θ P_θ[θ ∈ [L(X), U(X)]]$
  \end{defn}

  Generally, the idea behind an interval estimator is that we want to
  estimate $θ$ in a way that accounts for uncertainty.  So the
  researcher ``agrees'' not to distinguish between values in the
  interval.  These intervals can be one-sided or two-sided.

\item The idea of a confidence interval can be extended to
  \emph{confidence regions} or \emph{confidence sets} when the
  parameter has more than one element.  The extension is obvious.

\item If we have a statistic with known sampling distribution, we can
  often use that distribution to generate a confidence interval.  This
  is the idea: suppose you have a statistic $\θh$ and know that its
  distribution function, $F(t; θ)$, is a monotone function of $θ$.  We
  can define a $1-α$ confidence interval of the form $[θ_L(\θh),
  θ_U(\θh)]$ by solving the following equations for $θ_L, θ_U$:

  \begin{itemize}
  \item If $F(t; θ)$ is decreasing in $θ$:
    \begin{align*}
      F(\θh; θ_U) &= α/2 & F(\θh; θ_L) &= 1 - α/2
    \end{align*}
  \item If $F(t; θ)$ is increasing in $θ$:
    \begin{align*}
      F(\θh; θ_L) &= α/2 & F(\θh; θ_U) &= 1 - α/2
    \end{align*}
  \item Note that the $L$ and the $U$ are switched in the two formula.
  \end{itemize}

  \begin{ex} Let $X₁,...,X_n$ be iid $N(θ, 1)$ and we want to
    construct a two-sided 95\% interval for $θ$.  Let be $\Xb$, which
    is $N(θ, 1/n)$.  Then
    \begin{equation*}
      \Pr_θ[\Xb ≤ t] = Φ(\sqrt{n} (t - θ)) = F(t; θ)
    \end{equation*}
    which is decreasing in $θ$.

    To construct an interval, we need to solve
    \begin{align*}
      Φ(\sqrt{n} (\Xb - θ_U)) &= 0.025 \\
      Φ(\sqrt{n} (\Xb - θ_L)) &= 0.975
    \end{align*}
    which becomes
    \begin{align*}
      \Xb - θ_U &= Φ^{-1}(0.025) / \sqrt{n}
      \Xb - θ_L &= Φ^{-1}(0.975) / \sqrt{n}
    \end{align*}
    which then becomes
    \begin{align*}
      \Xb - Φ^{-1}(0.025) / \sqrt{n} &= θ_U \\
      \Xb - Φ^{-1}(0.975) / \sqrt{n} &= θ_L.
    \end{align*}
    Since $Φ^{-1}(0.025) = -1.96$ and $Φ^{-1}(0.975) = 1.96$, this
    finally becomes
    \begin{align*}
      θ_U &= \Xb + 1.96 / \sqrt{n} \\
      θ_L &= \Xb - 1.96 / \sqrt{n}
    \end{align*}
  \end{ex}

\end{itemize}

\section{Interval optimality}

\begin{itemize}

\item Generally, as you might assume from above, if you have several
  estimators to use, you should choose the estimator with the tighter
  sampling distribution—the more efficient estimator.

\item This is going to lead to a principle better for asymptotic
  intervals.

\item A second question, for a two sided interval, is how to spit the
  mass between the tails.  The following theorem can help (this
  version is taken from \citealp{CB02}):

  \begin{thm}
    Let $f(x)$ be a unimodal pdf with mode $x^*$.  If $[a,b]$
    satisfies
    \begin{enumerate}
    \item $∫_a^b f(x) dx = 1 - α$,
    \item $f(a) = f(b) > 0$, and
    \item $a ≤ x^* ≤ b$
    \end{enumerate}
    then $[a,b]$ is the shortest interval with coverage $1-α$.
  \end{thm}

  The proof works by showing that any shorter interval has coverage
  strictly less than $1-α$.  It's pretty easy to see from pictures,
  which I need to draw.

  \begin{itemize}
  \item Suppose $b'-a' < b-a$ and $a', b' ≤ a$
  \item Suppose $b'-a' < b-a$ and $a' ≤ a < b'$ (so then $b' < b$).
  \item Suppose $b'-a' < b-a$ and $a < a'$ and $b' < b$.
  \item Suppose $b'-a' < b-a$ and $a < a'$ and $b ≤ b'$.
  \end{itemize}

\end{itemize}

%%% Local Variables:
%%% mode: latex
%%% TeX-master: "../estimation"
%%% End:

% Copyright © 2013, authors of the "Econometrics Core" textbook; a
% complete list of authors is available in the file AUTHORS.tex.

% Permission is granted to copy, distribute and/or modify this
% document under the terms of the GNU Free Documentation License,
% Version 1.3 or any later version published by the Free Software
% Foundation; with no Invariant Sections, no Front-Cover Texts, and no
% Back-Cover Texts.  A copy of the license is included in the file
% LICENSE.tex and is also available online at
% <http://www.gnu.org/copyleft/fdl.html>.

\part*{Estimation of linear regression}%
\addcontentsline{toc}{part}{Estimation of linear regression}

\section{Properties of quadratic forms}

\begin{itemize}

\item Quadratic forms will come up a lot when we study linear
  regression.  If $X$ be a random $k$-vector and let $A$ be a $k × k$
  deterministic matrix then $X'A X$ is a (random) quadratic form.
  These are (relatively) easy to study because they can be written as
  summations:
  \begin{equation*}
    X'AX = ∑_{i,j} X_i X_j A_{ij}.
  \end{equation*}

\item Usually, when we discuss quadratic forms, we assume (at least
  implicitly) that $A$ is symmetric.  This assumption is usually
  without loss of generality because
  \begin{equation*}
    X' A X = X'(A/2 + A'/2)X
  \end{equation*}
  almost surely, and $X'(A/2 + A'/2)X$ is symmetric by construction.

\item Here's a representative result for quadratic forms that
  illustrates a useful trick.
  \begin{thm}
    Let $X$ be an $n × 1$ vector of random variables, and
    let $A$ be an $n × n$ symmetric matrix.  If $\E X = μ$ and $\var(X) =
    Σ$, then
    \begin{equation}
      \E X'AX = \tr(A Σ) + μ'Aμ.
    \end{equation}
  \end{thm}

  The proof uses three concepts that show up often: a scalar equals
  its own trace, linearity of the expectation, and a property of the
  trace: $\tr(XY) = \tr(YX)$ as long as both are conformable (the last
  property is difficult to see, so write out the matrix operations to
  convince yourself it is true).
  \begin{align*}
    \E(X'AX) &= \tr(\E(X'AX)) \\
    &= \E(\tr(X'AX)) \\
    &= \E(\tr(AXX')) \\
    &= \tr(\E(AXX')) \\
    &= \tr(A \E(XX')) \\
    &= \tr(A (\var(X) + μμ')) \\
    &= \tr(AΣ) + \tr(A μμ') \\
    &= \tr(AΣ) + μ'Aμ.
  \end{align*}

\item Similar results obviously exist for other moments, but the
  algebra isn't quite as nice (and doesn't illuminate any key
  technique other than raw tenacity).  Here's an example for the
  variance under a few simplifying assumptions (this particular
  statement comes from \citealp{SL03})
  \begin{thm}
    let $X₁,...,X_n$ be independent random variables with means
    $θ₁,...,θ_n$, and the same 2nd, 3rd and 4th central moments $μ₂$,
    $μ₃$, $μ₄$.  If $A$ is an $n × n$ symmetric matrix and $a$ is a
    column vector of the diagonal elements of $A$, then
    \begin{equation*}
      \var(X'AX) = 
      (μ₄ - 3 μ₂²)a'a + 2 μ₂² \tr(A²) + 4 μ₂ θ'A² θ + 4 μ₃ θ' A a.
    \end{equation*}
  \end{thm}
  To prove the result, take
  \begin{equation*}
    X'A X = ∑_{ij} ((X_i - θ_i) + θ_i) ((X_j - θ_j) + θ_j) A_{ij},
  \end{equation*}
  multiply out and take the variance.

\end{itemize}

\section{Algebraic and geometric properties of linear regression}

\begin{itemize}

\item Before worrying too much about statistical issues, we'll discuss
  a few issues that come up just because of the math involved in the
  linear regression estimator.  These issues have no statistical
  content, but are numeric identities (almost, that's not a fantastic
  way to say it).

\item Draw pictures in $\RR^n$ for $n = 3$.  A good set of values is
  \begin{align*}
    Y &= \begin{pmatrix}1.5 \\ 0.5 \\ 3\end{pmatrix}&
    X &=
    \begin{pmatrix}
      1  &  3 \\
      1  &  1 \\
      1  &  2
    \end{pmatrix}
  \end{align*}

\item The same pictures can be useful to understand the operations
  when adding a variable.  Suppose you regress $Y$ on $X₁$ and get an
  estimate $\βt₁$, then decide to regress $Y$ on $X₁$ and $X₂$, giving
  $\βh₁$ and $\βh₂$.  There is a relationship between these estimates
  \begin{thm}[Frisch-Waugh-Lovel Theorem]
    Let $e = Y - X₁\βt₁$ and define
    \begin{equation*}
      Z = (I - X₁(X₁'X₁)X₁')X₂,
    \end{equation*}
    (i.e. $Z$ consists of the residuals after regressing each column
    of $X₂$ on $X₁$.  Then $\βh₂ = (Z'Z)^{-1}Z'e$.
  \end{thm}
  Similar formulae exist for getting $\βh₁$ from $\βt₁$

  [Need to scan pictures and add them]

  This result is less useful than it used to be, because computers are
  far more powerful.  But the intuition is still just as useful—this
  is what we mean by ``partial out''—and the result itself can still
  be useful when dealing with very large data sets.

\item We may be interested in knowing how far $Y$ is from $\Yh$; a
  good comparison is how far relative to the distance between $Y$ and
  $ι \Yb$.  This is what the centered $R$-squared measures:
  \begin{equation*}
    R² = \frac{ \lVert Y - \Yh \rVert₂²}{\lVert Y - ι \Yb \rVert²}.
  \end{equation*}
  As long as the model contains a constant, this will be between zero
  and one.

  You should realize that $R²$ is almost useless as a guide to whether
  or not your model is ``good.''  There are a few specific
  applications (mostly forecasting) where I was somewhat interested in
  the model's $R²$, but that's it (knowing that it's high can
  sometimes reassure you that there's no need to look for a model with
  more predictive power).

\end{itemize}

\section{Gaussian random variables}

\begin{itemize}

\item The \emph{Normal distribution} is particularly important when
  studying linear regression.  To the extent that we have finite
  sample results, many will hold only if the errors are Normal.
  Moreover, whenever we use the Central Limit Theorem, we get an
  approximately normal random variable.

\item If $X$ is an $N(μ, Σ)$ random vector, its density is
  \begin{equation*}
    f(x) = \frac{1}{\sqrt{2ⁿ πⁿ \det(Σ)}} exp(-1/2(x - μ)'Σ^{-1}(x- μ)).
  \end{equation*}
  The parameters $μ$ and $Σ$ can be shown to be the mean and variance
  of the r.v. and completely determine its distribution.

  [you should draw the shape of the density and some elliptical
  contour plots]

  If $X$ is a $k$-dimensional multivariate normal with mean $μ$
  and variance $Σ$, $A X + b$ is also multivariate normal for any
  constant $n × k$ matrix $A$ and $n$-vector $b$.

  If $X$ is multivariate normal with mean $μ$ and variance $Σ$,
  and $Σ^{1/2}$ is a symmetric matrix such that $Σ^{1/2} Σ^{1/2} = Σ$,
  then $(Σ^{1/2})^{-1} (X - μ)$ is multivariate standard normal.

\item Independence of normal random variables is especially easy: if
  $X$ and $Y$ are uncorrelated normal random vectors, then they are
  independent.  The result follows from simply multplying out the
  square terms in the density function and then factoring it.

  Marginal and conditional distributions are also especially easy to
  work with.  If 
  \begin{equation*}
    (X₁,X₂) ∼ N((μ₁,μ₂), Σ)
  \end{equation*}
  with
  \begin{equation*}
    Σ = \begin{pmatrix}
      Σ_{11} & Σ_{12} \\ Σ_{12}' & Σ_{22}
    \end{pmatrix}
  \end{equation*}
  then $X₁ ∼ N(μ₁, Σ_{11})$, $X₂ ∼ N(μ₂, Σ_{22})$, and
  \begin{equation*}    
    X₁ ∣ X₂ ∼ N(μ₁ + Σ_{12} Σ_{22}^{-1} (X₂ - μ₂),
               Σ_{11} - Σ_{12}'Σ_{22}^{-1} Σ_{12}).
  \end{equation*}
  Notice that the conditional mean of $X₁$ depends on $X₂$, but the
  conditional variance doesn't.

\item The fact that zero correlation implies independence also makes
  it easy to determine whether functions of normal r.v.s are
  independent.  Linear combinations and quadratic forms of normal
  r.v.s will come up often, so we'll single those results out.  For
  the next results, let $X$ be a $k$-dimensional standard normal
  random vector (in practice, we will often work with $Σ^{-1/2} X$).
  \begin{enumerate}
  \item If $A$ is a $k × j$ matrix and $B$ a $k × l$ matrix such that
    $A'B = 0$, then $A'X$ and $B'X$ are independent.
  \item If $P$ is a $k × k$ idempotent matrix and $PB = 0$ then $X'PX$
    and $B'X$ are independent (an idempotent matrix is a square and
    symmetric matrix that satisfies $P = PP$).
  \item If $P$ and $Q$ are idempotent and $PQ = 0$ then $X'PX$ and
    $X'QX$ are independent.
  \end{enumerate}
  
\item Idempotent matrices come up often, as they represent projections
  in $\RRⁿ$.  If $z$ is a point in $\RRⁿ$ and $x₁,...,x_k$ is a set of
  vectors in $\RRⁿ$, we may want to project $z$ into the space spanned
  by $x₁,...,x_k$.  This amounts to finding a linear operator $P$ so
  that $P z$ is a linear combination of the $x_i$'s and $P z$ is as
  close as possible to $z$ subject to the first constraint.

  We'll worry about making sure that $Pz$ is a linear combination of
  the $x_i$'s later—that's going to be the focus of linear regression.
  But to ensure that $P z$ is as close as possible to $z$, we just
  need to have $z - Pz$ and $Pz$ perpendicular, so $(z - Pz)' Pz = 0$.
  But this requires that $z'(I - P)'P z = 0$ for any $z$, so $(I -
  P)'P = 0$, which amounts to our condition that $P = PP$.

  An idempotent matrix, $P$, has the property that all of its
  eigenvalues must be zero or one, so $\rank(P) = \tr(P)$.

\item Two statistics that are particularly relevant to the normal
  distribution are the sample mean, $\Xb = (1/n) ∑_{i=1}^n X_i$, and
  the sample variance, $S² = (1/(n-1)) ∑_{i=1}^n (X_i - \Xb)²$.  Note
  that we can write $\Xb$ as a linear combination of the $X_i$'s:
  \begin{equation*}
    \Xb = (1/n,...,1/n) · (X₁,...,X_n)' = (1/n) ι'X,
  \end{equation*}
  where $ι$ is a vector of $n$ ones, and
  \begin{equation*}
    S² = (1/(n-1)) (X - ι \Xb)'(X - ι \Xb) = (1/(n-1)) X'(I - (1/n) ιι')X.
  \end{equation*}
  If $X ∼ N(μ, σ² I)$ then we have $\Xb ∼ N(μ, (σ²/n) I)$.  Also,
  since $(I - (1/n) ιι')$ is idempotent, we know right away that $\Xb$
  and $S²$ are independent.

\item Some derived distributions are important

\item %
  \begin{defn}
    Let $X₁,...,X_k$ be independent standard normal.  The
    chi-squared distribution with $k$ degrees of freedom is the
    distribution of $∑_{i=1}^k X²_i$.
  \end{defn}

  \begin{thm}
    Let $X ∼ N(0, I_n)$ and let $P$ be a symmetric $n × n$ matrix.
    Then $X'PX$ is chi-squared with $k$ degrees of freedom iff $P$ is
    idempotent with rank $k$.
  \end{thm}

  \begin{thm}
    if $X$ and $Y$ are independent chi-squared with $df_X$ and $df_Y$
    degrees of freedom, then $X + Y$ is chi-square with $df_X + df_Y$
    degrees of freedom.
  \end{thm}

  If $X ∼ N(μ, σ² I)$ then $S²$ is chi-square with $n-1$ degrees of
  freedom.

\item %
  \begin{defn}
    If $X$ is a standard normal and $Y$ is a chi-squared with $n$ dof
    and $X$ and $Y$ are independent, then $X / \sqrt{Y/n}$ is a $t(n)$
    random variable.
  \end{defn}
  From the quadratic form results, this is going to come up when $Z$
  is multivariate standard normal and $X = B'Z$, $Y = Z'PZ$, and $B'P
  = 0$.

\item %
  \begin{defn}
    If $X$ and $Y$ are independent chi-squared with $df_X$ and $df_Y$
    degrees of freedom, then $\frac{X/df_X}{Y/df_Y}$ is an $F(df_X,
    df_Y)$ random variable.
  \end{defn}

  Expect this to come up when $Z$ is multivariate standard normal and
  $X = Z'QZ$, $Y = Z'PZ$ and $Q$ and $P$ are idempotent and orthogonal.

\end{itemize}

\section{Statistical aspects of linear regression}

\begin{itemize}
\item We typically will invoke some of the following assumptions:
  \begin{enumerate}
  \item $\E(\ep_i ∣ X) = 0$ for all $i$
  \item $\var(\ep_i ∣ X) = σ²$
  \item $\cov(\ep_i, \ep_j ∣ X) = 0$ for $i ≠ j$
  \item $\ep_i ∼ \iid(0,σ²)$ given $X$
  \item $\ep_i ∼ \iid\; N(0,σ²)$ given $X$
  \end{enumerate}
  note that 5 implies 4 which implies 1, 2, and 3.

\item To derive OLS as the MLE, we assume that $\ep ∼ N(0,σ² I)$.
  The mathematical derivation is presented in earlier sections, so
  we'll discuss some of the intuition now.

  This part is pretty terrible and it needs to be replaced with the
  actual drawings:
  \begin{itemize}
  \item draw plane for $n = 3$ and $k = 1$
  \item draw spheres for the different epsilons (sphere here is a
    consequence of normality and independence); each layer of the
    sphere is an iso-probability line
  \item draw a new picture to show estimating $Y$
    \begin{itemize}
    \item start with $Y$ and $X$ given
    \item choose some $β = B₁$
    \item draw different spheres for the ``epsilon'' corresponding to that mean
    \item draw a second picutre corresponding to $β = B₂$
    \item We can judge which of the values of $β$ is more plausible by
      looking at the density function, and choosing the one that's
      larger (which is just MLE); it should be clear that the
      coefficients that maximize the likelihood are going to be the
      coefficients that minimize the distance between $Y$ and $\Yh$.
    \end{itemize}
  \end{itemize}

\item Define a few terms: the OLS estimator is given by
  \begin{equation*}
    \βh = (X'X)^{-1}X'Y = \left(∑_i x_i x_i'\right) ∑_i x_i y_i;
  \end{equation*}
  The \emph{fitted values} are defined as $\Yh = X \βh$ and the OLS
  \emph{residuals} are defined as $\eph = Y - \Yh$.

\item Some of the finite sample properties are relatively easy to
  prove.  The first is unbiasedness; or really conditional
  unbiasedness.
  \begin{thm}
    Suppose that $Y = Xβ + \ep$ where $\E(\ep ∣ X) = 0$ and $X$ has
    full rank.  Then $\E(\βh ∣ X) = β$.
  \end{thm}
  This conditional result implies unconditional unbiasedness through
  the LIE: $\E \βh = \E \E(\βh ∣ X) = β$.  The proof is pretty
  straightforward:
  \begin{align*}
    \E(\βh ∣ X) &= \E((X'X)^{-1} X'Y ∣ X) \\
    &= (X'X)^{-1} X'\E(Y ∣ X) \\
    &= (X'X)^{-1} X'\E(Xβ + e ∣ X) \\
    &= (X'X)^{-1} X'X β \\
    &= β.
  \end{align*}
  One thing to notice is that this is a slightly stronger result than
  unbiasedness on its own: the expected value of the OLS coefficient
  estimator is $β$ \emph{regardless} of the value of $X$.

\item If we also assume that the errors are homoskedastic and
  uncorrelated, the same sort of approach gives us a formula for the
  variance of the OLS estimator.
  \begin{thm}
    Suppose that $Y = Xβ + \ep$ where $\E(\ep ∣ X) = 0$ and $X$ has
    full rank, and also assume that $\E(\ep \ep' ∣ X) = σ² I$ for some
    $σ²$.  Then $\var(\βh ∣ X) = σ²(X'X)^{-1}$.
  \end{thm}

  The proof is similar to before.  Notice that now the variance
  \emph{does} depend on the particular value of the $X$'s drawn;
  inspection will show you that the variance of the $i$th coefficient
  estimator increases with the variance of the $i$th regressor.

  Since $\βh - β = (X'X)^{-1}X'\ep$, we have
  \begin{align*}
    \var(\βh ∣ X) &= \E((\βh - β)(\βh-β)' ∣ X) \\
    &= \E((X'X)^{-1}X'\ep\ep'X(X'X)^{-1} ∣ X) \\
    &= (X'X)^{-1} X' \E(\ep\ep' ∣ X) X (X'X)^{-1} \\
    &= σ² (X'X)^{-1} X'X (X'X)^{-1} \\
    &= σ² (X'X)^{-1}
  \end{align*}

\item It's also worth thinking about optimality properties under only
  the exogeneity and heteroskedasticity assumtions; you've probably
  heard of the Gauss-Markov Theorem before.  Notice that OLS can be
  viewed as a weighted average: $\βh = (X'X) X'Y = ∑_{i=1}ⁿ w_i y_i$
  where $w_i$ is a $k$-vector, $w_i = (X'X)^{-1} x_i$.  The
  Gauss-Markov Theorem shows that OLS has the smallest variance of all
  of the unbiased linear estimators.
  \begin{thm}
    Assume that $Y = Xβ + \ep$ where $\ep ∼ (0, σ² I)$ given $X$.
    Then $\βh$ is the unique estimator with minimum variance (given
    $X$) among all linear, unbiased estimators (i.e. OLS is BLUE).
  \end{thm}

  Just as in our earlier proofs of optimaility, the trick here will be
  to show that any other linear, unbiased estimator can be written as
  $\βh$ plus some additional uncorrelated noise term.  Suppose that
  $V$ is a $k × n$ matrix s.t. $\E V'Y = β$ is unbiased for $β$, so
  $V'Y$ is a linear unbiased estimator.  We can write down immediately
  that
  \begin{equation*}
    V'Y = (X'X)^{-1}X'Y + [V'Y - (X'X)^{-1} X'Y],
  \end{equation*}
  and the proof then follows after showing that these two parts are
  uncorrelated.
  
  By construction, $V'Y = V'Xβ + V'e$, so
  \begin{equation*}
    β = \E(V'Y ∣ X) = \E(V'Xβ + V'e ∣ X) = V'X β + V'\E(e ∣ X) = V'Xβ.
  \end{equation*}
  This holds for any choice of $β$, so $V'X$ must be equal to $I$ and,
  as a result, $V'Y - β = V'e$.

  Now it is straightforward to calculate the covariance between
  $(X'X)^{-1}X'Y$ and $(V - (X'X)^{-1}X')Y$,
  \begin{align*}
    \cov((X'X)^{-1}X'Y, (V - (X'X)^{-1}X')Y ∣ X)
    &= \E((X'X)^{-1}X'e e'(V - X(X'X)^{-1}) ∣ X) \\
    &= σ² (X'X)^{-1}X'(V - X(X'X)^{-1}) \\
    &= σ² [(X'X)^{-1}X'V - (X'X)^{-1} X'X (X'X)^{-1}].
  \end{align*}
  Both equal $(X'X)^{-1}$, so the whole term is zero and we've shown
  that the covariance is zero, completing the proof.

\item We've proved this result conditional on $X$, but the
  unconditional extension is easy.  If $\var(\βh ∣ X) ≤ \var(V'Y ∣ X)$
  for all values of $X$, then $\E \var(\βh ∣ X) ≤ \E \var(V'Y ∣ X)$ as
  well.

\item The sampling distributions also matter.  Finite sample
  distributions are going to depend on the distribution of the error
  term, $\ep$.  $\βh$ is relatively easy to work with, since it equals
  $β + (X'X)^{-1} X'\ep$; if $\ep ∼ N(0, σ²)$ given $X$, clearly
  $\βh ∼ N(β, σ²(X'X)^{-1})$ given $X$.

  For $s²$, we can get results pretty quickly by relating it back to
  the quadratic form results. We have $s² = (n-k)^{-1} ∑_{i=1}ⁿ \eph²_i$
  which can also be written as
  \begin{align*}
    s² &= (n-k)^{-1} ∑_i (y_i - x_i'\βh)² \\
    &= (n-k)^{-1} (Y - X\βh)'(Y - X\βh) \\
    &= (n-k)^{-1} (Y - X(X'X)^{-1}X'Y)'(Y - X(X'X)^{-1}X'Y) \\
    &= (n-k)^{-1} Y'(I - X(X'X)^{-1}X')'(I - X(X'X)^{-1}X')Y \\
    &= (n-k)^{-1} Y'(I - X(X'X)^{-1}X')Y \\
    &= (n-k)^{-1} (Xβ + \ep)'(I - X(X'X)^{-1}X')(Xβ + \ep) \\
    &= (n-k)^{-1} \ep'(I - X(X'X)^{-1}X')\ep.
  \end{align*}
  Now $I - X(X'X)^{-1}X'$ is a projection matrix with rank $n-k$, so
  $s² ∼ (n-k)^{-1} χ²(n-k)$ given $X$.  Moreover, it is easy to verify
  that $s²$ and $\βh$ are independent given $X$.

\item Two unconditional results are more useful.  If we let $q_i$ be
  the $i,i$ element of $(X'X)^{-1}$ and look at the unconditional
  density of $(\βh_i - β_i)/\sqrt{q_i σ²}$, we see that
  \begin{align*}
    f_{(\βh_i - β_i)/\sqrt{q_i σ²}}(b)
    &= ∫ f_{(\βh_i - β_i)/\sqrt{q_i σ²}, q_i} (b, q) dq \\
    &= ∫ f_{(\βh_i - β_i)/\sqrt{q_i σ²}}(b ∣ q_i = q) f_{q_i}(q) dq \\
    &= ∫ φ(b) f_{q_i}(q) dq \\
    &= φ(b) ∫ f_{q_i}(q) dq \\
    &= φ(b).
  \end{align*}
  Similarly, replacing $σ²$ with $s²$ gives us a $t(n-k)$ distribution.

\end{itemize}

\section{Making predictions from regression models}

\subsection{motivation and setup}
\begin{itemize}
\item Say we've estimated a regression model.  What do we do with it?
\item Easiest possible use: make forecasts
\item Setup:
\begin{itemize}
\item have observations $(y_i, x_i)$ for $i = 1,...,n$
\item observe regressors for period $n+1$
\item want to predict $y_{n+1}$
\end{itemize}
\item model \[ y_i = x_i'β + ε_i \]
\item Estimated by OLS (or, maybe, GLS)
\end{itemize}

\subsection{natural forecast}

\begin{itemize}
\item The natural forecast for $y_{n+1}$ is $x_{n+1}'\βh$
\begin{itemize}
\item $x_{n+1}'β$ would be the forecast that minimizes expected
         squared error
\item $\βh$ is our best (minimum-variance) estimator of $β$
\item Expected value of $\hat y_{n+1}$ is $y_{n+1}$
\end{itemize}
\item How reliable is our forecast?
\begin{itemize}
\item more precisely, what is the variance of the forecast error?
\item Define the forecast error as 
  \[ e_{n+1} = y_{n+1} - \hat y_{n+1} = ε_{n+1} - x_{n+1}'(\βh - β) \]
\end{itemize}
\item Variance of forecast erro is going to reflect
\begin{itemize}
\item variance of $\βh$
\item variance of $ε_{n+1}$
\end{itemize}
\end{itemize}

\subsection{calculation of variance}

     We're going to assume that the errors are uncorrelated (like we
     have) and homoskedastic.
\begin{itemize}
\item conditionally heteroeskedastic errors can be dealt with by GLS
\item The usual calculation gives us
  \[ \var(e_{n+1} ∣ X, x_{n+1}) = \E(e_{n+1}² ∣ X_{n+1}) = \E(ε_{n+1}²
  ∣ X, x_{n+1}) + \E(((\βh - β)'x_{n+1})² ∣ X, x_{n+1})\]
\item the first term is just $σ²$
\item the second is 
  \[ x_{n+1}' \var(\βh ∣ X) x_{n+1} = σ² x_{n+1}'(X'X)^{-1}x_{n+1}\]
\item draw a scatterplot to illustrate why $\hat y$ will depend on
       the particular value of $x$.
\end{itemize}

\subsection{common use of this variance}

\begin{itemize}
\item \textbf{if the errors are normal}, we can construct a confidence
       interval for $y_{n+1}$ from this result
\item $\hat y_{n+1} ± t_{α/2} \sqrt{s² (1 + x_{n+1}'(X'X)^{-1}x_{n+1})}$
\end{itemize}

%%% Local Variables: 
%%% mode: latex
%%% TeX-master: "../../estimation"
%%% End: 

% Copyright © 2013, authors of the "Econometrics Core" textbook; a
% complete list of authors is available in the file AUTHORS.tex.

% Permission is granted to copy, distribute and/or modify this
% document under the terms of the GNU Free Documentation License,
% Version 1.3 or any later version published by the Free Software
% Foundation; with no Invariant Sections, no Front-Cover Texts, and no
% Back-Cover Texts.  A copy of the license is included in the file
% LICENSE.tex and is also available online at
% <http://www.gnu.org/copyleft/fdl.html>.

\part*{Problem set, point estimation}%
\addcontentsline{toc}{part}{Problem set, point estimation}

\begin{enumerate}

\item Write an R function that calculates the maximum likelihood
  estimator of $α$ and $β$ for the gamma$(α,β)$ distribution. Maximize
  the likelihood function numerically even if you are able to
  analytically derive the MLE.  You may find the R functions
  \texttt{optim} and \texttt{dgamma} helpful.  You should pay
  attention to the initial values of the numerical optimization
  procedure, and may want to use the method of moments to derive
  sensible initial values.

\item We often want to understand the properties of a statistical
  estimator that has too complicated of a distribution to calculate
  directly.  Or we might know that the distribution of a statistical
  estimator is approximately equal to particular formula, but don't
  know how accurate the approximation will be in any application.  For
  either situation, it can be useful to use a computer to simulate the
  estimator under different assumptions.

  It can be shown (i.e. \citet[p. 483]{CB02}) that if
  $X₁,…,X_n$ is an i.i.d. sample with distribution and density $F$
  and $f$ respectively, then the sample median of $X₁,…,X_n$ is
  approximately (i.e. this holds at $n = ∞$) distributed $\N(θ, (4 n
  f²(θ))^{-1})$, where $θ$ is the population
  median.
  \begin{enumerate}
  \item Simulate 1000 samples of size 10 with $X ∼ N(1,2)$ and plot
    a histogram of the sample median.  Also plot the density of the
    limiting normal distribution on the same graph.  How well does the
    approximation match the simulated density?  Hint: you will find it
    easier if you use the R function \texttt{replicate}.
  \item Repeat the previous question with $n=100$.
  \item Repeat the first question with different distribution
    functions.  What features of the distribution function affect the
    quality of the approximation?
  \end{enumerate}

\item Suppose that $X₁,…,X_n ∼ i.i.d. N(μ, σ²)$.  Is
  the sample median consistent for $μ$?  Is it asymptotically
  normal?  Do these answers require the $X_i$ to be Normal?

\item Let $\{e_t\}$ be a sequence of i.i.d. $(0,1)$ random variables
  and let $X_t = e_t + 0.5\, e_{t-1}$.  Such a process is called a
  Moving Average of order 1, abbreviated as MA(1).  What is the
  probability limit of $n^{-1} ∑_{t=1}ⁿ X_t$ as $n → ∞$?

\item Suppose that $Z$ is a random variable with mean $μ$ and variance
  $σ$ and let $e₁,e₂,…$ be a sequence of i.i.d. random variables with
  mean zero and variance 1 that are independent of $Z$.  Show that the
  sequence $X_i = Z + e_i$ obeys the following weak law of large
  numbers: as $n → ∞$, $n^{-1} ∑_{i=1}ⁿ ( X_i - \E(X_i ∣ Z) ) → 0$ in
  probability.  Note that $X₁,X₂,…$ is \emph{not} an i.i.d.\ sequence,
  but is an example of an \emph{exchangeable} sequence of random
  variables.

\item Suppose that $X_n →^p X$ and that $g$ is a continuous function.
  Prove that $g(X_n) →^p g(X)$.

\item Let $X₁,…,X_n ∼ N(θ, θ)$.  Derive the MLE of $θ$ and show that
  it is asymptotically normal.

\item Suppose that $X₁,…,X_n$ are an i.i.d. sample from the density
  $f(x) = 1/θ$, $0 ≤ x ≤ θ$.  Prove that $\max_i X_i$ is consistent
  for $θ$.

\end{enumerate}

%%% Local Variables: 
%%% mode: latex
%%% TeX-master: "../../estimation"
%%% End: 


% The files `AUTHORS_standalone.tex` and `LICENSE_standalone.tex` are
% available if you want to distribute the author list and the FDL on
% their own.
\addtocontents{toc}{\protect\setcounter{tocdepth}{0}}

\newpage
\part*{Complete list of authors}%
\addcontentsline{toc}{part}{Appendix A: Complete list of authors}
% Copyright © 2013, authors of the "Econometrics Core" textbook; a
% complete list of authors is available in the file AUTHORS.tex.

% Permission is granted to copy, distribute and/or modify this
% document under the terms of the GNU Free Documentation License,
% Version 1.3 or any later version published by the Free Software
% Foundation; with no Invariant Sections, no Front-Cover Texts, and no
% Back-Cover Texts.  A copy of the license is included in the file
% LICENSE.tex and is also available online at
% <http://www.gnu.org/copyleft/fdl.html>.

% Remove the next two lines if you are distributing the author list as
% a standalone pdf.
\noindent%
The following is a list of the contributors to the Econometrics Free
Library Project's \textit{Econometrics Core}, in order of their date
of first involvement (yes, I'm aware that it's a little ridiculous to
have this as a separate file when there is only a single contributor,
but let's dream big, shall we).

\begin{description}
\item[2009-07-01] Gray Calhoun, \email{gcalhoun@iastate.edu}
\end{description}

%%% Local Variables:
%%% mode: latex
%%% TeX-master: "AUTHORS_standalone"
%%% End:

\newpage
\part*{GNU Free Documentation License}%
\addcontentsline{toc}{part}{Appendix B: GNU Free Documentation License}
% Remove the next two lines if you are distributing the author list as
% a standalone pdf.
\part*{GNU Free Documentation License}%
\addcontentsline{toc}{part}{Appendix B: GNU Free Documentation License}
\setcounter{section}{-1}%
\renewcommand\thesection{\arabic{section}}%
\noindent Version 1.3, 3 November 2008

\noindent Copyright \copyright\ 2000, 2001, 2002, 2007, 2008 Free
Software Foundation, Inc.

\noindent \texttt{<http://fsf.org/>}
 
\noindent Everyone is permitted to copy and distribute verbatim copies
of this license document, but changing it is not allowed.

\section{Preamble}

The purpose of this License is to make a manual, textbook, or other
functional and useful document ``free'' in the sense of freedom: to
assure everyone the effective freedom to copy and redistribute it,
with or without modifying it, either commercially or noncommercially.
Secondarily, this License preserves for the author and publisher a way
to get credit for their work, while not being considered responsible
for modifications made by others.

This License is a kind of ``copyleft'', which means that derivative
works of the document must themselves be free in the same sense.  It
complements the GNU General Public License, which is a copyleft
license designed for free software.

We have designed this License in order to use it for manuals for free
software, because free software needs free documentation: a free
program should come with manuals providing the same freedoms that the
software does.  But this License is not limited to software manuals;
it can be used for any textual work, regardless of subject matter or
whether it is published as a printed book.  We recommend this License
principally for works whose purpose is instruction or reference.


\section{APPLICABILITY AND DEFINITIONS}

This License applies to any manual or other work, in any medium, that
contains a notice placed by the copyright holder saying it can be
distributed under the terms of this License.  Such a notice grants a
world-wide, royalty-free license, unlimited in duration, to use that
work under the conditions stated herein.  The ``\textbf{Document}'',
below, refers to any such manual or work.  Any member of the public is
a licensee, and is addressed as ``\textbf{you}''.  You accept the
license if you copy, modify or distribute the work in a way requiring
permission under copyright law.

A ``\textbf{Modified Version}'' of the Document means any work containing the
Document or a portion of it, either copied verbatim, or with
modifications and/or translated into another language.

A ``\textbf{Secondary Section}'' is a named appendix or a front-matter section of
the Document that deals exclusively with the relationship of the
publishers or authors of the Document to the Document's overall subject
(or to related matters) and contains nothing that could fall directly
within that overall subject.  (Thus, if the Document is in part a
textbook of mathematics, a Secondary Section may not explain any
mathematics.)  The relationship could be a matter of historical
connection with the subject or with related matters, or of legal,
commercial, philosophical, ethical or political position regarding
them.

The ``\textbf{Invariant Sections}'' are certain Secondary Sections whose titles
are designated, as being those of Invariant Sections, in the notice
that says that the Document is released under this License.  If a
section does not fit the above definition of Secondary then it is not
allowed to be designated as Invariant.  The Document may contain zero
Invariant Sections.  If the Document does not identify any Invariant
Sections then there are none.

The ``\textbf{Cover Texts}'' are certain short passages of text that are listed,
as Front-Cover Texts or Back-Cover Texts, in the notice that says that
the Document is released under this License.  A Front-Cover Text may
be at most 5 words, and a Back-Cover Text may be at most 25 words.

A ``\textbf{Transparent}'' copy of the Document means a machine-readable copy,
represented in a format whose specification is available to the
general public, that is suitable for revising the document
straightforwardly with generic text editors or (for images composed of
pixels) generic paint programs or (for drawings) some widely available
drawing editor, and that is suitable for input to text formatters or
for automatic translation to a variety of formats suitable for input
to text formatters.  A copy made in an otherwise Transparent file
format whose markup, or absence of markup, has been arranged to thwart
or discourage subsequent modification by readers is not Transparent.
An image format is not Transparent if used for any substantial amount
of text.  A copy that is not ``Transparent'' is called ``\textbf{Opaque}''.

Examples of suitable formats for Transparent copies include plain
ASCII without markup, Texinfo input format, LaTeX input format, SGML
or XML using a publicly available DTD, and standard-conforming simple
HTML, PostScript or PDF designed for human modification.  Examples of
transparent image formats include PNG, XCF and JPG.  Opaque formats
include proprietary formats that can be read and edited only by
proprietary word processors, SGML or XML for which the DTD and/or
processing tools are not generally available, and the
machine-generated HTML, PostScript or PDF produced by some word
processors for output purposes only.

The ``\textbf{Title Page}'' means, for a printed book, the title page itself,
plus such following pages as are needed to hold, legibly, the material
this License requires to appear in the title page.  For works in
formats which do not have any title page as such, ``Title Page'' means
the text near the most prominent appearance of the work's title,
preceding the beginning of the body of the text.

The ``\textbf{publisher}'' means any person or entity that distributes
copies of the Document to the public.

A section ``\textbf{Entitled XYZ}'' means a named subunit of the Document whose
title either is precisely XYZ or contains XYZ in parentheses following
text that translates XYZ in another language.  (Here XYZ stands for a
specific section name mentioned below, such as ``\textbf{Acknowledgements}'',
``\textbf{Dedications}'', ``\textbf{Endorsements}'', or ``\textbf{History}''.)  
To ``\textbf{Preserve the Title}''
of such a section when you modify the Document means that it remains a
section ``Entitled XYZ'' according to this definition.

The Document may include Warranty Disclaimers next to the notice which
states that this License applies to the Document.  These Warranty
Disclaimers are considered to be included by reference in this
License, but only as regards disclaiming warranties: any other
implication that these Warranty Disclaimers may have is void and has
no effect on the meaning of this License.


\section{VERBATIM COPYING}

You may copy and distribute the Document in any medium, either
commercially or noncommercially, provided that this License, the
copyright notices, and the license notice saying this License applies
to the Document are reproduced in all copies, and that you add no other
conditions whatsoever to those of this License.  You may not use
technical measures to obstruct or control the reading or further
copying of the copies you make or distribute.  However, you may accept
compensation in exchange for copies.  If you distribute a large enough
number of copies you must also follow the conditions in section~3.

You may also lend copies, under the same conditions stated above, and
you may publicly display copies.


\section{COPYING IN QUANTITY}

If you publish printed copies (or copies in media that commonly have
printed covers) of the Document, numbering more than 100, and the
Document's license notice requires Cover Texts, you must enclose the
copies in covers that carry, clearly and legibly, all these Cover
Texts: Front-Cover Texts on the front cover, and Back-Cover Texts on
the back cover.  Both covers must also clearly and legibly identify
you as the publisher of these copies.  The front cover must present
the full title with all words of the title equally prominent and
visible.  You may add other material on the covers in addition.
Copying with changes limited to the covers, as long as they preserve
the title of the Document and satisfy these conditions, can be treated
as verbatim copying in other respects.

If the required texts for either cover are too voluminous to fit
legibly, you should put the first ones listed (as many as fit
reasonably) on the actual cover, and continue the rest onto adjacent
pages.

If you publish or distribute Opaque copies of the Document numbering
more than 100, you must either include a machine-readable Transparent
copy along with each Opaque copy, or state in or with each Opaque copy
a computer-network location from which the general network-using
public has access to download using public-standard network protocols
a complete Transparent copy of the Document, free of added material.
If you use the latter option, you must take reasonably prudent steps,
when you begin distribution of Opaque copies in quantity, to ensure
that this Transparent copy will remain thus accessible at the stated
location until at least one year after the last time you distribute an
Opaque copy (directly or through your agents or retailers) of that
edition to the public.

It is requested, but not required, that you contact the authors of the
Document well before redistributing any large number of copies, to give
them a chance to provide you with an updated version of the Document.


\section{MODIFICATIONS}

You may copy and distribute a Modified Version of the Document under
the conditions of sections 2 and 3 above, provided that you release
the Modified Version under precisely this License, with the Modified
Version filling the role of the Document, thus licensing distribution
and modification of the Modified Version to whoever possesses a copy
of it.  In addition, you must do these things in the Modified Version:

\begin{itemize}
\item[A.] 
   Use in the Title Page (and on the covers, if any) a title distinct
   from that of the Document, and from those of previous versions
   (which should, if there were any, be listed in the History section
   of the Document).  You may use the same title as a previous version
   if the original publisher of that version gives permission.
   
\item[B.]
   List on the Title Page, as authors, one or more persons or entities
   responsible for authorship of the modifications in the Modified
   Version, together with at least five of the principal authors of the
   Document (all of its principal authors, if it has fewer than five),
   unless they release you from this requirement.
   
\item[C.]
   State on the Title page the name of the publisher of the
   Modified Version, as the publisher.
   
\item[D.]
   Preserve all the copyright notices of the Document.
   
\item[E.]
   Add an appropriate copyright notice for your modifications
   adjacent to the other copyright notices.
   
\item[F.]
   Include, immediately after the copyright notices, a license notice
   giving the public permission to use the Modified Version under the
   terms of this License, in the form shown in the Addendum below.
   
\item[G.]
   Preserve in that license notice the full lists of Invariant Sections
   and required Cover Texts given in the Document's license notice.
   
\item[H.]
   Include an unaltered copy of this License.
   
\item[I.]
   Preserve the section Entitled ``History'', Preserve its Title, and add
   to it an item stating at least the title, year, new authors, and
   publisher of the Modified Version as given on the Title Page.  If
   there is no section Entitled ``History'' in the Document, create one
   stating the title, year, authors, and publisher of the Document as
   given on its Title Page, then add an item describing the Modified
   Version as stated in the previous sentence.
   
\item[J.]
   Preserve the network location, if any, given in the Document for
   public access to a Transparent copy of the Document, and likewise
   the network locations given in the Document for previous versions
   it was based on.  These may be placed in the ``History'' section.
   You may omit a network location for a work that was published at
   least four years before the Document itself, or if the original
   publisher of the version it refers to gives permission.
   
\item[K.]
   For any section Entitled ``Acknowledgements'' or ``Dedications'',
   Preserve the Title of the section, and preserve in the section all
   the substance and tone of each of the contributor acknowledgements
   and/or dedications given therein.
   
\item[L.]
   Preserve all the Invariant Sections of the Document,
   unaltered in their text and in their titles.  Section numbers
   or the equivalent are not considered part of the section titles.
   
\item[M.]
   Delete any section Entitled ``Endorsements''.  Such a section
   may not be included in the Modified Version.
   
\item[N.]
   Do not retitle any existing section to be Entitled ``Endorsements''
   or to conflict in title with any Invariant Section.
   
\item[O.]
   Preserve any Warranty Disclaimers.
\end{itemize}

If the Modified Version includes new front-matter sections or
appendices that qualify as Secondary Sections and contain no material
copied from the Document, you may at your option designate some or all
of these sections as invariant.  To do this, add their titles to the
list of Invariant Sections in the Modified Version's license notice.
These titles must be distinct from any other section titles.

You may add a section Entitled ``Endorsements'', provided it contains
nothing but endorsements of your Modified Version by various
parties---for example, statements of peer review or that the text has
been approved by an organization as the authoritative definition of a
standard.

You may add a passage of up to five words as a Front-Cover Text, and a
passage of up to 25 words as a Back-Cover Text, to the end of the list
of Cover Texts in the Modified Version.  Only one passage of
Front-Cover Text and one of Back-Cover Text may be added by (or
through arrangements made by) any one entity.  If the Document already
includes a cover text for the same cover, previously added by you or
by arrangement made by the same entity you are acting on behalf of,
you may not add another; but you may replace the old one, on explicit
permission from the previous publisher that added the old one.

The author(s) and publisher(s) of the Document do not by this License
give permission to use their names for publicity for or to assert or
imply endorsement of any Modified Version.


\section{COMBINING DOCUMENTS}

You may combine the Document with other documents released under this
License, under the terms defined in section~4 above for modified
versions, provided that you include in the combination all of the
Invariant Sections of all of the original documents, unmodified, and
list them all as Invariant Sections of your combined work in its
license notice, and that you preserve all their Warranty Disclaimers.

The combined work need only contain one copy of this License, and
multiple identical Invariant Sections may be replaced with a single
copy.  If there are multiple Invariant Sections with the same name but
different contents, make the title of each such section unique by
adding at the end of it, in parentheses, the name of the original
author or publisher of that section if known, or else a unique number.
Make the same adjustment to the section titles in the list of
Invariant Sections in the license notice of the combined work.

In the combination, you must combine any sections Entitled ``History''
in the various original documents, forming one section Entitled
``History''; likewise combine any sections Entitled ``Acknowledgements'',
and any sections Entitled ``Dedications''.  You must delete all sections
Entitled ``Endorsements''.

\section{COLLECTIONS OF DOCUMENTS}

You may make a collection consisting of the Document and other documents
released under this License, and replace the individual copies of this
License in the various documents with a single copy that is included in
the collection, provided that you follow the rules of this License for
verbatim copying of each of the documents in all other respects.

You may extract a single document from such a collection, and distribute
it individually under this License, provided you insert a copy of this
License into the extracted document, and follow this License in all
other respects regarding verbatim copying of that document.


\section{AGGREGATION WITH INDEPENDENT WORKS}

A compilation of the Document or its derivatives with other separate
and independent documents or works, in or on a volume of a storage or
distribution medium, is called an ``aggregate'' if the copyright
resulting from the compilation is not used to limit the legal rights
of the compilation's users beyond what the individual works permit.
When the Document is included in an aggregate, this License does not
apply to the other works in the aggregate which are not themselves
derivative works of the Document.

If the Cover Text requirement of section~3 is applicable to these
copies of the Document, then if the Document is less than one half of
the entire aggregate, the Document's Cover Texts may be placed on
covers that bracket the Document within the aggregate, or the
electronic equivalent of covers if the Document is in electronic form.
Otherwise they must appear on printed covers that bracket the whole
aggregate.


\section{TRANSLATION}

Translation is considered a kind of modification, so you may
distribute translations of the Document under the terms of section~4.
Replacing Invariant Sections with translations requires special
permission from their copyright holders, but you may include
translations of some or all Invariant Sections in addition to the
original versions of these Invariant Sections.  You may include a
translation of this License, and all the license notices in the
Document, and any Warranty Disclaimers, provided that you also include
the original English version of this License and the original versions
of those notices and disclaimers.  In case of a disagreement between
the translation and the original version of this License or a notice
or disclaimer, the original version will prevail.

If a section in the Document is Entitled ``Acknowledgements'',
``Dedications'', or ``History'', the requirement (section~4) to Preserve
its Title (section~1) will typically require changing the actual
title.


\section{TERMINATION}

You may not copy, modify, sublicense, or distribute the Document
except as expressly provided under this License.  Any attempt
otherwise to copy, modify, sublicense, or distribute it is void, and
will automatically terminate your rights under this License.

However, if you cease all violation of this License, then your license
from a particular copyright holder is reinstated (a) provisionally,
unless and until the copyright holder explicitly and finally
terminates your license, and (b) permanently, if the copyright holder
fails to notify you of the violation by some reasonable means prior to
60 days after the cessation.

Moreover, your license from a particular copyright holder is
reinstated permanently if the copyright holder notifies you of the
violation by some reasonable means, this is the first time you have
received notice of violation of this License (for any work) from that
copyright holder, and you cure the violation prior to 30 days after
your receipt of the notice.

Termination of your rights under this section does not terminate the
licenses of parties who have received copies or rights from you under
this License.  If your rights have been terminated and not permanently
reinstated, receipt of a copy of some or all of the same material does
not give you any rights to use it.


\section{REVISIONS OF THIS LICENSE}

The Free Software Foundation may publish new, revised versions
of the GNU Free Documentation License from time to time.  Such new
versions will be similar in spirit to the present version, but may
differ in detail to address new problems or concerns.  See
\texttt{http://www.gnu.org/copyleft/}.

Each version of the License is given a distinguishing version number.
If the Document specifies that a particular numbered version of this
License ``or any later version'' applies to it, you have the option of
following the terms and conditions either of that specified version or
of any later version that has been published (not as a draft) by the
Free Software Foundation.  If the Document does not specify a version
number of this License, you may choose any version ever published (not
as a draft) by the Free Software Foundation.  If the Document
specifies that a proxy can decide which future versions of this
License can be used, that proxy's public statement of acceptance of a
version permanently authorizes you to choose that version for the
Document.

\section{RELICENSING}

``Massive Multiauthor Collaboration Site'' (or ``MMC Site'') means any
World Wide Web server that publishes copyrightable works and also
provides prominent facilities for anybody to edit those works.  A
public wiki that anybody can edit is an example of such a server.  A
``Massive Multiauthor Collaboration'' (or ``MMC'') contained in the
site means any set of copyrightable works thus published on the MMC
site.

``CC-BY-SA'' means the Creative Commons Attribution-Share Alike 3.0
license published by Creative Commons Corporation, a not-for-profit
corporation with a principal place of business in San Francisco,
California, as well as future copyleft versions of that license
published by that same organization.

``Incorporate'' means to publish or republish a Document, in whole or
in part, as part of another Document.

An MMC is ``eligible for relicensing'' if it is licensed under this
License, and if all works that were first published under this License
somewhere other than this MMC, and subsequently incorporated in whole
or in part into the MMC, (1) had no cover texts or invariant sections,
and (2) were thus incorporated prior to November 1, 2008.

The operator of an MMC Site may republish an MMC contained in the site
under CC-BY-SA on the same site at any time before August 1, 2009,
provided the MMC is eligible for relicensing.


\section*{ADDENDUM: How to use this License for your documents}
\addcontentsline{toc}{section}{ADDENDUM: How to use this License for your documents}

To use this License in a document you have written, include a copy of
the License in the document and put the following copyright and
license notices just after the title page:

\bigskip
\begin{quote}
    Copyright \copyright{}  YEAR  YOUR NAME.
    Permission is granted to copy, distribute and/or modify this document
    under the terms of the GNU Free Documentation License, Version 1.3
    or any later version published by the Free Software Foundation;
    with no Invariant Sections, no Front-Cover Texts, and no Back-Cover Texts.
    A copy of the license is included in the section entitled ``GNU
    Free Documentation License''.
\end{quote}
\bigskip
    
If you have Invariant Sections, Front-Cover Texts and Back-Cover Texts,
replace the ``with \dots\ Texts.''\ line with this:

\bigskip
\begin{quote}
    with the Invariant Sections being LIST THEIR TITLES, with the
    Front-Cover Texts being LIST, and with the Back-Cover Texts being LIST.
\end{quote}
\bigskip
    
If you have Invariant Sections without Cover Texts, or some other
combination of the three, merge those two alternatives to suit the
situation.

If your document contains nontrivial examples of program code, we
recommend releasing these examples in parallel under your choice of
free software license, such as the GNU General Public License,
to permit their use in free software.
\newpage
\part*{References}%
\addcontentsline{toc}{part}{References}
\bibliography{common/references,CITATION}
\end{document}

%%% Local Variables:
%%% mode: latex
%%% TeX-master: "estimation"
%%% End:
