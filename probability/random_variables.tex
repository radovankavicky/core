% Copyright © 2013, authors of the "Econometrics Core" textbook; a
% complete list of authors is available in the file AUTHORS.tex.

% Permission is granted to copy, distribute and/or modify this
% document under the terms of the GNU Free Documentation License,
% Version 1.3 or any later version published by the Free Software
% Foundation; with no Invariant Sections, no Front-Cover Texts, and no
% Back-Cover Texts.  A copy of the license is included in the file
% LICENSE.tex and is also available online at
% <http://www.gnu.org/copyleft/fdl.html>.

\part*{Random variables, distributions, and densities}%
\addcontentsline{toc}{part}{Random variables, distributions, and densities}

\section{Random variables as we might wish to introduce them to
  undergraduates.}

\begin{itemize}

\item We're trying to capture the idea of a ``hypothetical'' number
  that might take on many different values.  For example, if I were to
  go to Las Vegas next weekend, I might win \$10 while I'm there; I
  might win \$1,000; I might ``win'' --\$5,000; etc.  If we're going
  to talk meaningfully about what might happen if I go to Las Vegas,
  we're going to need a way to match each outcome (here how much money
  I win or lose) with the relative frequency of that outcome.

  The simplest\sidenote{``Simplest'' from a conceptual/intuitive
  standpoint; we'll see that this is far from the simplest way from
  a rigorous mathematical standpoint.}  way to do this is through a
  probability density function and, as in most of probability, the
  easiest way to start is to work with discrete events (in which case
  the ``density'' function us usually called a ``probability mass
  function'').  Suppose that we're interested in a countable number of
  possible outcomes on the real line; think of handful of dice and
  adding up all of the numbers, so the possible values that the sum
  can take on are in some set $S = \{x_i ∣ i ∈ \ZZ\}$, where \ZZ\ is
  the set of integers.  Then we can define a probability density
  function $f: \RR → \RR$ as the function such that
  \begin{enumerate}
  \item $f(x) ≥ 0$ for all $x ∈ \RR$,
  \item $f(x) = 0$ for all $x ∉ S$, and
  \item $∑_{x ∈ S} f(x_i) = 1$.
  \end{enumerate}

  So now we can think of a random variable $X$ that takes on values in
  the set $\mathcal{X}$ with probabilities given by $f$.  And we can
  build a probability function from a density $f$ for the random
  variable $X$ by asserting that, for any $A ⊂ S$,
  \begin{equation*}
    \Pr[ X ∈ A ] = ∑_{s ∈ A} f(s)
  \end{equation*}

\item If we want to let $X$ take on the entire range of values in \RR\
  we can proceed the same way.  Again define the density function $f:
  \RR → \RR$ such that
  \begin{enumerate}
  \item $f(x) ≥ 0$ for all $x ∈ \RR$ and
  \item $∫_{-∞}^∞ f(x) \dx = 1$.
  \end{enumerate}
  (The set $S ⊂ \RR$ s.t. $f(x) > 0$ for all $x ∈ S$ and $f(x) = 0$
  for all $x ∉ S$ is called the \emph{support} of the r.v. $X$.)  We
  can again build a probability function from the density function.
  For any closed interval $[a, b]$,\sidenote{Remember, $[a,b] = \{x ∈
  \RR ∣ a ≤ x ≤ b\}$.} we can define
  \begin{equation*}
    \Pr[ X ∈ [a,b] ] = ∫_a^b f(x) \dx,
  \end{equation*}
  and for any set $A$ that can be written as the union of a countable
  number of disjoint intervals, $A = ⋃_i [a_i, b_i]$, where $⋂_i [a_i,
  b_i] = ∅$, we can define
  \begin{equation*}
    \Pr[ X ∈ A ] = ∑_i \Pr[ X ∈ [a_i, b_i] ] = ∑_i ∫_{a_i}^{b_i} f(x) \dx
  \end{equation*}

\item It's natural to want to go even further, and define for any set
  $A ⊂ \RR$
  \begin{equation*}
    \Pr[ X ∈ A ] = ∫_A f(x) \dx ≡ ∫_{-∞}^∞ \1\{x ∈ A\} f(x) \dx,
  \end{equation*}
  but it turns out that this integral is not well-defined for every $A
  ⊂ \RR$ (and, consequently, it's possible to construct functions $f$
  that can't be used as densities, a fact that we glossed over
  earlier).  However, it should be obvious that we can construct a
  self-contained set of subsets of \RR, call it \BB, so that we can
  define $\Pr[X ∈ B] = ∫_B f(x) \dx$ for all $B ∈ \BB$.

  We can motivate some of properties that \BB\ needs from properties
  that seem necessary for the probability function $\Pr$.

  \begin{enumerate}
  \item We need $\Pr[X ∈ \RR] = ∫_{-∞}^∞ f(x) \dx = 1$, so \RR\ must be
    in \BB.
  \item If $\Pr[X ∈ B]$ is well defined then $\Pr[X ∉ B]$ should be as
    well, and we should have $\Pr[X ∉ B] = 1 - \Pr[X ∈ B]$.  Since $X
    ∉ B$ is equivalent to $X ∈ B^c$, $B^c$ must be in \BB\ whenever
    $B$ is.
  \item If $B₁$, $B₂$, $B₃$,… make up a countable sequence of sets and
    we can define $\Pr[X ∈ B_i]$ for each of them, we should be able
    to define $\Pr[X ∈ ⋃_i B_i]$ and $\Pr[X ∈ ⋂_i B_i]$.  So \BB\ must
    be closed under countable unions and intersections.
  \end{enumerate}

  Any set of sets \BB\ that satisfies those three properties is called
  a \emph{sigma-algebra} (or \emph{σ-algebra}).  For subsets of the
  real line, there's an especially useful and common σ-algebra, the
  \emph{Borel σ-algebra},which is defined as the smallest σ-algebra
  that has all of the intervals as elements.

\item The integral we're interested is a little different than the
  usual Riemann integral that you've seen in introductory calculus
  classes, but not in any way that's going to matter for our purposes.
  The key difference is that it works by taking different points in
  the function's height and mapping them back to Borel sets in the
  function's domain, while the Riemann integral works by taking
  different points in the function's domain and mapping them to the
  height.\sidenote{Remember, to get the Riemann integral of a function
  $g: \RR → \RR$ over $[a,b]$, first define an increasingly fine
  partition of $[a,b]$, $\{x_{i,n}, i = 1,…n, n = 1,2,…\}$ such that
  $a = x_{1,n}$, $b = x_{n,n}$, $x_{i,n} < x_{i+1,n}$ for all $i$
  and $\max_{i=1,…,n-1} |x_{i+1,n} - x_{i,n}| → 0$ as $n → ∞$.  If
  we let $M_i = \max_{x ∈ [x_{i,n}, x_{i+1,n}]} g(x)$ and $m_i =
  \min_{x ∈ [x_{i,n}, x_{i+1,n}]} g(x)$, then the Riemann integral
  exists if
  \begin{multline*}
    \lim_{n → ∞} ∑_{i=1}^{n-1} M_i (x_{i+1,n} - x_{i,n}) \\
    = \lim_{n → ∞} ∑_{i=1}^{n-1} m_i (x_{i+1,n} - x_{i,n})
  \end{multline*}
  for any partition $\{x_{n,i}\}$.} 
  So, if $g$ is any step function that can be written as a step
  function of Borel sets,
  \begin{equation*}
    g(x) = ∑_{i=1}^∞ c_i \1\{x ∈ B_i\}
  \end{equation*}
  where $B_i ∈ \BB$, we can define the Lebesgue integral as
  \begin{equation*}
    ∫ g dμ = ∑_{i=1}^∞ c_i μ(B_i),
  \end{equation*}
  where $μ(B_i)$ is the length of the set $B_i$.\sidenote{I should
  maybe add a definition or an Appendix on Lebesgue measure, but
  that will probably happen later.}  The integrals of functions that
  are not step functions can be evaluated as the limit of a sequence
  of integrals of functions that are step functions.  For example, if
  $g_n → f$ as $n → ∞$ where each $g_n = ∑_i c_{i,n} \1\{x ∈ B_{i,n}\}$
  then we often can show that
  $∫ f \dμ = \lim_{n → ∞} ∑_i c_{i,n} μ(B_{i,n})$ (I say ``often''
  because there are some conditions on the sequence $\{g_n\}$ that we
  haven't mentioned that need to be satisfied for this to hold)

  For all of this to work, we need to be able to find a Borel set in
  the domain of the function for any set of points in the range;
  i.e. for the step function above, we have, for any point $x$,
  \begin{equation*}
    g^{-1}(\{x\}) =
    \begin{cases}
      B_i & x = c_i \\
      ∅ & x ∉ \{c₁, c₂,…\}
    \end{cases}
  \end{equation*}
  and, for any $B ∈ \BB$, $g^{-1}(B) = ⋃_{i ∈ I} B_i$ where $I = \{i ∣
  c_i ∈ B\}$.\sidenote{You should prove on your own that this means
  that $\{g^{-1}(B) ∣ B ∈ \BB\}$ is a σ-algebra and is a subset of
  \BB.}

\item The key property, then, for this integral to work is that
  $g^{-1}(\BB)$ itself be a σ-algebra that is contained in \BB.  Such
  a function is called \emph{Borel-measurable}.  We're going to
  restrict our focus to densities $f$ that are Lebesgue measurable, so
  we can always write probabilities as
  \begin{equation*}
    \Pr[X ∈ B] = ∫_B f(x) dx
  \end{equation*}
  for Borel-sets $B$.  We won't be able to define these probabilities
  for non-Borel sets, but that's fine; in this case, every function
  that we'd want to work with is Lebesgue-measurable and every set
  that we'd want to work with is a Borel set.\sidenote{Add example of
  unmeasurable set here….}

  If we're clever, we can define probabilities for discrete random
  variables the same way.  Define the function $δ$, known as Dirac's
  δ-function, as $δ = \lim_{n → ∞} δ_n$ where
  \begin{equation*}
    δ_n(x) =
    \begin{cases}
      n + n² x & - \tfrac{1}{n} ≤ x ≤ 0 \\
      n - n² x & 0 < x ≤ \tfrac{1}{n} \\
      0        & \text{otherwise},
    \end{cases}
  \end{equation*}
  so each $δ_n$ is a very narrow, very tall spike at 0 with total area
  1, and as $n$ grows the spike gets narrower and higher.  The limit
  $δ$ is an infinitely tall, infinitely narrow spike at 0 with total
  area 1.

  Under Lebesgue-measure, we can now define the density of a
  Bernoulli(1/2) random variable $b$ as
  \begin{equation*}
    f_b(x) = \tfrac{1}{2} δ(x) + \tfrac{1}{2} δ(x - 1).
  \end{equation*}
  This lets us write $\Pr[b ∈ B] = ∫_B f_b(x) \dx$ for any Borel-set
  $B$, which is what we want.\sidenote{You should work through all of
  the steps necessary to prove this to yourself.}

\item Two key points to emphasize and remember: if the usual Riemann
  integral of a function exists, its value is the same as the Lebesgue
  integral. And it is very difficult to accidentally write down a
  function on \RRⁿ\ that is not Lebesgue measurable, so don't worry
  that you will.

\item The probabilities for a given random variable $X$ can be
  summarized by its \emph{cumulative distribution function} (CDF)
  which is typically written as $F_X$ and is defined as
  \begin{equation*}
    F_x(c) ≡ \Pr[X ≤ c] = \Pr[X ∈ (∞, c]]
  \end{equation*}
  for any $c ∈ \RR$.  The CDF of a random variable always exists and
  satisfies
  \begin{enumerate}
  \item $\lim_{x → -∞} F_X(x) = 0$
  \item $\lim_{x → ∞} F_X(x) = 1$
  \item $F_X$ is non-decreasing in $x$.
  \item $F_X$ is right-continuous.  For every number $x₀$,
    $\lim_{x ↓ x₀} F(x) = F(x₀)$.
  \end{enumerate}

  You should verify the property $\Pr[X ∈ (a, b]] = F_X(b) - F_X(a)$
  as homework.

\item Obviously we can have vectors of random variables (often called
  \emph{random vectors}).  These vectors have density functions too,
  which we can build up sequentially.  If $X₁ ∼ f₁$, we can imagine
  that a second r.v., $X₂$ has (potentially) a different density that
  depends on the value of $X₁$; write it as $f₂(· ∣ x₁)$, where $x₁$
  is a hypothetical value of $X₁$ (and the function $f₂(· ∣ x₁)$ is
  defined for all values in the support of $X₁$.  Then the joint
  density of $X₁$ and $X₂$, which we'll write as $f$, can be evaluated
  as
  \begin{equation}\label{eq:2}
    f(x₁,x₂) = f₂(x₂ ∣ x₁) × f₁(x₁).
  \end{equation}
  For discrete random variables, you can interpret this as saying
  ``the probability that $X₁$ and $X₂$ take the values $x₁$ and $x₂$
  is equal to the probability that $X₁$ takes the value $x₁$ and then
  $X₂$ takes the value $x₂$.''  But don't read too much into the word
  ``then'' too literally since the ordering isn't important.  We can
  also write
  \begin{equation}
    f(x₁,x₂) = f₁(x₁ ∣ x₂) f₂(x₂).
  \end{equation}

  If the conditional density of $X₁$ does not depend on $X₂$ (or vice
  versa) the random variables are said to be \emph{independent}.  This
  means that joint density can be written as
  \begin{equation*}
    f(x₁,x₂) = f₁(x₁) × f₂(x₂)
  \end{equation*}
  The elements of a $k$-vector of random variables are independent if
  \begin{equation*}
    f(x₁,…,x_k) = ∏_{i=1}^k f_i(x_i)
  \end{equation*}
  for all $x₁,…,x_k$ in the random variables' support.

\begin{figure}[t]
  [Empty for now.  Sometime soon, I'll add a figure.]
  \caption{Illustration of the conditional density math to go here.}
\end{figure}

  The functions $f₁(· ∣ x2)$ and $f₂(· ∣ x₁)$ are called \emph{the
  conditional densities of $X₁$ and $X₂$} respectively.  Given the
  joint and marginal densities, they can be calculated/defined as
  \begin{equation*}
    f₁(x₁ ∣ x₂) = f(x₁, x₂) / f₂(x₂)
  \end{equation*}
  Conditional probabilities of events and conditional distribution
  functions can be defined using this conditional density just as
  before; e.g.
  \begin{equation*}
    F₁(x ∣ x₂) = ∫_{-∞}^x f₁(z ∣ x₂) dz
  \end{equation*}
  defines the \emph{conditional CDF of $X₁$ given $X₂$}, where $x₂$ is
  a different hypothetical value in $X₂$'s support.

\item Note that we can recover the marginal density of $X₁$ from
  Equation~\eqref{eq:2} by integrating out the conditional density of
  $X₂$:
  \begin{equation*}
    ∫_{-∞}^{-∞} f₂(x₂ ∣ x₁) f₁(x₁) \dx_2 = f₁(x₁)
  \end{equation*}
  since $f₂(· ∣ x₁)$ is a density function and so integrates to 1.
  This gives a slightly less obvious relationship between the
  \emph{joint} and marginal densities:
  \begin{equation*}
    ∫_{-∞}^{-∞} f(x₁, x₂) \dx_2 = f₁(x₁).
  \end{equation*}
  There is clearly nothing special about the order of the variables.

\item We can define the joint CDF as
  \begin{equation*}
    F(x₁,…,x_k) = \Pr[X₁ ≤ x₁,…, X_k ≤ x_k].
  \end{equation*}
  and can be retrieved from the joint density as before:
  \begin{equation*}
    F(x₁,…,x_k) = ∫_{-∞}^{x₁} ⋯ ∫_{-∞}^{x_k} f(z₁,…,z_k) \dz_1 ⋯ \dz_k.
  \end{equation*}
  Writing this equation explicitly makes it clear that the CDF can be
  factored under independence just as the densities can.

  The marginal distribution and density functions can be derived
  directly from the joint distribution and density.  If we let $F₁$ be
  the marginal distribution of the first element and $f₁$ the marginal
  density, then we have
  \begin{equation*}
    F₁(x) = F(x, ∞,…,∞)
  \end{equation*}
  which is obvious in retrospect, because adding the additional
  restrictions that $X_j ∈ (-∞, ∞)$ for $j = 2,…,k$ imposes no new
  constraints on the vector $X$.

  Similarly, the relationship
  \begin{equation*}
    f₁(x) = ∫_{-∞}^∞ ⋯ ∫_{-∞}^∞ f(z₁,…,z_k) \dz_k ⋯ \dz_2.
  \end{equation*}
  can also seen by starting with the marginal CDF and then
  differentiating to get the density.

\end{itemize}

\section{Sigma-algebras as information sets}

\begin{itemize}

\item In the previous section, we motivated the Borel σ-algebra as a
  collection of sets that it makes sense to assign length to, and that
  it makes sense to talk about the ``Probability'' of a random
  variable landing in.  As you've probably seen elsewhere, we should
  think of a random variable is a way of numerically summarizing a
  random outcome.

\item Sometimes it doesn't make sense to assign numeric values right
  away, though.  If we want to think about modeling card games, the
  events that we might be concerned with are different hands that we
  might be dealt.  But it might not make sense to summarize that
  information numerically unless we have already set a particular
  game.  Fortunately, the key properties that we were discussing
  beforehand don't need to be specific to numbers.

  In general, we can call the set of all possible outcomes of a
  (possibly hypothetical) experiment the \emph{sample space} of the
  experiment, and let $Ω$ be the \emph{σ-algebra} of the experiment if
  it satisfies the rules we discussed above.  The elements of $Ω$ are
  usually referred to as \emph{events}.

  Since set theory plays a large role in this material,
  Table~\ref{tab:1} lists the definitions of common set operations.

\begin{table*}[t]
\begin{tabular}{lp{4in}}
  \toprule
  \multicolumn{2}{c}{Finite Set Operations} \\
  \midrule
  Union        & $A₁ ∪ A₂ ≡ \{x ∈ S : x ∈ A₁ \text{ or } x∈ A₂\}$ \\
  Intersection & $A₁ ∩ A₂ ≡ \{x ∈ S : x ∈ A₁ \text{ and } x ∈ A₂\}$ \\
  Subset       & $A₁ ⊂ A₂$ if $A₁ ∩ A₂ = A₁$ \\
  Equality     & $A₁ = A₂$ if $A₁ ⊂ A₂$ and $A₂ ⊂ A₁$ \\
  Complement   & $A₁^c ≡ \{x ∈ S : x ∉ A₁\}$ \\
  Difference   & $A₁ ∖ A₂ ≡ \{x ∈ S : x ∈ A₁ \text{ and } x ∉ A₂\}$ \\
  Symmetric difference & $A₁ Δ A₂ ≡ (A₁ ∖ A₂) ∪ (A₂ ∖ A₁)$ \\
  \midrule
  \multicolumn{2}{c}{Infinite Set Operations (can be defined for uncountable index sets as well)} \\
  \midrule
  Union        & $⋃_{k=1}^∞ A_k ≡ \{x ∈ S : x ∈ A_k$ for at least one $k ∈ 1,2,…\}$ \\
  Intersection & $⋂_{k=1}^∞ A_k ≡ \{x ∈ S : x ∈ A_k$ for all $k ∈ 1,2,…\}$ \\
  Infimum      & $\inf_{k ≥ n} A_k ≡ ⋂_{k=n}^∞ A_k$ \\
  Supremum     & $\sup_{k ≥ n} A_k ≡ ⋃_{k=n}^∞ A_k$ \\
  lim inf      & $\liminf_{n → ∞} A_n ≡ ⋃_{n=1}^∞ ⋂_{k=n}^∞ A_k$ \\
  lim sup      & $\limsup_{n → ∞} A_n ≡ ⋂_{n=1}^∞ ⋃_{k=n}^∞ A_k$ \\
  \midrule
  \multicolumn{2}{c}{Other definitions} \\
  \midrule
  limit 
  & Suppose that $A$, $A₁$, $A₂$,… is a sequence of subsets of $S$ and that $\limsup_{n → ∞} A_n = \liminf_{n → ∞} A_n = A$.
  Then $A$ is the limit of $\{A_n\}$. \\
  Disjoint 
  & $A₁$ and $A₂$ are disjoint if $A₁ ∩ A₂ = ∅$. \\
  Pairwise disjoint 
  & $A₁, A₂, …$ are pairwise disjoint if $A_i ∩ A_j = ∅$ for all $i$ and $j$ such that $i ≠ j$. \\
  Partition & If $A₁,A₂,…$ are pairwise disjoint and $⋃_{i=1}^∞ A_i = S$ then $\{A_i\}$ forms a partition of $S$. \\
  Power set & The power set of $S$, denoted $\mathcal{P}(S)$, is the set of all subsets of $S$.
  $\mathcal{P}(S) ≡ \{x : x ⊂ S\}$ \\
\bottomrule
\end{tabular}
  \caption{Collection of set operations; let $A₁$, $A₂$, $A₃$,… be subsets of another set $S$.}
  \label{tab:1}
\end{table*}

\item %
  \begin{defn} A collection $Ω$ of subsets of $S$ is a
    \emph{σ-algebra} if it satisfies the following
    \begin{enumerate}
    \item $S ∈ Ω$,
    \item $A^c ∈ Ω$ whenever $A ∈ Ω$, where $A^c = S ∖ A$,
    \item If $A₁$, $A₂$,… is a countable sequence with $A_i ∈ Ω$ for
      all $i$, then $⋃_{i=1}^∞ A_i ∈ Ω$ as well.
    \end{enumerate}
  \end{defn}

\item A pretty typical first example for σ-algebras involves coin
  flips.  But we're economists, so let's use a game.  One with the
  same structure as coin flips.
  \begin{ex}[Matching pennies]
    For \emph{matching pennies}, there are two players (A and B) and
    each has a coin.  Each player chooses ``Heads'' or ``Tails''
    (secretly) and then they simultaneously show each other their
    coin.  If the coins match, player A wins and if they don't, player
    B wins.

    Now, let's model a single round of the game.  The sample space is
    the set $S = \{HH, HT, TH, TT\}$.\sidenote{Where, for example,
    \emph{HT} means that player A plays ``Heads'' and player B plays
    ``Tails''} There are, of course, several σ-algebras that can be
    constructed on this sample space.  For example, the σ-algebra that
    represents whether or not A plays ``Heads'' is
    \begin{equation*}
      \{∅, S, \{HH,HT\}, \{TH,TT\}\}
    \end{equation*}
    and the σ-algebra representing whether or not A wins is
    \begin{equation*}
      \{∅, S, \{HH,TT\}, \{TH,HT\}\}.
    \end{equation*}
    We can form the second σ-algebra by starting with the event ``A
    wins'' (i.e. the set $\{HH, TT\}$), and the sample space $S$, and
    then adding more sets as necessary so that the σ-algebra is closed
    under complements and countable unions.\sidenote{You should verify
    as an exercise that both of these sets are actually σ-algebras.}
  \end{ex}

\item This example gets at why we can view a σ-algebra as an
  \emph{information set}.  The σ-algebra, ``A plays Heads,'' contains
  all of the information we learn if we learn whether or not the event
  ``A plays Heads'' has happened.\sidenote{I.e., we learn whether or
  not A played Heads, and whether or not A played Tails.}  We might
  also view this as the information available to player A after choosing
  H or T, but before revealing his or her choice to B.

  Also, if one σ-algebra contains another (the other is a subset) than
  it embodies more information than the smaller σ-algebra.

\item Now we can formally define a probability measure that obeys the
  same rules as before.
  \begin{defn} A function $\Pr$ is a \emph{probability measure} on a
    σ-algebra $Ω$ of subsets of $S$ if
    \begin{enumerate}
    \item $\Pr: Ω ↦ [0,1]$
    \item if $A₁$, $A₂$,… are disjoint events in $Ω$ then
      $\Pr(⋃_{i=1}^∞ A_i) = ∑_{i=1}^∞ \Pr(A_i)$
    \item $\Pr(S) = 1$.
    \end{enumerate}
  \end{defn}
  
\item We refer to a sample space $S$, a σ-algebra $Ω$, and a
  probability measure $\Pr$ as a \emph{probability space}, $(S, Ω,
  \Pr)$.

\item Let's go back to the example,
  \begin{ex}[Matching pennies, continued]
    We can define strategies through the probability measure on a
    generating class, then use the properties over countable unions
    and intersections to extend it to the rest of the σ-algebra.

    We'll suppose that A plays a mixed strategy and chooses ``Heads''
    with probability 1/3 (and Tails with probability 2/3), and that B
    plays ``Heads'' with probability 4/5.  We'll also assume that
    neither player is cheating, so these choices are independent of
    each other.  This means that we have
    \begin{align*}
      \Pr \{HH\} &= (1/3) × (4/5) = 4/15 \\
      \Pr \{HT\} &= (1/3) × (1/5) = 1/15 \\
      \Pr \{TH\} &= (2/3) × (4/5) = 8/15 \\
      \Pr \{TT\} &= (2/3) × (1/5) = 2/15.
    \end{align*}
    We can write any other event as the union of these sets, so this
    lets us calculate the probability of any event in the game.  For
    example, the event, \emph{A wins} is the set $\{HH,TT\}$, and
    \begin{equation*}
      \Pr \{HH, TT\} = \Pr \{HH\} + \Pr \{TT\} = (4/15) + (2/15) = 6/15
    \end{equation*}
    (these strategies are obviously not a Nash equilibrium).
  \end{ex}

\item Often, knowing that one event happens tells us something about
  how likely other events are.  In the Matching Pennies example we've
  used above, the probability that A wins is $6/15$.  If we know that
  A has played heads, we know that A wins if B plays heads too, which
  happens with probability $4/5$.

\item This was an example of conditioning on an event (e.g. the event
  that A plays heads).  The general formula for working out
  probabilities conditional on an event is as follows,

  \begin{defn}
    If $A$ and $B$ are events in some sample space $S$ and $\Pr[B]>0$,
    the \emph{conditional probability of $A$ given $B$} is written as
    $\Pr(A ∣ B)$ and defined to be
    \begin{equation*}
      \Pr(A ∣ B) ≡ \Pr(A ∩ B) / \Pr(B).
    \end{equation*}
    If $\Pr(B) = 0$ then we can define $\Pr(A ∣ B)$ arbitrarily to be
    zero.
  \end{defn}

  The equivalent formula
  \begin{equation*}
    \Pr(A ∩ B) = \Pr(A ∣ B) \Pr(B)
  \end{equation*}
  may make the interpretation easier: the probability that events $A$
  and $B$ happen is equal to the probability that $B$ happens times
  the probability that $A$ happened, knowing that $B$ already
  happened.\sidenote{Don't take the word ``already'' literally, since
  there is not time ordering implied by these formulas.}

\item Observe that if $A$ and $B$ are mutually exclusive, then $\Pr(A
  ∩ B) = 0$, and so $\Pr(A ∣ B) = 0$.

\item We can use the definitions and the simple observation that
  \begin{equation*}
    \Pr(A ∣ B) \Pr(B) = \Pr(B ∣ A) \Pr(A)
  \end{equation*}
  to justify \emph{Bayes's Rule}:
  \begin{thm}[Bayes's Rule]
    If $A$ and $B$ are sets in $S$ that have positive probability, then
    \begin{equation*}
      \Pr(A ∣ B) = \Pr(B ∣ A) \Pr(A) / \Pr(B).
    \end{equation*}
  \end{thm}

\item Naturally, we want to be able to discuss cases where two events
  do not affect each other at all; such events are called
  \emph{independent}.

  \begin{defn}
    Events $A$ and $B$ are \emph{independent} if $P(A ∩ B) = P(A)
    P(B)$.
  \end{defn}

  An immediate implication is that $\Pr(A ∣ B) = \Pr(A)$ if $A$ and
  $B$ are independent.

\end{itemize}

\section{Random variables are maps between information sets}

\begin{itemize}
\item The way to think about random variables, now, is that they
  summarize a probability space.  They take events that occur in a
  possibly abstract experiment and turn them into events in an easier
  to analyze experiment.  Often they map into the Borel σ-field.

  As an example, think back to the matching pennies example, but now
  imagine that players A and B play $n$ games in a row.  The sample
  space is now $\{HH, HT, TH, TT\}ⁿ$.  For some purposes, though, it
  might be sufficient to know how much money A wins from or loses to B
  (assuming they're gambling).  A random variable can map the
  different game outcomes to the real line and preserve the
  probability measure of the original experiment.

  \begin{defn} Let $\Fs$ be a sigma-field on a sample space $S$.  An
  $\Fs$-measurable function $X$ from $S$ to $\RR$ is a \emph{random
  variable}.
  \end{defn}

  Remember the definition of \emph{$\Fs$-measurable}: $X$ is an
  $\Fs$-measurable function from $S$ to $\RR$ if $X^{-1}(B) ∈ \Fs$ for
  all $B ∈ \BB$.  Intuitively, we want to use the random variable $X$
  to transfer the probability measure on $S$ to $\RR$.  This is only
  going to work if we can map events in $\RR$ (i.e., elements of
  $\BB$) back to the original sigma algebra $\Fs$.

  Formally, this means that $X$ is a function (it's worth repeating),
  and the random variable takes on values $X(ω)$, where $ω ∈ S$.  It
  can sometimes be useful to write it out so explicitly in proofs, but
  usually we will just write $X$ and suppress the argument.

\item This also reinforces the idea of $\Fs$ as an information set:
  $X$ is $\Fs$-measurable if knowledge of events in $\Fs$ is
  sufficient to know about $X$.  This concept is most clear in a
  time-series setting.  Let $\{\Fs_t\}$ be an expanding sequence of
  information sets, so $\Fs_t ⊂ \Fs_{t+1}$ for all $t$ (this means
  that $\Fs_t$ is something like ``the information available in period
  $t$''—in period $t$, we know/remember what happened in period $t-1$,
  contained in $\Fs_{t-1}$, but do not know what will happen in period
  $t+1$).  If $\{X_t\}$ is a sequence of random variables s.t. $X_t$ is
  $\Fs_t$-measurable for every $t$ (which is often written as $X_t ∈
  \Fs_t$), then $X_t$ is a random variable whose value is determined
  in period $t$—knowledge of the outcomes in period $t$ is sufficient
  to know the value of $X_t$, but not necessarily the value of
  $X_{t+1}$.

\item Measurable mappings from the sample space $S$ to $\RRⁿ$ are
  called \emph{random vectors}, more generally they're often called
  \emph{random elements}.

\item In this approach, the original probability space is the
  fundamental idea, and then we define random variables, distribution
  functions, and densities by referencing the original probability
  measure.  I.e., given a probability measure and a random variable
  $X$, we \emph{define} the CDF of $X$ as the function $F_X$ given by
  \begin{equation*}
    F_X(x) = \Pr(\{ω ∣ X(ω) ≤ x\}).
  \end{equation*}
  Note that the $ω$'s are elements of a sample space $S$, so the set
  $\{ω ∣ X(ω) ≤ x\}$ is an element of the sigma field that we've been
  calling $\Fs$.  We know that the set $\{ω ∣ X(ω) ≤ x\}$ must be in
  $\Fs$ by measurability: it is equivalent to the set $X^{-1}((-∞,x])$,
  and $(-∞,x] ∈ \BB$.

  The properties that we listed for CDFs can be derived from this
  definition by using some fundamental properties of probability
  measures we've skipped over.  And then, given a distribution
  function $F_X$, the density of $X$ can be defined as the function
  $f_X$ such that
  \begin{equation*}
    F_X(x) = ∫_{-∞}^{x} f_X(z) \dz.
  \end{equation*}

  As alluded to earlier, this is the more mathematically sound way to
  proceed, but it can give the wrong impression.  One almost never
  needs to work at this level of abstraction even in theoretical
  econometrics research—typically we start with the random variables
  and leave the deeper probability structure implicit.

\item A similar definition of independence holds for information sets
  and random variables as we defined for events (this is to some
  degree a restatement of what we discussed in the first section, but
  with a more justifiable mathematical underpinning).
  \begin{defn}
    Two sigma-fields $\Fs$ and $\Gs$ are independent if the events $F$
    and $G$ are \emph{independent} for any $F ∈ \Fs$ and $G ∈ \Gs$.
    Two random variables $X$ and $Y$ are \emph{independent} if the
    sigma-fields $X^{-1}(\BB)$ and $Y^{-1}(\BB)$ are independent.
  \end{defn}
  The random variable definition is essentially the same as
  \begin{equation*}
    \Pr[X ∈ I₁ ∩ Y ∈ I₂] = \Pr[X ∈ I₁] \Pr[Y ∈ I₂]
  \end{equation*}
  for any intervals $I₁$ and $I₂$.

\item It is then straightforward to verify that, for independent
  random variables $X$ and $Y$, the joint distribution has the
  property $F(x, y) = F_X(x) F_Y(y)$ and the joint density has the
  property $f(x,y) = f_X(x) f_Y(y)$.  These properties are easy to
  verify using the definition above, based on the underlying
  information sets, but are somewhat harder to prove if starting from
  a less abstract definition of independence.

  We also have, if $X₁$ and $X₂$ are independent, so are $g(X₁)$ and
  $h(X₂)$ for any (measurable) functions $g$ and $h$.  For a proof,
  observe
  \begin{align*}
    \Pr[g(X₁) ∈ I₁ ∩ h(X₂) ∈ I₂]
    &= \Pr[X₁ ∈ g^{-1}(X₁) ∩ X₂ ∈ h^{-1}(X₂)] \\
    &= \Pr[X₁ ∈ g^{-1}(X₁)] \Pr[X₂ ∈ h^{-1}(X₂)] \\
    &= \Pr[g(X₁) ∈ I₁] \Pr[h(X₂) ∈ I₂]
  \end{align*}
  where $g^{-1}(I₁) = \{x : g(x) ∈ I₁\}$, we don't need $g$ or $h$ to
  be invertible.

\end{itemize}

%%% Local Variables:
%%% mode: latex
%%% TeX-master: "../../probability"
%%% End:

%  LocalWords:  Borel CDF Supremum measurability CDFs differentiable
%  LocalWords:  Bayes's Infimum invertible
