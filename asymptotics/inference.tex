% Copyright © 2013, authors of the "Econometrics Core" textbook; a
% complete list of authors is available in the file AUTHORS.tex.

% Permission is granted to copy, distribute and/or modify this
% document under the terms of the GNU Free Documentation License,
% Version 1.3 or any later version published by the Free Software
% Foundation; with no Invariant Sections, no Front-Cover Texts, and no
% Back-Cover Texts.  A copy of the license is included in the file
% LICENSE.tex and is also available online at
% <http://www.gnu.org/copyleft/fdl.html>.

\part*{Inference}%
\addcontentsline{toc}{part}{Inference}

\section{Introduction and definitions}

\begin{itemize}
\item One problem with the testing results we've done so far
\begin{itemize}
\item need to know the DGP (up to unknown constants) to get the
        distribution of the statistic.
\begin{itemize}
\item An example would be nice (maybe X1+X2 from normal vs. uniform)
\end{itemize}
\item this is unrealistic in practice
\end{itemize}
\item CLT suggests taht, for a lot of DGPs and statistics, the exact
      distribution isn't going to matter much, as long as we have a
      lot of observations
\item So we can use the CLT to derive an approximate distribution for
      our statistics, and use approximate critical values.
    \item Definition: let $\{T_n\}$ be a sequence of test statistics
      for $θ ∈ θ_0$ against $θ ∈ θ_A$.  The sequence of tests has
      asymptotic size $α$ if
      \[\lim_{n → ∞} \sup_{θ ∈ θ_0} \Pr_θ[T_n \text{ rejects}] = α\]
\begin{itemize}
\item asymptotic level if the equality is an inequality.
\end{itemize}
\item The same idea holds for confidence intervals
\begin{itemize}
\item Let $\{[L_n, U_n\}$ be a sequence of interval estimators for a
        parameter $θ$.  The asymptotic confidence level for this
        sequence of statistics is
        \[\lim_{n → ∞} \inf_θ \Pr_{θ}[θ ∈ [L_n, U_n]]\]
\item important that the limit is outside the inf or sup
\item you should think of a good question or example to address that
\begin{itemize}
\item maybe from delta method?
\end{itemize}
\end{itemize}
\end{itemize}

\section{Wald Test}

\begin{itemize}
\item The Wald tests is built on the CLT explicitly
\end{itemize}

\subsection{Main idea}

\begin{itemize}
\item Suppose we want to test the null $θ = θ_0$ and we have an
  asymptotically normal estimator $\θh_n$: $\sqrt{n} (\θh_n - θ) →^d
  N(0,η²)$ for all $θ$.
\item Then we can base a test on $\frac{\sqrt{n}}{\ηh} (\θh - θ_0)$ as
  long as $\ηh$ is a consistent estimator of $η$. (i.e. a $t$-test or
  something derived from it.
\begin{itemize}
\item often get a chi-squared distribution
\end{itemize}
\end{itemize}

\subsection{Example}

\begin{itemize}
\item Suppose $(Y_i, X_i) ∼ i.i.d$ with unknown density
\begin{itemize}
\item $\E(Y_i ∣ X_i) = β_0 + β₁ X_i$
\item $\var(Y_i ∣ X_i) = σ²$
\end{itemize}
\item Want to test the null that $β_0 = β₁ = 0$ against
       two-sided alternatives.
\begin{itemize}
\item Use our MLE $\βh$ = $(X'X)^{-1} X'Y$
\end{itemize}
\item Step 1, prove that $\sqrt{n}(\βh - β) → N(0, Σ = σ² (\E x_i x_i')^{-1})$
\item step 2, prove that $\σh²$ converges in prob to $σ²$
\item Step 3, argue that $\frac{n}{\σh²}
       \βh'\Σh^{-1} \βh$ is asymptotically
       chi-square with 2 degrees of freedom.
\item Step 4, explain test statistic.
\end{itemize}

\section{LRT result}

\begin{itemize}
\item The chi-square limit distribution comes up a lot
\item A nice result for the LRT gives us a chi-square limit as well
\item Let $X₁,...,X_n ∼ i.i.d.\ f(x; θ)$ and let $λ(X)$ be the LR
  statistic for the test $θ ∈ θ_0$ vs. $θ ∉ θ_0$.  Under appropriate
  regularity conditions,
  \[-2 \log λ(X) →^d χ²_p\]
      under the null, where $p$ is the number of restrictions imposed
      by the null (in the wald test example, there wre two
      restrictions)
\end{itemize}

\section{Other miscellaneous points}

\begin{itemize}
\item You can also get a test from the first order conditions (called
      the ``score'' or ``LM'' test)
\item results for testing immediately translate into results for
      confidence intervals/sets.
\end{itemize}

%%% Local Variables:
%%% mode: latex
%%% TeX-master: "../asymptotics"
%%% End:
