% Copyright © 2013, authors of "Core Econometrics Notes;" a
% complete list of authors is available in the file AUTHORS.tex.

% Permission is granted to copy, distribute and/or modify this
% document under the terms of the GNU Free Documentation License,
% Version 1.3 or any later version published by the Free Software
% Foundation; with no Invariant Sections, no Front-Cover Texts, and no
% Back-Cover Texts.  A copy of the license is included in the file
% LICENSE.tex and is also available online at
% <http://www.gnu.org/copyleft/fdl.html>.

\part*{Point estimators}%
\addcontentsline{toc}{part}{Point estimators}

\section{Consistency}
     Consistency is an asymptotic version of unbiasedness

\paragraph{Definition}
an estimator $\θh$ is a consistent estimator of the parameter $θ$ if
$\θh →^d{p} θ$ for any value of $θ$.

\paragraph{Basic result for consistency \textbf{:hw:}}
\begin{itemize}
\item Statement of the result
\begin{itemize}
\item Let $\θh_n$ be a sequence of estimators of $θ$ such
          that, for every $θ$,
\begin{enumerate}
\item $\var_θ(\θh_n) → 0$
\item $\E_θ (\θh_n) → 0$
\end{enumerate}
\item then $W_n$ is consistent for $θ$
\end{itemize}
\item Proof:
\begin{itemize}
\item Follows immediately from Cauchy-Schwarz
\item Want to show $\Pr[|\θh_n - θ| > ε] → 0$ for all $ε > 0$
  \begin{align*}
    \Pr[|\θh_n - θ| > ε] 
    &≤ \dfrac{\E(\θh_n - θ)²}{ε²} \\
    &= \dfrac{\E((\θh_n - E\θh_n)+ (\E\θh_n - θ))²}{ε²} \\
    &= \dfrac{\E(\θh_n - E\θh_n)²}{ε²}
      + 2 \dfrac{\E((\θh_n - E\θh_n)(E\θh_n - θ))}{ε²} + \dfrac{(E\θh_n - θ)²}{ε²} \\
    &= o(1) + 0 + o(1)
  \end{align*}
\end{itemize}
\end{itemize}

\paragraph{Consistency of MLE (\citealp[Theorem 10.1.6]{CB02} and
  \citealp[Section 14.4]{Gre12})}
\begin{itemize}
\item We're just going to mention the result
\begin{itemize}
\item you'll see it in more detail next semester
\item we won't mention the assumptions
\item But, this is a big justification for using MLE as a general method
\end{itemize}
\item Suppose that $X₁,...,X_n ∼ i.i.d.\ f(x; θ)$ where
        $f$ satisfies some technical regularity conditions and suppose
        $\θh_n$ is the MLE of $θ$.  Then $\θh_n →^p θ$ as $n → ∞$.
\begin{itemize}
\item Note, doesn't always hold, but usually holds.
\item A good homework exercise---look up the regularity conditions
          in \citet{CB02} or \citet{Gre12} and see if common families (gamma,
          exponential, uniform, etc) satisfy them; then try to prove consistency
          directly.
\item There are weaker conditions than listed in \citet{CB02} (in particular,
          we can relax i.i.d)
\end{itemize}
\end{itemize}

\paragraph{Consistency of method of moments estimators}

Suppose that $X₁,...,X_n ∼ f(·; θ)$ where $θ$ is a $k$-vector, $(\E
X_i,...,\E X_i^k) = g(θ₁,...,θ_k)$, $g^{-1}$ is continuous, and that
$\frac{1}{n} ∑_i X_i^k$ obeys a weak LLN.  Then the Method of Moments
estimator of $θ$ is consistent.

\section{Asymptotic Efficiency}

\begin{itemize}
\item To talk about efficiency, we need to restrict our attention to
      asymptotically normal estimators
\begin{itemize}
\item rules out pathalogical counterexamples
\item I should add one next year (homework?)
\item kind of makes sense; for normal r.v. only the mean and
        variance matter; for other asymp. distributions other
        parameters are more important.
\end{itemize}
\item If $\θh_n$ satisfies $m_n (\θh_n - θ) →^d N(0, σ²)$ for some
  (increasing) sequence $m_n$ then $σ²$ is the asymptotic variance of
  $\θh_n$.
\item Suppose that $X_1,...,X_n$ have the likelihood fn $L(θ; X)$ and
  that $\θh_n$ is a sequence of estimators of $θ$.  $\θh_n$ is
  asymptotically efficient for $θ$ if $m_n (\θh_n - θ) \to^d N(0, σ²)$
  and
  \[σ² = \Big[\E_θ ((\tfrac{∂}{∂θ} \log L(θ; X))²)\big]^{-1}\].
\begin{itemize}
\item i.e. $\θh_n$ achieves the CR lower bound in the limit.
\end{itemize}
\item Nice result: under regularity conditions (like we saw before),
      MLEs are consistent and asymptotically normal as well as
      efficient.
\end{itemize}

%%% Local Variables: 
%%% mode: latex
%%% TeX-master: "../asymptotics"
%%% End: 
