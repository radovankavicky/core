% Copyright (c) 2013–2014, authors of "Core Econometrics;" a
% complete list of authors is available in the file AUTHORS.tex.

% Permission is granted to copy, distribute and/or modify this
% document under the terms of the GNU Free Documentation License,
% Version 1.3 or any later version published by the Free Software
% Foundation; with no Invariant Sections, no Front-Cover Texts, and no
% Back-Cover Texts. A copy of the license is included in the file
% LICENSE.tex and is also available online at
% <http://www.gnu.org/copyleft/fdl.html>.

\chapter{Problems}

\begin{hw}
  Suppose that $X_1,...,X_n \sim$ i.i.d. uniform($a$,$b$).
  \begin{enumerate}
  \item Derive the MLE of $\E X_i$ and prove that it is consistent.
  \item Please find the asymptotic distribution of the MLE of $\E
    X_i$. You will need to rescale the estimator to find an
    asymptotic distribution. Two hints: it is not asymptotically
    normal; and you may need to use the result $(1 + \tfrac{1}{n})^n \to
    e$ as $n \to \infty$ (``may'' meaning that my solution uses that result).
  \end{enumerate}
\end{hw}

\begin{hw}
  Suppose $X_1,...,X_n \sim$ uniform$(-\theta, \theta)$. Find the
  method-of-moments estimator of $\theta$ based on
  \begin{equation*}
    n^{-1} \sum_{i=1}^n X_i^2.
  \end{equation*}
  Is it consistent? Asymptotically normal?
\end{hw}

\begin{hw}
  Let $\{e_t\}$ be a sequence of i.i.d. $(0,1)$ random variables
  and let $X_t = e_t + 0.5\, e_{t-1}$. Such a process is called a
  Moving Average of order 1, abbreviated as MA(1). What is the
  probability limit of $n^{-1} \sum_{t=1}^n X_t$ as $n \to \infty$?
\end{hw}

\begin{hw}
  Suppose that $Z$ is a random variable with mean $\mu$ and variance
  $\sigma$ and let $e_1,e_2,...$ be a sequence of i.i.d. random
  variables with mean zero and variance 1 that are independent of $Z$.
  Show that the sequence $X_i = Z + e_i$ obeys the following weak law
  of large numbers: as $n \to \infty$, $n^{-1} \sum_{i=1}^n ( X_i -
  \E(X_i \mid Z) ) \to 0$ in probability. Note that $X_1,X_2,...$ is
  \emph{not} an i.i.d.\ sequence, but is an example of an
  \emph{exchangeable} sequence of random variables.
\end{hw}

\begin{hw}
  Suppose that $X_n \to^p X$ and that $g$ is a continuous function.
  Prove that $g(X_n) \to^p g(X)$.
\end{hw}

\begin{hw}
  Let $X_1,...,X_n \sim N(\theta, \theta)$. Derive the MLE of
  $\theta$ and show that it is asymptotically normal.
\end{hw}

\begin{hw}
  Suppose that $X_1,...,X_n$ are an i.i.d. sample from the density
  $f(x) = 1/\theta$, $0 \leq x \leq \theta$. Prove that $\max_i X_i$
  is consistent for $\theta$.
\end{hw}

\begin{hw}
  Let $\{e_t\}$ be a sequence of i.i.d. $(0,1)$ random variables
  and let $X_t = e_t + 0.5\, e_{t-1}$. Such a process is called a
  Moving Average of order 1, abbreviated as MA(1). What is the
  probability limit of $n^{-1} \sum_{t=1}^n X_t$ as $n \to \infty$?
\end{hw}

\begin{hw}
  Suppose that $X_1,...,X_n \sim$ i.i.d. uniform($a$,$b$).
  \begin{enumerate}
  \item Show that $(\min_i X_i, \max_i X_i)$ is the maximum likelihood
    estimator of $(a,b)$.
  \item Prove that the MLE is consistent.
  \item Please find the asymptotic distribution of the MLE. You will
    need to rescale the estimator to find an asymptotic distribution.
    Two hints: it is not asymptotically normal; and you may need to
    use the result $(1 + \tfrac{1}{n})^n \to e$ as $n \to \infty$.
  \item Use your answer to the previous question to construct a
    two-sided 90\% confidence interval for $b$.
  \item When we derived the asymptotic distribution earlier, we used
    only some aspects of the assumption that $X_i \sim$
    uniform($a$,$b$). Can you show that $\max_i X_i$ and $\min_i X_i$
    have the same asymptotic distribution that we derived above under
    weaker assumptions?
  \end{enumerate}
\end{hw}

\begin{hw}
  Suppose that $Z$ is a random variable with mean $\mu$ and variance
  $\sigma$ and let $e_1,e_2,\dots$ be a sequence of i.i.d. random
  variables with mean zero and variance 1 that are independent of $Z$.
  Show that the sequence $X_i = Z + e_i$ obeys the following weak law
  of large numbers: as $n \to \infty$, $n^{-1} \sum_{i=1}^n ( X_i -
  \E(X_i \mid Z) ) \to 0$ in probability. Note that $X_1,X_2,\dots$
  is \emph{not} an i.i.d.\ sequence, but is an example of an
  \emph{exchangeable} sequence of random variables.
\end{hw}

\begin{hw}
  Suppose that $X_n \to^p X$ and that $g$ is a continuous function.
  Prove that $g(X_n) \to^p g(X)$.
\end{hw}

\begin{hw}
  Let $X_1,\dots,X_n \sim N(\theta, \theta)$. Derive the MLE of
  $\theta$ and show that it is asymptotically normal.
\end{hw}

\begin{hw}
  Suppose that $X_1,\dots,X_n$ are an i.i.d. sample from the density
  $f(x) = 1/\theta$, $0 \leq x \leq \theta$. Prove that $\max_i X_i$
  is consistent for $\theta$.
\end{hw}

\begin{hw}
  Let $\{y_t\}$ be an iid $N(0,\sigma^2)$ sequence. Define $S_T =
  (T-1)^{-1} \sum_{t=1}^T (y_t - \bar y)^2$.
  \begin{enumerate}
  \item Prove that $\sqrt{n} (S_T - \sigma^2) \to N(0,2\sigma^4)$ in
    distribution.
  \item Calculate the asymptotic distribution of $\log S_T$.
  \item Do your results depend on the normality of $y_t$?
  \item Calculate the 90th and 95th percentile for $S_T$ with
    arbitrary values of $\sigma^2$ using both of these asymptotic
    distributions.
  \item Use each of these results to derive a test of the null
    hypothesis $\sigma^2 = \sigma_0^2$ against the null $\sigma^2 >
    \sigma_0^2$, where $\sigma_0^2$ is a known but arbitrary value.
    Your answer should give two formulas for the test's critical
    value---each one depends on $\sigma_0^2$ and $\alpha$ (the nominal
    size of the test) and you should have a separate answer for each
    asymptotic approximation.
  \item Let $c_1$ and $c_2$ denote the 90th percentiles that you
    calculated in question ?. Simulate 1000 i.i.d. standard normal
    samples with $n = 50$ and calculate the probability that $S_T$ is
    less than each of these percentiles.
  \item Repeat the previous question for the 95th percentiles.
  \item Plot a histogram of your 1000 simulated $S_T$ along with each
    approximate density for $S_T$. What do these results tell you
    about the quality of the approximations?
  \item You can also prove that $(T-1) S_T$ has a chi-square
    distribution with $T-1$ degrees of freedom in finite samples.
    Calculate the 90th and 95th percentiles of $S_T$ using this
    chi-square distribution and repeat the previous two simulations.
    How do these simulations compare to the previous simulations?
  \item Repeat the previous three questions using a skewed
    distribution and a heavy-tailed distribution (changing the
    approximation as necessary). How do the results change? How do
    they change if you use different values of $n$?
  \item What do these simulations tell you about using these
    approximations for testing. Focus on the usual confidence levels
    (i.e. 10\%, 5\%, and 1\% tests).
  \end{enumerate}
\end{hw}

%%% Local Variables: 
%%% mode: latex
%%% TeX-master: "../core_econometrics"
%%% End: 
