% Copyright © 2013, authors of the "Econometrics Core" textbook; a
% complete list of authors is available in the file AUTHORS.tex.

% Permission is granted to copy, distribute and/or modify this
% document under the terms of the GNU Free Documentation License,
% Version 1.3 or any later version published by the Free Software
% Foundation; with no Invariant Sections, no Front-Cover Texts, and no
% Back-Cover Texts.  A copy of the license is included in the file
% LICENSE.tex and is also available online at
% <http://www.gnu.org/copyleft/fdl.html>.

\part*{Problem Set}%
\addcontentsline{toc}{part}{Problem Set}

This is a collection of problems for self-study.

\begin{enumerate}
\item Prove that the OLS estimator $\βh$ and the OLS residuals $\hat
  u$ are uncorrelated.

\item Suppose that 
\begin{equation}
  y_i = β₁ + β₂ x_i + u_i,  
\end{equation}
with $u_i ∣ X ∼ (0, σ²)$ and $x_i ∼ (μ,τ²)$.  Please derive a test
statistic for the null hypothesis $β₁ = β₂²$ against the alternative
$β₁ ≠ β₂²$.

\item Suppose that $y₁,…,y_n$ are i.i.d. with $y_i ∣ x_i ∼ N(x_i'β,
  σ²)$ and each $x_i$ a $k × 1$ vector.  Prove that the OLS estimator
  is the best unbiased estimator of $β$.  Note that this is a stronger
  result than the Gauss-Markov theorem, since we are claiming that OLS
  is also better than nonlinear unbiased estimators.

\item We have the model $Y = Xβ + u$, where $n^{-1} X'X$ obeys a law
  of large numbers, $u$ is homoskedastic, and $n^{-1/2} ∑_i x_i u_i$
  obeys a central limit theorem.  Let $F$ be the F-statistic for the
  null hypothesis that $Rβ = 0$, where $R$ is an arbitrary $J × (K +
  1)$ vector
  \begin{enumerate}
  \item Suppose that the null hypothesis is true.  How does the
    variance of $F$ behave as $n → ∞$?  How does the mean of $F$
    behave?
  \item Suppose that $Rβ = δ$ for some nonzero vector $δ$, so the null
    hypothesis is false.  How do your answers from the first part
    change?
  \item What do your answers to the previous questions tell you about
    the power of the F-test as $n → ∞$?
  \end{enumerate}

\item Suppose that you are interested in the model $y_i = β₀ + x_iβ₁
  + u_i$ where $x_i$ is a random scalar, $\E(u_i ∣ X) = 0$ and the
  observations are i.i.d. But suppose that we do not observe $x_i$
  directly but instead we observe $w_i = x_i + u_i$, where $u_i$ is
  another error term.

  \begin{enumerate}
  \item Prove that, if $u_i$ is perfectly correlated with $x_i$, OLS
    is essentially consistent in the following sense: let $\dot x_i =
    (x_i - \E x_i) / sd(x_i)$ and let $\dot w_i = (w_i - \E w_i) /
    sd(w_i)$.  Then the OLS coefficient in the regression of $y_t$ on
    $\dot x_i$ is the same as that of the regression of $y_i$ on $\dot
    w_i$.
  \item Prove that, if $u_i$ is independent of $x_i$ and $u_i$, OLS is
    inconsistent even after standardizing the variables.
  \item Derive the MLE of $β₁$ under the assumption that $u_i$ and
    $u_i$ are independent mean-zero normal and determine whether it is
    consistent.
  \item Suppose now that $u_i$ and $u_i$ are independent, mean zero
    normal random variables, but now assume that we have another
    regressor $z_i = x_i + v_i$, where $v_i$ is another mean-zero,
    normal error term that's independent of $u_i$ and $u_i$.  Derive
    the MLE of $β₁$ and determine whether it is consistent.  For bonus
    points and well-deserved pride: is it asymptotically normal?
  \end{enumerate}

\item Suppose that you estimate the model $y_i = β₀ + β₁ x_{i1} + β₂
  x_{i2} + u_i$ with OLS and calculate the F-test for the null
  hypothesis $β₁ = 1$.  The $p$-value for this test is $0.03$ and
  $\βh₁$ is 0.54.  In the following questions, you can assume that all
  of the necessary OLS assumptions hold.

  \begin{enumerate}
  \item How would you use this information to test against the
    one-sided alternative that $β₁ > 1$ at the 5\% level?  Do you
    reject the null in favor of this alternative?
  \item How does your answer change if the alternative is $β₁ < 1$?
    Do you reject in favor of this alternative?
  \item Based on your answers to the previous two questions, please
    outline a procedure to test the general null hypothesis that $Rβ ≤
    q$, where the inequality holds element by element.
  \end{enumerate}

\item Let $y_i$ and $x_i$ be i.i.d. random scalars and $\E x_i = 0$.
  Suppose that you estimate two models: $y_i = μ + u_i$ and $y_i = β₀
  + β₁ x_i + v_i.$ Calculate the bias and variance of $\hat μ$ and
  $\βh₀$.  How do they depend on $β₁$?

\item Suppose that we have the model $y_i = β₀ + β₁ x_{1i} + β₂ x_{2i}
  + β₃ x_{3i} + u_i$ where $\E(u_i ∣ X) = 0$ but the errors may be
  heteroskedastic.

  \begin{enumerate}
  \item Suppose you want to test the null hypothesis $β_i ≤ 0$ for all
    $i = 1,…,3$ against the alternative that $β_i > 0$ for at least
    one $i$ (note that this is a single null hypothesis against a
    single alternative).  How could you use the multiple hypothesis
    techniques we discussed in class to test this hypothesis?
  \item If this procedure rejects the null that all of the $β_i$ are
    weakly negative, do we know why?  Specifically, do we know which
    of the $β_i$ is positive?  In what sense?
  \item Please write an R program to implement your answer to the
    first part of the question.
  \end{enumerate}

\item Suppose that we have two equations we want to estimate:
  \begin{equation}
    \label{eq:1}
    Y₁ = X₁ 'β₁ + u₁
  \end{equation}
  and
  \begin{equation}
    \label{eq:2}
    Y₂ = X₂'β₂ + u₂
  \end{equation}
  $\E(u ∣ X₁, X₂) = 0$ and $\E(u u' ∣ X₁, X₂) = σ ⊗ I$, $u = (u₁,
  u₂)'$.  The Gauss-Markov theorem states that the GLS estimator for
  $β₁$ and $β₂$ in the regression
  \begin{equation}
    \label{eq:3}
    \begin{pmatrix} Y₁ \\ Y₂ \end{pmatrix}
    = \begin{pmatrix} X₁ & 0 \\ 0 & X₂ \end{pmatrix} 
    \begin{pmatrix} β₁ \\ β₂ \end{pmatrix}
    + \begin{pmatrix} u₁ \\ u₂ \end{pmatrix}
  \end{equation}
  is BLUE.  But what if we only care about $β₁$, and so we just use
  the SUR estimate of $β₁$?  Equation \eqref{eq:1} on its own
  satisfies the Gauss-Markov assumptions, so the OLS estimator of $β₁$
  should be BLUE too.  But both estimators can't be BLUE, can they?
  They're different except in a few special cases.  What's going on?
\end{enumerate}

%%% Local Variables: 
%%% mode: latex
%%% TeX-master: "../../regression"
%%% End: 
