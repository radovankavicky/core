% Copyright (c) 2013-2014, Gray Calhoun.

% Permission is granted to copy, distribute and/or modify this
% document under the terms of the GNU Free Documentation License,
% Version 1.3 or any later version published by the Free Software
% Foundation; with no Invariant Sections, no Front-Cover Texts, and no
% Back-Cover Texts. A copy of the license is included in the file
% LICENSE.tex and is also available online at
% <http://www.gnu.org/copyleft/fdl.html>.

\section{Additional exercises}

\begin{hw}
  Prove that the OLS estimator $\betah$ and the OLS residuals $\hat u$
  are uncorrelated.
\end{hw}

\begin{hw}
  Suppose that
  \begin{equation}
    y_i = \beta_1 + \beta_2 x_i + u_i,
  \end{equation}
  with $u_i \mid X \sim (0, \sigma^2)$ and $x_i \sim (\mu,\tau^2)$.
  Please derive a test statistic for the null hypothesis $\beta_1 =
  \beta_2^2$ against the alternative $\beta_1 \neq \beta_2^2$.
\end{hw}

\begin{hw}
  Suppose that $y_1,\dots,y_n$ are i.i.d. with $y_i \mid x_i \sim
  N(x_i'\beta, \sigma^2)$ and each $x_i$ a $k \times 1$ vector. Prove
  that the OLS estimator is the best unbiased estimator of $\beta$.
  Note that this is a stronger result than the Gauss-Markov theorem,
  since we are claiming that OLS is also better than nonlinear
  unbiased estimators.
\end{hw}

\begin{hw}
  We have the model $Y = X\beta + u$, where $n^{-1} X'X$ obeys a law
  of large numbers, $u$ is homoskedastic, and $n^{-1/2} \sum_i x_i
  u_i$ obeys a central limit theorem. Let $F$ be the F-statistic for
  the null hypothesis that $R\beta = 0$, where $R$ is an arbitrary $J
  \times (K + 1)$ vector
  \begin{enumerate}
  \item Suppose that the null hypothesis is true. How does the
    variance of $F$ behave as $n \to \infty$? How does the mean of
    $F$ behave?
  \item Suppose that $R\beta = \delta$ for some nonzero vector
    $\delta$, so the null hypothesis is false. How do your answers
    from the first part change?
  \item What do your answers to the previous questions tell you about
    the power of the F-test as $n \to \infty$?
  \end{enumerate}
\end{hw}

\begin{hw}
  Suppose that you estimate the model $y_i = \beta_0 + \beta_1 x_{i1}
  + \beta_2 x_{i2} + u_i$ with OLS and calculate the F-test for the
  null hypothesis $\beta_1 = 1$. The $p$-value for this test is
  $0.03$ and $\betah_1$ is 0.54. In the following questions, you can
  assume that all of the necessary OLS assumptions hold.

  \begin{enumerate}
  \item How would you use this information to test against the
    one-sided alternative that $\beta_1 > 1$ at the 5\% level? Do you
    reject the null in favor of this alternative?
  \item How does your answer change if the alternative is $\beta_1 <
    1$? Do you reject in favor of this alternative?
  \item Based on your answers to the previous two questions, please
    outline a procedure to test the general null hypothesis that
    $R\beta \leq q$, where the inequality holds element by element.
  \end{enumerate}
\end{hw}

\begin{hw}
  Let $y_i$ and $x_i$ be i.i.d. random scalars and $\E x_i = 0$.
  Suppose that you estimate two models: $y_i = \mu + u_i$ and $y_i =
  \beta_0 + \beta_1 x_i + v_i.$ Calculate the bias and variance of
  $\hat \mu$ and $\betah_0$. How do they depend on $\beta_1$?
\end{hw}

\begin{hw}
  Suppose that we have the model $y_i = \beta_0 + \beta_1 x_{1i} +
  \beta_2 x_{2i} + \beta_3 x_{3i} + u_i$ where $\E(u_i \mid X) = 0$
  but the errors may be heteroskedastic.

  \begin{enumerate}
  \item Suppose you want to test the null hypothesis $\beta_i \leq 0$
    for all $i = 1,\dots,3$ against the alternative that $\beta_i > 0$
    for at least one $i$ (note that this is a single null hypothesis
    against a single alternative). How could you use the multiple
    hypothesis techniques we discussed in class to test this
    hypothesis?
  \item If this procedure rejects the null that all of the $\beta_i$
    are weakly negative, do we know why? Specifically, do we know
    which of the $\beta_i$ is positive? In what sense?
  \item Please write an R program to implement your answer to the
    first part of the question.
  \end{enumerate}
\end{hw}

%%% Local Variables: 
%%% mode: latex
%%% TeX-master: "../core_econometrics"
%%% End: 
