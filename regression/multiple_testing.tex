% Copyright © 2013, authors of "Core Econometrics;" a
% complete list of authors is available in the file AUTHORS.tex.

% Permission is granted to copy, distribute and/or modify this
% document under the terms of the GNU Free Documentation License,
% Version 1.3 or any later version published by the Free Software
% Foundation; with no Invariant Sections, no Front-Cover Texts, and no
% Back-Cover Texts.  A copy of the license is included in the file
% LICENSE.tex and is also available online at
% <http://www.gnu.org/copyleft/fdl.html>.

\part*{Multiple hypothesis testing in linear regression}%
\addcontentsline{toc}{part}{Multiple hypothesis testing in linear regression}


\section{Testing multiple hypotheses}

Some special issues come up when we have many hypotheses.  I need to
figure out some simulations to motivate this stuff.

\subsection{Intersection-Union Test}

\begin{itemize}
\item Suppose the null can be written as an intersection of simpler
  nulls:
  \begin{itemize}
  \item $H₀: θ ∈ ⋃_{γ ∈ Γ} θ_γ$ vs.  $H_A: θ ∉ ⋂_{γ ∈ Γ} θ_γ$
  \item It is easy to find tests for each individual set:
    \begin{itemize}
    \item $T_γ$ tests the null $θ ∈ θ_γ$ vs. $θ ∉ θ_γ$ for each $γ$.
    \end{itemize}
  \item If $T_γ$ is a level-$α$ test with rejection region $R_γ$ (ie
    it rejects if $(x₁,…,x_n) ∈ R$), then the test with rejection
    region $⋂_{γ ∈ Γ} R_γ$ is the IU Test and has level $α$
  \end{itemize}
\item Note that we don't need to correct the critical values here
\end{itemize}

\paragraph{Proof of validity}
\begin{itemize}
\item We know that, under the null, $θ ∈ θ_γ'$ for at least one $γ'$
\item $\Pr_θ[(X₁,…,X_n) ∈ ⋂_γ R_γ] ≤ \Pr_θ[(X₁,…, X_n) ∈ R_{γ'}] ≤ α$
\end{itemize}

\subsection{Union-Intersection Test}

\paragraph{Setup of UIT}
\begin{itemize}
\item Suppose the null can be written as an intersection of simpler
  nulls:
  \begin{itemize}
  \item $H₀: θ ∈ ⋂_{γ ∈ Γ} θ_γ$ vs.  $H_A: θ ∉ ⋂_{γ ∈ Γ} θ_γ$
  \item It is easy to find tests for each individual set:
    \begin{itemize}
    \item $T_γ$ tests the null $θ ∈ θ_γ$ vs. $θ ∉ θ_γ$ for each $γ$.
    \end{itemize}
  \item If $T_γ$ is a level-$\frac{α}{\# Γ}$ test with rejection
    region $R_γ$ (ie it rejects if $(x₁,…,x_n) ∈ R$), then the test
    with rejection region $⋃_{γ ∈ Γ} R_γ$ is the UI Test and has level
    $α$
  \end{itemize}
\item Note that we don't need to use a stepdown procedure here
\item Could still do better if we used the joint distribution of $T_γ$
\end{itemize}

\paragraph{Show that testing at $α$ doesn't work}

\paragraph{Proof of validity}
\begin{itemize}
\item Follows from the same argument as in multiple comparisons
  \begin{align*}
    \Pr_θ[(X₁,…,X_n) ∈ ⋃_{γ} R_γ] 
    &≤ ∑_{γ ∈ Γ} \Pr_θ[(X₁,…,X_n) ∈ R_γ] \\
    &≤ ∑_γ \frac{α}{\#Γ} \\
    &= α
  \end{align*}
\end{itemize}

\subsection{Multiple hypothesis testing}

\begin{itemize}
\item Suppose we have a bunch of hypotheses $θ ∈ θ_γ$, but instead of
  being interested in knowing about compound hypotheses, we're
  interested in all of them
  \begin{itemize}
  \item have many different parameters
  \item Put up empirical example where we're looking at regressors
  \end{itemize}
\item Now we want to control the rate of ``familywise error''
  \begin{itemize}
  \item Let $I$ index the true nulls: $I = \{γ ∈ Γ : θ_γ ∈ θ_γ\}$
  \item FWE is \[\sup_{θ: θ_γ ∈ θ_γ, γ ∈ I} \Pr_θ[T_γ \text{ rejects
      for at least one } γ ∈ I]\]
  \item want this error probability to be less than $α$
  \end{itemize}
\item As with UIT, testing each one at $α$ doesn't work
  \begin{itemize}
  \item show this
  \end{itemize}
\end{itemize}

\paragraph{Bonferroni bound}

\paragraph{Bonferroni-Holm stepdown}

\paragraph{Critical values from the joint distribution}

%%% Local Variables:
%%% mode: latex
%%% TeX-master: "../regression"
%%% End:
