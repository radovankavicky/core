% Copyright (c) 2013, authors of "Core Econometrics;" a
% complete list of authors is available in the file AUTHORS.tex.

% Permission is granted to copy, distribute and/or modify this
% document under the terms of the GNU Free Documentation License,
% Version 1.3 or any later version published by the Free Software
% Foundation; with no Invariant Sections, no Front-Cover Texts, and no
% Back-Cover Texts.  A copy of the license is included in the file
% LICENSE.tex and is also available online at
% <http://www.gnu.org/copyleft/fdl.html>.

\section{Introduction}
\subsection{How should we think about empirical applications?}

\begin{itemize}

\item What are we hoping to do with linear regression?  We probably
  have in mind changing one of the regressors.  Imagine a setting
  where we have data on
  \begin{itemize}
  \item some sort of school-average reading score on a state test
    (which is going to be our dependent variable)
  \item number of teachers per student at several schools
  \item money available per student at these schools
  \end{itemize}
  Conceivably, some policy could change number of teachers per student
  or could change the money available per student.  To know which (if
  either) of these is a good idea, we would need to know something like
  \begin{multline*}
    \E(\text{change in reading score} \mid \\
       \text{an independent body makes a 1\% change in expenditure})
  \end{multline*}

\item What makes econometrics distintive and interesting is that we
  don't always get to observe changes in expenditure and changes in
  test scores (although we do sometimes).  What we often have instead
  is sometimes we see levels of expenditure and levels of test scores
  in different schools.

  Moreover, we almost never get to observe changes made by an
  independent body.  We often will only see changes that were made for
  some reason---for example, if a school performed abnormally badly one
  year and was given funding to compensate, it would likely rebound
  next year (with or without the funding) and that rebound would be
  attributed to the funding.

\item When you're starting an empirical research project, probably the
  best thing to do is to think about what experiment you would want to
  design if you could.

  One approach if we want to understand how expenditure affects
  student reading test scores, we'd want to take all of the schools
  \emph{randomly} and \emph{independently} change their funding (leave
  some unchanged), then observe the change in test scores the next
  year.

  Next, think hard about how closely the process that created your
  data matches that experiment.

\item This can be an iterative process (though it is suboptimal to go
  back and forth to the data).  And you can't always think of an
  experiment.  If you can not think of an experiment, the question
  might not be an empirical question.  Sometimes this happens when you
  think about changing preference parameters, which is often
  interesting in the context of an economic model, but is more or less
  impossible in real life.

\end{itemize}

%%% Local Variables: 
%%% mode: latex
%%% TeX-master: "../core_econometrics"
%%% End: 
