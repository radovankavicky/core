% Copyright (c) 2013-2014, Gray Calhoun.

% Permission is granted to copy, distribute and/or modify this
% document under the terms of the GNU Free Documentation License,
% Version 1.3 or any later version published by the Free Software
% Foundation; with no Invariant Sections, no Front-Cover Texts, and no
% Back-Cover Texts.  A copy of the license is included in the file
% LICENSE.tex and is also available online at
% <http://www.gnu.org/copyleft/fdl.html>.

\section{Introduction}

\begin{itemize}[leftmargin=0pt]

\item This chapter gives a short preview of what we're going to do in
  Part~4. What we'll see, and what we've seen in earlier sections, is
  that the statistical properties of estimators are generally easy for
  a well-designed empirical study. (There are settings where things
  get complicated: time-series and weak instruments both come to mind,
  but this book doesn't deal with those settings in detail.) One will
  typically have an asymptotically normal estimator with a robust
  estimator of its variance, and constructing confidence intervals and
  test statistics is routine.  Most of this text, and most
  econometrics textbooks, can give you the appropriate formulas.

\item But there are hard parts that aren't fully formalized yet. In
  chronological order, they are roughly:
  \begin{enumerate}
  \item finding a research question and assembling a data set
    (material needs to be added)
  \item dealing with endogeneity and sample selection
  \item choosing a model to estimate
  \item checking the results for robustness and stability
  \item writing up the results as a research paper.
  \end{enumerate}
  Chapter~\ref{ch_causality} deals with the second issue and
  Chapter~\ref{ch_modeling} addresses the third. I need to write a
  chapter about specification testing to address the fourth point.

  This chapter will (in the future) have some additional material, but
  right now it just has a few references to writing guides. Read them
  all.\footnote{There are lots of online guides that could and should
    be included too.}

  \begin{itemize}
  \item \bibentry{StW99}.

    \emph{This is a short and clear guide to American English grammar
      and usage. Buy it and read it first.}

  \item \bibentry{Mil13}.

    \emph{Note that \citet{Mil04} is a similar book by the same
      author.  But you don't need to buy it since it is entirely
      contained in the \emph{Multivariate Analysis} book.}

  \item \bibentry{Tuf01}.

    \bibentry{Tuf90}.

    \emph{Tufte has two other books, but his first two are the
      best. These books discuss how to present data graphically.}

  \item \bibentry{WiC10}.

    \emph{This book gives step-by-step instructions on how to organize
      a research paper, from the entire document down to individual
      sentences. \citet{WiC10b} is a textbook-length treatment of this
      material and includes exercises, but you probably want
      \citet{WiC10}.}

  \end{itemize}

  I'll add more material later, but the last piece of advice is to use
  LaTeX. There are people who will be biased against your job-market
  paper if they see that it's written in Word.

\item Interpretation-wise, there's lots more to add here. In the
  meantime, look at the following in \citet{IW07}

  \begin{itemize}
  \item \textit{Estimation of average treatment effects under
      unconfoundedness}
    (\url{http://www.nber.org/WNE/lect_1_match_fig.pdf})
  \item \textit{Instrumental variables with treatment effect
      heterogeneity: local average treatment effects}
    (\url{http://www.nber.org/WNE/lect_13_weakmany_iv.pdf})
  \item \textit{Weak instruments and many instruments}
    (\url{http://www.nber.org/WNE/lect_5_late_fig.pdf})
  \item \textit{Discrete choice models}
    (\url{http://www.nber.org/WNE/lect_11_dc_fig.pdf})
  \item \textit{Missing data}
    (\url{http://www.nber.org/WNE/lect_12_missing.pdf})
  \end{itemize}
  as well as \citet{IW09}. The plan is to present mostly
  program-evaluation notation and results in the ``interpretation''
  section and let future coursework fill in estimating structural
  models.

\end{itemize}

%%% Local Variables: 
%%% mode: latex
%%% TeX-master: "../core_econometrics"
%%% End: 
