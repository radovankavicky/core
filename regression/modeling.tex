% Copyright (c) 2013, authors of "Core Econometrics;" a
% complete list of authors is available in the file AUTHORS.tex.

% Permission is granted to copy, distribute and/or modify this
% document under the terms of the GNU Free Documentation License,
% Version 1.3 or any later version published by the Free Software
% Foundation; with no Invariant Sections, no Front-Cover Texts, and no
% Back-Cover Texts.  A copy of the license is included in the file
% LICENSE.tex and is also available online at
% <http://www.gnu.org/copyleft/fdl.html>.

\section{Issues in modeling}

\subsection{Functional form of regression models}
\subsection{polynomial regression}

\paragraph{motivation}

\begin{itemize}
\item We can get some flexibility while keeping the OLS framework
\item Linear model is
  \[y_i = \beta_0 + \beta_1 x_{i1} + \cdots + \beta_k x_{ik} + \epsilon_i\]
\begin{itemize}
\item The regressors $x_{ij}$ can be transformations of the same
          underlying ``variable''
\end{itemize}
\end{itemize}

\paragraph{two degree polynomial}
\begin{itemize}
\item draw scatterplot of parabola
\item might think that the model
  \[y_i = \beta_0 + \beta_1 x_i + \beta_2 x^2_i + \vep_i\]
  is a good description of the data
\item This model can be estimated by OLS directly
\item More generally, models of the form
  \[y_i = \beta_0 + \beta_1 g_1(x_i) + \beta_2 g_2(x_i) + \cdots + \beta_k g_k(x_k) + \vep_i\]
  can be estimated by OLS if
\begin{itemize}
\item conditional mean of $\epsilon_i$ given X is zero
\item each $g_j$ is known.
\end{itemize}
\end{itemize}

\subsection{Binary regressors}

\paragraph{intro}
\begin{itemize}
\item This is basic, and you probably already know it
\item Non-numeric data can arrive as numbers
\begin{itemize}
\item just because the data arrive as numbers, doesn't mean that
          they are \emph{really} numbers.
\end{itemize}
\end{itemize}

\paragraph{Example from Greene: education data}
\begin{itemize}
\item $y_i$ is income
\item $x_{i1}$ (say) represents highest level of education and has
          values
\begin{enumerate}
\item if individual i has not completed high school
\item if individual i has completed high school but no higher
             degree
\item if individual i has completed college but no higher
             degree
\item if individual i has completed a master's degree (or
             advanced to candidacy in a Ph.D. program) but no higher
             degree
\item if individual i has completed a ph.d. program
\end{enumerate}
\end{itemize}

\paragraph{Common example from ``satisfaction surveys''}
\begin{itemize}
\item rate the following questions on a scale from 1 to 5
\begin{enumerate}
\item Strongly agree
\item Agree
\item neutral
\item Disagree
\item Strongly disagree
\end{enumerate}
\end{itemize}

\paragraph{How to proceed}
\begin{itemize}
\item \textbf{do not} regress $y$ on $x_1$! (draw scatterplot of $y$ vs. $x_1$)
\begin{itemize}
\item assumes that the difference between expected income w/
          Ph.D. vs Masters degree is the same as the difference for
          college degree vs. highschool.
\item Even worse, assumes that the expected difference between
          income with Ph.D. and high school degree is \underline{exactly} three
          times the difference between expected income with college
          vs. high school degree.
\end{itemize}
\item Instead, create four dummy variables
\begin{itemize}
\item remember from undergrad, creating 5 and including a constant
          means the design matrix doesn't have full rank
\item $D_1$ :: 1 if $x_{i1}$ = 2, 3, 4, or 5; zero otherwise
\item $D_2$ :: 1 if $x_{i1}$ = 3, 4, or 5; zero otherwise
\item $D_3$ :: 1 if $x_{i1}$ = 4, or 5; zero otherwise
\item $D_4$ :: 1 if $x_{i1}$ = 5; zero otherwise
\item note :: There are other ways that you could code it, but
                  this will make the comparisons we want to make easy.
\end{itemize}
\end{itemize}

\paragraph{Estimation}
\begin{itemize}
\item Now, regress $y_i$ on $D_{i1}$, $D_{i2}$, $D_{i3}$, $D_{i4}$, and a constant:
  \[ y_i = \beta_0 + \beta_1 D_{i1} + \beta_2 D_{i2} + \beta_3 D_{i3} + \beta_4 D_{i4} + \epsilon_i \]
\item Interpretation of this model
\begin{itemize}
\item Expected income of a worker with a masters degree but no
          higher degree:
\begin{itemize}
\item masters degree means that $D_1$, $D_2$, and $D_3$ all equal 1;
  $D_4$ equals zero is
  \[
  \E(y_i \mid D_{i1}=1,D_{i2}=1,D_{i3}=1,D_{i4}=0) = \beta_0 + \beta_1 + \beta_2 + \beta_3
  \]
\item Estimate of expected income for those workers is
  $\betah_0 + \betah_1 + \betah_2 + \betah_3$
\end{itemize}
\item Difference in expected income of people with college degree
          vs. those with masters degree:
\begin{itemize}
\item $\E(y_i \mid \text{college}) = \beta_0 + \beta_1 + \beta_2$
\item $\E(y_i \mid \text{masters}) = \beta_0 + \beta_1 + \beta_2 + \beta_3$
\item difference in expected values is just $\beta_3$
\item Estimated difference is $\betah_3$
\end{itemize}
\end{itemize}
\end{itemize}

\subsection{loosening ``linearity''}

\paragraph{splines}

\paragraph{motivation}
\begin{itemize}
\item Draw scatter plot for y vs. 1-5 (education) again
\item have a general relationship between expected value of income
         and observed level of education (not linear)
\item transformed that relationship to something we can estimate by
         OLS
\item Can we do the same thing if the regressors are \underline{not}
         discrete?
\end{itemize}

\paragraph{basic idea}
\begin{itemize}
\item Suppose x is unrestricted
\item draw nonlinear function $\E(y \mid x)$
\item We can discretize the regressors (draw blocks) and fit
         separate OLS regressions within each block
\begin{itemize}
\item gives a better fit than OLS alone
\item may be biased, but bias will be small if you have small
           blocks
\item decreasing the size of the block will increase the variance
           of each estimate, so there is a tradeoff.
\end{itemize}
\end{itemize}

\paragraph{improvements on the basic idea}
\begin{itemize}
\item may (probably) want to impose that $\E(y_i \mid x_i)$ is
         continuous in $x_i$.
\item formally, we can get our model from (assuming p different blocks)
  \[
  y_i = \delta_0 + \gamma_0x_i + \sum_{j=1}^{p-1} (\delta_j + \gamma_jx_i) \ind\{c_j \leq x_i\} + \epsilon_i
  \]
\begin{itemize}
\item continuity implies that
  \[ \delta_0 + \gamma_0 c_l + \sum_{j=1}^l (\delta_j + \gamma_j c_l) = \delta_0 + \gamma_0 c_l +
  \sum_{j=1}^{l+1} (\delta_j + \gamma_l c_l)\] for $l = 1,...,p-1$, which implies
  that $\delta_j = - \gamma_j c_j$ for all $l$
\item plugging this in to the initial equation gives
  \[
  y_i = \delta_0 + \gamma_0x_i + \sum_{j=1}^{p-1} \gamma_j(x_i-c_j) \ind\{c_j \leq x_i\} + \epsilon_i
  \]
\end{itemize}
\item we can estimate the model by regressing $y_i$ on a constant,
  $x_i$, and the variables $(x_i - c_j) \ind\{c_j \leq x_i\}$
\end{itemize}

\paragraph{another improvement}
\begin{itemize}
\item We can combine this idea with the polynomial regression we
         looked at earlier
\begin{itemize}
\item We could fit a quadratic or cubic polynomial in each block
           instead of a straight line
\item now we could impose that the function is continuous and
           continuously differentiable
\begin{itemize}
\item requires additional restrictions on the coefficients
\item no longer is OLS, but gives a similar estimator
\end{itemize}
\end{itemize}
\item approach is called ``splines''
\item formal discussion is beyond the scope of this class
\end{itemize}

\paragraph{local linear models}
\begin{itemize}
\item draw scatter plot again
\item instead of setting up boxes first, we might just try to
        estimate the regression model at this point
\item create a box around the point and fit
\item could do the same thing at each possible point and build up
        the conditional expectation that way
\item Obviously, could be improved by using a weighting scheme
\item example of ``local linear regression''
\item also beyond the scope of the class
\end{itemize}

\paragraph{common themes and discussion}
\begin{itemize}
\item Splines are an example of a ``semi-parametric'' estimator
\begin{itemize}
\item trying to estimate an unspecified function, but using a
          finite number of parameters to do so
\end{itemize}
\item Local linear regression is an example of a nonparametric estimator
\begin{itemize}
\item estimating the unknown function directly
\item different coefficient values for each possible value of the
          function
\end{itemize}
\item For both, the coefficients themselves aren't very useful
\begin{itemize}
\item want to interpret changes in the expected value given
          chagnges in the coefficient
\item can derive them from the unknown coefficients (like we saw
          with the binary regressor case
\item use tools like the delta method to calculate standard errors.
\end{itemize}
\end{itemize}

\subsection{model selection}
\subsection{problems in modeling}

\paragraph{Omitted variables}
\begin{itemize}
\item We've assumed so far that $Y = X\beta + \vep$ with $\E(\vep \mid X) = 0$.
\item Suppose that the true model is
  \[Y = X\beta + Z\alpha + \vep\] with $\E(\vep \mid X, Z) = 0$
\begin{itemize}
\item but you regress $Y$ on $X$ (and not $Z$)
\item e.g. $Y$ is income, $X$ is education and experience, and
          $Z$ is some numeric measure of ability.
\item maybe $Z$ is unobserved
\item maybe there are other reasons you omit it (ignorance, possibly)
\end{itemize}
\item Is the OLS estimator consistent for $\beta$?
\end{itemize}

\paragraph{mathematical consequences}
\begin{itemize}
\item Rewrite the true relationship
  \[ Y = X\beta + (Z\alpha + \vep) \equiv X\beta + u \]
\item we know that (under reasonable assumptions ensuring LLNs)
  \begin{align*}
    \betah - \beta
    &= (n^{-1} X'X)^{-1}(n^{-1} X'u) \\
    &\to (\E(x_{i}x_{i}'))^{-1} \lim n^{-1} \sum_{i=1}^{n} x_{i} \E(u_{i} \mid X)
  \end{align*}
\item \textbf{But} we can't assume that $\E(u_i \mid X) = 0$ (a.s.)
\begin{itemize}
\item $\E(u_{i} \mid X) = \E(\alpha z_{i} \mid X) + \E(\vep_{i} \mid X)$
\begin{itemize}
\item The second term is zero since (by the L.I.E.)
  \[\E(\vep_{i} \mid X) = \E(\E(\vep_{i}\mid X, Z) \mid X) = 0\]
\item The first term is zero only if
\begin{itemize}
\item $\alpha = 0$
\item $\E(x_{i} z_{i}) = 0$
\end{itemize}
\end{itemize}
\end{itemize}
\item Think back to our example
\begin{itemize}
\item $\E(ability \mid education, experience)$ is probably not zero
\item $\alpha = 0$ only if ability doesn't affect income, after
           taking education and experience into account
\item both are probably not zero
\item OLS on just $X$ is going to give you a biased and
           inconsistent estimate.
\end{itemize}
\end{itemize}

\paragraph{remedies}

\paragraph{include the variable!}
\begin{itemize}
\item If you observe $Z$, it's usually a good idea to include it
          in the regression.
\begin{itemize}
\item a caveat with small sample sizes: when $\alpha$ is close
            to zero, this can increase the variance of your estimator
            enough that including the variable is a bad idea
\item when $\alpha$ is close to zero, the bias is small too.
\end{itemize}
\item we'll discuss ways to determine whether or not you should
          include a variable soon
\end{itemize}

\paragraph{proxies -- see \citet[Section 12.5]{Gre12}}
\begin{itemize}
\item What if you can't include the variable?
\begin{itemize}
\item ability might be impossible to directly observe, for
            example.
\end{itemize}
\item Suppose you included a different variable that is a measure
          of the missing variable
\begin{itemize}
\item A test score instead of ability
\item Call that variable $W$; moreover, suppose that
            \[W = \gamma_0 + Z \gamma_1 + \delta\]
            so $W$ is unrelated to the $X$ variables
\end{itemize}
\item What happens if you regress $Y$ on $X$ and $W$?
\end{itemize}
\begin{itemize}

\item mathematical representation of proxies
\begin{itemize}
\item We can solve for $Z$ in the previous relationship:
  \[ Z = (W - \gamma_0 - \delta ) / \gamma_1 \]
\item We can rewrite the true relationship as
  \begin{align*} 
    Y &= X\beta + Z\alpha + \vep \\
    &= \beta_0 + X\beta + (W - \gamma_0 - \delta) \alpha/\gamma_1 \\
    &= (\beta_0 - \gamma_0\alpha/\gamma_1) + X\beta + W \alpha/\gamma_1 + (\vep - \alpha/\gamma_1\delta)
  \end{align*}
\item If $\E(\delta \mid X, W) = 0$ (which may be reasonable), we
           can use OLS to get consistent estimates of
\begin{itemize}
\item $\beta-\gamma_0\alpha/\gamma_1$ (the new intercept)
\item $\beta$
\item $\alpha/\gamma_1$ (the slope on W)
\end{itemize}
\item In our setup, we only care about $\beta$, and we can
           estimate it consistently by OLS (and unbiased, too).
\item The same concept applies with more than one variable
\begin{itemize}
\item Note that $\alpha/\gamma_1$ is complicated
\item Will be more complicated with more variables
\item In particular, it can have a \textbf{different sign} than $\alpha$.
\end{itemize}
\end{itemize}
\end{itemize}

\paragraph{Inclusion of too many variables}
\begin{itemize}
\item We saw this earlier in the lecture
\item If we include variables that have population coefficients equal
        to zero, the variance of our estimator increases and the model
        is less accurate
\item The coefficients are still unbiased, though.
\end{itemize}

\subsection{choosing a model}

\paragraph{$R^2$}
\begin{itemize}
\item We know that $R^2$ increases if we include more regressors:
  \[R^2 = \frac{SSR}{SST} = 1 - \frac{\veph'\veph}{y'M_0y}\]
  with $M_0 = I - n^{-1} \i \i'$
\begin{itemize}
\item If we add regressors, the numerator,
  $\veph'\veph$, decreases.
\end{itemize}
\item motivates adjusted $R^2$:
  \[ \bar{R}^2 = 1 - \frac{n-1}{n-K}(1-R^2)\]
\begin{itemize}
\item if we add a regressor, $K$ increases too, so the decrease in
  $R^2$ is offset.
\end{itemize}
\end{itemize}

\paragraph{Information criteria}
\begin{itemize}
\item In general, it's nice to have a somewhat automatic approach.
\item Suppose you have a set of regressors you think is plausible,
  $x_1,...,x_p$
\item Is there a general (ie popularly accepted) procedure to get a
        set of regressors to apply OLS?
\item In practice, both the AIC and BIC amount to increasing the
        penalty term on $\bar{R}^2$ (they have more formal
        derivations, but we don't have time to go through them
\item AIC :: $s^2_y (1 - R^2) e^{2K/n}$
\item BIC :: $s^2_y (1 - R^2) n^{2K/n}$
\item note :: \underline{smaller} values mean that the model is a better fit
\end{itemize}

\paragraph{use}
\begin{itemize}
\item presentation
\begin{itemize}
\item usually these are presented in logs (so the numbers don't get
           too large)
\item AIC is often reported as $\ln(\veph'\veph/n) + 2K/n$
\item BIC is often reported as $\ln(\veph'\veph/n) + (K/n) \ln n$
\item read the help of the software you're using to be sure
\end{itemize}
\item workflow
\begin{itemize}
\item calculate the AIC (or BIC) for all of the models you're considering.
\item choose the one that minimizes the criterion.
\item these are commonly used in timeseries/macro, less so (that
           I'm aware of) in cross-section/micro
\end{itemize}
\end{itemize}

\paragraph{homework \textbf{:hw:}}
\begin{itemize}
\item Consider this situation: you are interested in the effect of a
        variable $X_1$ (say, education level) on $Y$ (say, earnings),
        so you want to test whether the coefficient $\beta_1$ is
        significant in the equation
        \[ Y_{i} = \beta_0 + \beta_1 X_{i1} + \vep_i. \]
        But, you don't know whether or not you should include a second
        regressor, $X_2$ (maybe some measure of ability, like a test
        score).

        Three of the approaches we've discussed are:
\begin{enumerate}
\item regress $Y$ on $X_1$ and test for significance.
\item regress $Y$ on $X_1$ and $X_2$ and test $\beta_1$ for
           significance.
\item regress $Y$ on $X_1$ and $X_2$ and test $\beta_2$ for
           significance.
\begin{itemize}
\item If $\beta_2$ is significant, test $\beta_1$ for
             significance.
\item If $\beta_2$ is \textbf{not} significant, regress $Y$ on only
             $X_1$ and test $\beta_1$ for significance in that
             model.
\end{itemize}
\end{enumerate}
\item Please design a monte carlo experiment in R that would let you
        study how well each of these approaches would work.
\end{itemize}

\subsection{Shrinkage}
\begin{itemize}
\item Why is having irrelevant regressors bad?
\begin{itemize}
\item increases the variance of our estimators
\end{itemize}
\item One approach: drop regressors
\begin{itemize}
\item sets the coefficients on some of the regressors equal to zero
\item introduces bias
\end{itemize}
\item Another approach
\begin{itemize}
\item shift the coefficients on all of the regressors closer to zero
\item decreases the variance on the coefficients
\item introduces bias as well
\end{itemize}
\end{itemize}

\paragraph{intuition}
\begin{itemize}
\item For OLS:
\begin{itemize}
\item $\betah = (X'X)^{-1}X'Y$
\item $\E(\betah \mid X) = (X'X)^{-1}X'X\beta$
\item $\var(\betah \mid X) = \sigma^2(X'X)^{-1} X'X (X'X)^{-1}$
\end{itemize}
\item Another possibility
\begin{itemize}
\item $\betah_2 = (X'X + V)^{-1}X'Y$
\begin{itemize}
\item $V$ is some positive definite matrix
\item Assume $V$ is a function of $X$
\item Say $V = X'X$ (for example); then $\betah_2 = \betah / 2$
\end{itemize}
\item $\E(\betah_2 \mid X) = (X'X + V)^{-1}X'X\beta$
\begin{itemize}
\item So the estimator is biased
\item in our example, $\E(\betah \mid X) = \beta/2$
\end{itemize}
\item $\var(\betah_2 \mid X) = \sigma^2 (X'X + V)^{-1} X'X (X'X + V)^{-1}$
\begin{itemize}
\item in our example, $\var(\betah_2 \mid X) = (\sigma^2/4) (X'X)^{-1}$
\end{itemize}
\end{itemize}
\end{itemize}

\paragraph{variance reduction of this estimator}
\begin{itemize}
\item We can see that $\var(\betah_2 \mid X) \leq \var(\betah \mid X)$:
\begin{itemize}
\item Take inverses and look at difference:
  \begin{align*}
    (\var(\betah_2 \mid X)^{-1} - (\var(\betah \mid X))^{-1}
    &= \sigma^2((X'X + V) (X'X)^{-1} (X'X + V) - X'X) ] \\
    &= \sigma^2 (X'X + V + V(X'X)^{-1} (X'X + V)) \\
    &= \sigma^2 (X'X + 2 V + V(X'X)^{-1}V) 
  \end{align*}
\item This last term is positive definite
\end{itemize}
\end{itemize}

\paragraph{The ridge regression estimator}
\begin{itemize}
\item To get a convenient expression for our estimator, we should center and scale the regressors
\begin{description}
\item[$\Xt$] centered and scaled $X$
\begin{itemize}
\item $X = [1, X_1, X_2, ..., X_{k}]$
\item $\Xt = [1, (X_1 - \i \Xb_1) / \sd(X_1), ...]$
\end{itemize}
\end{description}
\item The ridge regression estimator is
  \[ \betah_{ridge} = (\Xt'\Xt + \l I)^{-1} \Xt'Y\]
\begin{itemize}
\item For uncentered/unscaled regressors, either
\begin{itemize}
\item adjust the formula
\item transform the scaled/centered coefficients
\end{itemize}
\end{itemize}
\end{itemize}

%%% Local Variables:
%%% mode: latex
%%% TeX-master: "../core_econometrics"
%%% End:
