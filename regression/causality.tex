% Copyright (c) 2013–2014, authors of "Core Econometrics;" a
% complete list of authors is available in the file AUTHORS.tex.

% Permission is granted to copy, distribute and/or modify this
% document under the terms of the GNU Free Documentation License,
% Version 1.3 or any later version published by the Free Software
% Foundation; with no Invariant Sections, no Front-Cover Texts, and no
% Back-Cover Texts.  A copy of the license is included in the file
% LICENSE.tex and is also available online at
% <http://www.gnu.org/copyleft/fdl.html>.

\chapter{Estimating causal relationships and dealing with endogeneity}
\label{ch_causality}
Two reviews by \citet{IW07,IW09} expand on the material covered in
this chapter.
\section{How should we think about empirical applications?}

\begin{itemize}[leftmargin=0pt]

\item What are we hoping to do with linear regression?  We probably
  have in mind changing one of the regressors.  Imagine a setting
  where we have data on
  \begin{itemize}
  \item some sort of school-average reading score on a state test
    (which is going to be our dependent variable)
  \item number of teachers per student at several schools
  \item money available per student at these schools
  \end{itemize}
  Conceivably, some policy could change number of teachers per student
  or could change the money available per student.  To know which (if
  either) of these is a good idea, we would need to know something like
  \begin{equation*}
    \E(\text{change in reading score} \mid
       \text{an independent body makes a 1\% change in expenditure})
  \end{equation*}

\item What makes econometrics distintive and interesting is that we
  usually don't get to observe changes in expenditure and changes in
  test scores (although we do sometimes).  What we often have instead
  is sometimes we see levels of expenditure and levels of test scores
  in different schools.

  Moreover, we almost never get to observe changes made by an
  independent body.  We often will only see changes that were made for
  some reason---for example, if a school performed abnormally badly one
  year and was given funding to compensate, it would likely rebound
  next year (with or without the funding) and that rebound would be
  attributed to the funding.

\item When you're starting an empirical research project, probably the
  best thing to do is to think about what experiment you would want to
  design if you could.

  One approach if we want to understand how expenditure affects
  student reading test scores, we'd want to take all of the schools
  \emph{randomly} and \emph{independently} change their funding (leave
  some unchanged), then observe the change in test scores the next
  year.

  Next, think hard about how closely the process that created your
  data matches that experiment.

\item This can be an iterative process (though it is suboptimal to go
  back and forth to the data).  And you can't always think of an
  experiment.  If you can not think of an experiment, the question
  might not be an empirical question.  Sometimes this happens when you
  think about changing preference parameters, which is often
  interesting in the context of an economic model, but is more or less
  impossible in real life.

\end{itemize}

\section{Relaxing the complete exogeneity assumption}

\begin{itemize}[leftmargin=0pt]

\item The assumption that all of the regressors are exogenous is
  usually unjustified. Suppose that we have a DGP
  \begin{equation*}
    y_i = \beta_0 + \beta_1 \text{education}_i
    + \beta_2 \text{ability}_i + \varepsilon_i
  \end{equation*}
  where education is observed but ability is not, and they're
  potentially correlated. We want to estimate $\beta_1$. Here we
  assume that
  \begin{equation*}
    \E(\vep_i \mid \text{education}_i, \text{ability}_i) = 0 \ a.s.
  \end{equation*}
  The regression of $y_i$ on $\text{education}_i$ would then be biased
  for $\beta_1$.

  A common approach is to add variables related to ability to the
  regression and then estimate the model
  \begin{equation*}
    y_i = \beta_0 + \beta_1 \text{education}_i
    + \gamma'\text{controls}_i
    + \underbrace{\beta_2 \text{ability}_i - 
                  \gamma'\text{controls}_i + \varepsilon_i}_{u_i}
  \end{equation*}
  where $u_t$ is the unobserved error in the regression of $y_i$ on
  education and the additional controls. Obviously, if
  \begin{equation*}
    \gamma'\text{controls}_i = \beta_2 \text{ability}_i
  \end{equation*}
  almost surely for a particular value of $\gamma$, there are no
  problems. But that is very unlikely.

  A more likely situation is that the expected value of
  $\text{ability}_i$ is independent of $\text{education}_i$, once we
  condition on the controls. Mathematically, this means that
  \begin{align*}
    \E(\text{ability}_i \mid \text{education}_i, \text{controls}_i)
    &= \E(\text{ability}_i \mid \text{controls}_i) && a.s. \\
    &= \alpha_0 + \alpha_1'\text{controls}_i && a.s.
  \end{align*}
  where we impose linearity in the second equation, but that the last
  conditional expectation is not zero.

  Under this setup, we can work out the parameter values identified by
  OLS. Assume the controls are univariate (for simplicity) and that
  $\beta_0 = 0$, so OLS identifies the parameters
  \newcommand{\ab}{\text{ab}}
  \newcommand{\ed}{\text{ed}}
  \newcommand{\co}{\text{c}}
  \begin{align*}
    \begin{pmatrix}
      \beta_1^* \\ \gamma^*
    \end{pmatrix} &=
    \begin{pmatrix}
      \E \ed_i^2 & \E \ed_i \co_i \\ \E \ed_i \co_i & \E \co_i^2
    \end{pmatrix}^{-1}
    \begin{pmatrix}
      \E \ed_i y_i \\ \E \co_i y_i
    \end{pmatrix}
    \\ &= \frac{1}{\E \ed_i^2 \E \co_i^2 - (\E \ed_i \co_i)^2}
    \begin{pmatrix}
      \E \co_i^2 & -\E \ed_i \co_i \\ -\E \ed_i \co_i & \E \ed_i^2
    \end{pmatrix}
    \begin{pmatrix}
      \E \ed_i (\beta_1 \ed_i + \beta_2 \ab_i + \vep_i) \\
      \E \co_i (\beta_1 \ed_i + \beta_2 \ab_i + \vep_i)
    \end{pmatrix} \\
    &= \frac{1}{\E \ed_i^2 \E \co_i^2 - (\E \ed_i \co_i)^2}
    \begin{pmatrix}
      \E \co_i^2 & -\E \ed_i \co_i \\ -\E \ed_i \co_i & \E \ed_i^2
    \end{pmatrix}
    \begin{pmatrix}
      \beta_1 \E \ed_i^2 + \beta_2 \E \ed_i \ab_i \\
      \beta_1 \E \ed_i \co_i + \beta_2 \E \co_i \ab_i
    \end{pmatrix}
  \end{align*}
  where we have used the assumption that $\ed_i$ and $\vep_i$ are
  uncorrelated and an implicit assumption that $\co_i$ and $\vep_i$
  are uncorrelated. ($\vep_i$ is being treated here as irreducible
  error; otherwise include the part correlated with $\co_i$ in with
  $\ab_i$.)

  We can then see that
  \begin{equation*}
    \beta_i^* = \beta_1 + \beta_2 
    \frac{\E \co_i^2 \E \ed_i \ab_i - \E\ed_i \co_i \E \co_i \ab_i}%
    {\E \ed_i^2 \E \co_i^2 - (\E \ed_i \co_i)^2},
  \end{equation*}
  so $\beta_i^* = \beta_1$ only if
  \begin{align*}
    0
    &= \E \ed_i \ab_i - \E\ed_i \co_i \E \co_i \ab_i / \E \co_i^2 \\
    &= \E\big( \ab_i \big(\ed_i - \co_i (\E \co_i^2)^{-1}\E(\co_i
    \ed_i) \big) \big).
  \end{align*}
  Now, notice that
  \begin{equation*}
    \ed_i - \co_i (\E \co_i^2)^{-1}\E(\co_i \ed_i)
  \end{equation*}
  is the component of $\ed_i$ orthogonal to $\co_i$ and so is
  uncorrelated with $\ab_i$, ensuring that $\beta_1^* = \beta_1$
  (elaborate).

\item This argument means that OLS is able to estimate $\beta_1$ even
  though the other controls are necessarily endogenous.

\item This would be a good place to explain how the same issue arises
  in true experiments.

\end{itemize}

\section{Using too many regressors}

This would be a good place to discuss having regressors that interrupt
the causal effect. An example: suppose that you want to know the
effect that taking a college test prep class has on college admissions
(at the individual level). One might imagine including SAT scores as a
regressor, if they were available. But the main mechanism that we'd
expect this sort of class to affect college admissions is through test
scores, so ``controlling for'' test scores probably doesn't make sense.

\section{Instrumental variable estimators}

\begin{itemize}[leftmargin=0pt]

\item Sometimes, even if the important variables of $x_i$ are not
  plausibly exogenous, there is some degree to which it is influenced
  by exogenous factors. (Horrible sentence.) Again, assume that
  \begin{equation*}
    Y = X\beta + \vep
  \end{equation*}
  where $X$ and $\vep$ are potentially correlated, but now suppose
  that there exists an $l \times 1$ random vector $z_i$ with $l \geq
  k$ such that
  \begin{equation*}
    \E(\vep_i \mid z_i) = 0
  \end{equation*}
  and $\E z_i z_i'$, $\E x_i x_i'$, and $\E z_i x_i'$ all have full
  rank. This gives the moment conditions
  \begin{equation*}
    \E (y_i - x_i'\beta) z_i = 0
  \end{equation*}
  or
  \begin{equation}\label{eq:2}
    \E (y_i z_i) = \E (z_i x_i')\beta
  \end{equation}
  which can be used as the foundation of an estimator.

\item If $k = l$, the assumptions listed above imply that we can
  solve~\eqref{eq:2} for $\beta$ directly, and
  \begin{equation*}
    \beta = \E (z_i x_i')^{-1} \E (y_i z_i).
  \end{equation*}
  The natural estimator for $\beta$ is then
  \begin{equation*}
    \betah = (Z'X)^{-1} Z'Y.
  \end{equation*}
  It is easy to see that this estimator is consistent and
  asymptotically normal as long as LLNs and CLTs hold:
  \begin{align*}
    \betah
    &= \beta + (Z'X)^{-1}Z'\vep \\
    &= \beta + \Big((1/n) \sum_{i=1}^n z_i x_i'\Big)^{-1}
    (1/n) \sum_{i=1}^n z_i \vep_i \\
    &\to^p \beta
  \end{align*}
  and
  \begin{align*}
    \sqrt{n}(\betah - \beta)
    &= \Big((1/n) \sum_{i=1}^n z_i x_i'\Big)^{-1}
    (1/\sqrt{n}) \sum_{i=1}^n z_i \vep_i \\
    &\to^d N\big(0, (\E z_i x_i')^{-1} \E \vep_i^2 z_i z_i' (\E x_i z_i')^{-1}\big).
  \end{align*}

  The expectations can be estimated with their sample analogues as
  usual:
  \begin{equation*}
    (1/n) \sum_{i=1}^n z_i x_i' \to^p \E z_i x_i'
  \end{equation*}
  and
  \begin{equation*}
    (1/n) \sum_{i=1}^n \veph_i^2 z_i z_i' \to^p \E \vep_i^2 z_i z_i',
  \end{equation*}
  where $\veph_i = y_i - x_i'\betah$.

\item If $k > l$, then $Z'X$ will not be square or invertible. But we can
  derive a consistent estimator as follows. The sample analogue
  of~\eqref{eq:2} is still
  \begin{equation*}
    (1/n) Z'X = (1/n) Z'Y\betah
  \end{equation*}
  and premultiplying both sides by $n X'Z (Z'Z)^{-1}$ gives
  \begin{equation*}
    X'Z (Z'Z)^{-1} Z'X = X'Z (Z'Z)^{-1} Z'Y\betah
  \end{equation*}
  which can now be solved for $\betah$,
  \begin{equation*}
    \betah = (X'Z (Z'Z)^{-1} Z'X)^{-1} X'Z (Z'Z)^{-1} Z'Y,
  \end{equation*}
  which can be shown to be consistent and asymptotically normal just
  as above.

  Note that $Z (Z'Z)^{-1} Z'Y$ is the projection of $Y$ onto the
  column space of $Z$ and $Z (Z'Z)^{-1} Z'X$ is $X$'s projection.

\item Remember that the variance of the OLS estimator decreases as the
  dispersion of the regressors increases. Since we can think of
  two-stage least squares as the OLS regression of the component of on
  the component of $X$ driven by $Z$, which is going to have much
  smaller dispersion than $X$ itself, it is intuitive that the
  variance of the two-stage least squares estimator is larger than
  OLS's.

  When there is very little correlation between $z_i$ and $x_i$, more
  serious problems arise. This is called the ``weak instruments problem.''
  See \citet{SS97}, \citet{SWY02}, and \citet{SY05}.

\item When we have more instruments than regressors, we can check
  whether the instruments are internally consistent. Observe that we
  could use any $k$ of the instruments to generate a consistent
  estimator of $\beta$, so it is a useful check to verify that all of
  these identified $\beta$s are actually the same.

  If all of the exogeneity assumptions hold, then 
  \begin{equation*}
    (1/n) \sum_{i=1}^n z_i \veph_i \to^p 0
  \end{equation*}
  (since $\veph_i \to \vep_i$ by consistency) and
  \begin{equation*}
    (1/\sqrt{n}) \sum_{i=1}^n z_i \veph_i \to^d N(0, V)
  \end{equation*}
  with $V = \E \vep_i^2 z_i z_i'$. So then
  \begin{equation*}
    (1/n) \Big(\sum_{i=1}^n \veph_i z_i'\Big) \Vh^{-1}
    \Big(\sum_{i=1}^n z_i \veph_i\Big)
  \end{equation*}
  converges in distribution to a chi-square random variable, under the
  null.  What's interesting is that this chi-square has $l-k$ degrees
  of freedom and not $l$ (need to add an explanation; I don't like the
  one I have right now).

\item Another check is to verify that the instruments are uncorrelated
  with variables similar to what you're concerned about in the error
  term. This statement needs to be elaborated on, obviously.

\end{itemize}

%%% Local Variables: 
%%% mode: latex
%%% TeX-master: "../core_econometrics"
%%% End: 

%  LocalWords:  endogeneity exogeneity regressors DGP OLS univariate
%  LocalWords:  regressor CLTs
