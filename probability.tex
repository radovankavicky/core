\documentclass[nohyper]{external/tufte-handout}
% The hyperref is loaded in the preamble with additional arguments to
% avoid some xelatex warnings.

\title[Probability overview]%
{Probability Overview \\
  Core Econometrics Notes, part 1}
% Copyright © 2013, authors of the "Core Econometrics;" a
% complete list of authors is available in the file AUTHORS.tex.

% Permission is granted to copy, distribute and/or modify this
% document under the terms of the GNU Free Documentation License,
% Version 1.3 or any later version published by the Free Software
% Foundation; with no Invariant Sections, no Front-Cover Texts, and no
% Back-Cover Texts.  A copy of the license is included in the file
% LICENSE.tex and is also available online at
% <http://www.gnu.org/copyleft/fdl.html>.

\newcommand{\version}{0.7.1}
\newcommand{\releasedate}{10 Dec. 2013}

%%% Local Variables:
%%% mode: latex
%%% TeX-master: "core_econometrics"
%%% End:

\author{Gray Calhoun} 
% (This comment is repeated in the Makefile)
% I'm still not sure the best way to do author information; I'm much
% more concerned in the long run about how different attributation
% styles would make someone more or less likely to contribute to an
% existing text or to license an existing draft.  For now, there's
% only one author, so I'll put myself as the author.  If someone else
% contributes any edits, etc., I'll change it to {Gray Calhoun and
% EFLP}.  If anyone wants to contribute a lot of original material and
% wants named authorship, please email the mailing list so we can
% discuss merging projects.

\usepackage{amssymb,amsmath,amsthm,verbatim}
\usepackage{fontspec,unicode-math,xltxtra,xunicode,booktabs}
\setromanfont[Ligatures=TeX]{TeX Gyre Pagella}
\setsansfont[Ligatures=TeX,Scale=MatchLowercase]{TeX Gyre Heros}
% \setmonofont[Scale=MatchLowercase]{Inconsolata}
\setmathfont{Asana-Math}

\frenchspacing
\setcounter{secnumdepth}{1}
\setcounter{tocdepth}{1}
\renewcommand\bibname{}
\renewcommand\refname{}
\renewcommand\contentsname{}
\bibliographystyle{abbrvnat}
\setcitestyle{round}
\newcommand{\email}[1]{\href{mailto:#1}{\nolinkurl{#1}}}
\newcommand{\homepage}{\url{http://www.econometricslibrary.org}}
\newcommand{\maillist}{\email{econometricslibrary@librelist.com}}
\newcommand{\bugtrack}%
{\url{https://github.com/EconometricsLibrary/CoreEconometricsText/issues}}

% Workaround for bugs in the tufte-latex class
\renewcommand\smallcapsspacing[1]{{\addfontfeature{LetterSpace = 8}\scshape#1}}
\renewcommand\allcapsspacing[1]{{\addfontfeature{LetterSpace = 15}#1}}
% Getting waringings from latexmk with default tufte-latex hyperref
\usepackage[unicode,pdfencoding=auto,hyperfootnotes=false,hidelinks]{hyperref}

\newcommand{\BibTeX}{Bib\!\TeX}
\newcommand{\pvalue}{\ensuremath{p}-value}
\newcommand{\ftest}{\ensuremath{F}-test}
\newcommand{\ttest}{\ensuremath{t}-test}

% Math shortcuts
\renewcommand{\Pr}{\operatorname{Pr}}

\DeclareMathOperator{\1}{1}
\DeclareMathOperator{\abs}{abs}
\DeclareMathOperator{\avar}{avar}
\DeclareMathOperator{\bias}{bias}
\DeclareMathOperator{\corr}{corr}
\DeclareMathOperator{\cov}{cov}
\DeclareMathOperator{\E}{E}
\DeclareMathOperator{\median}{median}
\DeclareMathOperator{\mse}{mse}
\DeclareMathOperator{\rank}{rank}
\DeclareMathOperator{\range}{range}
\DeclareMathOperator{\sd}{sd}
\DeclareMathOperator{\tr}{tr}
\DeclareMathOperator{\var}{var}

\DeclareMathOperator*{\argmax}{arg\,max}
\DeclareMathOperator*{\argmin}{arg\,min}
\DeclareMathOperator*{\plim}{plim}

\DeclareMathOperator{\binomial}{binomial}
\DeclareMathOperator{\bernoulli}{bernoulli}
\DeclareMathOperator{\invWishart}{inverse\ Wishart}
\DeclareMathOperator{\N}{N}
\DeclareMathOperator{\uniform}{uniform}

\newcommand{\BB}{\ensuremath{\mathbb{B}}}
\newcommand{\NN}{\ensuremath{\mathbb{N}}}
\newcommand{\PP}{\ensuremath{\mathbb{P}}}
\newcommand{\QQ}{\ensuremath{\mathbb{Q}}}
\newcommand{\RR}{\ensuremath{\mathbb{R}}}
\newcommand{\RRᵏ}{\ensuremath{\mathbb{R}ᵏ}}
\newcommand{\RRⁿ}{\ensuremath{\mathbb{R}ⁿ}}
\newcommand{\RRb}{\ensuremath{\bar{\mathbb{R}}}}
\newcommand{\ZZ}{\ensuremath{\mathbb{Z}}}

\newcommand{\Fs}{\ensuremath{\mathcal{F}}}
\newcommand{\Gs}{\ensuremath{\mathcal{G}}}
\newcommand{\Ps}{\ensuremath{\mathcal{P}}}

\newcommand{\ov}[2][1]{\tfrac{#1}{#2}}
\newcommand{\iid}{i.i.d.}

\newcommand{\ep}{\varepsilon}
\newcommand{\eph}{\hat{\varepsilon}}

\newcommand{\ah}{\hat{a}}
\newcommand{\αh}{\hat{α}}
\newcommand{\bh}{\hat{b}}
\newcommand{\βb}{\bar{β}}
\newcommand{\βh}{\hat{β}}
\newcommand{\βt}{\tilde{β}}
\newcommand{\eh}{\hat{e}}
\newcommand{\εb}{\bar{ε}}
\newcommand{\εh}{\hat{ε}}
\newcommand{\εt}{\tilde{ε}}
\newcommand{\ηh}{\hat{η}}
\newcommand{\Fh}{\hat{F}}
\newcommand{\λh}{\hat{λ}}
\newcommand{\μb}{\bar{μ}}
\newcommand{\μh}{\hat{μ}}
\newcommand{\Ωh}{\hat{Ω}}
\newcommand{\Qh}{\hat{Q}}
\newcommand{\Σh}{\hat{Σ}}
\newcommand{\sh}{\hat{s}}
\newcommand{\σh}{\hat{σ}}
\newcommand{\σb}{\bar{σ}}
\newcommand{\θh}{\hat{θ}}
\newcommand{\Vh}{\hat{V}}
\newcommand{\Xb}{\bar{X}}
\newcommand{\Xc}{\mathcal{X}}
\newcommand{\Xh}{\hat{X}}
\newcommand{\Xt}{\tilde{X}}
\newcommand{\Yb}{\bar{Y}}
\newcommand{\Yh}{\hat{Y}}
\newcommand{\Yt}{\tilde{Y}}
\newcommand{\yh}{\hat{y}}

\newcommand{\dx}{\,dx}
\newcommand{\dy}{\,dy}
\newcommand{\dμ}{\,dμ}
\newcommand{\dθ}{\,dθ}
\renewcommand{\dz}{\,dz}

\newtheorem{thm}{Theorem}[section]
\newtheorem{defn}{Definition}[section]
\newtheorem{ex}{Example}[section]
\newtheorem{asmp}{Assumption}[section]


\begin{document}
\maketitle

\bigskip\noindent%
Copyright © 2013, the authors of \textit{Core Econometrics Notes}.
A complete author list is included in Appendix A of this document.

Permission is granted to copy, distribute and/or modify this document
under the terms of the GNU Free Documentation License, Version 1.3 or
any later version published by the Free Software Foundation; with no
Invariant Sections, no Front-Cover Texts, and no Back-Cover Texts.  A
copy of the license is included in Appendix B of this document and is
also available online at \url{http://www.gnu.org/copyleft/fdl.html}.

This text was produced as part of the Econometrics Free Library
Project.  The project's goal is to produce high-quality free and
open access econometrics textbooks and reference material.  More
information about this project is available at its homepage,
\homepage, including information on how
to participate.  This text is typeset in \LaTeX\ and there is a link
to the source of the document at the project's homepage as well.

Please cite this document as
\begin{itemize}
\item[] \bibentry{eflp-core}
\item[] \verbatiminput{CITATION.bib}
\end{itemize}
The \BibTeX\ record is provided for your convenience.
If the year and version are not specified, please see the files
\texttt{README.md} or \texttt{VERSION.tex} for directions on how to
find them.

When you find errors in this text, please email the project mailing
list, \maillist, or log the error in the project issue tracker,
\bugtrack.  It would be helpful if you could refer to the version
number listed above as well.  Thanks!

\addcontentsline{toc}{part}{Table of Contents}
\tableofcontents
\listoftables
\listoffigures

% Copyright (c) 2013, authors of "Core Econometrics;" a
% complete list of authors is available in the file AUTHORS.tex.

% Permission is granted to copy, distribute and/or modify this
% document under the terms of the GNU Free Documentation License,
% Version 1.3 or any later version published by the Free Software
% Foundation; with no Invariant Sections, no Front-Cover Texts, and no
% Back-Cover Texts.  A copy of the license is included in the file
% LICENSE.tex and is also available online at
% <http://www.gnu.org/copyleft/fdl.html>.

\chapter{Random variables, distributions, and densities}

\section{Random variables as we might wish to introduce them to
  undergraduates.}

\begin{itemize}

\item We're trying to capture the idea of a ``hypothetical'' number
  that might take on many different values.  For example, if I were to
  go to Las Vegas next weekend, I might win \$10 while I'm there; I
  might win \$1,000; I might ``win'' --\$5,000; etc.  If we're going
  to talk meaningfully about what might happen if I go to Las Vegas,
  we're going to need a way to match each outcome (here how much money
  I win or lose) with the relative frequency of that outcome.

  The simplest\sidenote{``Simplest'' from a conceptual/intuitive
  standpoint; we'll see that this is far from the simplest way from
  a rigorous mathematical standpoint.}  way to do this is through a
  probability density function and, as in most of probability, the
  easiest way to start is to work with discrete events (in which case
  the ``density'' function us usually called a ``probability mass
  function'').  Suppose that we're interested in a countable number of
  possible outcomes on the real line; think of handful of dice and
  adding up all of the numbers, so the possible values that the sum
  can take on are in some set $S = \{x_i ∣ i ∈ \ZZ\}$, where \ZZ\ is
  the set of integers.  Then we can define a probability density
  function $f: \RR → \RR$ as the function such that
  \begin{enumerate}
  \item $f(x) ≥ 0$ for all $x ∈ \RR$,
  \item $f(x) = 0$ for all $x ∉ S$, and
  \item $∑_{x ∈ S} f(x_i) = 1$.
  \end{enumerate}

  So now we can think of a random variable $X$ that takes on values in
  the set $\mathcal{X}$ with probabilities given by $f$.  And we can
  build a probability function from a density $f$ for the random
  variable $X$ by asserting that, for any $A ⊂ S$,
  \begin{equation*}
    \Pr[ X ∈ A ] = ∑_{s ∈ A} f(s)
  \end{equation*}

\item If we want to let $X$ take on the entire range of values in \RR\
  we can proceed the same way.  Again define the density function $f:
  \RR → \RR$ such that
  \begin{enumerate}
  \item $f(x) ≥ 0$ for all $x ∈ \RR$ and
  \item $∫_{-∞}^∞ f(x) \dx = 1$.
  \end{enumerate}
  (The set $S ⊂ \RR$ s.t. $f(x) > 0$ for all $x ∈ S$ and $f(x) = 0$
  for all $x ∉ S$ is called the \emph{support} of the r.v. $X$.)  We
  can again build a probability function from the density function.
  For any closed interval $[a, b]$,\sidenote{Remember, $[a,b] = \{x ∈
  \RR ∣ a ≤ x ≤ b\}$.} we can define
  \begin{equation*}
    \Pr[ X ∈ [a,b] ] = ∫_a^b f(x) \dx,
  \end{equation*}
  and for any set $A$ that can be written as the union of a countable
  number of disjoint intervals, $A = ⋃_i [a_i, b_i]$, where $⋂_i [a_i,
  b_i] = ∅$, we can define
  \begin{equation*}
    \Pr[ X ∈ A ] = ∑_i \Pr[ X ∈ [a_i, b_i] ] = ∑_i ∫_{a_i}^{b_i} f(x) \dx
  \end{equation*}

\item It's natural to want to go even further, and define for any set
  $A ⊂ \RR$
  \begin{equation*}
    \Pr[ X ∈ A ] = ∫_A f(x) \dx ≡ ∫_{-∞}^∞ \1\{x ∈ A\} f(x) \dx,
  \end{equation*}
  but it turns out that this integral is not well-defined for every $A
  ⊂ \RR$ (and, consequently, it's possible to construct functions $f$
  that can't be used as densities, a fact that we glossed over
  earlier).  However, it should be obvious that we can construct a
  self-contained set of subsets of \RR, call it \BB, so that we can
  define $\Pr[X ∈ B] = ∫_B f(x) \dx$ for all $B ∈ \BB$.

  We can motivate some of properties that \BB\ needs from properties
  that seem necessary for the probability function $\Pr$.

  \begin{enumerate}
  \item We need $\Pr[X ∈ \RR] = ∫_{-∞}^∞ f(x) \dx = 1$, so \RR\ must be
    in \BB.
  \item If $\Pr[X ∈ B]$ is well defined then $\Pr[X ∉ B]$ should be as
    well, and we should have $\Pr[X ∉ B] = 1 - \Pr[X ∈ B]$.  Since $X
    ∉ B$ is equivalent to $X ∈ B^c$, $B^c$ must be in \BB\ whenever
    $B$ is.
  \item If $B₁$, $B₂$, $B₃$,… make up a countable sequence of sets and
    we can define $\Pr[X ∈ B_i]$ for each of them, we should be able
    to define $\Pr[X ∈ ⋃_i B_i]$ and $\Pr[X ∈ ⋂_i B_i]$.  So \BB\ must
    be closed under countable unions and intersections.
  \end{enumerate}

  Any set of sets \BB\ that satisfies those three properties is called
  a \emph{sigma-algebra} (or \emph{σ-algebra}).  For subsets of the
  real line, there's an especially useful and common σ-algebra, the
  \emph{Borel σ-algebra},which is defined as the smallest σ-algebra
  that has all of the intervals as elements.

\item The integral we're interested is a little different than the
  usual Riemann integral that you've seen in introductory calculus
  classes, but not in any way that's going to matter for our purposes.
  The key difference is that it works by taking different points in
  the function's height and mapping them back to Borel sets in the
  function's domain, while the Riemann integral works by taking
  different points in the function's domain and mapping them to the
  height.\sidenote{Remember, to get the Riemann integral of a function
  $g: \RR → \RR$ over $[a,b]$, first define an increasingly fine
  partition of $[a,b]$, $\{x_{i,n}, i = 1,…n, n = 1,2,…\}$ such that
  $a = x_{1,n}$, $b = x_{n,n}$, $x_{i,n} < x_{i+1,n}$ for all $i$
  and $\max_{i=1,…,n-1} |x_{i+1,n} - x_{i,n}| → 0$ as $n → ∞$.  If
  we let $M_i = \max_{x ∈ [x_{i,n}, x_{i+1,n}]} g(x)$ and $m_i =
  \min_{x ∈ [x_{i,n}, x_{i+1,n}]} g(x)$, then the Riemann integral
  exists if
  \begin{multline*}
    \lim_{n → ∞} ∑_{i=1}^{n-1} M_i (x_{i+1,n} - x_{i,n}) \\
    = \lim_{n → ∞} ∑_{i=1}^{n-1} m_i (x_{i+1,n} - x_{i,n})
  \end{multline*}
  for any partition $\{x_{n,i}\}$.} 
  So, if $g$ is any step function that can be written as a step
  function of Borel sets,
  \begin{equation*}
    g(x) = ∑_{i=1}^∞ c_i \1\{x ∈ B_i\}
  \end{equation*}
  where $B_i ∈ \BB$, we can define the Lebesgue integral as
  \begin{equation*}
    ∫ g dμ = ∑_{i=1}^∞ c_i μ(B_i),
  \end{equation*}
  where $μ(B_i)$ is the length of the set $B_i$.\sidenote{I should
  maybe add a definition or an Appendix on Lebesgue measure, but
  that will probably happen later.}  The integrals of functions that
  are not step functions can be evaluated as the limit of a sequence
  of integrals of functions that are step functions.  For example, if
  $g_n → f$ as $n → ∞$ where each $g_n = ∑_i c_{i,n} \1\{x ∈ B_{i,n}\}$
  then we often can show that
  $∫ f \dμ = \lim_{n → ∞} ∑_i c_{i,n} μ(B_{i,n})$ (I say ``often''
  because there are some conditions on the sequence $\{g_n\}$ that we
  haven't mentioned that need to be satisfied for this to hold)

  For all of this to work, we need to be able to find a Borel set in
  the domain of the function for any set of points in the range;
  i.e. for the step function above, we have, for any point $x$,
  \begin{equation*}
    g^{-1}(\{x\}) =
    \begin{cases}
      B_i & x = c_i \\
      ∅ & x ∉ \{c₁, c₂,…\}
    \end{cases}
  \end{equation*}
  and, for any $B ∈ \BB$, $g^{-1}(B) = ⋃_{i ∈ I} B_i$ where $I = \{i ∣
  c_i ∈ B\}$.\sidenote{You should prove on your own that this means
  that $\{g^{-1}(B) ∣ B ∈ \BB\}$ is a σ-algebra and is a subset of
  \BB.}

\item The key property, then, for this integral to work is that
  $g^{-1}(\BB)$ itself be a σ-algebra that is contained in \BB.  Such
  a function is called \emph{Borel-measurable}.  We're going to
  restrict our focus to densities $f$ that are Lebesgue measurable, so
  we can always write probabilities as
  \begin{equation*}
    \Pr[X ∈ B] = ∫_B f(x) dx
  \end{equation*}
  for Borel-sets $B$.  We won't be able to define these probabilities
  for non-Borel sets, but that's fine; in this case, every function
  that we'd want to work with is Lebesgue-measurable and every set
  that we'd want to work with is a Borel set.\sidenote{Add example of
  unmeasurable set here….}

  If we're clever, we can define probabilities for discrete random
  variables the same way.  Define the function $δ$, known as Dirac's
  δ-function, as $δ = \lim_{n → ∞} δ_n$ where
  \begin{equation*}
    δ_n(x) =
    \begin{cases}
      n + n² x & - \tfrac{1}{n} ≤ x ≤ 0 \\
      n - n² x & 0 < x ≤ \tfrac{1}{n} \\
      0        & \text{otherwise},
    \end{cases}
  \end{equation*}
  so each $δ_n$ is a very narrow, very tall spike at 0 with total area
  1, and as $n$ grows the spike gets narrower and higher.  The limit
  $δ$ is an infinitely tall, infinitely narrow spike at 0 with total
  area 1.

  Under Lebesgue-measure, we can now define the density of a
  Bernoulli(1/2) random variable $b$ as
  \begin{equation*}
    f_b(x) = \tfrac{1}{2} δ(x) + \tfrac{1}{2} δ(x - 1).
  \end{equation*}
  This lets us write $\Pr[b ∈ B] = ∫_B f_b(x) \dx$ for any Borel-set
  $B$, which is what we want.\sidenote{You should work through all of
  the steps necessary to prove this to yourself.}

\item Two key points to emphasize and remember: if the usual Riemann
  integral of a function exists, its value is the same as the Lebesgue
  integral. And it is very difficult to accidentally write down a
  function on \RRⁿ\ that is not Lebesgue measurable, so don't worry
  that you will.

\item The probabilities for a given random variable $X$ can be
  summarized by its \emph{cumulative distribution function} (CDF)
  which is typically written as $F_X$ and is defined as
  \begin{equation*}
    F_x(c) ≡ \Pr[X ≤ c] = \Pr[X ∈ (∞, c]]
  \end{equation*}
  for any $c ∈ \RR$.  The CDF of a random variable always exists and
  satisfies
  \begin{enumerate}
  \item $\lim_{x → -∞} F_X(x) = 0$
  \item $\lim_{x → ∞} F_X(x) = 1$
  \item $F_X$ is non-decreasing in $x$.
  \item $F_X$ is right-continuous.  For every number $x₀$,
    $\lim_{x ↓ x₀} F(x) = F(x₀)$.
  \end{enumerate}

  You should verify the property $\Pr[X ∈ (a, b]] = F_X(b) - F_X(a)$
  as homework.

\item Obviously we can have vectors of random variables (often called
  \emph{random vectors}).  These vectors have density functions too,
  which we can build up sequentially.  If $X₁ ∼ f₁$, we can imagine
  that a second r.v., $X₂$ has (potentially) a different density that
  depends on the value of $X₁$; write it as $f₂(· ∣ x₁)$, where $x₁$
  is a hypothetical value of $X₁$ (and the function $f₂(· ∣ x₁)$ is
  defined for all values in the support of $X₁$.  Then the joint
  density of $X₁$ and $X₂$, which we'll write as $f$, can be evaluated
  as
  \begin{equation}\label{eq:2}
    f(x₁,x₂) = f₂(x₂ ∣ x₁) × f₁(x₁).
  \end{equation}
  For discrete random variables, you can interpret this as saying
  ``the probability that $X₁$ and $X₂$ take the values $x₁$ and $x₂$
  is equal to the probability that $X₁$ takes the value $x₁$ and then
  $X₂$ takes the value $x₂$.''  But don't read too much into the word
  ``then'' too literally since the ordering isn't important.  We can
  also write
  \begin{equation}
    f(x₁,x₂) = f₁(x₁ ∣ x₂) f₂(x₂).
  \end{equation}

  If the conditional density of $X₁$ does not depend on $X₂$ (or vice
  versa) the random variables are said to be \emph{independent}.  This
  means that joint density can be written as
  \begin{equation*}
    f(x₁,x₂) = f₁(x₁) × f₂(x₂)
  \end{equation*}
  The elements of a $k$-vector of random variables are independent if
  \begin{equation*}
    f(x₁,…,x_k) = ∏_{i=1}^k f_i(x_i)
  \end{equation*}
  for all $x₁,…,x_k$ in the random variables' support.

\begin{figure}[t]
  [Empty for now.  Sometime soon, I'll add a figure.]
  \caption{Illustration of the conditional density math to go here.}
\end{figure}

  The functions $f₁(· ∣ x2)$ and $f₂(· ∣ x₁)$ are called \emph{the
  conditional densities of $X₁$ and $X₂$} respectively.  Given the
  joint and marginal densities, they can be calculated/defined as
  \begin{equation*}
    f₁(x₁ ∣ x₂) = f(x₁, x₂) / f₂(x₂)
  \end{equation*}
  Conditional probabilities of events and conditional distribution
  functions can be defined using this conditional density just as
  before; e.g.
  \begin{equation*}
    F₁(x ∣ x₂) = ∫_{-∞}^x f₁(z ∣ x₂) dz
  \end{equation*}
  defines the \emph{conditional CDF of $X₁$ given $X₂$}, where $x₂$ is
  a different hypothetical value in $X₂$'s support.

\item Note that we can recover the marginal density of $X₁$ from
  Equation~\eqref{eq:2} by integrating out the conditional density of
  $X₂$:
  \begin{equation*}
    ∫_{-∞}^{-∞} f₂(x₂ ∣ x₁) f₁(x₁) \dx_2 = f₁(x₁)
  \end{equation*}
  since $f₂(· ∣ x₁)$ is a density function and so integrates to 1.
  This gives a slightly less obvious relationship between the
  \emph{joint} and marginal densities:
  \begin{equation*}
    ∫_{-∞}^{-∞} f(x₁, x₂) \dx_2 = f₁(x₁).
  \end{equation*}
  There is clearly nothing special about the order of the variables.

\item We can define the joint CDF as
  \begin{equation*}
    F(x₁,…,x_k) = \Pr[X₁ ≤ x₁,…, X_k ≤ x_k].
  \end{equation*}
  and can be retrieved from the joint density as before:
  \begin{equation*}
    F(x₁,…,x_k) = ∫_{-∞}^{x₁} ⋯ ∫_{-∞}^{x_k} f(z₁,…,z_k) \dz_1 ⋯ \dz_k.
  \end{equation*}
  Writing this equation explicitly makes it clear that the CDF can be
  factored under independence just as the densities can.

  The marginal distribution and density functions can be derived
  directly from the joint distribution and density.  If we let $F₁$ be
  the marginal distribution of the first element and $f₁$ the marginal
  density, then we have
  \begin{equation*}
    F₁(x) = F(x, ∞,…,∞)
  \end{equation*}
  which is obvious in retrospect, because adding the additional
  restrictions that $X_j ∈ (-∞, ∞)$ for $j = 2,…,k$ imposes no new
  constraints on the vector $X$.

  Similarly, the relationship
  \begin{equation*}
    f₁(x) = ∫_{-∞}^∞ ⋯ ∫_{-∞}^∞ f(z₁,…,z_k) \dz_k ⋯ \dz_2.
  \end{equation*}
  can also seen by starting with the marginal CDF and then
  differentiating to get the density.

\end{itemize}

\section{Sigma-algebras as information sets}

\begin{itemize}

\item In the previous section, we motivated the Borel σ-algebra as a
  collection of sets that it makes sense to assign length to, and that
  it makes sense to talk about the ``Probability'' of a random
  variable landing in.  As you've probably seen elsewhere, we should
  think of a random variable is a way of numerically summarizing a
  random outcome.

\item Sometimes it doesn't make sense to assign numeric values right
  away, though.  If we want to think about modeling card games, the
  events that we might be concerned with are different hands that we
  might be dealt.  But it might not make sense to summarize that
  information numerically unless we have already set a particular
  game.  Fortunately, the key properties that we were discussing
  beforehand don't need to be specific to numbers.

  In general, we can call the set of all possible outcomes of a
  (possibly hypothetical) experiment the \emph{sample space} of the
  experiment, and let $Ω$ be the \emph{σ-algebra} of the experiment if
  it satisfies the rules we discussed above.  The elements of $Ω$ are
  usually referred to as \emph{events}.

  Since set theory plays a large role in this material,
  Table~\ref{tab:1} lists the definitions of common set operations.

\begin{table*}[t]
\begin{tabular}{lp{4in}}
  \toprule
  \multicolumn{2}{c}{Finite Set Operations} \\
  \midrule
  Union        & $A₁ ∪ A₂ ≡ \{x ∈ S : x ∈ A₁ \text{ or } x∈ A₂\}$ \\
  Intersection & $A₁ ∩ A₂ ≡ \{x ∈ S : x ∈ A₁ \text{ and } x ∈ A₂\}$ \\
  Subset       & $A₁ ⊂ A₂$ if $A₁ ∩ A₂ = A₁$ \\
  Equality     & $A₁ = A₂$ if $A₁ ⊂ A₂$ and $A₂ ⊂ A₁$ \\
  Complement   & $A₁^c ≡ \{x ∈ S : x ∉ A₁\}$ \\
  Difference   & $A₁ ∖ A₂ ≡ \{x ∈ S : x ∈ A₁ \text{ and } x ∉ A₂\}$ \\
  Symmetric difference & $A₁ Δ A₂ ≡ (A₁ ∖ A₂) ∪ (A₂ ∖ A₁)$ \\
  \midrule
  \multicolumn{2}{c}{Infinite Set Operations (can be defined for uncountable index sets as well)} \\
  \midrule
  Union        & $⋃_{k=1}^∞ A_k ≡ \{x ∈ S : x ∈ A_k$ for at least one $k ∈ 1,2,…\}$ \\
  Intersection & $⋂_{k=1}^∞ A_k ≡ \{x ∈ S : x ∈ A_k$ for all $k ∈ 1,2,…\}$ \\
  Infimum      & $\inf_{k ≥ n} A_k ≡ ⋂_{k=n}^∞ A_k$ \\
  Supremum     & $\sup_{k ≥ n} A_k ≡ ⋃_{k=n}^∞ A_k$ \\
  lim inf      & $\liminf_{n → ∞} A_n ≡ ⋃_{n=1}^∞ ⋂_{k=n}^∞ A_k$ \\
  lim sup      & $\limsup_{n → ∞} A_n ≡ ⋂_{n=1}^∞ ⋃_{k=n}^∞ A_k$ \\
  \midrule
  \multicolumn{2}{c}{Other definitions} \\
  \midrule
  limit 
  & Suppose that $A$, $A₁$, $A₂$,… is a sequence of subsets of $S$ and that $\limsup_{n → ∞} A_n = \liminf_{n → ∞} A_n = A$.
  Then $A$ is the limit of $\{A_n\}$. \\
  Disjoint 
  & $A₁$ and $A₂$ are disjoint if $A₁ ∩ A₂ = ∅$. \\
  Pairwise disjoint 
  & $A₁, A₂, …$ are pairwise disjoint if $A_i ∩ A_j = ∅$ for all $i$ and $j$ such that $i ≠ j$. \\
  Partition & If $A₁,A₂,…$ are pairwise disjoint and $⋃_{i=1}^∞ A_i = S$ then $\{A_i\}$ forms a partition of $S$. \\
  Power set & The power set of $S$, denoted $\mathcal{P}(S)$, is the set of all subsets of $S$.
  $\mathcal{P}(S) ≡ \{x : x ⊂ S\}$ \\
\bottomrule
\end{tabular}
  \caption{Collection of set operations; let $A₁$, $A₂$, $A₃$,… be subsets of another set $S$.}
  \label{tab:1}
\end{table*}

\item %
  \begin{defn} A collection $Ω$ of subsets of $S$ is a
    \emph{σ-algebra} if it satisfies the following
    \begin{enumerate}
    \item $S ∈ Ω$,
    \item $A^c ∈ Ω$ whenever $A ∈ Ω$, where $A^c = S ∖ A$,
    \item If $A₁$, $A₂$,… is a countable sequence with $A_i ∈ Ω$ for
      all $i$, then $⋃_{i=1}^∞ A_i ∈ Ω$ as well.
    \end{enumerate}
  \end{defn}

\item A pretty typical first example for σ-algebras involves coin
  flips.  But we're economists, so let's use a game.  One with the
  same structure as coin flips.
  \begin{ex}[Matching pennies]
    For \emph{matching pennies}, there are two players (A and B) and
    each has a coin.  Each player chooses ``Heads'' or ``Tails''
    (secretly) and then they simultaneously show each other their
    coin.  If the coins match, player A wins and if they don't, player
    B wins.

    Now, let's model a single round of the game.  The sample space is
    the set $S = \{HH, HT, TH, TT\}$.\sidenote{Where, for example,
    \emph{HT} means that player A plays ``Heads'' and player B plays
    ``Tails''} There are, of course, several σ-algebras that can be
    constructed on this sample space.  For example, the σ-algebra that
    represents whether or not A plays ``Heads'' is
    \begin{equation*}
      \{∅, S, \{HH,HT\}, \{TH,TT\}\}
    \end{equation*}
    and the σ-algebra representing whether or not A wins is
    \begin{equation*}
      \{∅, S, \{HH,TT\}, \{TH,HT\}\}.
    \end{equation*}
    We can form the second σ-algebra by starting with the event ``A
    wins'' (i.e. the set $\{HH, TT\}$), and the sample space $S$, and
    then adding more sets as necessary so that the σ-algebra is closed
    under complements and countable unions.\sidenote{You should verify
    as an exercise that both of these sets are actually σ-algebras.}
  \end{ex}

\item This example gets at why we can view a σ-algebra as an
  \emph{information set}.  The σ-algebra, ``A plays Heads,'' contains
  all of the information we learn if we learn whether or not the event
  ``A plays Heads'' has happened.\sidenote{I.e., we learn whether or
  not A played Heads, and whether or not A played Tails.}  We might
  also view this as the information available to player A after choosing
  H or T, but before revealing his or her choice to B.

  Also, if one σ-algebra contains another (the other is a subset) than
  it embodies more information than the smaller σ-algebra.

\item Now we can formally define a probability measure that obeys the
  same rules as before.
  \begin{defn} A function $\Pr$ is a \emph{probability measure} on a
    σ-algebra $Ω$ of subsets of $S$ if
    \begin{enumerate}
    \item $\Pr: Ω ↦ [0,1]$
    \item if $A₁$, $A₂$,… are disjoint events in $Ω$ then
      $\Pr(⋃_{i=1}^∞ A_i) = ∑_{i=1}^∞ \Pr(A_i)$
    \item $\Pr(S) = 1$.
    \end{enumerate}
  \end{defn}
  
\item We refer to a sample space $S$, a σ-algebra $Ω$, and a
  probability measure $\Pr$ as a \emph{probability space}, $(S, Ω,
  \Pr)$.

\item Let's go back to the example,
  \begin{ex}[Matching pennies, continued]
    We can define strategies through the probability measure on a
    generating class, then use the properties over countable unions
    and intersections to extend it to the rest of the σ-algebra.

    We'll suppose that A plays a mixed strategy and chooses ``Heads''
    with probability 1/3 (and Tails with probability 2/3), and that B
    plays ``Heads'' with probability 4/5.  We'll also assume that
    neither player is cheating, so these choices are independent of
    each other.  This means that we have
    \begin{align*}
      \Pr \{HH\} &= (1/3) × (4/5) = 4/15 \\
      \Pr \{HT\} &= (1/3) × (1/5) = 1/15 \\
      \Pr \{TH\} &= (2/3) × (4/5) = 8/15 \\
      \Pr \{TT\} &= (2/3) × (1/5) = 2/15.
    \end{align*}
    We can write any other event as the union of these sets, so this
    lets us calculate the probability of any event in the game.  For
    example, the event, \emph{A wins} is the set $\{HH,TT\}$, and
    \begin{equation*}
      \Pr \{HH, TT\} = \Pr \{HH\} + \Pr \{TT\} = (4/15) + (2/15) = 6/15
    \end{equation*}
    (these strategies are obviously not a Nash equilibrium).
  \end{ex}

\item Often, knowing that one event happens tells us something about
  how likely other events are.  In the Matching Pennies example we've
  used above, the probability that A wins is $6/15$.  If we know that
  A has played heads, we know that A wins if B plays heads too, which
  happens with probability $4/5$.

\item This was an example of conditioning on an event (e.g. the event
  that A plays heads).  The general formula for working out
  probabilities conditional on an event is as follows,

  \begin{defn}
    If $A$ and $B$ are events in some sample space $S$ and $\Pr[B]>0$,
    the \emph{conditional probability of $A$ given $B$} is written as
    $\Pr(A ∣ B)$ and defined to be
    \begin{equation*}
      \Pr(A ∣ B) ≡ \Pr(A ∩ B) / \Pr(B).
    \end{equation*}
    If $\Pr(B) = 0$ then we can define $\Pr(A ∣ B)$ arbitrarily to be
    zero.
  \end{defn}

  The equivalent formula
  \begin{equation*}
    \Pr(A ∩ B) = \Pr(A ∣ B) \Pr(B)
  \end{equation*}
  may make the interpretation easier: the probability that events $A$
  and $B$ happen is equal to the probability that $B$ happens times
  the probability that $A$ happened, knowing that $B$ already
  happened.\sidenote{Don't take the word ``already'' literally, since
  there is not time ordering implied by these formulas.}

\item Observe that if $A$ and $B$ are mutually exclusive, then $\Pr(A
  ∩ B) = 0$, and so $\Pr(A ∣ B) = 0$.

\item We can use the definitions and the simple observation that
  \begin{equation*}
    \Pr(A ∣ B) \Pr(B) = \Pr(B ∣ A) \Pr(A)
  \end{equation*}
  to justify \emph{Bayes's Rule}:
  \begin{thm}[Bayes's Rule]
    If $A$ and $B$ are sets in $S$ that have positive probability, then
    \begin{equation*}
      \Pr(A ∣ B) = \Pr(B ∣ A) \Pr(A) / \Pr(B).
    \end{equation*}
  \end{thm}

\item Naturally, we want to be able to discuss cases where two events
  do not affect each other at all; such events are called
  \emph{independent}.

  \begin{defn}
    Events $A$ and $B$ are \emph{independent} if $P(A ∩ B) = P(A)
    P(B)$.
  \end{defn}

  An immediate implication is that $\Pr(A ∣ B) = \Pr(A)$ if $A$ and
  $B$ are independent.

\end{itemize}

\section{Random variables are maps between information sets}

\begin{itemize}
\item The way to think about random variables, now, is that they
  summarize a probability space.  They take events that occur in a
  possibly abstract experiment and turn them into events in an easier
  to analyze experiment.  Often they map into the Borel σ-field.

  As an example, think back to the matching pennies example, but now
  imagine that players A and B play $n$ games in a row.  The sample
  space is now $\{HH, HT, TH, TT\}ⁿ$.  For some purposes, though, it
  might be sufficient to know how much money A wins from or loses to B
  (assuming they're gambling).  A random variable can map the
  different game outcomes to the real line and preserve the
  probability measure of the original experiment.

  \begin{defn} Let $\Fs$ be a sigma-field on a sample space $S$.  An
  $\Fs$-measurable function $X$ from $S$ to $\RR$ is a \emph{random
  variable}.
  \end{defn}

  Remember the definition of \emph{$\Fs$-measurable}: $X$ is an
  $\Fs$-measurable function from $S$ to $\RR$ if $X^{-1}(B) ∈ \Fs$ for
  all $B ∈ \BB$.  Intuitively, we want to use the random variable $X$
  to transfer the probability measure on $S$ to $\RR$.  This is only
  going to work if we can map events in $\RR$ (i.e., elements of
  $\BB$) back to the original sigma algebra $\Fs$.

  Formally, this means that $X$ is a function (it's worth repeating),
  and the random variable takes on values $X(ω)$, where $ω ∈ S$.  It
  can sometimes be useful to write it out so explicitly in proofs, but
  usually we will just write $X$ and suppress the argument.

\item This also reinforces the idea of $\Fs$ as an information set:
  $X$ is $\Fs$-measurable if knowledge of events in $\Fs$ is
  sufficient to know about $X$.  This concept is most clear in a
  time-series setting.  Let $\{\Fs_t\}$ be an expanding sequence of
  information sets, so $\Fs_t ⊂ \Fs_{t+1}$ for all $t$ (this means
  that $\Fs_t$ is something like ``the information available in period
  $t$''—in period $t$, we know/remember what happened in period $t-1$,
  contained in $\Fs_{t-1}$, but do not know what will happen in period
  $t+1$).  If $\{X_t\}$ is a sequence of random variables s.t. $X_t$ is
  $\Fs_t$-measurable for every $t$ (which is often written as $X_t ∈
  \Fs_t$), then $X_t$ is a random variable whose value is determined
  in period $t$—knowledge of the outcomes in period $t$ is sufficient
  to know the value of $X_t$, but not necessarily the value of
  $X_{t+1}$.

\item Measurable mappings from the sample space $S$ to $\RRⁿ$ are
  called \emph{random vectors}, more generally they're often called
  \emph{random elements}.

\item In this approach, the original probability space is the
  fundamental idea, and then we define random variables, distribution
  functions, and densities by referencing the original probability
  measure.  I.e., given a probability measure and a random variable
  $X$, we \emph{define} the CDF of $X$ as the function $F_X$ given by
  \begin{equation*}
    F_X(x) = \Pr(\{ω ∣ X(ω) ≤ x\}).
  \end{equation*}
  Note that the $ω$'s are elements of a sample space $S$, so the set
  $\{ω ∣ X(ω) ≤ x\}$ is an element of the sigma field that we've been
  calling $\Fs$.  We know that the set $\{ω ∣ X(ω) ≤ x\}$ must be in
  $\Fs$ by measurability: it is equivalent to the set $X^{-1}((-∞,x])$,
  and $(-∞,x] ∈ \BB$.

  The properties that we listed for CDFs can be derived from this
  definition by using some fundamental properties of probability
  measures we've skipped over.  And then, given a distribution
  function $F_X$, the density of $X$ can be defined as the function
  $f_X$ such that
  \begin{equation*}
    F_X(x) = ∫_{-∞}^{x} f_X(z) \dz.
  \end{equation*}

  As alluded to earlier, this is the more mathematically sound way to
  proceed, but it can give the wrong impression.  One almost never
  needs to work at this level of abstraction even in theoretical
  econometrics research—typically we start with the random variables
  and leave the deeper probability structure implicit.

\item A similar definition of independence holds for information sets
  and random variables as we defined for events (this is to some
  degree a restatement of what we discussed in the first section, but
  with a more justifiable mathematical underpinning).
  \begin{defn}
    Two sigma-fields $\Fs$ and $\Gs$ are independent if the events $F$
    and $G$ are \emph{independent} for any $F ∈ \Fs$ and $G ∈ \Gs$.
    Two random variables $X$ and $Y$ are \emph{independent} if the
    sigma-fields $X^{-1}(\BB)$ and $Y^{-1}(\BB)$ are independent.
  \end{defn}
  The random variable definition is essentially the same as
  \begin{equation*}
    \Pr[X ∈ I₁ ∩ Y ∈ I₂] = \Pr[X ∈ I₁] \Pr[Y ∈ I₂]
  \end{equation*}
  for any intervals $I₁$ and $I₂$.

\item It is then straightforward to verify that, for independent
  random variables $X$ and $Y$, the joint distribution has the
  property $F(x, y) = F_X(x) F_Y(y)$ and the joint density has the
  property $f(x,y) = f_X(x) f_Y(y)$.  These properties are easy to
  verify using the definition above, based on the underlying
  information sets, but are somewhat harder to prove if starting from
  a less abstract definition of independence.

  We also have, if $X₁$ and $X₂$ are independent, so are $g(X₁)$ and
  $h(X₂)$ for any (measurable) functions $g$ and $h$.  For a proof,
  observe
  \begin{align*}
    \Pr[g(X₁) ∈ I₁ ∩ h(X₂) ∈ I₂]
    &= \Pr[X₁ ∈ g^{-1}(X₁) ∩ X₂ ∈ h^{-1}(X₂)] \\
    &= \Pr[X₁ ∈ g^{-1}(X₁)] \Pr[X₂ ∈ h^{-1}(X₂)] \\
    &= \Pr[g(X₁) ∈ I₁] \Pr[h(X₂) ∈ I₂]
  \end{align*}
  where $g^{-1}(I₁) = \{x : g(x) ∈ I₁\}$, we don't need $g$ or $h$ to
  be invertible.

\end{itemize}

\section{Almost sure convergence of sequences of random variables}

\begin{itemize}

\item Remember our mathematical definition of random variables: a
  random variable is a (measurable) function from a sample space $S$
  to the real line.  So, for any $ω ∈ S$, the sequence $X₁(ω),
  X₂(ω),...$ is just a sequence of numbers.  For a given $ω$, the
  sequence might converge and for a different $ω$, the sequence might
  not.  Informally, the sequence of random variables $X₁,X₂,...$
  converges \emph{almost surely} if there is probability 1 of drawing
  an $ω$ such that $X_n(ω)$ converges.
  \begin{defn}
    The sequence of random variables $X_n$ converges \emph{almost
    surely} to the random variable $X$ if, for every $ε > 0$,
    \begin{equation}\label{eq:1}
      \Pr[\lim_{n → ∞} | X_n - X | < ε] = 1.
    \end{equation}
  \end{defn}
  This will typically be written as $X_n → X$ a.s. or $X_n →^{a.s.}
  X$.

  We can write equation~\eqref{eq:1} more explicitly as
  \begin{equation*}
    \Pr(\{ ω : \lim_{n → ∞} | X_n(ω) - X(ω) | < ε \}) = 1.
  \end{equation*}

  You should prove that almost sure convergence implies convergence in
  probability, but not vice versa.

\item The \emph{Strong LLN} deals with almost sure convergence.  One
  example is \emph{Kolmogorov's SLLN}: if $\{X_n\}$ is a sequence of
  i.i.d. random variables with mean $μ$ then $\Xb → μ$ almost surely.

\item For statistical purposes, we pretty much only need convergence
  in probability, not almost sure convergence.

\end{itemize}

%%% Local Variables:
%%% mode: latex
%%% TeX-master: "../core_econometrics"
%%% End:

%  LocalWords:  Borel CDF Supremum measurability CDFs differentiable
%  LocalWords:  Bayes's Infimum invertible

% Copyright (c) 2013, authors of "Core Econometrics;" a
% complete list of authors is available in the file AUTHORS.tex.

% Permission is granted to copy, distribute and/or modify this
% document under the terms of the GNU Free Documentation License,
% Version 1.3 or any later version published by the Free Software
% Foundation; with no Invariant Sections, no Front-Cover Texts, and no
% Back-Cover Texts.  A copy of the license is included in the file
% LICENSE.tex and is also available online at
% <http://www.gnu.org/copyleft/fdl.html>.

\chapter{Properties of random variables}

\section{Expectation}

\begin{itemize}[leftmargin=0pt]
\item There are a few ways to think about the \emph{expectation} of a
  random variable.  One of the more natural is to think about many
  independent repeated draws, $X_1, X_2,\dots,X_n$, from the same density
  function $f$---think of them as repeated measurements of some physical
  quantity, like a weight or a height.  Under some often plausible
  assumptions (that will be discussed later) these measurements
  will cluster around a particular point $\mu$ in the following sense:
  as the number of observations grows ($n \to \infty$), the average of the
  observations will settle down and converge to $\mu$.  If the
  measurements are unbiased, then $\mu$ will be the true quantity being
  measured; if they are unbiased, $\mu$ will reflect both the true
  quantity and the systematic measurement error.  In either case, $\mu$
  is the \emph{expected value} of the density $f$.

  Formally, write the expected value of the density $f$ as $\E X$
  where $X \sim f$ is a representative draw from the distribution, and
  define the expected value as
  \begin{equation}\label{p3}
    \E X = \int x f(x) \dx.
  \end{equation}
  For a random vector $X = (X_1,\dots,X_k)$, the expected value is just the
  vector of each element's expected value:
  $\E X = (\E X_1,\dots \E X_k)'$

  Now, the expected value does not necessarily exist for a given
  random variable: if the density's tails are too fat, the integral
  in~\eqref{p3} can be infinite.  Generally, if $\E |X| = \infty$ then
  the mean of $X$ is said to not exist.

\item The expected value of a function of $X$, $\E g(X)$, is
  surprisingly straightforward and does not require the density of
  $g(X)$ to be calculated:
  \begin{equation*}
    \E g(X) = \int g(x) f(x) \d x.
  \end{equation*}
  This can be verified in special cases from the formula for the
  density of $g(X)$.  For the general case, this result follows from
  taking the limit of the expectation of step functions (it can be
  shown that the expectation operator is closed under certain limits,
  but we won't cover those properties here).

\item Note that the expectation of an indicator function is just the
  probability of the event represented by that indicator function:
  \begin{equation*}
    \E \ind\{ X \in A \} = \Pr[X \in A]
  \end{equation*}

  Also notice a fundamental and important property of the expectation
  operator, linearity.  For any constants $a$ and $b$, we have
  \begin{equation*}
    \E(a X + b) = a \E X + b
  \end{equation*}

\item For a random vector, we define the expecation as the vector of
  expectations of the individual elements:
  \begin{equation*}
    \E X = (\E X_1,\dots,\E X_k)
  \end{equation*}

\item It is straightforward to see that if $X$ and $Y$ are
  independent then $\E XY = \E X \E Y$.  For proof:
  \begin{align*}
    \E XY
    &= \int\int x y f_{X,Y}(x,y) \dx \dy \\
    &= \int\int xy f_X(x) f_Y(y) \dx \dy \\
    &= \int x f_X(x) \dx \int y f_Y(y) \dy \\
    &= \E X \E Y.
  \end{align*}

\item An especially useful inequality based on the expectation is
  \emph{Jensen's inequality}.
  \begin{thm}
    For any random variable $X$, $\E g(X) \geq g(\E X)$ for any convex
    $g$.
  \end{thm}

  Remember the definition of a convex function:
  \begin{defn}
    A function $g(x)$ is convex if 
    \begin{equation*}
      g( \lambda x + (1-\lambda) y ) \leq \lambda g(x) + (1-\lambda) g(y) 
    \end{equation*}
    for all $\lambda \in [0,1]$ and any $x$ and $y$.
  \end{defn}

  [Add a picture.]

\end{itemize}

\section{Conditional expectation}

\begin{itemize}[leftmargin=0pt]

\item We can also (obviously) take the expectation of conditional
  densities.  If $X$ and $Y$ are two random vectors, the conditional
  expectation of $Y$ given $X$ is
  \begin{equation*}
    \E(Y \mid x) = \int y f_Y(y \mid x) \dy
  \end{equation*}
  where $f_Y$ is the density of $Y$.\footnote{There are other, more
  fundamental ways to derive conditional expectations that we're not
  going to worry about.}  When we don't want to focus on a specific
  value of $X$, we just write $\E(Y \mid X)$ (note that this is now a
  function of the random variable $X$).

\item The conditional expectation has some important properties (these
  properties actually make a lot more sense if you view the
  conditional expectation as a projection):
  \begin{enumerate}
  \item If $Y = g(X)$ for some function $g$ then $\E(Y \mid X) = Y$.
    This occurs because the conditional density of $Y$ given $X$ is
    degenerate.
  \item Similarly, If $X = (X_1, X_2)$ then we have the relationships
    \begin{equation*}
      \E(\E(Y \mid X) \mid X_1) = \E(Y \mid X_1) \quad\text{a.s.}
    \end{equation*}
    and
    \begin{equation*}
      \E(\E(Y \mid X_1) \mid X) = \E(Y \mid X_1) \quad\text{a.s.}
    \end{equation*}
    Here it's important to remember the connection between random
    variables and information sets.  The r.v. $X_1$ contains less
    information than $X$, so when we first condition on $X_1$ to get
    $\E(Y \mid X_1)$, that step removes (integrates out) the information
    in $X_2$.  A second conditioning step can not restore that
    information.
  \item $\E(X_1 Y \mid X) = X_1 \E(Y \mid X)$; again, given the information in
    $X$, we can view $X_1$ as a known constant.  The result then
    follows from linearity of the expectation.
  \item The Law of Iterated Expectations
    \begin{thm}
      Suppose that $X$ and $Y$ are random variables and $Y$ has finite
      mean; then $\E \E(Y \mid X) = \E Y$.
    \end{thm}
    The proof (assuming densities, etc) is pretty straightforward
    \begin{align*}
      \E Y &= \int y f_Y(y) \dy \\
      &= \int \int y f_{X,Y}(x,y) \dx \dy \\
      &= \int \int y f_Y(y \mid X = x) f_X(x) \dx \dy \\
      &= \int \int y f_Y(y \mid X = x) \dy f_X(x) \dx \\
      &= \int \E(Y \mid X = x) f_X(x) \dx \\
      &= \E \E(Y \mid X)
    \end{align*}
  \end{enumerate}

\end{itemize}

\section{General measures of the center of a density}

\begin{itemize}[leftmargin=0pt]

\item The mean has lots of advantages; it is mathematically convenient
  to work with and shows up often because of the Law of Large Numbers.
  But it has disadvantages too; it is extremely sensitive to events in
  the tails of the distribution.  So sometimes other measures of
  location are useful

\item %
  \begin{defn}
    The \emph{median} of a distribution $F$ is the value $c$ such that
    $\Pr[X \leq c] \geq 1/2$ and $\Pr[X \geq c] \geq 1/2$.
  \end{defn}
  Note that this definition means that the median may not be unique if
  $X$ is discrete.

\item %
  \begin{defn}
    The \emph{mode} of a distribution is the value that maximizes
    the density function $f_X$.
  \end{defn}
  The mode is not necessarily unique either.

\end{itemize}

\section{Measures of dispersion and other moments}

\begin{itemize}[leftmargin=0pt]

\item The variance measures a random variable's dispersion about its
  mean, and is defined as
  \begin{equation*}
    \var(X) = \E(X - \E X)^2.
  \end{equation*}
  An equivalent formula is $\var(X) = \E(X^2) - (\E X)^2$, which can be
  easier to use at times.  Obviously the variance is nonnegative (it
  will be zero if $X$ is a constant).

  The standard deviation is the square root of the variance, which has
  the advantage that it's in the same units as $X$.

\item We can generalize the variance to measure the linear
  relationship between two variables.  Define the \emph{covariance} as
  \begin{equation*}
    \cov(X,Y) = \E(X - \E X) (Y - \E Y) = \E XY - \E X \E Y
  \end{equation*}
  and the \emph{correlation} as
  \begin{equation*}
    \corr(X, Y) = \cov(X,Y) / \sqrt{\var(X)} \sqrt{\var(Y)}.
  \end{equation*}

  The Cauchy-Schwarz inequality ensures that the correlation is
  between $-1$ and 1.
  \begin{thm}[Cauchy-Schwarz inequality]
    Let $X$ and $Y$ be random variables.  Then 
    \begin{equation}\label{p5}
      \E (|X Y|) \leq \sqrt{\E X^2} \sqrt{\E Y^2}
    \end{equation}
  \end{thm}
  To prove this result, we're going to show that it is equivalent to
  something like $0 \leq \text{something}^2$;~\eqref{p5} is equivalent to
  \begin{equation*}
    0 \leq \E X^2 - (\E |X Y|)^2 / \E Y^2
  \end{equation*}
  as long as $Y$ is not identically zero (if $Y$ is identically zero
  then the result holds trivially).  Then
  \begin{align*}
    \E X^2 - (\E |X Y|)^2 / \E Y^2
    &= \E X^2 - 2 (\E |XY|)^2/ \E Y^2 + (\E |XY|)^2/ \E Y^2 \\
    &= \E X^2 - 2(\E |XY| / \E Y^2) \E |XY| + (\E |XY|/\E Y^2)^2 \E Y^2 \\
    &= \E(X^2 - 2(\E |XY| / \E Y^2) |X| |Y| + (\E |XY|/\E Y^2)^2 Y^2) \\
    &= \E(|X| - (\E |XY|/\E Y^2) |Y|)^2
  \end{align*}
  and this last quantity is clearly nonnegative.

\item The variance obeys a relationship similar to the Law of the
  Iterated Expectation:
  \begin{equation*}
    \var(X) = \var(\E(X|Y)) + \E(\var(X|Y)).
  \end{equation*}
  There are two implications of this formula: first, that a
  conditional expectation has lower variance than its original random
  variable; and second, that conditioning decreases the variance on
  average, but not almost surely.

\item We can define the variance for vectors as well: $\var(X) = \E(X
  - \E X) (X - \E X)'$.  For vectors, $\var(X)$ is called the
  \emph{variance-covariance matrix}.  Observe that if $X$ has
  variance-covariance matrix $\Sigma$ and is $A$ a deterministic $j \times k$
  matrix then $A X$ has variance-covariance matrix $A \Sigma A'$.

\item We can verify that the variance-covariance matrix is always
  positive semi-definite, i.e. that $\alpha'\var(X)\alpha \geq 0$ for any nonzero
  $\alpha$:
  \begin{align*}
    \alpha'\var(X)\alpha &= \alpha' \E((X - \E X)(X - \E X)') \alpha \\
    &= \E \alpha'(X - \E X)(X - \E X)'\alpha \\
    &= \E (\alpha'X - \E(\alpha'X))^2 \\
    &= \var(\alpha'X) \geq 0
  \end{align*}

\item In general, we can refer to the $k$th moment or central moment.
  The $k$th moment of a random variable $X$ is just $\E X^k$ and it is
  said to exist if $\E |X|^k$ is finite.  The $k$th central moment is
  defined as $\E |X - \E X|^k$.

\item As a consequence of Jensen's inequality, we have the following
  result: if the $p$th moment of a random variable is finite, so are
  all moments between 0 and $p$.

  For a proof, let $q$ be between 0 and $p$ and observe that $g(x) \equiv
  x^{p/q}$ is convex.  So, for any $Y$, Jensen's inequality implies
  that $\E g(Y) \geq g(\E Y)$.  Now, let $Y = |X|^q$, so
  \begin{equation}\label{p4}
    \E g(|X|^q) \geq g(\E |X|^q).
  \end{equation}
  But $\E g(|X|^q) = \E |X|^p$ and $g(\E |X|^q) = (\E |X|^q)^{p/q}$.
  Raising both sides of Equation~\eqref{p4} to the $q/p$ power
  completes the proof.

\item The first four moments can be interpreted to some degree; we
  have already discussed the first two.  The random variable $X$'s
  \emph{skewness} is its normalized third moment, $\E(X - \E X)^3 /
  \sigma^3$, but you will sometimes see slightly different definitions.
  Skewness measures the asymmetry of the density function: if the
  density is perfectly symmetric, it is zero; if there is a long right
  tail, the skewness is positive, etc.

  $X$'s \emph{kurtosis} is its normalized fourth moment, $\E(X - \E
  X)^4 / \sigma^4$, which measures to some extent the thickness of the tails
  (higher kurtosis implies thicker tails).  It is sometimes useful to
  discuss the \emph{excess kurtosis}, which is $\E(X - \E X)^4 / \sigma^4 -
  3$; ``$3$'' is the kurtosis of the normal distribution.

\end{itemize}

\section{General transformations of random variables}

\begin{itemize}[leftmargin=0pt]

\item Notice that, if $g$ is a Borel-measurable function from $\RR$ to
  $\RR$, then $g(X)$ is a random variable as well.  For any $B \in \BB$,
  we know that $g^{-1}(B) \in \BB$ by measurability of $g$, and so
  $X^{-1}(g^{-1}(B)) \in \Fs$ by measurability of $X$.

  This can be useful when $g$ is smooth and when $X$ has a known
  density function.  If we start with the distribution functions, let
  $H$ be the CDF of $g(X)$ and $F$ the CDF of $X$.  Then, if $g$ is
  invertible and monotone increasing,
  \begin{equation*}
    H(c) = \Pr[g(X) \leq c] = \Pr[X \leq g^{-1} c] = F(g^{-1}(c)).
  \end{equation*}
  Now, assuming all of these operations go through, we can write the
  density of $g(X)$ as
  \begin{align*}
    h(z) &= (d/dz) H(z) \\
    &= (d/dz) F(g^{-1}(z)) \\
    &= f(g^{-1}(z)) (d/dz) g^{-1}(z).
  \end{align*}

  If $g$ is monotone decreasing, you can do almost exactly the same
  operations but need to account for the different sign.  More
  generally, it might be necessary to partition the domain of $g$ into
  segments where $g$ is monotone and work with the individual pieces.

  A reasonably formal but concise statement of this result is:
  \begin{thm}
    Suppose that $Y = g(X)$, where $X$ is a random variable with density
    $f_X$ and $g$ is a continuous, monotone function with continuously
    differentiable inverse.  The density of $Y$, $f_Y$, is given by the
    equation
    \begin{equation}
      f_Y(y) = f_X(g^{-1}(y)) | (d/dy) g^{-1}(y) |
    \end{equation}
    for $y$ in the range of $g$ and zero elsewhere.
  \end{thm}

\item \emph{Location-scale families} are a useful special case of the
  transformation formula.  If $X$ is a random variable with density
  $f_X$ and $f_Y$ is the density of $Y = \sigma X + \mu$ for some constant
  $\mu$ and positive constant $\sigma$, then
  \begin{equation*}
    f_Y(x) = \tfrac{1}{\sigma} f_X\big( \tfrac{x - \mu}{\sigma} \big)
  \end{equation*}
  for all $x$.\footnote{Verify on your own that this follows from the
  general transformation theorem.}

  This family of pdfs is called the ``location-scale family with
  standard pdf $f_X(x)$.''
  \begin{itemize}
  \item $\mu$ is the \emph{location parameter}
  \item $\sigma$ is the \emph{scale parameter}.
  \end{itemize}
  If only $\mu$ or $\sigma$ varies then we have either a \emph{location
  family} or a \emph{scale family}.

\item The math also becomes more painful if we are dealing with random
  vectors.  A general theory is very painful to work with because of
  dimension changes in particular, but working with one-to-one
  transformations from $\RR^k$ to $\RR^k$ is reasonably straightforward.
  First we need a definition:
  
  \begin{defn}
    Let $g: \RR^k \to \RR^k$ be a differentiable function.  The Jacobian
    of this function is the matrix of partial derivatives
    \begin{equation*}
      J(x) =
      \begin{pmatrix} \partial g_1(x)/\partial x_1 & \partial g_1(x)/\partial x_2 & \cdots & \partial g_1/\partial x_k \\
      \vdots \\
      \partial g_k/\partial x_1 & \cdots & \cdots & \partial g_k/\partial x_k
      \end{pmatrix}.
    \end{equation*}
  \end{defn}
      
  Then the result is (note that this is the version presented by
  \citealp[B.7.7]{Gre12})
  \begin{thm}
    Suppose that $X$ is a random vector in $\RR^k$ with joint density
    $f_X$ and that $g: \RR^k \to \RR^k$ is one-to-one from the support of
    $X$ to its image and let $Y = g(X)$.  Then the joint density of $Y$ is
    \begin{equation}
      f_Y(y) = f_X(g^{-1}(y)) \abs(\det( J(y) ))
    \end{equation}
    (where $J(y)$ is the Jacobian of $g^{-1}$) in the image of the support
    of $X$ and is zero elsewhere.
  \end{thm}

\item For transformations from a higher dimension to a lower
  dimension, one can sometimes create an artificial transformation
  that's easy to work out, then integrate out the other dimension.

\end{itemize}

%%% Local Variables: 
%%% mode: latex
%%% TeX-master: "../core_econometrics"
%%% End: 
%  LocalWords:  nonnegative kth skewness

% Copyright (c) 2013, authors of "Core Econometrics;" a
% complete list of authors is available in the file AUTHORS.tex.

% Permission is granted to copy, distribute and/or modify this
% document under the terms of the GNU Free Documentation License,
% Version 1.3 or any later version published by the Free Software
% Foundation; with no Invariant Sections, no Front-Cover Texts, and no
% Back-Cover Texts.  A copy of the license is included in the file
% LICENSE.tex and is also available online at
% <http://www.gnu.org/copyleft/fdl.html>.

\section{Problems}

\begin{enumerate}

\item When $X$ has a continuous distribution $F$ and $U \sim$
  uniform(0,1), $F(X) =^d U$ and, consequently, $X =^d F^{-1}(U)$ (see
  \citealt[Theorem 2.1.10]{CB02} for a proof).  Use this result to
  write an R function called \texttt{rtan} that generates $n$ random
  variables from the location-scale family $\mu + \sigma X$, where $X
  \sim F$ and
  \begin{equation*}
    F(x) = 0.5 + \tan^{-1}(x)/\pi.
  \end{equation*}
  The arguments of the function should include $n$, $\mu$, and $\sigma$.

\item You can also generate random variables indirectly.  If $f_X$ has
  support $(0,1)$ and $c = \max_x f_X(x)$, then the following
  algorithm will draw $X$ from the density $f_X$ \citep[see][Section
  5.6.2]{CB02}
  \begin{enumerate}
  \item Draw $U$ and $V$ as independent uniform(0,1).
  \item If $U \leq f_X(V) / c$, set $X = V$.  Otherwise draw $U$ and $V$
    again.
  \item One case where we might need to use this algorithm is if we
    wanted to draw $X$ and $Y$ that satisfy
    \begin{align*}
      \binom{X}{Y} &\sim N(\mu,\Sigma) & \text{s.t.}&& X^2 + Y^2 &= r^2,
    \end{align*}
    where $r$ is some known scalar.  Note that the probability that
    the constraint holds is zero, so we can't just simulate $(X,Y)$
    from the normal distribution and discard pairs that don't satisfy
    it.  Use the accept-reject algorithm to write an $r$ function
    \texttt{rpair} that simulates $n$ observations from this
    conditional distribution ($\mu$, $\Sigma$ and $r$ should all be arguments
    of the function).
  \end{enumerate}

\item Let $X$ and $Y$ be two random variables with the joint density
  function
  \begin{equation*}
    f(x, y) = \begin{cases}
      \min(6 y, 6 - 6y, 6 x, 6 - 6 x) & \text{if } (x, y) \in [0,1]^2 \\
      0 & \text{otherwise.}
    \end{cases}
  \end{equation*}
  (this looks like a pyramid centered at (1/2, 1/2)).
  \begin{enumerate}
  \item Derive the joint distribution of $X$ and $Y$.
  \item Derive the marginal distribution and density of $X$
  \item Derive the conditional distribution and density of $Y$ given
    $X$.
  \item We're going to write two different R functions to generate
    draws from this density function. The first is going to use the
    analytic work you've done in the previous few questions, and the
    second uses a trick. For both, please write a function that takes
    the number of observations to generate as its only argument and
    generate a contour plot of 10000 draws from the function.
    \begin{enumerate}
    \item Invert the distribution function of $X$ and the conditional
      distribution function of $Y$ given $X$. You can then generate
      two independent uniform random variables $U$ and $V$ and and
      pass them to the inverted distribution functions. Probably the
      best way to do this is to create a function \texttt{rX} that
      generates $X$ and another function \texttt{rY} that takes $X$ as
      an argument and generates a draw from the conditional
      distribution of $Y$.
    \item You can also generate random variables indirectly using the
      accept-reject algorithm \citep[Section 5.6.2]{CB02}. Let $c =
      \max_{x,y} f(x,y)$ and do the following
      \begin{enumerate}
      \item Draw $U$, $V$ and $W$ as independent uniform(0,1).
      \item If $U \leq f(V,W) / c$, set $X=V$ and $Y=W$.  Otherwise draw
        $U$, $V$, and $W$ again.
      \item Repeat until you accept candidate values of $X$ and
        $Y$. That is a single draw from the target density.
      \end{enumerate}
    \end{enumerate}

    The first approach will execute faster, but can be impractical (or
    even impossible) when the density is complicated. The second
    approach is sometimes slow but can usually be applied (and when it
    can't, there are other similar algorithms that can be applied more
    generally).
  \end{enumerate}

\item Suppose that $X$ and $Y$ are random variables with finite second
  moments such that $\E(Y \mid X) = \alpha + \beta X$.  Prove that
  \begin{align}
    \alpha &= \E Y - \beta \E X &\text{and}&& \beta &= \frac{\cov(X, Y)}{\var(X)}.
  \end{align}

\item Suppose that $X_1,\dots,X_n$ are an i.i.d. sample from the continuous
  distribution $F$.  What is the distribution of $\max_i X_i$?  Derive
  the joint distribution of $\min_i X_i$ and $\max_i X_i$.

\item Let $X$ and $Y$ be random variables.  Prove that 
  \begin{equation}
    \var(Y) = \var(\E(Y \mid X)) + \E \var(Y \mid X)
  \end{equation}

\item Suppose that $X_1$ and $X_2$ are independent univariate Normal
  random variables.  Prove that $X_1 + X_2$ is Normal and find its mean
  and variance.

\item Let $X \sim N(0,I_n)$ and let $P$ be a symmetric $n \times n$ matrix.
  Prove that $X'PX$ is chi-square with $k$ degrees of freedom if and
  only if $P$ is idempotent with rank $k$ \citep{SL03}.

\item Suppose that $X_1 \sim$ beta$(\alpha, \beta)$ and $X_2 \sim$ beta$(\gamma, \delta)$ and
  they are independent.  Derive the density of $X_1 \cdot X_2$.

\end{enumerate}

%%% Local Variables: 
%%% mode: latex
%%% TeX-master: "../core_econometrics"
%%% End: 

% The files `AUTHORS_standalone.tex` and `LICENSE_standalone.tex` are
% available if you want to distribute the author list and the FDL on
% their own.
\addtocontents{toc}{\protect\setcounter{tocdepth}{0}}

\newpage
\part*{Complete list of authors}%
\addcontentsline{toc}{part}{Appendix A: Complete list of authors}
% Copyright © 2013, authors of the "Econometrics Core" textbook; a
% complete list of authors is available in the file AUTHORS.tex.

% Permission is granted to copy, distribute and/or modify this
% document under the terms of the GNU Free Documentation License,
% Version 1.3 or any later version published by the Free Software
% Foundation; with no Invariant Sections, no Front-Cover Texts, and no
% Back-Cover Texts.  A copy of the license is included in the file
% LICENSE.tex and is also available online at
% <http://www.gnu.org/copyleft/fdl.html>.

% Remove the next two lines if you are distributing the author list as
% a standalone pdf.
\noindent%
The following is a list of the contributors to the Econometrics Free
Library Project's \textit{Econometrics Core}, in order of their date
of first involvement (yes, I'm aware that it's a little ridiculous to
have this as a separate file when there is only a single contributor,
but let's dream big, shall we).

\begin{description}
\item[2009-07-01] Gray Calhoun, \email{gcalhoun@iastate.edu}
\end{description}

%%% Local Variables:
%%% mode: latex
%%% TeX-master: "AUTHORS_standalone"
%%% End:

\newpage
\part*{GNU Free Documentation License}%
\addcontentsline{toc}{part}{Appendix B: GNU Free Documentation License}
% Remove the next two lines if you are distributing the author list as
% a standalone pdf.
\part*{GNU Free Documentation License}%
\addcontentsline{toc}{part}{Appendix B: GNU Free Documentation License}
\setcounter{section}{-1}%
\renewcommand\thesection{\arabic{section}}%
\noindent Version 1.3, 3 November 2008

\noindent Copyright \copyright\ 2000, 2001, 2002, 2007, 2008 Free
Software Foundation, Inc.

\noindent \texttt{<http://fsf.org/>}
 
\noindent Everyone is permitted to copy and distribute verbatim copies
of this license document, but changing it is not allowed.

\section{Preamble}

The purpose of this License is to make a manual, textbook, or other
functional and useful document ``free'' in the sense of freedom: to
assure everyone the effective freedom to copy and redistribute it,
with or without modifying it, either commercially or noncommercially.
Secondarily, this License preserves for the author and publisher a way
to get credit for their work, while not being considered responsible
for modifications made by others.

This License is a kind of ``copyleft'', which means that derivative
works of the document must themselves be free in the same sense.  It
complements the GNU General Public License, which is a copyleft
license designed for free software.

We have designed this License in order to use it for manuals for free
software, because free software needs free documentation: a free
program should come with manuals providing the same freedoms that the
software does.  But this License is not limited to software manuals;
it can be used for any textual work, regardless of subject matter or
whether it is published as a printed book.  We recommend this License
principally for works whose purpose is instruction or reference.


\section{APPLICABILITY AND DEFINITIONS}

This License applies to any manual or other work, in any medium, that
contains a notice placed by the copyright holder saying it can be
distributed under the terms of this License.  Such a notice grants a
world-wide, royalty-free license, unlimited in duration, to use that
work under the conditions stated herein.  The ``\textbf{Document}'',
below, refers to any such manual or work.  Any member of the public is
a licensee, and is addressed as ``\textbf{you}''.  You accept the
license if you copy, modify or distribute the work in a way requiring
permission under copyright law.

A ``\textbf{Modified Version}'' of the Document means any work containing the
Document or a portion of it, either copied verbatim, or with
modifications and/or translated into another language.

A ``\textbf{Secondary Section}'' is a named appendix or a front-matter section of
the Document that deals exclusively with the relationship of the
publishers or authors of the Document to the Document's overall subject
(or to related matters) and contains nothing that could fall directly
within that overall subject.  (Thus, if the Document is in part a
textbook of mathematics, a Secondary Section may not explain any
mathematics.)  The relationship could be a matter of historical
connection with the subject or with related matters, or of legal,
commercial, philosophical, ethical or political position regarding
them.

The ``\textbf{Invariant Sections}'' are certain Secondary Sections whose titles
are designated, as being those of Invariant Sections, in the notice
that says that the Document is released under this License.  If a
section does not fit the above definition of Secondary then it is not
allowed to be designated as Invariant.  The Document may contain zero
Invariant Sections.  If the Document does not identify any Invariant
Sections then there are none.

The ``\textbf{Cover Texts}'' are certain short passages of text that are listed,
as Front-Cover Texts or Back-Cover Texts, in the notice that says that
the Document is released under this License.  A Front-Cover Text may
be at most 5 words, and a Back-Cover Text may be at most 25 words.

A ``\textbf{Transparent}'' copy of the Document means a machine-readable copy,
represented in a format whose specification is available to the
general public, that is suitable for revising the document
straightforwardly with generic text editors or (for images composed of
pixels) generic paint programs or (for drawings) some widely available
drawing editor, and that is suitable for input to text formatters or
for automatic translation to a variety of formats suitable for input
to text formatters.  A copy made in an otherwise Transparent file
format whose markup, or absence of markup, has been arranged to thwart
or discourage subsequent modification by readers is not Transparent.
An image format is not Transparent if used for any substantial amount
of text.  A copy that is not ``Transparent'' is called ``\textbf{Opaque}''.

Examples of suitable formats for Transparent copies include plain
ASCII without markup, Texinfo input format, LaTeX input format, SGML
or XML using a publicly available DTD, and standard-conforming simple
HTML, PostScript or PDF designed for human modification.  Examples of
transparent image formats include PNG, XCF and JPG.  Opaque formats
include proprietary formats that can be read and edited only by
proprietary word processors, SGML or XML for which the DTD and/or
processing tools are not generally available, and the
machine-generated HTML, PostScript or PDF produced by some word
processors for output purposes only.

The ``\textbf{Title Page}'' means, for a printed book, the title page itself,
plus such following pages as are needed to hold, legibly, the material
this License requires to appear in the title page.  For works in
formats which do not have any title page as such, ``Title Page'' means
the text near the most prominent appearance of the work's title,
preceding the beginning of the body of the text.

The ``\textbf{publisher}'' means any person or entity that distributes
copies of the Document to the public.

A section ``\textbf{Entitled XYZ}'' means a named subunit of the Document whose
title either is precisely XYZ or contains XYZ in parentheses following
text that translates XYZ in another language.  (Here XYZ stands for a
specific section name mentioned below, such as ``\textbf{Acknowledgements}'',
``\textbf{Dedications}'', ``\textbf{Endorsements}'', or ``\textbf{History}''.)  
To ``\textbf{Preserve the Title}''
of such a section when you modify the Document means that it remains a
section ``Entitled XYZ'' according to this definition.

The Document may include Warranty Disclaimers next to the notice which
states that this License applies to the Document.  These Warranty
Disclaimers are considered to be included by reference in this
License, but only as regards disclaiming warranties: any other
implication that these Warranty Disclaimers may have is void and has
no effect on the meaning of this License.


\section{VERBATIM COPYING}

You may copy and distribute the Document in any medium, either
commercially or noncommercially, provided that this License, the
copyright notices, and the license notice saying this License applies
to the Document are reproduced in all copies, and that you add no other
conditions whatsoever to those of this License.  You may not use
technical measures to obstruct or control the reading or further
copying of the copies you make or distribute.  However, you may accept
compensation in exchange for copies.  If you distribute a large enough
number of copies you must also follow the conditions in section~3.

You may also lend copies, under the same conditions stated above, and
you may publicly display copies.


\section{COPYING IN QUANTITY}

If you publish printed copies (or copies in media that commonly have
printed covers) of the Document, numbering more than 100, and the
Document's license notice requires Cover Texts, you must enclose the
copies in covers that carry, clearly and legibly, all these Cover
Texts: Front-Cover Texts on the front cover, and Back-Cover Texts on
the back cover.  Both covers must also clearly and legibly identify
you as the publisher of these copies.  The front cover must present
the full title with all words of the title equally prominent and
visible.  You may add other material on the covers in addition.
Copying with changes limited to the covers, as long as they preserve
the title of the Document and satisfy these conditions, can be treated
as verbatim copying in other respects.

If the required texts for either cover are too voluminous to fit
legibly, you should put the first ones listed (as many as fit
reasonably) on the actual cover, and continue the rest onto adjacent
pages.

If you publish or distribute Opaque copies of the Document numbering
more than 100, you must either include a machine-readable Transparent
copy along with each Opaque copy, or state in or with each Opaque copy
a computer-network location from which the general network-using
public has access to download using public-standard network protocols
a complete Transparent copy of the Document, free of added material.
If you use the latter option, you must take reasonably prudent steps,
when you begin distribution of Opaque copies in quantity, to ensure
that this Transparent copy will remain thus accessible at the stated
location until at least one year after the last time you distribute an
Opaque copy (directly or through your agents or retailers) of that
edition to the public.

It is requested, but not required, that you contact the authors of the
Document well before redistributing any large number of copies, to give
them a chance to provide you with an updated version of the Document.


\section{MODIFICATIONS}

You may copy and distribute a Modified Version of the Document under
the conditions of sections 2 and 3 above, provided that you release
the Modified Version under precisely this License, with the Modified
Version filling the role of the Document, thus licensing distribution
and modification of the Modified Version to whoever possesses a copy
of it.  In addition, you must do these things in the Modified Version:

\begin{itemize}
\item[A.] 
   Use in the Title Page (and on the covers, if any) a title distinct
   from that of the Document, and from those of previous versions
   (which should, if there were any, be listed in the History section
   of the Document).  You may use the same title as a previous version
   if the original publisher of that version gives permission.
   
\item[B.]
   List on the Title Page, as authors, one or more persons or entities
   responsible for authorship of the modifications in the Modified
   Version, together with at least five of the principal authors of the
   Document (all of its principal authors, if it has fewer than five),
   unless they release you from this requirement.
   
\item[C.]
   State on the Title page the name of the publisher of the
   Modified Version, as the publisher.
   
\item[D.]
   Preserve all the copyright notices of the Document.
   
\item[E.]
   Add an appropriate copyright notice for your modifications
   adjacent to the other copyright notices.
   
\item[F.]
   Include, immediately after the copyright notices, a license notice
   giving the public permission to use the Modified Version under the
   terms of this License, in the form shown in the Addendum below.
   
\item[G.]
   Preserve in that license notice the full lists of Invariant Sections
   and required Cover Texts given in the Document's license notice.
   
\item[H.]
   Include an unaltered copy of this License.
   
\item[I.]
   Preserve the section Entitled ``History'', Preserve its Title, and add
   to it an item stating at least the title, year, new authors, and
   publisher of the Modified Version as given on the Title Page.  If
   there is no section Entitled ``History'' in the Document, create one
   stating the title, year, authors, and publisher of the Document as
   given on its Title Page, then add an item describing the Modified
   Version as stated in the previous sentence.
   
\item[J.]
   Preserve the network location, if any, given in the Document for
   public access to a Transparent copy of the Document, and likewise
   the network locations given in the Document for previous versions
   it was based on.  These may be placed in the ``History'' section.
   You may omit a network location for a work that was published at
   least four years before the Document itself, or if the original
   publisher of the version it refers to gives permission.
   
\item[K.]
   For any section Entitled ``Acknowledgements'' or ``Dedications'',
   Preserve the Title of the section, and preserve in the section all
   the substance and tone of each of the contributor acknowledgements
   and/or dedications given therein.
   
\item[L.]
   Preserve all the Invariant Sections of the Document,
   unaltered in their text and in their titles.  Section numbers
   or the equivalent are not considered part of the section titles.
   
\item[M.]
   Delete any section Entitled ``Endorsements''.  Such a section
   may not be included in the Modified Version.
   
\item[N.]
   Do not retitle any existing section to be Entitled ``Endorsements''
   or to conflict in title with any Invariant Section.
   
\item[O.]
   Preserve any Warranty Disclaimers.
\end{itemize}

If the Modified Version includes new front-matter sections or
appendices that qualify as Secondary Sections and contain no material
copied from the Document, you may at your option designate some or all
of these sections as invariant.  To do this, add their titles to the
list of Invariant Sections in the Modified Version's license notice.
These titles must be distinct from any other section titles.

You may add a section Entitled ``Endorsements'', provided it contains
nothing but endorsements of your Modified Version by various
parties---for example, statements of peer review or that the text has
been approved by an organization as the authoritative definition of a
standard.

You may add a passage of up to five words as a Front-Cover Text, and a
passage of up to 25 words as a Back-Cover Text, to the end of the list
of Cover Texts in the Modified Version.  Only one passage of
Front-Cover Text and one of Back-Cover Text may be added by (or
through arrangements made by) any one entity.  If the Document already
includes a cover text for the same cover, previously added by you or
by arrangement made by the same entity you are acting on behalf of,
you may not add another; but you may replace the old one, on explicit
permission from the previous publisher that added the old one.

The author(s) and publisher(s) of the Document do not by this License
give permission to use their names for publicity for or to assert or
imply endorsement of any Modified Version.


\section{COMBINING DOCUMENTS}

You may combine the Document with other documents released under this
License, under the terms defined in section~4 above for modified
versions, provided that you include in the combination all of the
Invariant Sections of all of the original documents, unmodified, and
list them all as Invariant Sections of your combined work in its
license notice, and that you preserve all their Warranty Disclaimers.

The combined work need only contain one copy of this License, and
multiple identical Invariant Sections may be replaced with a single
copy.  If there are multiple Invariant Sections with the same name but
different contents, make the title of each such section unique by
adding at the end of it, in parentheses, the name of the original
author or publisher of that section if known, or else a unique number.
Make the same adjustment to the section titles in the list of
Invariant Sections in the license notice of the combined work.

In the combination, you must combine any sections Entitled ``History''
in the various original documents, forming one section Entitled
``History''; likewise combine any sections Entitled ``Acknowledgements'',
and any sections Entitled ``Dedications''.  You must delete all sections
Entitled ``Endorsements''.

\section{COLLECTIONS OF DOCUMENTS}

You may make a collection consisting of the Document and other documents
released under this License, and replace the individual copies of this
License in the various documents with a single copy that is included in
the collection, provided that you follow the rules of this License for
verbatim copying of each of the documents in all other respects.

You may extract a single document from such a collection, and distribute
it individually under this License, provided you insert a copy of this
License into the extracted document, and follow this License in all
other respects regarding verbatim copying of that document.


\section{AGGREGATION WITH INDEPENDENT WORKS}

A compilation of the Document or its derivatives with other separate
and independent documents or works, in or on a volume of a storage or
distribution medium, is called an ``aggregate'' if the copyright
resulting from the compilation is not used to limit the legal rights
of the compilation's users beyond what the individual works permit.
When the Document is included in an aggregate, this License does not
apply to the other works in the aggregate which are not themselves
derivative works of the Document.

If the Cover Text requirement of section~3 is applicable to these
copies of the Document, then if the Document is less than one half of
the entire aggregate, the Document's Cover Texts may be placed on
covers that bracket the Document within the aggregate, or the
electronic equivalent of covers if the Document is in electronic form.
Otherwise they must appear on printed covers that bracket the whole
aggregate.


\section{TRANSLATION}

Translation is considered a kind of modification, so you may
distribute translations of the Document under the terms of section~4.
Replacing Invariant Sections with translations requires special
permission from their copyright holders, but you may include
translations of some or all Invariant Sections in addition to the
original versions of these Invariant Sections.  You may include a
translation of this License, and all the license notices in the
Document, and any Warranty Disclaimers, provided that you also include
the original English version of this License and the original versions
of those notices and disclaimers.  In case of a disagreement between
the translation and the original version of this License or a notice
or disclaimer, the original version will prevail.

If a section in the Document is Entitled ``Acknowledgements'',
``Dedications'', or ``History'', the requirement (section~4) to Preserve
its Title (section~1) will typically require changing the actual
title.


\section{TERMINATION}

You may not copy, modify, sublicense, or distribute the Document
except as expressly provided under this License.  Any attempt
otherwise to copy, modify, sublicense, or distribute it is void, and
will automatically terminate your rights under this License.

However, if you cease all violation of this License, then your license
from a particular copyright holder is reinstated (a) provisionally,
unless and until the copyright holder explicitly and finally
terminates your license, and (b) permanently, if the copyright holder
fails to notify you of the violation by some reasonable means prior to
60 days after the cessation.

Moreover, your license from a particular copyright holder is
reinstated permanently if the copyright holder notifies you of the
violation by some reasonable means, this is the first time you have
received notice of violation of this License (for any work) from that
copyright holder, and you cure the violation prior to 30 days after
your receipt of the notice.

Termination of your rights under this section does not terminate the
licenses of parties who have received copies or rights from you under
this License.  If your rights have been terminated and not permanently
reinstated, receipt of a copy of some or all of the same material does
not give you any rights to use it.


\section{REVISIONS OF THIS LICENSE}

The Free Software Foundation may publish new, revised versions
of the GNU Free Documentation License from time to time.  Such new
versions will be similar in spirit to the present version, but may
differ in detail to address new problems or concerns.  See
\texttt{http://www.gnu.org/copyleft/}.

Each version of the License is given a distinguishing version number.
If the Document specifies that a particular numbered version of this
License ``or any later version'' applies to it, you have the option of
following the terms and conditions either of that specified version or
of any later version that has been published (not as a draft) by the
Free Software Foundation.  If the Document does not specify a version
number of this License, you may choose any version ever published (not
as a draft) by the Free Software Foundation.  If the Document
specifies that a proxy can decide which future versions of this
License can be used, that proxy's public statement of acceptance of a
version permanently authorizes you to choose that version for the
Document.

\section{RELICENSING}

``Massive Multiauthor Collaboration Site'' (or ``MMC Site'') means any
World Wide Web server that publishes copyrightable works and also
provides prominent facilities for anybody to edit those works.  A
public wiki that anybody can edit is an example of such a server.  A
``Massive Multiauthor Collaboration'' (or ``MMC'') contained in the
site means any set of copyrightable works thus published on the MMC
site.

``CC-BY-SA'' means the Creative Commons Attribution-Share Alike 3.0
license published by Creative Commons Corporation, a not-for-profit
corporation with a principal place of business in San Francisco,
California, as well as future copyleft versions of that license
published by that same organization.

``Incorporate'' means to publish or republish a Document, in whole or
in part, as part of another Document.

An MMC is ``eligible for relicensing'' if it is licensed under this
License, and if all works that were first published under this License
somewhere other than this MMC, and subsequently incorporated in whole
or in part into the MMC, (1) had no cover texts or invariant sections,
and (2) were thus incorporated prior to November 1, 2008.

The operator of an MMC Site may republish an MMC contained in the site
under CC-BY-SA on the same site at any time before August 1, 2009,
provided the MMC is eligible for relicensing.


\section*{ADDENDUM: How to use this License for your documents}
\addcontentsline{toc}{section}{ADDENDUM: How to use this License for your documents}

To use this License in a document you have written, include a copy of
the License in the document and put the following copyright and
license notices just after the title page:

\bigskip
\begin{quote}
    Copyright \copyright{}  YEAR  YOUR NAME.
    Permission is granted to copy, distribute and/or modify this document
    under the terms of the GNU Free Documentation License, Version 1.3
    or any later version published by the Free Software Foundation;
    with no Invariant Sections, no Front-Cover Texts, and no Back-Cover Texts.
    A copy of the license is included in the section entitled ``GNU
    Free Documentation License''.
\end{quote}
\bigskip
    
If you have Invariant Sections, Front-Cover Texts and Back-Cover Texts,
replace the ``with \dots\ Texts.''\ line with this:

\bigskip
\begin{quote}
    with the Invariant Sections being LIST THEIR TITLES, with the
    Front-Cover Texts being LIST, and with the Back-Cover Texts being LIST.
\end{quote}
\bigskip
    
If you have Invariant Sections without Cover Texts, or some other
combination of the three, merge those two alternatives to suit the
situation.

If your document contains nontrivial examples of program code, we
recommend releasing these examples in parallel under your choice of
free software license, such as the GNU General Public License,
to permit their use in free software.
\newpage
\part*{References}%
\addcontentsline{toc}{part}{References}
\bibliography{common/references,CITATION}
\end{document}

%%% Local Variables:
%%% mode: latex
%%% TeX-master: "probability"
%%% End:
