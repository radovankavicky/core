\documentclass[nohyper]{external/tufte-handout}
% The hyperref is loaded in the preamble with additional arguments to
% avoid some xelatex warnings.

\title{Core Econometrics Textbook Part 5: Linear Regression}
% Copyright © 2013, authors of the "Core Econometrics;" a
% complete list of authors is available in the file AUTHORS.tex.

% Permission is granted to copy, distribute and/or modify this
% document under the terms of the GNU Free Documentation License,
% Version 1.3 or any later version published by the Free Software
% Foundation; with no Invariant Sections, no Front-Cover Texts, and no
% Back-Cover Texts.  A copy of the license is included in the file
% LICENSE.tex and is also available online at
% <http://www.gnu.org/copyleft/fdl.html>.

\newcommand{\version}{0.7.1}
\newcommand{\releasedate}{10 Dec. 2013}

%%% Local Variables:
%%% mode: latex
%%% TeX-master: "core_econometrics"
%%% End:

\author{Gray Calhoun} 
% (This comment is repeated in the Makefile)
% I'm still not sure the best way to do author information; I'm much
% more concerned in the long run about how different attributation
% styles would make someone more or less likely to contribute to an
% existing text or to license an existing draft.  For now, there's
% only one author, so I'll put myself as the author.  If someone else
% contributes any edits, etc., I'll change it to {Gray Calhoun and
% EFLP}.  If anyone wants to contribute a lot of original material and
% wants named authorship, please email the mailing list so we can
% discuss merging projects.

\usepackage{amssymb,amsmath,amsthm,verbatim}
\usepackage{fontspec,unicode-math,xltxtra,xunicode,booktabs}
\setromanfont[Ligatures=TeX]{TeX Gyre Pagella}
\setsansfont[Ligatures=TeX,Scale=MatchLowercase]{TeX Gyre Heros}
% \setmonofont[Scale=MatchLowercase]{Inconsolata}
\setmathfont{Asana-Math}

\frenchspacing
\setcounter{secnumdepth}{1}
\setcounter{tocdepth}{1}
\renewcommand\bibname{}
\renewcommand\refname{}
\renewcommand\contentsname{}
\bibliographystyle{abbrvnat}
\setcitestyle{round}
\newcommand{\email}[1]{\href{mailto:#1}{\nolinkurl{#1}}}
\newcommand{\homepage}{\url{http://www.econometricslibrary.org}}
\newcommand{\maillist}{\email{econometricslibrary@librelist.com}}
\newcommand{\bugtrack}%
{\url{https://github.com/EconometricsLibrary/CoreEconometricsText/issues}}

% Workaround for bugs in the tufte-latex class
\renewcommand\smallcapsspacing[1]{{\addfontfeature{LetterSpace = 8}\scshape#1}}
\renewcommand\allcapsspacing[1]{{\addfontfeature{LetterSpace = 15}#1}}
% Getting waringings from latexmk with default tufte-latex hyperref
\usepackage[unicode,pdfencoding=auto,hyperfootnotes=false,hidelinks]{hyperref}

\newcommand{\BibTeX}{Bib\!\TeX}
\newcommand{\pvalue}{\ensuremath{p}-value}
\newcommand{\ftest}{\ensuremath{F}-test}
\newcommand{\ttest}{\ensuremath{t}-test}

% Math shortcuts
\renewcommand{\Pr}{\operatorname{Pr}}

\DeclareMathOperator{\1}{1}
\DeclareMathOperator{\abs}{abs}
\DeclareMathOperator{\avar}{avar}
\DeclareMathOperator{\bias}{bias}
\DeclareMathOperator{\corr}{corr}
\DeclareMathOperator{\cov}{cov}
\DeclareMathOperator{\E}{E}
\DeclareMathOperator{\median}{median}
\DeclareMathOperator{\mse}{mse}
\DeclareMathOperator{\rank}{rank}
\DeclareMathOperator{\range}{range}
\DeclareMathOperator{\sd}{sd}
\DeclareMathOperator{\tr}{tr}
\DeclareMathOperator{\var}{var}

\DeclareMathOperator*{\argmax}{arg\,max}
\DeclareMathOperator*{\argmin}{arg\,min}
\DeclareMathOperator*{\plim}{plim}

\DeclareMathOperator{\binomial}{binomial}
\DeclareMathOperator{\bernoulli}{bernoulli}
\DeclareMathOperator{\invWishart}{inverse\ Wishart}
\DeclareMathOperator{\N}{N}
\DeclareMathOperator{\uniform}{uniform}

\newcommand{\BB}{\ensuremath{\mathbb{B}}}
\newcommand{\NN}{\ensuremath{\mathbb{N}}}
\newcommand{\PP}{\ensuremath{\mathbb{P}}}
\newcommand{\QQ}{\ensuremath{\mathbb{Q}}}
\newcommand{\RR}{\ensuremath{\mathbb{R}}}
\newcommand{\RRᵏ}{\ensuremath{\mathbb{R}ᵏ}}
\newcommand{\RRⁿ}{\ensuremath{\mathbb{R}ⁿ}}
\newcommand{\RRb}{\ensuremath{\bar{\mathbb{R}}}}
\newcommand{\ZZ}{\ensuremath{\mathbb{Z}}}

\newcommand{\Fs}{\ensuremath{\mathcal{F}}}
\newcommand{\Gs}{\ensuremath{\mathcal{G}}}
\newcommand{\Ps}{\ensuremath{\mathcal{P}}}

\newcommand{\ov}[2][1]{\tfrac{#1}{#2}}
\newcommand{\iid}{i.i.d.}

\newcommand{\ep}{\varepsilon}
\newcommand{\eph}{\hat{\varepsilon}}

\newcommand{\ah}{\hat{a}}
\newcommand{\αh}{\hat{α}}
\newcommand{\bh}{\hat{b}}
\newcommand{\βb}{\bar{β}}
\newcommand{\βh}{\hat{β}}
\newcommand{\βt}{\tilde{β}}
\newcommand{\eh}{\hat{e}}
\newcommand{\εb}{\bar{ε}}
\newcommand{\εh}{\hat{ε}}
\newcommand{\εt}{\tilde{ε}}
\newcommand{\ηh}{\hat{η}}
\newcommand{\Fh}{\hat{F}}
\newcommand{\λh}{\hat{λ}}
\newcommand{\μb}{\bar{μ}}
\newcommand{\μh}{\hat{μ}}
\newcommand{\Ωh}{\hat{Ω}}
\newcommand{\Qh}{\hat{Q}}
\newcommand{\Σh}{\hat{Σ}}
\newcommand{\sh}{\hat{s}}
\newcommand{\σh}{\hat{σ}}
\newcommand{\σb}{\bar{σ}}
\newcommand{\θh}{\hat{θ}}
\newcommand{\Vh}{\hat{V}}
\newcommand{\Xb}{\bar{X}}
\newcommand{\Xc}{\mathcal{X}}
\newcommand{\Xh}{\hat{X}}
\newcommand{\Xt}{\tilde{X}}
\newcommand{\Yb}{\bar{Y}}
\newcommand{\Yh}{\hat{Y}}
\newcommand{\Yt}{\tilde{Y}}
\newcommand{\yh}{\hat{y}}

\newcommand{\dx}{\,dx}
\newcommand{\dy}{\,dy}
\newcommand{\dμ}{\,dμ}
\newcommand{\dθ}{\,dθ}
\renewcommand{\dz}{\,dz}

\newtheorem{thm}{Theorem}[section]
\newtheorem{defn}{Definition}[section]
\newtheorem{ex}{Example}[section]
\newtheorem{asmp}{Assumption}[section]


\begin{document}
\maketitle

\bigskip\noindent%
Copyright © 2013, the authors of the \textit{Core Econometrics Textbook}.
A complete author list is included in Appendix A of this document.

Permission is granted to copy, distribute and/or modify this document
under the terms of the GNU Free Documentation License, Version 1.3 or
any later version published by the Free Software Foundation; with no
Invariant Sections, no Front-Cover Texts, and no Back-Cover Texts.  A
copy of the license is included in Appendix B of this document and is
also available online at \url{http://www.gnu.org/copyleft/fdl.html}.

This text was produced as part of the Econometrics Free Library
Project.  The project's goal is to produce high-quality free (and
open-source) econometrics textbooks and reference material.  More
information about this project is available at its homepage,
\homepage, including information on how
to participate.  This text is typeset in \LaTeX\ and there is a link
to the source of the document at the project's homepage as well.

Please cite this document as
\begin{itemize}
\item[] \bibentry{eflp-core}
\item[] \verbatiminput{CITATION.bib}
\end{itemize}
The \BibTeX\ record is provided for your convenience.
If the year and version are not specified, please see the files
\texttt{README.md} or \texttt{VERSION.tex} for directions on how to
find them.

\addcontentsline{toc}{part}{Table of Contents}
\tableofcontents

% Copyright (c) 2013, authors of "Core Econometrics;" a
% complete list of authors is available in the file AUTHORS.tex.

% Permission is granted to copy, distribute and/or modify this
% document under the terms of the GNU Free Documentation License,
% Version 1.3 or any later version published by the Free Software
% Foundation; with no Invariant Sections, no Front-Cover Texts, and no
% Back-Cover Texts.  A copy of the license is included in the file
% LICENSE.tex and is also available online at
% <http://www.gnu.org/copyleft/fdl.html>.

\chapter{Introduction}
\section{Overview of linear regression}

\paragraph{description of the goals}
\begin{itemize}
\item What are we hoping to do?
\begin{itemize}
\item we probably have in mind changing one of these variables
\begin{itemize}
\item conceivably, could change number of teachers per student
\item conceivably, could change money available per student
\end{itemize}
\item so, we want something like (informally) E(change in reading
          score | an independent body makes a 1\% change in expenditure)
\end{itemize}
\item what makes econometrics distinctive?
\begin{itemize}
\item we \underline{don't} always get to observe changes in expenditure and
          changes in scores
\begin{itemize}
\item sometimes we do
\item sometimes we see levels of expenditure and levels scores
\end{itemize}
\item we almost never get to observe \underline{changes made by an independent           body}.
\begin{itemize}
\item often, you'll see changes that were made for some reason
\end{itemize}
\end{itemize}
\end{itemize}

\paragraph{\underline{the key strategy for empirical research}}
\begin{itemize}
\item an important message before we start
\begin{itemize}
\item think about what experiment you would \textbf{want to design} if you
          could
\begin{itemize}
\item one approach if we want to understand how expenditure affects student
            reading test scores, we'd want to take all of the schools
\item \textbf{randomly} and \textbf{independently} change their funding (leave
            some unchanged)
\item observe the change in test scores the next year
\end{itemize}
\item then think hard about how closely the process that created
          your data matches that experiment
\begin{itemize}
\item this can be an iterative process
\end{itemize}
\item if you can not think of an experiment, the question might not
          be an empirical question
\begin{itemize}
\item sometimes this happens when you think about changing
            preference parameters
\end{itemize}
\end{itemize}
\end{itemize}

\paragraph{description of the ``environment''}
\begin{itemize}
\item have n data points (observations)
\item have k+1 regressors
\item believe that there's a relationship between the dependent
        variable and the regressors:
        \[\begin{pmatrix} Y₁ \\ ⋮ \\ Y_n\end{pmatrix}
        = \begin{pmatrix} 
        X_{1,0} & X_{1,1} & ⋯ & X_{1,k} \\
        ⋮ \\
        X_{n,0} & X_{n,1} & ⋯ & X_{n,k} \end{pmatrix}
        \begin{pmatrix}  β₀ \\ ⋮ \\ β_k \end{pmatrix}+ 
        \begin{pmatrix} \ep₁ \\ ⋮ \\ \ep_n \end{pmatrix} \]
\begin{itemize}
\item $Y$ is the response vector
\item $X$ is the design matrix
\item $β$ is the vector of unknown coefficients
\item $\ep$ is the error
\item so, the ith row of this relationship is
  \[Y_i = X_{i,0} β₀ + X_{i,1} β₁ + ⋯ + X_{i,k} β_k + \ep_i\]
\end{itemize}
\end{itemize}

\subsection{informal motivation}
\begin{itemize}
\item In matrix form: $Y = X β + \ep$
\item if the error is small, then $Y ≈ Xβ$
\begin{itemize}
\item so $X'Y ≈ X'X β$
\begin{itemize}
\item $X'X$ will be invertible if $X$ has full rank
\end{itemize}
\item and then $β ≈ (X'X)^{-1} X'Y$
\end{itemize}
\end{itemize}

\subsection{outline of the second half of the course}

\begin{itemize}
\item we're going to talk mostly about this estimator and slight
       variations of it
\begin{itemize}
\item what it does (under what assumptions)
\item what it \textbf{doesn't} do
\end{itemize}
\item next semester, you'll discuss improvements
\end{itemize}

%%% Local Variables: 
%%% mode: latex
%%% TeX-master: "../core_econometrics"
%%% End: 

% Copyright (c) 2013, authors of "Core Econometrics;" a
% complete list of authors is available in the file AUTHORS.tex.

% Permission is granted to copy, distribute and/or modify this
% document under the terms of the GNU Free Documentation License,
% Version 1.3 or any later version published by the Free Software
% Foundation; with no Invariant Sections, no Front-Cover Texts, and no
% Back-Cover Texts.  A copy of the license is included in the file
% LICENSE.tex and is also available online at
% <http://www.gnu.org/copyleft/fdl.html>.

\chapter{Modeling with linear regression}
\label{ch_modeling}
\tryinput{regression/modeling_macros.tex}
\providecommand{\exvar}{?}

\section{What happens when the model is wrong}

\begin{itemize}[leftmargin=0pt]

\item You've heard of \emph{omitted variable bias} but we need to
  think about what that means.  Suppose that we have a single
  regressor, the DGP is
  \begin{equation*}
    y_i = \alpha_0 + \alpha_1 x_i + \alpha_2 x^2_i + \vep_i
  \end{equation*}
  where (for argument's sake) $(x_i, \vep_i) \sim \iid\ (0,I)$ with
  $\vep_i$ and $x_i$ independent.  But now suppose we estimate the
  model
  \begin{equation*}
    y_i = \betah_0 + \betah_1 x_i
  \end{equation*}
  with OLS.  What are the properties of $\betah_0$ and $\betah_1$?  We know
  that
  \begin{equation*}
    \betah =
    \begin{pmatrix}
      1 & (1/n) \sum_{i=1}^n x_i \\
      (1/n) \sum_{i=1}^n x_i & (1/n) \sum_{i=1}^n x^2_i
    \end{pmatrix}^{-1}
    \begin{pmatrix}
      (1/n) \sum_{i=1}^n (\alpha_0 + \alpha_1 x_i + \alpha_2 x_i^2 + \vep_i ) \\
      (1/n) \sum_{i=1}^n x_i (\alpha_0 + \alpha_1 x_i + \alpha_2 x_i^2 + \vep_i)
    \end{pmatrix}.
  \end{equation*}
  Since we've assumed finite variance (not really an important part of
  this example) we can apply LLNs to everything, so
  \begin{gather*}
     (1/n) \sum_{i=1}^n x_i \to^p 0 \\
     (1/n) \sum_{i=1}^n x^2_i \to^p 1 \\
     (1/n) \sum_{i=1}^n (\alpha_0 + \alpha_1 x_i + \alpha_2 x_i^2 + \vep_i ) \to^p  \alpha_0 + \alpha_1 \\
     (1/n) \sum_{i=1}^n x_i (\alpha_0 + \alpha_1 x_i + \alpha_2 x_i^2 + \vep_i) \to^p
     \alpha_1 + \alpha_2 \E x_i^3
  \end{gather*}
  so
  \begin{equation*}
    \betah \to^p
    \begin{pmatrix} 1 & 0 \\ 0 & 1 \end{pmatrix}^{-1}
    \begin{pmatrix} \alpha_0 + \alpha_2 \\ \alpha_1 + \alpha_2 \end{pmatrix}
    = \begin{pmatrix} \alpha_0 + \alpha_2 \\ \alpha_1 + \alpha_2 \end{pmatrix}
  \end{equation*}

  Usually we don't care about the intercept, but we almost always care
  about the slope, and the OLS estimator for the slope is
  asymptotically biased.

\item Now, how do we interpret those coefficients?  They minimize the
  SSR among the class of models linear in $x_i$.  We can show that the
  first order conditions for
  \begin{equation*}
    \argmin_{\beta_0, \beta_1} \E (y_i - \beta_0 - \beta_1 x_i)^2 
  \end{equation*}
  are satisfied by the above $\plim$s:
  \begin{align*}
    0
    &= (\partial /\partial \beta) \E (y_i - \beta_0 - \beta_1 x_i)^2 \\
    &= - 2
    \begin{pmatrix}
      \E (y_i - \beta_0 - \beta_1 x_i) \\
      \E (y_i - \beta_0 - \beta_1 x_i) x_i
    \end{pmatrix} \\
    &= - 2
    \begin{pmatrix}
      \E (\alpha_0 + \alpha_1 x_i + \alpha_2 x_i^2 + \vep_i - \beta_0 - \beta_1 x_i) \\
      \E (\alpha_0 + \alpha_1 x_i + \alpha_2 x_i^2 + \vep_i - \beta_0 - \beta_1 x_i) x_i
    \end{pmatrix} \\
    &= - 2
    \begin{pmatrix}
      \alpha_0 + \alpha_2 - \beta_0 \\
      \alpha_1 + \alpha_2 \E x_i^3 - \beta_1
    \end{pmatrix}
  \end{align*}
  and this is satisfied only when $\beta_0 = \alpha_0 + \alpha_2$ and
  $\beta_1 = \alpha_1 + \alpha_2 \E x_i^3$.

\item Now, is this what you want?  Sometimes (for prediction) this
  isn't too bad on average (especially if $\alpha_2$ is small).  But
  usually this isn't what we want.

  But the thing to notice is that the moment and weak dependence
  conditions ensure that the OLS estimator converges to
  \emph{something} and is asymptotically normal around that parameter
  value, even if it's not what we want to estimate.

\item But we can also look at the variance of the OLS estimator here
  (where $X$ is just a column of ones and the column of $x_i$'s)
  \begin{align*}
    \E(\betah \mid X)
    &= ((1/n) X'X)^{-1}
    \begin{pmatrix}
      (1/n) \sum_{i=1}^n \E(\alpha_0 + \alpha_1 x_i + \alpha_2 x_i^2 + \vep_i \mid X) \\
      (1/n) \sum_{i=1}^n \E(x_i (\alpha_0 + \alpha_1 x_i + \alpha_2 x_i^2 + \vep_i) \mid X)
    \end{pmatrix} \\
    &= ((1/n) X'X)^{-1}
    \begin{pmatrix}
      (1/n) \sum_{i=1}^n (\alpha_0 + \alpha_1 x_i + \alpha_2 x_i^2) \\
      (1/n) \sum_{i=1}^n (\alpha_0 x_i + \alpha_1 x_i^2 + \alpha_2 x_i^3)
    \end{pmatrix}
  \end{align*}
  and so
  \begin{align*}
    \var(\betah \mid X)
    &= ((1/n) X'X)^{-1}
      (1/n^2) \sum_{i=1}^n \E(\vep_i^2 \mid X)
      \begin{pmatrix} 1 & x_i \\ x_i & x_i^2 \end{pmatrix}
      ((1/n) X'X)^{-1} \\
    &= \sigma^2 (X'X)^{-1}
  \end{align*}
  under conditional homoskedasticity.  The natural question is how
  this variance compares to that of the full model.

  To simplify comparisons, impose that $\beta_0$ is known to be zero.  Then
  the restricted variance just becomes $\sigma^2 / \sum_{i=1}^n x_i^2$ and the full
  model's variance-covariance matrix is
  \begin{equation*} \sigma^2
    \begin{pmatrix}
      \sum_{i=1}^n x_i^2 & \sum_{i=1}^n x_i^3 \\
      \sum_{i=1}^n x_i^3 & \sum_{i=1}^n x_i^4
    \end{pmatrix}^{-1}
    = \frac{\sigma^2}{\sum_{i=1}^n x_i^2 \sum_{i=1}^n x_i^4 - \big( \sum_{i=1}^n x_i^3 \big)^2}
    \begin{pmatrix}
      \sum_{i=1}^n x_i^4 & -\sum_{i=1}^n x_i^3 \\
      -\sum_{i=1}^n x_i^3 & \sum_{i=1}^n x_i^3
    \end{pmatrix}.
  \end{equation*}
  Now, we're just interested in the top left element of that matrix,
  which is
  \begin{equation*}
    \frac{\sigma^2 \sum_{i=1}^n x_i^4}%
    {\sum_{i=1}^n x_i^2 \sum_{i=1}^n x_i^4 - \big( \sum_{i=1}^n x_i^3 \big)^2}.
  \end{equation*}

  We can compare their reciprocals easily; the reciprocal of the full
  model's variance is
  \begin{equation*}
    \sum_{i=1}^n x_i^2 / \sigma^2 -
    \big( \sum_{i=1}^n x_i^3 \big)^2 \Big/ \sigma^2 \sum_{i=1}^n x_i^4
  \end{equation*}
  which is clearly less than $\sum_{i=1}^n x_i^2 / \sigma^2$, the
  reciprocal of the variance of the restricted estimator, so the
  restricted estimator has smaller conditional variance.

\item Don't go overboard.  We can come up with examples where the
  unrestricted model has lower variance too.
  \begin{hw}[Regression model with omitted exogenous variables]
    Suppose that $(Y_i, X_i, Z_i)$ are \iid\ such that
    \begin{equation*}
      \E(Y_i \mid X_i, Z_i) = \alpha' X_i + \beta' Z_i
    \end{equation*}
    and assume that $X_i$ and $Z_i$ have finite second moments and
    that $\E(Z_i \mid X_i) = 0$.  This second condition makes the OLS
    estimator of $Y$ on $X$ unbiased for $\alpha$.  Find parameter
    values for $\beta$ and $\alpha$ and $\var(Y_i \mid X_i, Z_i)$ so
    that the OLS estimator of the regression of $Y$ on $X$ has higher
    variance than the OLS estimator from the regression on $X$ and
    $Z$.

    Then verify your results by writing a short program to conduct a
    Monte Carlo study comparing the variance of each estimator.
  \end{hw}

\item There's nothing special about the original variables we're
  given, and often they need to be transformed.  We can use OLS
  whenever we have models of the form
  \begin{equation*}
    y_i = \beta_0 + \beta_1 g_1(z_i) + \cdots + \beta_k g_k(z_i) + \vep_i
  \end{equation*}
  as long as the moment and dependence conditions hold.  And, of
  course, the coefficients estimated are going to be the ones that
  minimize the expected SSR, regardless of the true DGP.

\item An extreme example of this is for categorical data.  You will
  often see this in marketing-type surveys where the numbers 1--5
  represent ``strongly agree'' through ``strongly disagree.''

  \begin{ex}[Categorical regressor]
    Suppose that $y_i$ is income for individual $i$ and $x_i$
    represents highest level of education achieved by individual $i$,
    where
    \begin{equation*}
      x_i =
      \begin{cases}
        0 & \text{if individual $i$ has not completed high school} \\
        1 & \text{completed high school but no higher degree} \\
        2 & \text{completed college but no higher degree} \\
        3 & \text{completed a master's degree or entered a PhD
          program, but no higher degree} \\
        4 & \text{completed a PhD program}.
      \end{cases}
    \end{equation*}

    It would be a mistake to regress $y_i$ on the raw values of $x_i$.
    The numbers in $x_i$ \emph{don't mean anything}, but that
    regression would impose an assumption that the difference in
    expected income between someone with no degree ($x_i = 0$) and
    someone with a high school diploma ($x_i = 1$) is exactly half the
    difference between someone with a college diploma and someone with
    a PhD.  So, obviously, estimating that sort of model is basically
    useless for interpretation, etc.  You could still interpret the
    model as a best linear predictor, just like we did earlier, but
    why?

    Instead, for this sort of model you want to create separate
    indicator variables:
    \begin{align*}
      D_{i1} &= \ind\{x_i = 1\} &
      D_{i2} &= \ind\{x_i = 2\} \\
      D_{i3} &= \ind\{x_i = 3\} &
      D_{i4} &= \ind\{x_i = 4\}
    \end{align*}
    and then estimate the model
    \begin{equation*}
      y_i = \beta_0 + \sum_{j=1}^4 \beta_j D_{ij} + \vep_i.
    \end{equation*}
    
    Now each $\beta_j$ can be interpreted as the conditional
    expectation given each educational level, and the difference in
    the coefficients gives the difference in conditional expectations.
  \end{ex}

  NB: even if you estimate the correct model (with indicator
  variables) it's likely that you'll only be able to estimate the mean
  of the population of people with college degrees, etc.  It's
  unlikely that you're going to be able to estimate the expected
  change in someone's earnings, etc., that would result from a change
  in their education level.
  
\end{itemize}

\section{Model selection}

[need to add material on graphics, residual plots, etc.]

\begin{itemize}[leftmargin=0pt]

\item There are lots of tools for choosing a model, ranging from
  informal tools, like eyeballing scatter plots, to formal tools that
  try to give the researcher a precise measure of ``model fit.''  Some
  are very bad, but most of them can be helpful in discovering gross
  features of a dataset.  Generally speaking, the informal tools
  (graphing, plotting, etc.) work better but take longer, so they
  might be prohibitively time consuming for datasets with many
  potential regressors.

\item As we'll see, pretty much all of the model selection tools
  perform alarmingly badly at discovering detailed features of a
  dataset.  This is easiest to show for the formal tools, obviously,
  but there's no reason to think that informal tools do any better.
  It's just hard to prove results about them.

  We'll talk about this issue later in the section, but if you are
  impatient or want more details, see the review article by
  \citet{LP05}.

\item The first model-selection statistic that anyone sees is
  $R$-square, so let's get it out of the way.

  \begin{defn}[Centered $R$-square] $R$-square measures the highest
    fraction of variance in the dependent variable that can be removed
    by a linear combination of the regressors:
    \begin{equation*}
      R^2 = 1 - SSR/SST = 1 - \veph'\veph/y'M_0y
    \end{equation*}
    with $M_0 = I - \iota \iota' / n$.
  \end{defn}

  This measure has the obvious shortcoming that adding any regressor
  will allow no less of the variance of the dependent variable to be
  removed, so $R^2$ will go up (weakly) if you add more regressors.

\item One response, then, is to use $R^2$ but add an additional
  penalty term to offset the increase that one would expect if adding
  an irrelevant regressor.  This gives \emph{adjusted $R$-square},
  $\Rb^2$, which is defined as
  \begin{equation*}
    \Rb^2 = 1 - \frac{n-1}{n-K}(1-R^2).
  \end{equation*}
  Now if we add a regressor, $K$ increases too and $\Rb^2$ may go
  down.

\item We can also increase the penalty even more, giving the AIC and
  the BIC:
  \begin{equation*}
    AIC = s^2_y (1 - R^2) e^{2K/n}
  \end{equation*}
  and
  \begin{equation*}
    BIC = s^2_y (1 - R^2) n^{2K/n}
  \end{equation*}
  where $s_y^2 = \sum_{i=1}^n (y_i - \yb)^2 / n$.\footnote{The penalty
    is higher for the BIC, so it will typically select a smaller model
    than the AIC.}
  For these measures, smaller values mean the model is a better fit.

  Usually these are presented in logs, so the AIC is often reported as
  $\ln(\veph'\veph/n) + 2K/n$ and the BIC is often reported as
  $\ln(\veph'\veph/n) + (K/n) \ln n$.

  The imagined workflow is that the researcher calculates the AIC (or
  BIC) for all of the models under consideration, then chooses the one
  that minimizes the criterion.

\item There is a problem with all of this, though.  Imagine the
  setting where you're choosing between two models and the only
  difference between the two is that the larger model has an
  additional regressor---for a specific example, imagine that your
  model selection method is to do a \ttest\ on its coefficient and
  choose the smaller model if the test fails to reject.  Step back for
  a second and remember: you always have the option of just using the
  larger model.  If the coefficient on the additional regressor is not
  zero, this model will have lower bias and may have lower variance as
  well.

  So the pre-test can only be helpful when the coefficients are small
  (but should be harmless for very large coefficients).
  Unfortunately, there is enough noise in the coefficient estimates
  that the \ttest s do extremely badly here.  The next example is
  heavily stylized but illustrates this point.

  \begin{ex}
    Consider the simplest possible setting: $X_1,\dots,X_n \sim \iid
    N(\mu,\sigma^2)$ where $\mu$ is unknown and the two models are
    $\Xb$ and
    \begin{equation*}
      \muh =
      \begin{cases}
        0   & |\Xb| \leq n^{-1/4} \\
        \Xb & |\Xb| > n^{-1/4}.
      \end{cases}
    \end{equation*}
    (this is ``Hodges's estimator''\footnote{Need to track down the
      real citation, but see \citet{LP08}.})

    The first question is to look at the asymptotic variance of each
    estimator.  Obviously, $\var(\sqrt{n} \Xb) = \sigma^2$.  But the
    asymptotic variance of $\muh$ depends on $mu$.  If $\mu \neq 0$
    then
    \begin{equation*}
      \Pr[ \Xb > n^{-1/4} ] \to 1
    \end{equation*}
    so $\muh = \Xb + o_p/\sqrt{n}$ and $\var(\sqrt{n} \muh) \to
    \sigma^2$.  On the other hand, if $\mu = 0$ then
    \begin{equation*}
      \Pr[ \Xb > n^{-1/4} ] = \Pr[ \sqrt{n} \Xb > n^{1/4} ] \to 0
    \end{equation*}
    and so $\var(\sqrt{n} \muh) \to 0$.  Combined, we have
    \begin{equation*}
      \var(\sqrt{n} \muh) \to
      \begin{cases}
        0        & \mu = 0 \\
        \sigma^2 & \mu \neq 0,
      \end{cases}
    \end{equation*}
    which looks like a clear improvement: the asymptotic variance is
    the same everywhere except at $\mu = 0$ and the estimator is much
    more precise there.

    More sophisticated mathematical arguments show that this
    conclusion is wrong.  But we'll just get some intuition from
    simulations.

    The R code for the second estimator is straightforward:
    \begin{lstlisting}[gobble=6]
      mh <- function(x) {
        xbar <- mean(x)
        crit <- 1 / length(x)^0.25
        if (abs(xbar) > crit) {
          return(xbar)
        } else {
          return(0)
        }
      }
    \end{lstlisting}
    We can also write a simple function to estimate the relative
    variance of the two estimators for any $n$ and $\mu$ through
    simulation:
    \begin{lstlisting}[gobble=6]
      relvariance <- function(mu, n, nsims = 2000) {
        variances <- apply(replicate(nsims, {
          x <- rnorm(n, mu)
          c(xbar = mean(x), mhat = mh(x))
        }), 1, var)
        return(variances[2] / variances[1])
      }
    \end{lstlisting}
    So, for example, calling
    \begin{lstlisting}[firstline=3,lastline=3,gobble=6]
      set.seed(647893)
      exvar <-
        relvariance(1, 100)
    \end{lstlisting}
    returns~\exvar, which is the simulated estimate of $\var(\muh)
    / \var(\Xb)$ when $\mu$ is 1 and $n$ is 100.

    We can repeat this call for $\mu$ between $-1$ and $1$:
    \begin{lstlisting}[firstline=2,lastline=2,gobble=8]
      mcvars <-
        sapply(seq(-1, 1, 0.01), function(m) relvariance(m, 100))
      pdf(file = "regression/modeling_fig1.pdf", width=5, height=5)
      plot(mcvars ~ seq(-1, 1, 0.01), type = "l")
      lines(c(1,1) ~ c(-1,1), col = "light gray")
      dev.off()
    \end{lstlisting}
    and the results are plotted in Figure~\ref{fig:m1}.  For $\mu$
    very near zero the selection model is more efficient, and for
    $\mu$ far from zero the models are essentially equivalent (the
    ratio is about 1).  But for $\mu$ in a neighborhood of 1 the
    selection model is much, much worse than the naive sample average.

    Figure~\ref{fig:m2} plots the same graph for 1000 observations,
    called by
    \begin{lstlisting}[firstline=2,lastline=2,gobble=8]
      mcvars2 <-
        sapply(seq(-1, 1, 0.01), function(m) relvariance(m, 1000))
      pdf(file = "regression/modeling_fig2.pdf", width=5, height=5)
      plot(mcvars2 ~ seq(-1, 1, 0.01), type = "l")
      lines(c(1,1) ~ c(-1,1), col = "light gray")
      dev.off()
    \end{lstlisting}
    and notice that the results are qualitatively the same.  But the
    neighborhood is smaller and the relative inefficiency of the
    selection model is larger.
  \end{ex}

  \begin{figure}[t]
    \centering
    \tryincludegraphics{regression/modeling_fig1.pdf}
    \caption{Relative variance of model-selection method to mean for
      100 observations.}
    \label{fig:m1}
  \end{figure}

  \begin{figure}[t]
    \centering
    \tryincludegraphics{regression/modeling_fig2.pdf}
    \caption{Relative variance of model-selection method to mean for
      1000 observations.}
    \label{fig:m2}
  \end{figure}

\item To discuss further: do we care about relative or absolute
  efficiency; show that this applies even to a realistic selection
  model.
  
\end{itemize}

\section{Shrinkage}
\begin{itemize}[leftmargin=0pt]
\item Why is having irrelevant regressors bad?
\begin{itemize}
\item increases the variance of our estimators
\end{itemize}
\item One approach: drop regressors
\begin{itemize}
\item sets the coefficients on some of the regressors equal to zero
\item introduces bias
\end{itemize}
\item Another approach
\begin{itemize}
\item shift the coefficients on all of the regressors closer to zero
\item decreases the variance on the coefficients
\item introduces bias as well
\end{itemize}
\end{itemize}

\paragraph{intuition}
\begin{itemize}
\item For OLS:
\begin{itemize}
\item $\betah = (X'X)^{-1}X'Y$
\item $\E(\betah \mid X) = (X'X)^{-1}X'X\beta$
\item $\var(\betah \mid X) = \sigma^2(X'X)^{-1} X'X (X'X)^{-1}$
\end{itemize}
\item Another possibility
\begin{itemize}
\item $\betah_2 = (X'X + V)^{-1}X'Y$
\begin{itemize}
\item $V$ is some positive definite matrix
\item Assume $V$ is a function of $X$
\item Say $V = X'X$ (for example); then $\betah_2 = \betah / 2$
\end{itemize}
\item $\E(\betah_2 \mid X) = (X'X + V)^{-1}X'X\beta$
\begin{itemize}
\item So the estimator is biased
\item in our example, $\E(\betah \mid X) = \beta/2$
\end{itemize}
\item $\var(\betah_2 \mid X) = \sigma^2 (X'X + V)^{-1} X'X (X'X + V)^{-1}$
\begin{itemize}
\item in our example, $\var(\betah_2 \mid X) = (\sigma^2/4) (X'X)^{-1}$
\end{itemize}
\end{itemize}
\end{itemize}

\paragraph{variance reduction of this estimator}
\begin{itemize}
\item We can see that $\var(\betah_2 \mid X) \leq \var(\betah \mid X)$:
\begin{itemize}
\item Take inverses and look at difference:
  \begin{align*}
    (\var(\betah_2 \mid X)^{-1} - (\var(\betah \mid X))^{-1}
    &= \sigma^2((X'X + V) (X'X)^{-1} (X'X + V) - X'X) ] \\
    &= \sigma^2 (X'X + V + V(X'X)^{-1} (X'X + V)) \\
    &= \sigma^2 (X'X + 2 V + V(X'X)^{-1}V) 
  \end{align*}
\item This last term is positive definite
\end{itemize}
\end{itemize}

\paragraph{The ridge regression estimator}
\begin{itemize}
\item To get a convenient expression for our estimator, we should center and scale the regressors
\begin{description}
\item[$\Xt$] centered and scaled $X$
\begin{itemize}
\item $X = [1, X_1, X_2, ..., X_{k}]$
\item $\Xt = [1, (X_1 - \i \Xb_1) / \sd(X_1), ...]$
\end{itemize}
\end{description}
\item The ridge regression estimator is
  \[ \betah_{ridge} = (\Xt'\Xt + \l I)^{-1} \Xt'Y\]
\begin{itemize}
\item For uncentered/unscaled regressors, either
\begin{itemize}
\item adjust the formula
\item transform the scaled/centered coefficients
\end{itemize}
\end{itemize}
\end{itemize}

\begin{lstlisting}[print=false]
  texcommand <- function(macro, tex, fmt = "%s")
    sprintf("\\newcommand{\\%s}{%s}\n", macro, sprintf(fmt, tex))

  cat(file = "regression/modeling_macros.tex",
      texcommand("exvar", exvar, "%.2f"))
\end{lstlisting}

%%% Local Variables:
%%% mode: latex
%%% TeX-master: "../core_econometrics"
%%% End:

% Copyright (c) 2013, authors of "Core Econometrics;" a
% complete list of authors is available in the file AUTHORS.tex.

% Permission is granted to copy, distribute and/or modify this
% document under the terms of the GNU Free Documentation License,
% Version 1.3 or any later version published by the Free Software
% Foundation; with no Invariant Sections, no Front-Cover Texts, and no
% Back-Cover Texts.  A copy of the license is included in the file
% LICENSE.tex and is also available online at
% <http://www.gnu.org/copyleft/fdl.html>.

\section{Multiple hypothesis testing in linear regression}

\subsection{Testing multiple hypotheses}

Some special issues come up when we have many hypotheses.  I need to
figure out some simulations to motivate this stuff.

\subsection{Intersection-Union Test}

\begin{itemize}
\item Suppose the null can be written as an intersection of simpler
  nulls:
  \begin{itemize}
  \item $H_0: \theta \in \bigcup_{\gamma \in \Gamma} \theta_\gamma$ vs.  $H_A: \theta \notin \bigcap_{\gamma \in \Gamma} \theta_\gamma$
  \item It is easy to find tests for each individual set:
    \begin{itemize}
    \item $T_\gamma$ tests the null $\theta \in \theta_\gamma$ vs. $\theta \notin \theta_\gamma$ for each $\gamma$.
    \end{itemize}
  \item If $T_\gamma$ is a level-$\alpha$ test with rejection region $R_\gamma$ (ie
    it rejects if $(x_1,\dots,x_n) \in R$), then the test with rejection
    region $\bigcap_{\gamma \in \Gamma} R_\gamma$ is the IU Test and has level $\alpha$
  \end{itemize}
\item Note that we don't need to correct the critical values here
\end{itemize}

\paragraph{Proof of validity}
\begin{itemize}
\item We know that, under the null, $\theta \in \theta_\gamma'$ for at least one $\gamma'$
\item $\Pr_\theta[(X_1,\dots,X_n) \in \bigcap_\gamma R_\gamma] \leq \Pr_\theta[(X_1,\dots, X_n) \in R_{\gamma'}] \leq \alpha$
\end{itemize}

\subsection{Union-Intersection Test}

\paragraph{Setup of UIT}
\begin{itemize}
\item Suppose the null can be written as an intersection of simpler
  nulls:
  \begin{itemize}
  \item $H_0: \theta \in \bigcap_{\gamma \in \Gamma} \theta_\gamma$ vs.  $H_A: \theta \notin \bigcap_{\gamma \in \Gamma} \theta_\gamma$
  \item It is easy to find tests for each individual set:
    \begin{itemize}
    \item $T_\gamma$ tests the null $\theta \in \theta_\gamma$ vs. $\theta \notin \theta_\gamma$ for each $\gamma$.
    \end{itemize}
  \item If $T_\gamma$ is a level-$\frac{\alpha}{\# \Gamma}$ test with rejection
    region $R_\gamma$ (ie it rejects if $(x_1,\dots,x_n) \in R$), then the test
    with rejection region $\bigcup_{\gamma \in \Gamma} R_\gamma$ is the UI Test and has level
    $\alpha$
  \end{itemize}
\item Note that we don't need to use a stepdown procedure here
\item Could still do better if we used the joint distribution of $T_\gamma$
\end{itemize}

\paragraph{Show that testing at $\alpha$ doesn't work}

\paragraph{Proof of validity}
\begin{itemize}
\item Follows from the same argument as in multiple comparisons
  \begin{align*}
    \Pr_\theta[(X_1,\dots,X_n) \in \bigcup_{\gamma} R_\gamma] 
    &\leq \sum_{\gamma \in \Gamma} \Pr_\theta[(X_1,\dots,X_n) \in R_\gamma] \\
    &\leq \sum_\gamma \frac{\alpha}{\#\Gamma} \\
    &= \alpha
  \end{align*}
\end{itemize}

\subsection{Multiple hypothesis testing}

\begin{itemize}
\item Suppose we have a bunch of hypotheses $\theta \in \theta_\gamma$, but instead of
  being interested in knowing about compound hypotheses, we're
  interested in all of them
  \begin{itemize}
  \item have many different parameters
  \item Put up empirical example where we're looking at regressors
  \end{itemize}
\item Now we want to control the rate of ``familywise error''
  \begin{itemize}
  \item Let $I$ index the true nulls: $I = \{\gamma \in \Gamma : \theta_\gamma \in \theta_\gamma\}$
  \item FWE is \[\sup_{\theta: \theta_\gamma \in \theta_\gamma, \gamma \in I} \Pr_\theta[T_\gamma \text{ rejects
      for at least one } \gamma \in I]\]
  \item want this error probability to be less than $\alpha$
  \end{itemize}
\item As with UIT, testing each one at $\alpha$ doesn't work
  \begin{itemize}
  \item show this
  \end{itemize}
\end{itemize}

\paragraph{Bonferroni bound}

\paragraph{Bonferroni-Holm stepdown}

\paragraph{Critical values from the joint distribution}

%%% Local Variables:
%%% mode: latex
%%% TeX-master: "../core_econometrics"
%%% End:

% Copyright © 2013, authors of the "Core Econometrics Textbook;" a
% complete list of authors is available in the file AUTHORS.tex.

% Permission is granted to copy, distribute and/or modify this
% document under the terms of the GNU Free Documentation License,
% Version 1.3 or any later version published by the Free Software
% Foundation; with no Invariant Sections, no Front-Cover Texts, and no
% Back-Cover Texts.  A copy of the license is included in the file
% LICENSE.tex and is also available online at
% <http://www.gnu.org/copyleft/fdl.html>.

\part*{Heteroskedasticity}%
\addcontentsline{toc}{part}{Heteroskedasticity}
\section{Heteroskedasticity-robust standard errors}

\begin{itemize}
\item what happens if $\E(ε ε' ∣ X) ≠ σ² I$?
\item maybe instead we have
\begin{description}
\item[zero correlation] $\E(ε_i ε_j ∣ X) = 0$
\item[heteroskedasticity] $\E(ε²_i ∣ X) = σ²_i(x_i)$ where the $σ²_i$
  are all (uniformly) finite and positive.
\end{description}
\item we want to know which of our previous conclusions hold
\end{itemize}

\paragraph{examples}
\begin{itemize}
\item may have (say) quarterly earnings data on different firms, and
        you might expect that earnings in certain industries are more
        volatile than others
\item may have GDP data; we'd expect that percent-change in GDP is
        going to be more volatile some quarters compared to others
\end{itemize}

\paragraph{unbiasedness}
\begin{itemize}
\item For unbiasedness, we just used the fact that $\E(ε ∣ X) = 0$
\begin{itemize}
\item exogeneity of the regressors
\end{itemize}
\item unbiasedness holds with heteroskedasticy errors
\end{itemize}

\paragraph{consistency}
\begin{itemize}
\item To prove consistency, we just needed to show that
        $n^{-1} ∑_{i=1}^n x_i ε_i$ has mean zero and obeys a
        law of large numbers
\item zero mean holds just like it does for unbiasedness
\item for LLN, we can show that each $x_i ε_i$ has finite
        variance;
\item observe that
  \[\var(x_i ε_i) = \E \var(x_i ε_i ∣ X) = \E(x_i x_i') σ²_i\]
  which is finite
\item so $\βh$ is consistent even under heteroskedasticity
\end{itemize}

\paragraph{formula for the variance}
\begin{itemize}
\item our original formula for the variance of $\βh$ is
  $σ² (X'X)^{-1}$
\begin{itemize}
\item obviously, this is a problem since we don't have a constant
  value of $σ²$ any more.
\end{itemize}
\item we can calculate the variance of $\βh$ under our new
  assumption (with $D$ the diagonal matrix with elements $σ²_i$)
  \begin{align*}
    E((\βh - β)(\βh - β)' ∣ X)
    &= (X'X)^{-1} \E(X'ε ε'X ∣ X) (X'X)^{-1} \\
    &= (X'X)^{-1} X'DX (X'X)^{-1} \\
    &= (X'X)^{-1} (∑_{i=1}^n x_i x_i' σ²_i)(X'X)^{-1}
  \end{align*}
\item This matrix is different than $σ²(X'X)^{-1}$
\item any hypothesis tests that we conduct assuming homoskedasticity
        will be invalid under heteroskedasticity
\end{itemize}

\paragraph{finite sample normality}
\begin{itemize}
\item Suppose now that $ε_i ∼ N(0, σ²_i)$ given X
\item We know that $\βh - β = (X'X)^{-1}X'ε$ which
        is still just a linear transformation of normal random variables
\item So $\βh$ is still normal (given X)
\begin{itemize}
\item variance-covariance matrix is obviously different than it was
          earlier
\item $\βh ∼ N\left(0, (X'X)^{-1} \left(∑_{i=1}^n x_i x_i' σ²_i \right) (X'X)^{-1}\right)$
  given $X$
\end{itemize}
\end{itemize}

\paragraph{asymptotic normality}
\begin{itemize}
\item To prove asymptotic normality, we needed to show that
\begin{itemize}
\item $n^{-1} X'X$ obeyed a law of large numbers
\begin{itemize}
\item we assumed this
\item doesn't have anything to do with the errors
\end{itemize}
\item $n^{-1/2} ∑_{i=1}^n x_i ε_i$ obeys a central limit
          theorem
\end{itemize}
\item If you look back at the proof, $x_i ε_i$ has
  heteroskedasticity even if $ε_i$ has constant variances
\item The exact same CLT we used earlier applies here
\begin{itemize}
\item we used the Lindeberg-Feller CLT earlier
\end{itemize}
\item So we have asymptotic normality under heteroskedasticity
\begin{itemize}
\item let $Q = \lim n^{-1} X'X$
\item let $Ω = \lim n^{-1} ∑_i x_i x_i' σ²_i$
\item $\sqrt{n}(\βh - β) → N(0, Q^{-1} Ω Q^{-1})$ in
          distribution.
\end{itemize}
\end{itemize}

\paragraph{summary}
\begin{itemize}
\item heteroskedasticity only affects the variance of $\βh$
\item it will make the tests we've studied invalid
\item if we can estimate the new variance
\begin{itemize}
\item we can replace the old variance in formulae for test statistics
\item these new tests will be (asymptotically) valid
\end{itemize}
\end{itemize}

\subsection{White's heteroscedasticity robust estimator}
\begin{itemize}
\item The goal is going to be getting new test statistics
\item To do that, we're going to work out an estimator for the
       asymptotic variance-covariance matrix, $Q^{-1} Ω Q^{-1}$
\item Derived by \citet{Whi80}
\item We can replace $s²(X'X)^{-1}$ with this new estimator to get
       aysmptotically valid versions of the t-test and the F-test
\item reading is 4.3 of \citet{KZ08}.
\end{itemize}

\paragraph{split up the sandwich}
\begin{itemize}
\item goals
\begin{itemize}
\item We need to estimate $Q^{-1}$
\item We need to estimate $Ω$.
\end{itemize}
\item $Q^{-1}$ is easy: we know that $n^{-1} X'X$ is a consistent
        estimator of $Q$
\item For $Ω$, we're going to apply our usual trick, the LLN.
\begin{itemize}
\item $Ω_n = n^{-1} ∑ \E(ε²_i ∣ x_i) x_i x_i'$
\item $\Ωh = \lim_n n^{-1} ∑ \εh²_i x_i x_i'$
\end{itemize}
\item want to prove that $\Ωh - Ω_n → 0$ in probability
\end{itemize}

\paragraph{consistency of $\Ωh$}
We can rewrite the difference:
  \begin{align*}
    \Ωh - Ω_n &= n^{-1} ∑_i x_i x_i' (\εh²_i - σ²_i) \\
    &= n^{-1} ∑_i x_i x_i'(\εh²_i - ε²_i) 
       + n^{-1} ∑_i x_i x_i'(ε²_i - σ²_i)
  \end{align*}

\paragraph{LLN for the second term}
\begin{itemize}
\item We know that each summand has mean zero
  \[\E(x_i x_i'(ε²_i - σ²_i)) = \E(x_i x_i' \E(ε²_i - σ²_i ∣ x_i)) = 0\]
\item Under conditions similar to those we used earlier, we know
         that $x_i x_i' (ε²_i - σ²_i)$ has finite and
         positive variance.
\item so the sum obeys (for example) Chebychev's LLN
\end{itemize}

\paragraph{convergence of the first term}
\begin{itemize}
\item note that
  \[ \εh²_i = (y_i - x_i'\βh)² = (ε_i + x_i'β - x_i'\βh)²
  = ε²_i - 2 ε_i x_i'(\βh - β) + (x_i'(\βh - β))²\]
\item so the first average becomes
  \[n^{-1}∑ x_i x_i' (x_i'(\βh - β))²
  - 2 n^{-1} ∑  x_i x_i' (ε_i x_i'(\βh - β)) \]
\item (informally) we can see why these terms converge to zero
\begin{itemize}
\item look the (j,k) element of the first matrix
  \[ n^{-1} ∑_i x_{ij} x_{ik}(x_i'(\βh - β))² =
  (\βh - β)' \Big(n^{-1} ∑_i x_i x_{ij} x_{ik} x_{i}'\Big) (\βh- β)\]
\begin{itemize}
\item under usual conditions, the term inside the sum is going to
             converge to its average
\item each $\βh - β$ is going to converge to zero
\end{itemize}
\item look at the (j,k) element of the second matrix
  \[ n^{-1} ∑_i x_{ij} x_{ik}(ε_i x_i'(\βh - β)) =
  \Big(n^{-1} ∑_i ε_i x_{ij} x_{ik} x_{i}'\Big) (\βh- β)\]
\begin{itemize}
\item the term inside the sum is going to converge to its
             average (zero, in this case)
\item $\βh - β$ is going to converge to zero
\end{itemize}
\end{itemize}
\end{itemize}

\paragraph{conclusion}
\begin{itemize}
\item consequently $Ω - \Ωh = Ω - Ω_n + Ω_n - \Ωh → 0$ in probability
\end{itemize}

\subsection{uses of the robust variance-covariance matrix}

\begin{itemize}
\item Define $\Σh$ as 
  \[\Σh = (n^{-1} X'X)^{-1} n^{-1} ∑_i x_i x_i' \εh²_i (n^{-1} X'X)^{-1}\]
\item variation of the t-test
\begin{itemize}
\item $\sqrt{n} \frac{\βh_j - β_j}{\Σh_{jj}} → N(0,1)$ in distribution
\end{itemize}
\item variation of the F-test (called the wald test)
\begin{itemize}
\item suppose that $R$ is a $J × K$ matrix with full rank
\item $n (R(\βh - β))'(R\Σh R')^{-1} (R(\βh - β))$ is asymptotically
  chi-square with $J$ degrees of freedom
\end{itemize}
\item obviously, you can use this for a hypothesis test by imposing
  $Rβ  = q$
\end{itemize}

\section{testing for heteroskedasticity}

\begin{itemize}
\item reading is 8.5 of \citet{Gre12} and 4.2 of \citet{KZ08}
\item we're not going to work through the proofs of this section, but
      we will explain the intuition behind them
\end{itemize}

\subsection{heuristic idea for test}

\begin{itemize}
\item $\βh$ is consistent even under heteroskedasticity
\item $\εh_i → ε_i$ for each $i$.
\item If $σ²_i$ is different for different individuals, it implies
  that $ε²_i$ should be correlated with $x_i$.
\item Why not regress $ε²_i$ on $x_i$ and do an F-test for
       the significance of the overall regression?
\end{itemize}

\subsection{more formal idea for test}

\begin{itemize}
\item Want to test the null hypothesis
  \[ H₀: \quad σ²_i = σ \qquad i = 1,...,n \]
\item If the null hypothesis is true, then $s² (n^{-1} X'X)^{-1} → Ω$
  in probability as $n → ∞$
\begin{itemize}
\item $Ω = \lim n^{-1} ∑_i x_i x_i' σ²_i$.
\item this equals $σ² \lim n^{-1} ∑_i x_i x_i'$ for
         homoskedasticity
\end{itemize}
\item We know that $n^{-1} ∑_i \εh²_i x_i x_i' → Ω$ as well
\begin{itemize}
\item holds with or without heteroskedasticity.
\end{itemize}
\item So, if the null hypothesis is true, $n^{-1} (∑_i \εh²_i - s²)
  x_i x_i' → Ω$ in probability.
\begin{itemize}
\item if the null is false, this doesn't happen.
\end{itemize}
\item This looks like an average
\end{itemize}

\subsection{explanation of the test}

\begin{itemize}
\item Let $ψ_i$ denote the vector of unique elements of $x_ix_i'$,
       along with a constant term (if not in $x_i$).
\begin{itemize}
\item say that $ψ_i$ has length $p$.
\end{itemize}
\item We'd expect that $n^{-1/2} ∑_i (\εh²_i - s²)
       ψ_i$ should obey a central limit theorem, so $→ N(0, V)$
       (say) in distribution (where $V$ is a pretty large matrix).
\item Then we'd have 
  \[
  \Big(n^{-1/2} ∑_i (\εh²_i - s²)ψ_i\Big)' \Vh^{-1} \Big(n^{-1/2} ∑_i (\εh²_i - s²)ψ_i\Big)
  →^d χ²_p
  \]
\begin{itemize}
\item holds under the null hypothesis
\item fails under the alternative (which gives us power).
\end{itemize}
\end{itemize}

\subsection{implementation of the test}

     Statistic is quite easy to implement (derived by \citealp{Whi80})

\begin{enumerate}
\item Regress $y_i$ on $x_i$ and save the OLS residuals.
\item Regress $\εh_i$ on $φ_i$ and calculate the $R²$
        from this regression
\item $n R²$ is asymptotically $χ²_{p-1}$ under the null
        hypothesis.
\end{enumerate}

\section{Generalized Least Squares}

Reading is \citet[8.3]{Gre12}

\subsection{motivation}

\begin{itemize}
\item Suppose that instead of having $\E( ε ε' ∣ X) = σ² I$ we knew
  that $\E(ε ε ∣ X) = σ² Ω$ for some other known matrix $Ω$.
\item we could still do MLE to estimate $β$ and $σ$
\item Would generally give you a different estimate of $β$ than OLS.
\end{itemize}n

\subsection{Infeasible GLS}

\paragraph{Draw pictures}

\paragraph{mathematics}
\begin{itemize}
\item We know that $Ω^{-1/2}ε ∼ (0, σ² I)$ given $X$
\item What happens if we regress $Ω^{-1/2} Y$ on $Ω^{-1/2} X$?
\begin{itemize}
\item Let $\Yt = Ω^{-1/2} Y$
\item Let $\Xt = Ω^{-1/2} X$
\item $\εt = \Yt - \Xt β$
\end{itemize}
\item If $\E(Y ∣ X) = Xβ$ then $\E(\Yt ∣ X) = \Xt β$
\item If we regress $\Yt$ on $\Xt$, we're back in standard
        OLS with homoskedastic errors.
\item Estimator is $\βh_{GLS} = (X'Ω^{-1}X)^{-1}X'Ω^{-1}Y$
\begin{itemize}
\item Unbiased (obviously)
\item Satisfies Gauss-Markov (BLUE)
\item Variance is $σ² (\Xt' \Xt)^{-1} = σ² (X' Ω^{-1} X)^{-1}$
\item Normal if the error terms are normal
\item Asymptotically normal otherwise.
\end{itemize}
\end{itemize}

\subsection{Weighted Least Squares (ie example)}

\begin{itemize}
\item Suppose that we believe that $σ²_i = σ² x_{ij}²$ for one of the
  regressors $j$.
\begin{itemize}
\item textbook gives an example: dependent variable is firm profits
         and indep variable is size.
\end{itemize}
\item To implement GLS, we want to regress
  \[ y/x_k = β + β₁(x₁/x_k) + β₂(x₂/x_k) + ⋯ + ε/x_k \]
\begin{itemize}
\item intuitively, if the firm is bigger, profit is more volatile
         and so we want to count that observation less in our estimation.
\end{itemize}
\end{itemize}

\subsection{Feasible GLS}

\begin{itemize}
\item Suppose that $Ω$ has a few unknown parameters:
\begin{itemize}
\item Two examples from TS:
\begin{itemize}
\item $Ω = (1, ρ, ρ², ...)$ (as a matrix)
\item $Ω = (1, γ, 0, 0; γ, 1, ...)$ (as a matrix)
\end{itemize}
\end{itemize}
\item We could just estimate those parameters along with the others.
\item Two different approaches
\begin{itemize}
\item MLE (next semester)
\item two-step least squares (FGLS)
\end{itemize}
\end{itemize}

\paragraph{Explanation of approach}
\begin{itemize}
\item Remember how we tested for heteroskedasticity
\item regress $\εh$ on the (squared) regressors and test
        for significance
\item use a similar approach here, but use for modeling.
\item Suppose that
        \[ \E(ε²_i ∣ X) = z_i'α \]
        where $z_i$ is some function of the regressors.
\item If we could regress $ε²_i$ on $z_i$, we could estimate $α$ consistently.
\item Since $\εh_i$ is consistent for $ε_i$, we can regress $ε²_i$ on
  $z_i$ to estimate $α$.
\end{itemize}

\paragraph{Algorithm}
\begin{itemize}
\item Regress $y$ on $x$ to get $\εh$
\item Regress $\εh$ on $z$ to estimate $α$
\item Regress $y_i/w_i$ on $x_i/w_i$ to estimate $\βh_{FGLS}$ where
  $w_i = sqrt{z_i'\αh}$
\end{itemize}

\paragraph{some more math}
\begin{itemize}
\item if we let $\Ωh$ be the estimate of $Ω$ that uses $\αh$,
  consistency of $\αh$ ensures that $\Ωh$ behaves asymptotically like
  $Ω$.
\item All of the asymptotic results from the GLS estimator apply to
        the Feasible GLS estimator
\item Finite sample results don't necessarily hold.
\end{itemize}

%%% Local Variables:
%%% mode: latex
%%% TeX-master: "../regression"
%%% End:

% Copyright (c) 2013, authors of "Core Econometrics;" a
% complete list of authors is available in the file AUTHORS.tex.

% Permission is granted to copy, distribute and/or modify this
% document under the terms of the GNU Free Documentation License,
% Version 1.3 or any later version published by the Free Software
% Foundation; with no Invariant Sections, no Front-Cover Texts, and no
% Back-Cover Texts.  A copy of the license is included in the file
% LICENSE.tex and is also available online at
% <http://www.gnu.org/copyleft/fdl.html>.

\chapter{Systems of equations}

\begin{itemize}[leftmargin=0pt]

\item The basic idea here is that it is sometimes more efficient to
  estimate multiple equations simultaneously than individually.  For
  example, suppose you have two relationships you are interested in
  (returns data, say)
  \begin{equation*}
    r_{jt} - r_{ft} = \alpha_j + \beta_j (r_{mt} - r_{ft}) + \vep_{it}
  \end{equation*}
  where the variables are
  \begin{itemize}
  \item $r_{it}$ is the return on asset $i$
  \item $r_{ft}$ is the risk-free rate of return (i.e. T-bills)
  \item $r_{mt}$ is the return on the entire market.
  \end{itemize}

  In this setting, should you do OLS on each equation individually, or
  do something different? This section will present the
  \emph{Seemingly Unrelated Regressions} estimator and will discuss
  when it is possible to do better than OLS.

\item More generally, suppose we have two models
  \begin{equation}\label{eq:3}
    Y_1 = X_1 \beta_1 + \vep_1
  \end{equation}
  and
  \begin{equation}\label{eq:4}
    Y_2 = X_2 \beta_2 + \vep_2
  \end{equation}
  where each $Y_i$ is $n \times 1$.  Alternatively, we can write
  \begin{equation*}
    \binom{y_{1i}}{y_{2i}}
    = \binom{x_{1i}'\beta_1}{x_{2i}'\beta_2} +  \binom{\vep_{1i}}{ \vep_{2i}}
  \end{equation*}
  for $i = 1,...,n$.

\item We're going to make standard assumptions from our treatment of
  OLS:
  \begin{description}
  \item[Exogeneity.] For all $i$,
    \begin{equation*}
      E\Bigg(\binom{\vep_{1i}}{\vep_{2i}} \ \Big|\ X\Bigg) = 0
    \end{equation*}
  \item[Homoskedasticity.] For all $i$,
    \begin{equation*}
      E\Bigg(\binom{\vep_{1i}}{\vep_{2i}} \binom{\vep_{1i}}{\vep_{2i}}'
      \ \Big|\ X \Bigg) = \Sigma
    \end{equation*}
  \item[Uncorrelated errors.] For $i \neq j$,
    \begin{equation*}
      E\Bigg(\binom{\vep_{1i}}{\vep_{2i}}
      \binom{\vep_{1j}}{\vep_{2j}}' \ \Big|\ X \Bigg) = 0
    \end{equation*}
    This assumption allows errors to be correlated across equations
    but not across observations.
  \end{description}
  
  Also assume that the usual technical conditions on the regressors
  and errors hold to allow for us to apply laws of large numbers, etc.

\end{itemize}

\section{Estimation of $\beta$}

\begin{itemize}[leftmargin=0pt]
\item We can rewrite Equations~\eqref{eq:3} and~\eqref{eq:4} as
  \begin{equation}\label{eq:5}
    \underbrace{\binom{Y_1}{Y_2}}_Y
    = \underbrace{\begin{pmatrix} X_1 & 0 \\ 0 & X_2 \end{pmatrix}
    \binom{\beta_1}{\beta_2}}_{X\beta} +
    \underbrace{\binom{\vep_1}{\vep_2}}_\vep.
  \end{equation}

  Under the assumptions listed earlier, we can estimate the
  ``stacked'' model by OLS or GLS.
  Since $E(\vep \mid X) = 0$ and
  \begin{equation*}
    E(\vep \vep' \mid X) = \Sigma \otimes I =
    \begin{pmatrix}
      \sigma^2_{11} I & \sigma_{12} I \\ \sigma_{21} I & \sigma^2_{22}
      I
    \end{pmatrix},
  \end{equation*}
  the model satisfies the requirements for GLS regression to be BLUE,
  even though the dependent variable is different for the two
  equations.

  \begin{hw}
    Verify that the direct OLS estimator of~\eqref{eq:5} is
    numerically equivalent to the individual OLS estimators
    of~\eqref{eq:3} and~\eqref{eq:4} individually.
  \end{hw}

\item Under the assumption that $\Sigma$ is known, 
  \begin{equation*}
    \betah_{GLS} = ( X' (\Sigma \otimes I)^{-1} X )^{-1} 
                  X' (\Sigma \otimes I)^{-1} Y.
  \end{equation*}
  This expression can be simplified further. Using the equality
  $(\Sigma \otimes I)^{-1} = \Sigma^{-1} \otimes I$ and letting
  $\gamma_{ij}$ denote the $i,j$ element of $\Sigma^{-1}$, we have
  \begin{equation*}
    \betah_{SUR} = \begin{pmatrix}
      \gamma_{11} X_1'X_1 & \gamma_{12} X_1'X_2 \\
      \cdot & \gamma_{22} X_2'X_2
    \end{pmatrix}^{-1}
    \begin{pmatrix}
      \gamma_{11} X_1'y_1 + \gamma_{12} X_1'y_2 \\
      \gamma_{12} X_2'y_1 + \gamma_{22} X_2'y_2
    \end{pmatrix}.
  \end{equation*}

\item The variance-covariance matrix is found as usual:
  \begin{equation*}
    \var(\betah_{SUR} \mid X) = (X'(\Sigma \otimes I)^{-1}X)^{-1}.
  \end{equation*}

\item Of course, the assumption that $\Sigma$ is known is
  unreasonable. It can be estimated, pretty easily here, though,
  because OLS is consistent (if inefficient) and so $(\veph_{i1},
  \veph_{i2})$ is consistent for $(\vep_{i1}, \vep_{i2})$.

  The LLN then implies that
  \begin{equation*}
    (1/n) \sum_i (\veph_{i1}, \veph_{i2})' (\veph_{i1}, \veph_{i2})
    \equiv \Sigmah \to^p \Sigma
  \end{equation*}
  and we can plug $\Sigmah$ into the GLS formula for a feasible
  estimator.

\end{itemize}

\section{Simplifications of the SUR estimator}

\begin{itemize}[leftmargin=0pt]

\item The SUR becomes the OLS estimator in two specific and
  informative cases. First, if the errors are uncorrelated across
  models, and second, if all of the same regressors appear in both
  models (which is the case for Vector Autoregressions in time series,
  for example).

\item First, suppose that the errors in the two equations are
  uncorrelated. Remember the formula for SUR estimator:
  \begin{equation*}
    \betah_{SUR} = (X'(\Sigma \otimes I)^{-1}X)^{-1}X'(\Sigma \otimes I)^{-1}Y
  \end{equation*}
  If $\sigma_{12} = 0$, the first term is
  \begin{equation*}
    (X'( \Sigma \otimes I)^{-1} X)^{-1} =
    \begin{pmatrix}
      X_1'X_1 / \sigma_{11}^2 & 0 \\ 0 & X_2'X_2 / \sigma_{22}^2
    \end{pmatrix}^{-1}.
  \end{equation*}
  The second term is
  \begin{equation*}
    X'(\Sigma \otimes I)^{-1}Y =
    \begin{pmatrix}      
      X_1' Y_1 / \sigma_{11}^2 \\ X_2'Y_2 / \sigma_{22}^2
    \end{pmatrix}.
  \end{equation*}
  And the estimator becomes
  \begin{equation*}
    \betah_{SUR} =
    \begin{pmatrix}
      (X_1'X_1)^{-1}X_1'Y_1 \\ (X_2'X_2)^{-1}X_2'Y_2
    \end{pmatrix}
    =
    \begin{pmatrix}
      \betah_{1,OLS} \\ \betah_{2,OLS}
    \end{pmatrix}.
  \end{equation*}

\item Now suppose that $X_1$ and $X_2$ both equal $X$.
  The SUR estimator becomes
  \begin{align*}
    \betah_{SUR}
    &= \begin{pmatrix}
      \gamma_{11} X_1'X_1 & \gamma_{12} X_1'X_2 \\ & \gamma_{22} X_2'X_2 
    \end{pmatrix}^{-1}
    \begin{pmatrix}
      \gamma_{11} X_1'Y_1 + \gamma_{12} X_1'Y_2 \\
      \gamma_{12} X_2'Y_1 + \gamma_{22} X_2'Y_2
    \end{pmatrix} \\
    &= \begin{pmatrix}
      \gamma_{11} X'X & \gamma_{12} X'X \\ & \gamma_{22}X'X
    \end{pmatrix}^{-1}
    \begin{pmatrix}
      \gamma_{11} X'Y_1 + \gamma_{12}X'Y_2 \\
      \gamma_{12} X'Y_1 + \gamma_{22} X'Y_2
    \end{pmatrix}
  \end{align*}

  The first term becomes $(\Sigma^{-1} \otimes X'X)^{-1}$ which equals
  $\Sigma \otimes (X'X)^{-1}$. The second term becomes
  \[\begin{pmatrix}
    X'X (\gamma_{11}\betah_{1,OLS} + \gamma_{12}\betah_{2,OLS}) \\
    X'X (\gamma_{21}\betah_{1,OLS} + \gamma_{22}\betah_{2,OLS})
  \end{pmatrix}\] after premultiplying each $X'Y_j$ with
  $X'X(X'X)^{-1}$. Now Multiply through and everything cancels, giving
  \begin{equation*}
    \betah_{SUR} =
    \begin{pmatrix} \betah_{1,OLS} \\ \betah_{2,OLS} \end{pmatrix}
  \end{equation*}
\end{itemize}

\section{Hypothesis testing}

\begin{itemize}[leftmargin=0pt]
\item Testing is exactly like in OLS or GLS: we have an asymptotically
  normal estimator with a covariance matrix that can be consistently
  estimated.  The (robust, usually) \ttest\ and \ftest\ work fine.
\end{itemize}

%%% Local Variables:
%%% mode: latex
%%% TeX-master: "../core_econometrics"
%%% End:

% Copyright (c) 2013, authors of "Core Econometrics;" a
% complete list of authors is available in the file AUTHORS.tex.

% Permission is granted to copy, distribute and/or modify this
% document under the terms of the GNU Free Documentation License,
% Version 1.3 or any later version published by the Free Software
% Foundation; with no Invariant Sections, no Front-Cover Texts, and no
% Back-Cover Texts. A copy of the license is included in the file
% LICENSE.tex and is also available online at
% <http://www.gnu.org/copyleft/fdl.html>.

\chapter{Problems}

\begin{hw}
  Prove that the OLS estimator $\betah$ and the OLS residuals $\hat u$
  are uncorrelated.
\end{hw}

\begin{hw}
  Suppose that
  \begin{equation}
    y_i = \beta_1 + \beta_2 x_i + u_i,
  \end{equation}
  with $u_i \mid X \sim (0, \sigma^2)$ and $x_i \sim (\mu,\tau^2)$.
  Please derive a test statistic for the null hypothesis $\beta_1 =
  \beta_2^2$ against the alternative $\beta_1 \neq \beta_2^2$.
\end{hw}

\begin{hw}
  Suppose that $y_1,\dots,y_n$ are i.i.d. with $y_i \mid x_i \sim
  N(x_i'\beta, \sigma^2)$ and each $x_i$ a $k \times 1$ vector. Prove
  that the OLS estimator is the best unbiased estimator of $\beta$.
  Note that this is a stronger result than the Gauss-Markov theorem,
  since we are claiming that OLS is also better than nonlinear
  unbiased estimators.
\end{hw}

\begin{hw}
  We have the model $Y = X\beta + u$, where $n^{-1} X'X$ obeys a law
  of large numbers, $u$ is homoskedastic, and $n^{-1/2} \sum_i x_i
  u_i$ obeys a central limit theorem. Let $F$ be the F-statistic for
  the null hypothesis that $R\beta = 0$, where $R$ is an arbitrary $J
  \times (K + 1)$ vector
  \begin{enumerate}
  \item Suppose that the null hypothesis is true. How does the
    variance of $F$ behave as $n \to \infty$? How does the mean of
    $F$ behave?
  \item Suppose that $R\beta = \delta$ for some nonzero vector
    $\delta$, so the null hypothesis is false. How do your answers
    from the first part change?
  \item What do your answers to the previous questions tell you about
    the power of the F-test as $n \to \infty$?
  \end{enumerate}
\end{hw}

\begin{hw}
  Suppose that you are interested in the model $y_i = \beta_0 +
  x_i\beta_1 + u_i$ where $x_i$ is a random scalar, $\E(u_i \mid X) =
  0$ and the observations are i.i.d. But suppose that we do not
  observe $x_i$ directly but instead we observe $w_i = x_i + u_i$,
  where $u_i$ is another error term.
  \begin{enumerate}
  \item Prove that, if $u_i$ is perfectly correlated with $x_i$, OLS
    is essentially consistent in the following sense: let $\dot x_i =
    (x_i - \E x_i) / sd(x_i)$ and let $\dot w_i = (w_i - \E w_i) /
    sd(w_i)$. Then the OLS coefficient in the regression of $y_t$ on
    $\dot x_i$ is the same as that of the regression of $y_i$ on $\dot
    w_i$.
  \item Prove that, if $u_i$ is independent of $x_i$ and $u_i$, OLS is
    inconsistent even after standardizing the variables.
  \item Derive the MLE of $\beta_1$ under the assumption that $u_i$ and
    $u_i$ are independent mean-zero normal and determine whether it is
    consistent.
  \item Suppose now that $u_i$ and $u_i$ are independent, mean zero
    normal random variables, but now assume that we have another
    regressor $z_i = x_i + v_i$, where $v_i$ is another mean-zero,
    normal error term that's independent of $u_i$ and $u_i$. Derive
    the MLE of $\beta_1$ and determine whether it is consistent. For bonus
    points and well-deserved pride: is it asymptotically normal?
  \end{enumerate}
\end{hw}

\begin{hw}
  Suppose that you estimate the model $y_i = \beta_0 + \beta_1 x_{i1}
  + \beta_2 x_{i2} + u_i$ with OLS and calculate the F-test for the
  null hypothesis $\beta_1 = 1$. The $p$-value for this test is
  $0.03$ and $\betah_1$ is 0.54. In the following questions, you can
  assume that all of the necessary OLS assumptions hold.

  \begin{enumerate}
  \item How would you use this information to test against the
    one-sided alternative that $\beta_1 > 1$ at the 5\% level? Do you
    reject the null in favor of this alternative?
  \item How does your answer change if the alternative is $\beta_1 <
    1$? Do you reject in favor of this alternative?
  \item Based on your answers to the previous two questions, please
    outline a procedure to test the general null hypothesis that
    $R\beta \leq q$, where the inequality holds element by element.
  \end{enumerate}
\end{hw}

\begin{hw}
  Let $y_i$ and $x_i$ be i.i.d. random scalars and $\E x_i = 0$.
  Suppose that you estimate two models: $y_i = \mu + u_i$ and $y_i =
  \beta_0 + \beta_1 x_i + v_i.$ Calculate the bias and variance of
  $\hat \mu$ and $\betah_0$. How do they depend on $\beta_1$?
\end{hw}

\begin{hw}
  Suppose that we have the model $y_i = \beta_0 + \beta_1 x_{1i} +
  \beta_2 x_{2i} + \beta_3 x_{3i} + u_i$ where $\E(u_i \mid X) = 0$
  but the errors may be heteroskedastic.

  \begin{enumerate}
  \item Suppose you want to test the null hypothesis $\beta_i \leq 0$
    for all $i = 1,\dots,3$ against the alternative that $\beta_i > 0$
    for at least one $i$ (note that this is a single null hypothesis
    against a single alternative). How could you use the multiple
    hypothesis techniques we discussed in class to test this
    hypothesis?
  \item If this procedure rejects the null that all of the $\beta_i$
    are weakly negative, do we know why? Specifically, do we know
    which of the $\beta_i$ is positive? In what sense?
  \item Please write an R program to implement your answer to the
    first part of the question.
  \end{enumerate}
\end{hw}

\begin{hw}
  Consider this situation: you are interested in the effect of a
  variable $X_1$ (say, education level) on $Y$ (say, earnings), so you
  want to test whether the coefficient $\beta_1$ is significant in the
  equation
  \begin{equation*}
    Y_{i} = \beta_0 + \beta_1 X_{i1} + \vep_i.
  \end{equation*}
  But, you don't know whether or not you should include a second
  regressor, $X_2$ (maybe some measure of ability, like a test score).

  Three of the approaches we've discussed are:
  \begin{enumerate}
  \item regress $Y$ on $X_1$ and test for significance.
  \item regress $Y$ on $X_1$ and $X_2$ and test $\beta_1$ for
    significance.
  \item regress $Y$ on $X_1$ and $X_2$ and test $\beta_2$ for
    significance.
    \begin{itemize}
    \item If $\beta_2$ is significant, test $\beta_1$ for
      significance.
    \item If $\beta_2$ is \emph{not} significant, regress $Y$ on only
      $X_1$ and test $\beta_1$ for significance in that model.
    \end{itemize}
  \end{enumerate}

  Please design a Monte Carlo experiment in R that would let you study
  how well each of these approaches would work.
\end{hw}

%%% Local Variables: 
%%% mode: latex
%%% TeX-master: "../core_econometrics"
%%% End: 


% The files `AUTHORS_standalone.tex` and `LICENSE_standalone.tex` are
% available if you want to distribute the author list and the FDL on
% their own.
\addtocontents{toc}{\protect\setcounter{tocdepth}{0}}

\newpage
\part*{Complete list of authors}%
\addcontentsline{toc}{part}{Appendix A: Complete list of authors}
% Copyright © 2013, authors of the "Econometrics Core" textbook; a
% complete list of authors is available in the file AUTHORS.tex.

% Permission is granted to copy, distribute and/or modify this
% document under the terms of the GNU Free Documentation License,
% Version 1.3 or any later version published by the Free Software
% Foundation; with no Invariant Sections, no Front-Cover Texts, and no
% Back-Cover Texts.  A copy of the license is included in the file
% LICENSE.tex and is also available online at
% <http://www.gnu.org/copyleft/fdl.html>.

% Remove the next two lines if you are distributing the author list as
% a standalone pdf.
\noindent%
The following is a list of the contributors to the Econometrics Free
Library Project's \textit{Econometrics Core}, in order of their date
of first involvement (yes, I'm aware that it's a little ridiculous to
have this as a separate file when there is only a single contributor,
but let's dream big, shall we).

\begin{description}
\item[2009-07-01] Gray Calhoun, \email{gcalhoun@iastate.edu}
\end{description}

%%% Local Variables:
%%% mode: latex
%%% TeX-master: "AUTHORS_standalone"
%%% End:

\newpage
\part*{GNU Free Documentation License}%
\addcontentsline{toc}{part}{Appendix B: GNU Free Documentation License}
% Remove the next two lines if you are distributing the author list as
% a standalone pdf.
\part*{GNU Free Documentation License}%
\addcontentsline{toc}{part}{Appendix B: GNU Free Documentation License}
\setcounter{section}{-1}%
\renewcommand\thesection{\arabic{section}}%
\noindent Version 1.3, 3 November 2008

\noindent Copyright \copyright\ 2000, 2001, 2002, 2007, 2008 Free
Software Foundation, Inc.

\noindent \texttt{<http://fsf.org/>}
 
\noindent Everyone is permitted to copy and distribute verbatim copies
of this license document, but changing it is not allowed.

\section{Preamble}

The purpose of this License is to make a manual, textbook, or other
functional and useful document ``free'' in the sense of freedom: to
assure everyone the effective freedom to copy and redistribute it,
with or without modifying it, either commercially or noncommercially.
Secondarily, this License preserves for the author and publisher a way
to get credit for their work, while not being considered responsible
for modifications made by others.

This License is a kind of ``copyleft'', which means that derivative
works of the document must themselves be free in the same sense.  It
complements the GNU General Public License, which is a copyleft
license designed for free software.

We have designed this License in order to use it for manuals for free
software, because free software needs free documentation: a free
program should come with manuals providing the same freedoms that the
software does.  But this License is not limited to software manuals;
it can be used for any textual work, regardless of subject matter or
whether it is published as a printed book.  We recommend this License
principally for works whose purpose is instruction or reference.


\section{APPLICABILITY AND DEFINITIONS}

This License applies to any manual or other work, in any medium, that
contains a notice placed by the copyright holder saying it can be
distributed under the terms of this License.  Such a notice grants a
world-wide, royalty-free license, unlimited in duration, to use that
work under the conditions stated herein.  The ``\textbf{Document}'',
below, refers to any such manual or work.  Any member of the public is
a licensee, and is addressed as ``\textbf{you}''.  You accept the
license if you copy, modify or distribute the work in a way requiring
permission under copyright law.

A ``\textbf{Modified Version}'' of the Document means any work containing the
Document or a portion of it, either copied verbatim, or with
modifications and/or translated into another language.

A ``\textbf{Secondary Section}'' is a named appendix or a front-matter section of
the Document that deals exclusively with the relationship of the
publishers or authors of the Document to the Document's overall subject
(or to related matters) and contains nothing that could fall directly
within that overall subject.  (Thus, if the Document is in part a
textbook of mathematics, a Secondary Section may not explain any
mathematics.)  The relationship could be a matter of historical
connection with the subject or with related matters, or of legal,
commercial, philosophical, ethical or political position regarding
them.

The ``\textbf{Invariant Sections}'' are certain Secondary Sections whose titles
are designated, as being those of Invariant Sections, in the notice
that says that the Document is released under this License.  If a
section does not fit the above definition of Secondary then it is not
allowed to be designated as Invariant.  The Document may contain zero
Invariant Sections.  If the Document does not identify any Invariant
Sections then there are none.

The ``\textbf{Cover Texts}'' are certain short passages of text that are listed,
as Front-Cover Texts or Back-Cover Texts, in the notice that says that
the Document is released under this License.  A Front-Cover Text may
be at most 5 words, and a Back-Cover Text may be at most 25 words.

A ``\textbf{Transparent}'' copy of the Document means a machine-readable copy,
represented in a format whose specification is available to the
general public, that is suitable for revising the document
straightforwardly with generic text editors or (for images composed of
pixels) generic paint programs or (for drawings) some widely available
drawing editor, and that is suitable for input to text formatters or
for automatic translation to a variety of formats suitable for input
to text formatters.  A copy made in an otherwise Transparent file
format whose markup, or absence of markup, has been arranged to thwart
or discourage subsequent modification by readers is not Transparent.
An image format is not Transparent if used for any substantial amount
of text.  A copy that is not ``Transparent'' is called ``\textbf{Opaque}''.

Examples of suitable formats for Transparent copies include plain
ASCII without markup, Texinfo input format, LaTeX input format, SGML
or XML using a publicly available DTD, and standard-conforming simple
HTML, PostScript or PDF designed for human modification.  Examples of
transparent image formats include PNG, XCF and JPG.  Opaque formats
include proprietary formats that can be read and edited only by
proprietary word processors, SGML or XML for which the DTD and/or
processing tools are not generally available, and the
machine-generated HTML, PostScript or PDF produced by some word
processors for output purposes only.

The ``\textbf{Title Page}'' means, for a printed book, the title page itself,
plus such following pages as are needed to hold, legibly, the material
this License requires to appear in the title page.  For works in
formats which do not have any title page as such, ``Title Page'' means
the text near the most prominent appearance of the work's title,
preceding the beginning of the body of the text.

The ``\textbf{publisher}'' means any person or entity that distributes
copies of the Document to the public.

A section ``\textbf{Entitled XYZ}'' means a named subunit of the Document whose
title either is precisely XYZ or contains XYZ in parentheses following
text that translates XYZ in another language.  (Here XYZ stands for a
specific section name mentioned below, such as ``\textbf{Acknowledgements}'',
``\textbf{Dedications}'', ``\textbf{Endorsements}'', or ``\textbf{History}''.)  
To ``\textbf{Preserve the Title}''
of such a section when you modify the Document means that it remains a
section ``Entitled XYZ'' according to this definition.

The Document may include Warranty Disclaimers next to the notice which
states that this License applies to the Document.  These Warranty
Disclaimers are considered to be included by reference in this
License, but only as regards disclaiming warranties: any other
implication that these Warranty Disclaimers may have is void and has
no effect on the meaning of this License.


\section{VERBATIM COPYING}

You may copy and distribute the Document in any medium, either
commercially or noncommercially, provided that this License, the
copyright notices, and the license notice saying this License applies
to the Document are reproduced in all copies, and that you add no other
conditions whatsoever to those of this License.  You may not use
technical measures to obstruct or control the reading or further
copying of the copies you make or distribute.  However, you may accept
compensation in exchange for copies.  If you distribute a large enough
number of copies you must also follow the conditions in section~3.

You may also lend copies, under the same conditions stated above, and
you may publicly display copies.


\section{COPYING IN QUANTITY}

If you publish printed copies (or copies in media that commonly have
printed covers) of the Document, numbering more than 100, and the
Document's license notice requires Cover Texts, you must enclose the
copies in covers that carry, clearly and legibly, all these Cover
Texts: Front-Cover Texts on the front cover, and Back-Cover Texts on
the back cover.  Both covers must also clearly and legibly identify
you as the publisher of these copies.  The front cover must present
the full title with all words of the title equally prominent and
visible.  You may add other material on the covers in addition.
Copying with changes limited to the covers, as long as they preserve
the title of the Document and satisfy these conditions, can be treated
as verbatim copying in other respects.

If the required texts for either cover are too voluminous to fit
legibly, you should put the first ones listed (as many as fit
reasonably) on the actual cover, and continue the rest onto adjacent
pages.

If you publish or distribute Opaque copies of the Document numbering
more than 100, you must either include a machine-readable Transparent
copy along with each Opaque copy, or state in or with each Opaque copy
a computer-network location from which the general network-using
public has access to download using public-standard network protocols
a complete Transparent copy of the Document, free of added material.
If you use the latter option, you must take reasonably prudent steps,
when you begin distribution of Opaque copies in quantity, to ensure
that this Transparent copy will remain thus accessible at the stated
location until at least one year after the last time you distribute an
Opaque copy (directly or through your agents or retailers) of that
edition to the public.

It is requested, but not required, that you contact the authors of the
Document well before redistributing any large number of copies, to give
them a chance to provide you with an updated version of the Document.


\section{MODIFICATIONS}

You may copy and distribute a Modified Version of the Document under
the conditions of sections 2 and 3 above, provided that you release
the Modified Version under precisely this License, with the Modified
Version filling the role of the Document, thus licensing distribution
and modification of the Modified Version to whoever possesses a copy
of it.  In addition, you must do these things in the Modified Version:

\begin{itemize}
\item[A.] 
   Use in the Title Page (and on the covers, if any) a title distinct
   from that of the Document, and from those of previous versions
   (which should, if there were any, be listed in the History section
   of the Document).  You may use the same title as a previous version
   if the original publisher of that version gives permission.
   
\item[B.]
   List on the Title Page, as authors, one or more persons or entities
   responsible for authorship of the modifications in the Modified
   Version, together with at least five of the principal authors of the
   Document (all of its principal authors, if it has fewer than five),
   unless they release you from this requirement.
   
\item[C.]
   State on the Title page the name of the publisher of the
   Modified Version, as the publisher.
   
\item[D.]
   Preserve all the copyright notices of the Document.
   
\item[E.]
   Add an appropriate copyright notice for your modifications
   adjacent to the other copyright notices.
   
\item[F.]
   Include, immediately after the copyright notices, a license notice
   giving the public permission to use the Modified Version under the
   terms of this License, in the form shown in the Addendum below.
   
\item[G.]
   Preserve in that license notice the full lists of Invariant Sections
   and required Cover Texts given in the Document's license notice.
   
\item[H.]
   Include an unaltered copy of this License.
   
\item[I.]
   Preserve the section Entitled ``History'', Preserve its Title, and add
   to it an item stating at least the title, year, new authors, and
   publisher of the Modified Version as given on the Title Page.  If
   there is no section Entitled ``History'' in the Document, create one
   stating the title, year, authors, and publisher of the Document as
   given on its Title Page, then add an item describing the Modified
   Version as stated in the previous sentence.
   
\item[J.]
   Preserve the network location, if any, given in the Document for
   public access to a Transparent copy of the Document, and likewise
   the network locations given in the Document for previous versions
   it was based on.  These may be placed in the ``History'' section.
   You may omit a network location for a work that was published at
   least four years before the Document itself, or if the original
   publisher of the version it refers to gives permission.
   
\item[K.]
   For any section Entitled ``Acknowledgements'' or ``Dedications'',
   Preserve the Title of the section, and preserve in the section all
   the substance and tone of each of the contributor acknowledgements
   and/or dedications given therein.
   
\item[L.]
   Preserve all the Invariant Sections of the Document,
   unaltered in their text and in their titles.  Section numbers
   or the equivalent are not considered part of the section titles.
   
\item[M.]
   Delete any section Entitled ``Endorsements''.  Such a section
   may not be included in the Modified Version.
   
\item[N.]
   Do not retitle any existing section to be Entitled ``Endorsements''
   or to conflict in title with any Invariant Section.
   
\item[O.]
   Preserve any Warranty Disclaimers.
\end{itemize}

If the Modified Version includes new front-matter sections or
appendices that qualify as Secondary Sections and contain no material
copied from the Document, you may at your option designate some or all
of these sections as invariant.  To do this, add their titles to the
list of Invariant Sections in the Modified Version's license notice.
These titles must be distinct from any other section titles.

You may add a section Entitled ``Endorsements'', provided it contains
nothing but endorsements of your Modified Version by various
parties---for example, statements of peer review or that the text has
been approved by an organization as the authoritative definition of a
standard.

You may add a passage of up to five words as a Front-Cover Text, and a
passage of up to 25 words as a Back-Cover Text, to the end of the list
of Cover Texts in the Modified Version.  Only one passage of
Front-Cover Text and one of Back-Cover Text may be added by (or
through arrangements made by) any one entity.  If the Document already
includes a cover text for the same cover, previously added by you or
by arrangement made by the same entity you are acting on behalf of,
you may not add another; but you may replace the old one, on explicit
permission from the previous publisher that added the old one.

The author(s) and publisher(s) of the Document do not by this License
give permission to use their names for publicity for or to assert or
imply endorsement of any Modified Version.


\section{COMBINING DOCUMENTS}

You may combine the Document with other documents released under this
License, under the terms defined in section~4 above for modified
versions, provided that you include in the combination all of the
Invariant Sections of all of the original documents, unmodified, and
list them all as Invariant Sections of your combined work in its
license notice, and that you preserve all their Warranty Disclaimers.

The combined work need only contain one copy of this License, and
multiple identical Invariant Sections may be replaced with a single
copy.  If there are multiple Invariant Sections with the same name but
different contents, make the title of each such section unique by
adding at the end of it, in parentheses, the name of the original
author or publisher of that section if known, or else a unique number.
Make the same adjustment to the section titles in the list of
Invariant Sections in the license notice of the combined work.

In the combination, you must combine any sections Entitled ``History''
in the various original documents, forming one section Entitled
``History''; likewise combine any sections Entitled ``Acknowledgements'',
and any sections Entitled ``Dedications''.  You must delete all sections
Entitled ``Endorsements''.

\section{COLLECTIONS OF DOCUMENTS}

You may make a collection consisting of the Document and other documents
released under this License, and replace the individual copies of this
License in the various documents with a single copy that is included in
the collection, provided that you follow the rules of this License for
verbatim copying of each of the documents in all other respects.

You may extract a single document from such a collection, and distribute
it individually under this License, provided you insert a copy of this
License into the extracted document, and follow this License in all
other respects regarding verbatim copying of that document.


\section{AGGREGATION WITH INDEPENDENT WORKS}

A compilation of the Document or its derivatives with other separate
and independent documents or works, in or on a volume of a storage or
distribution medium, is called an ``aggregate'' if the copyright
resulting from the compilation is not used to limit the legal rights
of the compilation's users beyond what the individual works permit.
When the Document is included in an aggregate, this License does not
apply to the other works in the aggregate which are not themselves
derivative works of the Document.

If the Cover Text requirement of section~3 is applicable to these
copies of the Document, then if the Document is less than one half of
the entire aggregate, the Document's Cover Texts may be placed on
covers that bracket the Document within the aggregate, or the
electronic equivalent of covers if the Document is in electronic form.
Otherwise they must appear on printed covers that bracket the whole
aggregate.


\section{TRANSLATION}

Translation is considered a kind of modification, so you may
distribute translations of the Document under the terms of section~4.
Replacing Invariant Sections with translations requires special
permission from their copyright holders, but you may include
translations of some or all Invariant Sections in addition to the
original versions of these Invariant Sections.  You may include a
translation of this License, and all the license notices in the
Document, and any Warranty Disclaimers, provided that you also include
the original English version of this License and the original versions
of those notices and disclaimers.  In case of a disagreement between
the translation and the original version of this License or a notice
or disclaimer, the original version will prevail.

If a section in the Document is Entitled ``Acknowledgements'',
``Dedications'', or ``History'', the requirement (section~4) to Preserve
its Title (section~1) will typically require changing the actual
title.


\section{TERMINATION}

You may not copy, modify, sublicense, or distribute the Document
except as expressly provided under this License.  Any attempt
otherwise to copy, modify, sublicense, or distribute it is void, and
will automatically terminate your rights under this License.

However, if you cease all violation of this License, then your license
from a particular copyright holder is reinstated (a) provisionally,
unless and until the copyright holder explicitly and finally
terminates your license, and (b) permanently, if the copyright holder
fails to notify you of the violation by some reasonable means prior to
60 days after the cessation.

Moreover, your license from a particular copyright holder is
reinstated permanently if the copyright holder notifies you of the
violation by some reasonable means, this is the first time you have
received notice of violation of this License (for any work) from that
copyright holder, and you cure the violation prior to 30 days after
your receipt of the notice.

Termination of your rights under this section does not terminate the
licenses of parties who have received copies or rights from you under
this License.  If your rights have been terminated and not permanently
reinstated, receipt of a copy of some or all of the same material does
not give you any rights to use it.


\section{REVISIONS OF THIS LICENSE}

The Free Software Foundation may publish new, revised versions
of the GNU Free Documentation License from time to time.  Such new
versions will be similar in spirit to the present version, but may
differ in detail to address new problems or concerns.  See
\texttt{http://www.gnu.org/copyleft/}.

Each version of the License is given a distinguishing version number.
If the Document specifies that a particular numbered version of this
License ``or any later version'' applies to it, you have the option of
following the terms and conditions either of that specified version or
of any later version that has been published (not as a draft) by the
Free Software Foundation.  If the Document does not specify a version
number of this License, you may choose any version ever published (not
as a draft) by the Free Software Foundation.  If the Document
specifies that a proxy can decide which future versions of this
License can be used, that proxy's public statement of acceptance of a
version permanently authorizes you to choose that version for the
Document.

\section{RELICENSING}

``Massive Multiauthor Collaboration Site'' (or ``MMC Site'') means any
World Wide Web server that publishes copyrightable works and also
provides prominent facilities for anybody to edit those works.  A
public wiki that anybody can edit is an example of such a server.  A
``Massive Multiauthor Collaboration'' (or ``MMC'') contained in the
site means any set of copyrightable works thus published on the MMC
site.

``CC-BY-SA'' means the Creative Commons Attribution-Share Alike 3.0
license published by Creative Commons Corporation, a not-for-profit
corporation with a principal place of business in San Francisco,
California, as well as future copyleft versions of that license
published by that same organization.

``Incorporate'' means to publish or republish a Document, in whole or
in part, as part of another Document.

An MMC is ``eligible for relicensing'' if it is licensed under this
License, and if all works that were first published under this License
somewhere other than this MMC, and subsequently incorporated in whole
or in part into the MMC, (1) had no cover texts or invariant sections,
and (2) were thus incorporated prior to November 1, 2008.

The operator of an MMC Site may republish an MMC contained in the site
under CC-BY-SA on the same site at any time before August 1, 2009,
provided the MMC is eligible for relicensing.


\section*{ADDENDUM: How to use this License for your documents}
\addcontentsline{toc}{section}{ADDENDUM: How to use this License for your documents}

To use this License in a document you have written, include a copy of
the License in the document and put the following copyright and
license notices just after the title page:

\bigskip
\begin{quote}
    Copyright \copyright{}  YEAR  YOUR NAME.
    Permission is granted to copy, distribute and/or modify this document
    under the terms of the GNU Free Documentation License, Version 1.3
    or any later version published by the Free Software Foundation;
    with no Invariant Sections, no Front-Cover Texts, and no Back-Cover Texts.
    A copy of the license is included in the section entitled ``GNU
    Free Documentation License''.
\end{quote}
\bigskip
    
If you have Invariant Sections, Front-Cover Texts and Back-Cover Texts,
replace the ``with \dots\ Texts.''\ line with this:

\bigskip
\begin{quote}
    with the Invariant Sections being LIST THEIR TITLES, with the
    Front-Cover Texts being LIST, and with the Back-Cover Texts being LIST.
\end{quote}
\bigskip
    
If you have Invariant Sections without Cover Texts, or some other
combination of the three, merge those two alternatives to suit the
situation.

If your document contains nontrivial examples of program code, we
recommend releasing these examples in parallel under your choice of
free software license, such as the GNU General Public License,
to permit their use in free software.
\newpage
\part*{References}%
\addcontentsline{toc}{part}{References}
\bibliography{common/references,CITATION}
\end{document}

%%% Local Variables:
%%% mode: latex
%%% TeX-master: "regression"
%%% End:
